\documentclass{../../mathnotes}

\usepackage{enumerate}

\title{Modern Algebra II: Problem Set 11}
\author{Nilay Kumar}
\date{Last updated: \today}


\begin{document}

\maketitle

\subsection*{Problem 1}

Let $F$ be a field. Let $F(t)$ be the field of rational functions with coefficients in $F$, where $t$
is an indeterminate. We wish to show that $F$ is algebraically closed in $F(t)$, i.e. that if $r(t)\in F(t)$
and $r(t)$ is algebraic over $F$, then $r(t)\in F$. So suppose that $r=p(t)/q(t)$ where $p,q$ are relatively prime
in $F[t]$, as well as that $f(r)=0$ for some monic $f(x)\in F[x]$. If $r=0$, it's clear that $r\in F$, and thus
we may assume that $r\neq 0$. Now note that $F[x]$ is a subring of $F[t][x]$ and thus of $F(t)[x]$. Hence, we may
view $f(x)=\sum_{i=0}^n a_ix^i$ as a polynomial in $F[t][x]$ with $a_i\in F$. The rational roots test then tells us that
any rational root (in $F(t)[x]$, now) must have its numerator divide $a_0$ and its denominator divide $a_n=1$ (we
have assumed without loss of generality that $f(x)$ is monic). This means that out chosen $r(t)$ must be of the form
$r(t)=p$ where $p\in F$ divides $a_0$ as the only elements of $F[t]$ that divide $a_0\in F$ must be in $F$.
Hence, $r\in F$ and thus $F$ is algebraically closed in $F(t)$.

\subsection*{Problem 2}


Let $R$ be a UFD and let $f(x),g(x)\in R[x]$ be two nonzero polynomials.
Then we can write $f(x)=c(f)f_0(x)$ and $g(x)=c(g)g_0(x)$ where $f_0,g_0$ are primitive.
Taking the product, we find $f(x)g(x)=c(f)c(g)f_0(x)g_0(x)$. By the lemma we proved in class,
$f_0(x)g_0(x)$ is primitive, and thus the content $c(fg)=c(f)c(g)$.

\subsection*{Problem 3}

Let $E$ be a finite extension field of a field $F$, and let $\sigma:E\to E$ be a homomorphism such that $\sigma(a)=a$
for all $a\in F$. Note that $E$ is an $F$-vector space, and thus, since $\sigma$ is an $F$-linear map (as we discussed in class),
$\text{Im } \sigma$ must be a vector subspace; this follows simply by linearity of $\sigma$ and the fact that $\text{Im }\sigma$ contains
the identity $0\in F$: if $f(a),f(b)\in\text{Im }\sigma$, then $f(a)+f(b)=f(a+b)\in\text{Im }\sigma$ as well. Using the fact that
$E$ is a field and all homorphisms between fields are injective (and $E$ is a finite $F$-vector space), we must have that 
$\sigma$ is surjective as well as injective by dimension-counting, as injectivity implies that $\dim_F\sigma(E)=\dim_F E$.

\subsection*{Problem 4}

Consider the field $\Q(\sqrt{2},\sqrt{3})$ with $\Q$-basis $1,\sqrt{2},\sqrt{3},\sqrt{6}$. Using the definitions
for $\Gal(\Q(\sqrt{2},\sqrt{3})/\Q)=\left\{ 1,\sigma_1,\sigma_2,\sigma_3 \right\}$ from class, it is clear that the fixed
fields can be found as follows. For $\sigma_1$, which flips the sign of the $\sqrt{2}$, the fixed field consists of the
elements that are mapped to themselves under $\sigma_1$, i.e. elements that are independent of $\sqrt{2}$, i.e.
the subfield $\Q(\sqrt{3})\in\Q(\sqrt{3},\sqrt{2})$. For $\sigma_2$ we have those that are independent of the $\sqrt{3}$,
i.e. the subfield $\Q(\sqrt{2})\in\Q(\sqrt{2},\sqrt{3})$. Finally, for $\sigma_3$ we need elements that don't change
when we change the sign of both $\sqrt{2}$ and $\sqrt{3}$, i.e. elements in the subfield $\Q(\sqrt{6})\in\Q(\sqrt{3},\sqrt{2})$.

\subsection*{Problem 5}

Let $\sigma$ be complex conjugation acting on the field $\Q(\sqrt[3]{2},\omega)$, where $\omega=e^{2\pi i/3}$
is a cube root of unity, and hence a root of $x^2+x+1$. If we look at this field as an extension of $Q(\sqrt[3]{2})$,
it's clearly a finite extension of degree 2, as the degree of the irreducible polynomial for $\omega$ is 2. For
the same reason, $1$ and $\omega$ are linearly independent, as no $\Q(\sqrt[3]{2})$-linear combination of them
can yield zero (all such combinations would be necessarily quadratic). By linear algebra, then, $1,\omega$ must be
a $\Q(\sqrt[3]{2})$-basis for $\Q(\sqrt[3]{2},\omega)$. To find the fixed field $\Q(\sqrt[3]{2},\omega)^{\langle\sigma\rangle}$,
we must find the subfield of elements that is fixed by complex conjugation. Clearly any element with an imaginary part cannot
be in the fixed field, as it would suffer a sign change. Consequently the fixed field cannot have a component in the $\omega$ direction,
so to speak, and thus must be wholly in $\Q(\sqrt[3]{2})$, which is the fixed field.

\subsection*{Problem 6}

Take the polynomial $\Phi_5(x)=(x^5-1)/(x-1)$, irreducible in $\Q[x]$ of degree 4, where $\zeta=e^{2\pi i/5}$ is a root of $\Phi_5(x)$.

\begin{enumerate}[(a)]
    \item Given $\zeta$ as above, we can check that $\zeta^\alpha$ for $\alpha=1,2,3,4$ is a root of $\Phi_5(x)$:
        \begin{align*}
            \Phi_5(\zeta)&=\frac{e^{2\pi i}-1}{e^{2\pi i/5}-1}=0\\
            \Phi_5(\zeta^2)&=\frac{e^{4\pi i}-1}{e^{4\pi i/5}-1}=0\\
            \Phi_5(\zeta^3)&=\frac{e^{6\pi i}-1}{e^{6\pi i/5}-1}=0\\
            \Phi_5(\zeta^4)&=\frac{e^{8\pi i}-1}{e^{8\pi i/5}-1}=0
        \end{align*}
        simply because the numerator goes to zero. Now take the Galois group $\Gal(\Q(\zeta)/\Q)$, which is the group
        of automorphisms of $\Q(\zeta)$ that fixes $\Q$. By the theorem we proved in class, then, since $\Q(\zeta)$ is generated
        over $\Q$ by the roots of $\Phi_5(x)$ that lie in $\Q(\zeta)$, the homomorphism from the Galois group to the symmetric
        group $S_4$, viewed as the set of permutations of the set $\left\{ \zeta_1,\zeta_2,\zeta_3,\zeta_4 \right\}$, is injective.
    \item Given $\sigma\in\Gal(\Q(\zeta)/\Q)$, since $\sigma$ must fix elements of $\Q$, it must only act non-trivially
        on powers of $\zeta$. Note however, that since $\sigma$ is a homomorphism, once we know how it acts on $\zeta$, namely,
        $\sigma(\zeta)$, it follows trivially that $\sigma(\zeta^2)=\sigma^2(\zeta)$ and similarly for higher powers. Thus, since
        any element of $\Q(\zeta)$ can be written as a $\Q$-linear combination of elements of $\Q$ and powers of $\zeta$, $\sigma$ is
        completely determined by its value on $\zeta$. Of course, even this value is restricted, as from the previous part of the problem,
        $\sigma$ can only take $\zeta$ to one of $\zeta,\zeta^2,\zeta^3,\zeta^4$, and thus there are at most four possibilities
        for $\sigma(\zeta)$.
\end{enumerate}

\subsection*{Problem 7}

Suppose such an automorphism existed.
Then consider $\sigma(\sqrt[4]{2})\in \R$. We must have that $\sigma^2(\sqrt[4]{2})>0$ as it is the square of a real number.
But by the homomorphism property, we know that $\sigma^2(\sqrt[4]{2})=\sigma(\sqrt{2})=-\sqrt{2}<0$, which is a contradiction.


\end{document}
