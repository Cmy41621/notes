\documentclass{../../mathnotes}


\title{Modern Algebra II: Problem Set 4}
\author{Nilay Kumar}
\date{Last updated: \today}


\begin{document}

\maketitle

\subsection*{Problem 1}

Let $R$ be a ring with $R\neq\left\{ 0 \right\}$. We wish to show that $R$ is a field if and only if every ideal of $R$
is either $\left\{ 0 \right\}$ or $R$. If $R$ is a field, we know that every nonzero $r\in R$ has an inverse $r^{-1}$.
Take any ideal $I\subset R$. If $I$ is empty, we are done. Otherwise, $I$ contains at least one element, call it $r$.
By the ideal's absorbing property, $rr^{-1}=1$ must also be in $I$. However, we know that if $I$ contains 1, it must
contain the whole ring $R$.

Conversely, let $R$ be a ring with ideals only $\left\{ 0 \right\}$ and $R$. We wish to show that every non-zero element 
$R$ has a multiplicative inverse, $r^{-1}$. Take the ideal generated by some $r\in R$: $(r)=\left\{ rs | s\in R \right\}$.
By hypothesis, $(r)=\left\{ 0 \right\}$ or $(r)=R$. We are not interested in the former case, as it requires that $r=0$.
If $(r)=R$, on the other hand, $(r)$ must contain unity, i.e. 1 can be written as a multiple of $r$. It follows, then, that
$rs=1$ for some $s\in R$, and thus we have found a multiplicative inverse for any non-zero element $r\in R$, and thus $R$
must be a field.

\subsection*{Problem 2}

Let $F$ be a field and let $\rho:F\to R$ be a ring homomorphism. We wish to show that either $\rho$ is injective or
$R=\left\{ 0 \right\}$ and hence $\rho(a)=0$ for all $a\in F$.

Since $\ker\rho$ is an ideal of a field $F$,  $\ker F$ must either be $\left\{ 0 \right\}$ or $F$. If the kernel is
the zero element, $\rho$ must be injective. Otherwise, if $\ker\rho=F$, every element in $F$ gets mapped to zero in
$R$. However, a ring homomorphism always maps $1\to 1$, so the ring must not have a unity. As $R$ is assumed to be a
commutative ring with unity, it must be the zero ring.

\subsection*{Problem 3}

Let $I$ and $J$ be ideals of a ring $R$. We take the ideal sum to be $I+J=\left\{ r+s:r\in I,s\in J \right\}$.
Note that $I+J$ satisfies the absorbing property, because for any $r\in R$ and $k=i+j\in I+J$, the product
$rk=ri+rj\in I+J$ since the first term is in $I$ and the second is in $J$. That $R$ is an additive subgroup follows
directly from the additive properties of $I$ and $J$, so $I+J$ is an ideal in $R$.

In fact, every ideal $K$ containing both $I$ and $J$ must contain $I+J$. In other words, for any $i\in I$ and $j\in J$,
the sum $i+j$ must be in $K$. This follows from the fact that $K$ must form an additive subgroup -- i.e. the sum of
two elements in $K$ must be in $K$. As both $i,j\in K$, it is clear that $i+j\in K$, and consequently $K$ must contain
$I+J$.

\subsection*{Problem 4}

Let $I$ and $J$ be ideals in a ring $R$. We define the ideal product to be
\begin{align*}
    I\cdot J=\left\{ \sum_{i=1}^n r_is_i:r_i\in I,s_i\in J \right\}.
\end{align*}
In other words, $I\cdot J$ contains all finite sums of products of two elements, one each from $I$ and $J$. We wish
to show that $I\cdot J$ is contained in $I\cap J$, i.e. that every element of the ideal product is in $I$ as well as
$J$. First note that for all $i$, we know that $r_is_i\in I$ by the absorbing property of $r_i\in I$ and that $r_is_i\in J$
by the absorbing property of $s_i\in J$. Since both $I$ and $J$ are additive subgroups of $R$, the sum $\sum_{i=1}^nr_is_i$
must also be in both $I$ and $J$, and we are done.

\subsection*{Problem 5}

Let $r$ and $n$ be elements of the ring $\mathbb{Z}$ and let $(n)$ be the principal ideal generated by $n$. We wish to
show that $r\in(n)$ if and only if $n$ divides $r$. If $r\in (n)$, it can be written as $r=ns$ for some $s\in\mathbb{Z}$,
by definition of the ideal $(n)$, and thus $n$ divides $r$.
Conversely, if $n$ divides $r$, there exists some $s\in R$ such that $r=ns$. Since $(n)$ contains every multiple of $n$,
$r\in(n)$, and we are done.

The ideal sum $(n)+(m)$ is the set of all sums of multiples of $n$ or $m$. It contains elements such as $n, m, n+m, 2n+m,
n+2m, 2n+2m,\cdots$. The intersection $(n)\cap(m)$ is simply the set of elements of $\mathbb{Z}$ that are divisble by
both $n$ and $m$. The ideal product $(n)\cdot(m)$ on the other hand, is the set of all elements that are divisible by
$nm$. Note carefully that divisibility by $nm$ is not equivalent to divisibility by $m$ and $n$. Take, for example, the
ideals $(2)$ and $(4)$ -- the product ideal consists of all multiples of 8, whereas the intersection of the two ideals
is the set of multiples of 4; these two sets are \textit{not} the same.

\subsection*{Problem 6}

Let $S$ be a ring and $R$ a subring of $S$. On the last problem set, we showed that if $J$ is an ideal in $S$, then
$I=R\cap J$ is an ideal in $R$. Let $g$ be a map from $R$ to $S/J$ such that $g(r)=r+J$, for any $r\in R$. What is the
kernel of $f$? It is the set of all elements $r\in R$ for which $r+J=0+J$: i.e. $R\cap J=I$. By the fundamental theorem
for homomorphisms, then, we know that there exists an isomorphism $\phi:R/I\to{\rm Im} g$. Thus there is an injective map
from $R/I$ to $S/J$, namely the composition of the injective inclusion map $\iota:{\rm Im}g\to S/J$ with the isomorphsim:
$f=\iota\circ\phi:R/I\to S/J$. This map is, of course, a homomorphism, as it is the composition of an isomorphism and the
inclusion homomorphism.

We now wish to show that $f$ is surjective if and only if for every $s\in S$, there exists $r\in R$ such that
$s\equiv r\mod J$, i.e. $s-r\in J$. First note that $f$ is surjective if and only if for every $s+J\in S/J$, there exists
and $r\in R$ such that $f(r+I)=r+J$ is equal to $s+J$. This, in turn, holds if and only if $(r+J)-(s+J)=0+J$; i.e.
$s-r\in J$.

\subsection*{Problem 7}

Let $R$ be the subring $\mathbb{Z}[i]=\left\{ a+bi:a,b\in\mathbb{Z} \right\}$ of $\mathbb{C}$. Let $I=(2+3i)$ be the
principal ideal in $\mathbb{Z}[i]$ generated by $2+3i$.

It should be clear that I contains $2+3i$ and $-3+2i$, as $1(2+3i)=2+3i$ and $i(2+3i)=-3+2i$. In fact, the additive subgroup
$(I,+)$ of the group $(\mathbb{Z}[i],+)$ is generated by $2+3i$ and $-3+2i$, because any $\mathbb{Z}[i]$-multiple of $2+3i$
can be written as a sum of multiples of $2+3i$ and $-3+2i$:
\begin{align*}
    (a+bi)(2+3i)=(2+3i)a+(-3+2i)b.
\end{align*}

To determine whether an arbitrary element of $\mathbb{Z}[i]$ such as $i+5$ is in $I$, we can divide:
\begin{align*}
    \frac{i+5}{2+3i}=\frac{i+5}{2+3i}\cdot\frac{2-3i}{2-3i}=1-i,
\end{align*}
and so $i+5\in I$ as it can be written as the product of $2+3i$ and $1-i$. Consequently, we can write $i\equiv-5\mod I$.

Now consider the homomorphism $f:\mathbb{Z}\to\mathbb{Z}[i]/I$ such that $f(n)=n+I=n+(2+3i)$. To see that $f$ is
surjective, take any $n+I\in\mathbb{Z}[i]/I$. Any $m$ such that $m\equiv n\mod I$ will satisfy $f(m)=n+I$ simply
because $f(m)=m+I=n+I$ (using the equivalence), and thus, since there exist such $m$'s (a perfectly legitimate candidate
is $m(2+3i)$), $f$ must be surjective.

Note that for an integer such as 13 to be in the intersection $\mathbb{Z}\cap I$, it must be a multiple of $2+3i$.
Again, we can check this via division:
\begin{align*}
    \frac{13}{2+3i}=\frac{13}{2+3i}\cdot\frac{2-3i}{2-3i}=2-3i,
\end{align*}
and so $13\in\mathbb{Z}\cap I$. It turns out, in fact, that $\mathbb{Z}\cap I=13\mathbb{Z}$. To show this, let us
first show that $13\mathbb{Z}\subset \mathbb{Z}\cap I$ and then show that $\mathbb{Z}\cap I\subset 13\mathbb{Z}$.
Note that the computation above, after the addition of an arbitrary integer $n$ in the numerator, proves that every
integer multiple of 13 is in $(2+3i)$, and so is in $\mathbb{Z}\cap I$.
The converse, that every integer in $I$ is a multiple of 13, is checked by the usual division for any $n\in\mathbb{Z}$:
\begin{align*}
    \frac{n}{2+3i}\cdot\frac{2-3i}{2-3i}=\frac{2n-3ni}{13}.
\end{align*}
Since we know $n\in I$, the above fraction must be in $\mathbb{Z}[i]$. Consequently, $2n$ and $3n$ must be (integer)
divisible by 13. As 2, 3, and 13 are relatively prime, it follows that $n$ must be divisible by 13 as well, and
we are done.

Since $\mathbb{Z}$ is a subring of $\mathbb{Z}[i]$, and the $f$ defined earlier is a homomorphism from $\mathbb{Z}$
to $\mathbb{Z}[i]$, the previous problem tells us that
$\mathbb{Z}/\left( \mathbb{Z}\cap I \right)=\mathbb{Z}/13\mathbb{Z}\cong \mathbb{Z}[i]/(2+3i)$. In other words,
we have reached the result that $\mathbb{Z}/13\mathbb{Z}\cong \mathbb{Z}[i]/I$. As $\mathbb{Z}/13\mathbb{Z}$ is a
field, $\mathbb{Z}[i]/I$ must be a field as well, and thus $I$ is a maximal ideal (recall that an ideal $I$ of a ring
$R$ is maximal if and only if $R/I$ is a field). Of course, any maximal ideal is prime, so $I$ is a prime ideal as well.

\end{document}
