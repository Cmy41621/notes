\documentclass{../mathnotes}

\title{Notes on Algebra II}
\author{Nilay Kumar}
\date{Last updated: \today}


\begin{document}

\maketitle

\section{Reducibility}

February 25, 2013

Let $F$ be a field, and $F[x]$ be the ring of polynomials over $F$. Recall we have already shown that every ideal
in $F[x]$ is principal, and that there exists a unique gcd of two non-zero polynomials. Additionally, we showed that
if $f$ and $g$ are two relatively prime polynomials, then $f|gh\implies f|h$.

\begin{defn}
    A polynomial $p(x)\in F[x]$ is \textbf{irreducible} if $\deg p(x)>0$, i.e. $p$ is not zero and not a unit,
    and if $p=fg$ implies that one of $f,g$ is a unit and the other is a unit times $p$. In words, $p(x)$ is
    irreducible if it does not factor into a product of two polynomials with strictly smaller (non-zero)
    degree. A polynomial is said to be $\textbf{reducible}$ if it is not irreducible.
\end{defn}

\begin{exmp}(Reducibility)
    \begin{enumerate}[(i)]
        \item Any linear polynomial $x+a$ is obviously irreducible.
        \item Any quadratic polynomial is clearly reducible if and only if it has two linear factors.
            This is equivalent to the polynomial having a root, as long division will yield the second factor.
        \item Similarly, a cubic polynomial is reducible if and only if it has a root.
        \item For higher degrees, the existence of a root is not equivalent to reducibility, as we will
            see in the next example.
    \end{enumerate}
\end{exmp}

\begin{exmp}(Simple examples)
    \begin{itemize}
        \item $x^2-2$ is irreducible in $\mathbb{Q}[x]$, as it has no roots in $\mathbb{Q}$. It is, however,
            reducible in $\mathbb{R}[x]$: $x^2-2=(x-\sqrt{2})(x+\sqrt{2})$.
        \item $x^2+1$ is irreducible in $\mathbb{R}[x]$ but reducible in $\mathbb{C}[x]$: $x^2+1=(x-i)(x+i)$.
        \item $x^3-2$ is irreducible in $\mathbb{Q}[x]$, but reducible in $\mathbb{R}[x]$, where we can write
            it as a product of $x-\sqrt[3]{2}$ and an irreducible quadratic.
        \item $x^4-4=(x^2-2)(x^2+2)$ is reducible in $\mathbb{Q}[x]$ but has no roots!
    \end{itemize}
\end{exmp}
In fact, it is generally a hard problem to determine whether an arbitrary polynomial $f(x)\in\mathbb{Q}[x]$ is
irreducible. Note, however, that we can think of irreducibility in analogy to that for natural numbers, as
the following dichotomy illustrates.
\begin{rem}
    If $p(x)\in F[x]$ is irreducible, then for any polynomial $f\in F[x]$, either $p|f$ or $p$ and $f$ are
    relatively prime.
\end{rem}
\begin{proof}
    Let $d=\gcd(p,f)$. By definition, $d$ divides $p$. However, as $p$ is irreducible, $d$ must either be
    a unit or $d$ must be $cp$ for $c$ a unit. In the first case, since the gcd of $p$ and $f$ is a unit,
    $p$ and $f$ must be relatively prime. In the second case, since $d=cp$ by construction divides $f$,
    $p$ must divide $f$.
\end{proof}

\begin{cor}
    If $p\in F[x]$ is irreducible and $p|fg$, then either $p|f$ or $p|g$.
\end{cor}
\begin{proof}
    By the above remark, either $p|f$ or $p$ and $f$ are relatively prime.
    If $p|f$, we are done. Otherwise, $p$ is relatively prime to $f$, and by what we showed last class,
    $p|g$.
\end{proof}

\begin{thm}[Unique factorization of polynomials]
    Let $f(x)\in F[x]$ with $\deg f(x)>0$. Then there exist $k$ irreducible polynomials in $F[x]$ such
    that
    \[f(x)=\prod_{i=1}^k p_i(x).\]
    Additionally, if it is also true that $f(x)=\prod_{i=1}^lq_i$, then $k=l$, and after some reordering,
    there exist nonzero constants such that $q_i=c_ip_i$.

    In other words, for any polynomial with degree greater than zero, there always exists a unique
    factorization into a product of irreducible polynomials.
\end{thm}
\begin{proof}
    Let us first show existence. We proceed by complete induction on the degree of $f$.
    If $\deg f=1$, $f$ is irreducible, and we are done. Otherwise, we assume that the theorem
    holds for all degrees less than $n$. Let $\deg f=n$. If $f$ is irreducible, we are done.
    Otherwise, $f=g_1g_2$ with $\deg g_1<n$ and $\deg g_2<n$. By the inductive hypothesis,
    $g_1$ and $g_2$ are products of irreducible polynomials, and thus $f$ must be as well, and we are done.

    The real muscle of this theorem comes in the form of uniqueness. Suppose $f=\prod_{i=1}^kp_i=\prod_{j=1}^lq_j$,
    with $p_i,q_j$ reducible. We proceed by induction on $k$. If $k=1$, $p_1=q_1\cdots q_l$. Clearly, then,
    $p_1|q_1\cdots q_l$, and thus (by induction over the statement at the beginning of lecture), $p_1$ must divide
    $q_i$ for some $i$. But the $q_i$ are irreducible and $p_1$ is not a constant, so $p_1=cq_i$ for some unit
    $c$. If we now reorder terms, we can assume that $i=1$ and we can cancel:
    \begin{align*}
        p_1=cq_1&=q_1q_2\cdots q_l\\
        c&=q_2\cdots q_l.
    \end{align*}
    But this is impossible, as the product of $q$'s has degree greater than zero. Consequently, $l$ must be 1, and
    thus $p_1=q_1$ and we have shown that $k=l$. The general case is similar; we write $p_1\cdots p_k=q_1\cdots q_l$. Then $p_1|q_1\cdots q_l$,
    and so for some $i$, $p_1=cq_i$. After reordering, we can write
    \begin{align*}
        cq_1p_2\cdots p_k&=q_1\cdots q_l\\
        cp_2\cdots p_k&=q_2 \cdots q_l,
    \end{align*}
    and by induction, we know that $k-1=l-1$. Reordering, we can write $p_=c_iq_i$ for $i=2\cdots k$,
    and we are done.
\end{proof}
Note that the irreducible factors need not be distinct.

\begin{thm}
    Let $F$ be a field. Let $I$ be an ideal in $F[x]$. Then the following are equivalent:
    \begin{enumerate}[(i)]
        \item $I$ is a maximal ideal.
        \item $I$ is a prime ideal and $I\neq \left\{ 0 \right\}$.
        \item $I=(p)$, where $p$ is a irreducible polynomial.
    \end{enumerate}
\end{thm}
\begin{proof}
    Let us first show that $(i)\implies(ii)$.
    Say $I$ is maximal. Then, $I$ must be prime. Additionally, $I$ cannot be the zero ideal, as it is not
    maximal, and so we are done.

    Showing $(ii)\implies(iii)$ is a little trickier. Suppose $I$ is a prime ideal with $I\neq \left\{ 0 \right\}$.
    We want to show that the ideal is generated by an irreducible element. Since every ideal in $F[x]$ is principal,
    $I=(p)$ for some $p\in F[x]$. Let us show that $p$ is irreducible. First note that $p$ cannot be a unit, because
    otherwise $1\in(p)$ which implies that $(p)=F[x]$, which is not possible for prime ideals. Furthermore,
    $p\neq0$, as $I$ is assumed not to be the zero ideal. To show that $p$ is irreducible, we need to show that if
    $p=fg$ then one of $f,g$ is a unit and the other is a unit times $p$. So take $p=fg$. Then, $fg\in (p)=I$.
    Since $I$ is prime, either $f\in I$ or $g\in I$. Take the first case, $f\in(p)$. Then, $f=hp$ for some $h\in F[x]$,
    and so $p=hpg\implies 1=hg$, i.e. $h,g$ are units, and thus $f$ is a unit times $p$.
    Thus, $p$ is irreducible.

    Finally, we show that $(iii)\implies(i)$. Let $I=(p)$, with $p$ irreducible. We wish to show that $I$ is maximal,
    i.e. $(p)\neq F[x]$ and if $(p)\subset J$ then either $J=(p)$ or $J=F[x]$. First note that $(p)\neq F[x]$ because
    $\deg p>1$ and so it can't generate constants. Next, since $J$ is necessarily a principal ideal, $J=(f)$, for some
    $f\in F[x]$. If $(p)\subset (f)$, then $p\in(f)$, so $p=fg$ for some $g\in F[x]$. But $p$ is irreducible, so either
    $f$ is a unit, in which case $J=(f)=F[x]$, or $f=cp$, for $c$ a unit, in which case $J=(f)=(p)$. Hence, $I$ is
    maximal.
\end{proof}
This theorem is quite handy in constructing interesting fields, as the following corollary shows.
\begin{cor}
    $F[x]/(f)$ is a field if and only if $f$ is irreducible.
\end{cor}
\begin{proof}
    This follows from above theorem and the fact that $F[x]/(f)$ is a field if and only if $(f)$ is a maximal ideal.
\end{proof}

This allows us to show that certain rings are, in fact, fields -- something that may not have been obvious -- or, in fact,
to find wholly new fields.
\begin{exmp}{\ \\} 
    \begin{itemize}
        \item $\mathbb{Q}[x]/(x^2-2)$ is a field, as $x^2-2$ is irreducible in $\mathbb{Q}[x]$, and its elements,
            by what we know about long division, are of the form $c+d\alpha$, where $\alpha=x+(x^2-2)$. In addition,
            $\alpha^2=2$.
        \item $\mathbb{R}[x]/(x^2+1)$ is a field, as $x^2+1$ is irreducible in $\mathbb{R}[x]$, and its elements are of
            the form $c+d\alpha$ where $\alpha=x+(x^2+1)$ satisfies $\alpha^2=-1$.
        \item $\mathbb{Q}[x]/(x^3-2)$ is a field, as $x^3-2$ is irreducible in $\mathbb{Q}[x]$, and its elements are of
            the form $c+d\alpha+e\alpha^2$, where $\alpha=x+(x^3-2)$ satisfies $\alpha^3=2$. We often rewrite the elements
            as $c+d\sqrt[3]{2}+e\sqrt[3]{2}^2$.
        \item Take the finite field $\mathbb{F}_2$ and the polynomial $x^2+x+1\in\mathbb{F}_2$. Since the only members of
            $\mathbb{F}_2$ are 0 and 1, it should be clear that this polynomial has no roots, and thus is irreducible in $\mathbb{F}_2[x]$.
            Consequently, $E=\mathbb{F}_2[x]/(x^2+x+1)$ is a field. Its elements are of the form $c+d\alpha$, where of course $c,d\in\mathbb{F}_2$
            and $\alpha=x+(x^2+x+1)$, which satisfies the property that $\alpha^2=-\alpha-1=\alpha+1$. $E$ has four elements
            (since $c$ and $d$ can each take 2 values).
    \end{itemize}
\end{exmp}


\section{Field extensions}
February 27, 2013

In general, a problem in algebra is to enlarge the domain of discourse, i.e. $\mathbb{R}\to\mathbb{C}$, so that one can solve
equations that were hitherto unsolvable. So given a polynomial $f(x)\in F[x]$, we wish to find a root of $f(x)$. Maybe there is
not root of $f(x)$ in $F$, so we wish to enlarge $F$.
A simple idea that we saw earlier was that if $R$ is any ring with $f(x)\in R[x]$, there was a way to enlarge $R$.
We consider $R[x]/(f(x))$, which always has a root $x+(f(x))=\alpha$. By construction, $f(\alpha)=0$.

There is a problem with this -- we don't know much about the algebraic structure of this new quotient ring, $R[x]/I$.
The solution is: if $R$ is a field, and $f(x)$ is irreducible, then $R[x]/I$ is good, i.e. it is a field, as we
saw last lecture. But really, even if $f(x)$ is reducible, we should think in analogy to the world of
$\mathbb{Z}/n\mathbb{Z}$, where $n>0$. The full details of this analogy are fleshed out in the file \texttt{analogy.pdf}.

\begin{thm}
    Let $f(x)\in F[x]$ and suppose $\deg f(x)>0$, i.e. $f$ is nonconstant. Then, there exists a field $E$ that
    contains (a subfield isomorphic to) $F$, and an element $\alpha\in E$ such that $f(\alpha)=0$.
\end{thm}
\begin{proof}
    Take $f(x)$ and find an irreducible factor (we know these exist from last time) $p(x)$. We consider $E=F[x]/(p(x))$;
    $E$ is a field. Additionally, there is a map $F\to E$ that sends $a\in F\mapsto a+(p(x))$. This map is injective (why?),
    and we identify $F$ with the image subfield. We know that if $\alpha=x+(p(x))$, then $p(\alpha)=0$. But
    $p(x)|f(x)$, so $f(\alpha)=0$ as well, and we are done.
\end{proof}

\begin{cor}
    Let $f(x)\in F[x]$, $\deg f(x)>0$. Then there exists a field $E$ such that, in $E[x]$, $f(x)$ is a product
    of linear factors.
\end{cor}
\begin{proof}
    We apply the above theorem to find $E_1$ and $\alpha_1\in E_1$ such that $f(\alpha_1)=0$ ($F\leq E_1$).
    In $E_1[x]$, $f(x)=(x-\alpha_1)g_1(x)$, where $\deg g_1(x)=\deg f(x)-1$. Informally speaking, now all we have to do
    is to keep going! We find $E_2$ with $E_1\leq E_2$ such that we can write $g_1(x)=(x-\alpha_2)g_2(x)$ with $\alpha_2\in E_2$ with
    $\deg g_2=\deg f(x)-2$. We continute until we run out of degrees, and clearly we have factored $f$ into linear factors.
\end{proof}

Let us now switch perspectives. Let's consider the situation where $E$ and $F$ are fields, and $F\leq E$, i.e. $F$ is a subfield
of $E$. We also say that $E$ is an \textbf{extension field} of $F$. Note that we used the machinery of prime and maximal ideals to
construct extensions. Now, given an extension, let us use these tools to analyze these fields.

Consider $E$ an extension field of $F$, and let $\alpha\in E$. Look at ${\rm ev}_\alpha: F[x]\to E$. Note that ${\rm Im\;} {\rm ev}_\alpha=F[\alpha]\leq E$.
At this point, all we know is that $F[\alpha]$ is an integral domain. We claim that there are exactly 2 possibilities.
\begin{enumerate}
    \item $\ker {\rm ev}_\alpha=\left\{ 0 \right\}$, i.e. that if $f(x)\in F[x]$, $f(x)\neq 0$, then $f(\alpha)\neq 0$.
        In this case, we say that $\alpha$ is \textbf{transcendental} over $F$. Additionally, ${\rm ev}_\alpha:F[x]\to E$
        is injective, which suggests that it extends to an injection $F(x)\to E$, whose image we call $F(\alpha)$. Elements
        of this image have the form $f(\alpha)/g(\alpha)$ where $f,g\in F[x]$ and $g\neq 0$. This is the smallest subfield of
        $E$ containing $F$ and $\alpha$.

        A famous example is that $\pi \in \mathbb{R}$ is transcendental over $\mathbb{Q}$ (Lindemann, 1880). The same holds
        for $e$. Note carefully, that $\pi\in \mathbb{R}$ is not transcendental over $\mathbb{R}$, as $\pi$ is a root of $x-\pi$.c w
    \item $\ker {\rm ev}_\alpha\neq \left\{ 0 \right\},$ i.e. there exists an $f(x)\in F[x]$, $f\neq 0$ such that $f(\alpha)=0$.
        In this case, we say that $\alpha$ is \textbf{algebraic} over $F$. What can we say here? An incredible amount, it turns out.

        First note that $\ker {\rm ev}_\alpha$ is a principal ideal in $F[x]$: $\ker {\rm ev}_\alpha=(p(x))$. Additionally,
        we know that $F[\alpha]={\rm Im\;}{\rm ev}_\alpha\cong F[x]/\ker{\rm ev}_\alpha$. But since $F[x]$ is an integral domain
        (subring of a field), $(p(x))$ is a prime ideal that is not $\left\{ 0 \right\}$. This means that $(p(x))$ is a maximal ideal
        and $p(x)$ is irreducible (by theorem proved last time). Consequently, $F[x]/(p(x))=F[x]/\ker{\rm ev}_\alpha$ is a field, and
        $F[\alpha]$ is a field. Recall the example of $\mathbb{Q}[\sqrt[3]{2}]$ being a field. Thus, we now write $F[\alpha]=F(\alpha)$,
        which is the smallest subfield of $E$ containing $F$ and $\alpha$.

        In particular, there is a unique monic generator of $(p(x))=\ker{\rm ev}_\alpha$. It is denoted $\irr(\alpha,F,x)$, which
        is read ``the irreducible polynomial for $\alpha$ over $F$.'' It satisfies $\irr(\alpha,F,\alpha)=0$. Let us do a few examples:
        \begin{exmp}
            \[\irr(\frac{1}{2},\mathbb{Q},x)=x-\frac{1}{2}\]
            \[\irr(\sqrt[3]{2},\mathbb{Q},x)=x^3-2\]
            \[\irr(\sqrt[3]{2},\mathbb{Q}(\sqrt[3]{2}),x)=x-\sqrt[3]{2}\]
            \[\irr(\sqrt[3]{2},\mathbb{Q}(\sqrt{2}),x)=?\]
        \end{exmp}

        \begin{rem}
        Note that if $f(x)\in F[x]$ is any polynomial such that $f(\alpha)=0$, then the $\irr(a,F,x)|f(x)$.
        \end{rem}
        \begin{proof}
            $f(\alpha)=0\iff f(x)\in\ker {\rm ev}_\alpha=(\irr(\alpha,F,x))$. By definition, then, $\irr(\alpha,F,x)|f(x)$.
        \end{proof}
        For example, take $f(x)\in \mathbb{R}[x]$. If $f(i)=0$, then $x^2+1|f(x)$.
\end{enumerate}

Mostly we will be working with the algebraic case, as the transcendental case belongs in a separate course.

\begin{defn}
    Given $F\leq E$ and $\alpha\in E$ algebraic over $F$, we define the \textbf{degree} of $\alpha$ over $F$ as $\deg\irr(\alpha,F,x)$.
\end{defn}
For example, $\deg_{\mathbb{Q}}\sqrt[3]{2}=3$ and $\deg_{\mathbb{R}}\sqrt[3]{2}=1$ and $\deg_{\mathbb{R}}i=2$.

\begin{defn}
    Let $F\leq E$. Then we say that $E$ is a \textbf{simple extension} of $F$ if $E=F(\alpha)$ for some $\alpha\in E$.
\end{defn}
Roughly speaking, this means that we can extend $F$ to $E$ by throwing in only one more element -- i.e. we can do this if $\alpha$ is transcendental.
Take, for example, $\mathbb{Q}(\sqrt{2})$. $\sqrt{3}\notin\mathbb{Q}(\sqrt{2})$. Then, $\mathbb{Q}(\sqrt{2})\leq \mathbb{Q}(\sqrt{2})(\sqrt{3})=\mathbb{Q}(\sqrt{2},\sqrt{3}).$
Then, $\mathbb{Q}\leq \mathbb{Q}(\sqrt{2}+\sqrt{3})$, one can find that $\mathbb{Q}(\sqrt{2}+\sqrt{3})=\mathbb{Q}(\sqrt{2},\sqrt{3})$. This shows
that simple extensions are not always obviously simple.

\section{$F$-vector spaces}
Let us take a detour through vector spaces.

Let $F$ be a field. An \textbf{$F$-vector space} is an abelian group $(V,+)$ and a function $F\times V\to V$ called \textbf{scalar multiplication}, which
we write as $av$ such that:
\begin{enumerate}
    \item $a(bv)=(ab)v$
    \item $a(v+w)=av+aw$
    \item $(a+b)v=av+bv$
    \item $1\cdot v=v$
\end{enumerate}

A very useful example is $V=F^n=\left\{ (a_1\cdots a_n): a_i\in F \right\}$, i.e. the Cartesian product of $F$ with itself $n$ times.
We define addition componentwise, and multiply the scalar through each component, as usual. It is easy to check that $F^n$ is a vector space.
The $n=0$ case is allowed, as it is the zero vector space with only 0.

Another example is the space of functions $X\to F$ on any set $X$, which we denote by $F^X$. Functions are added pointwise as usual,
and scalar multiplication is done pointwise as well.

The important example for us is actually a bit unexpected. Suppose $E$ is an extension field of $F$. Then $E$ is an $F$-vector space.
$E$ is already an abelian group and scalar multiplication is defined in the ordinary sense of multiplication. The rest of the axioms
follow straightforwardly. Now, this is not as strange as it might look. The complex numbers, for example, are an extension field of the reals,
and we are used to going back and forth between numbers/vectors: $a+bi\iff (a,b)$. Similarly, $\mathbb{Q}(\sqrt[3]{2})$ is a $\mathbb{Q}$-vector
space as $a+b\sqrt[3]{2}+c\sqrt[3]{2}^2\iff (a,b,c)$. Furthermore, $\mathbb{R}$ is a $\mathbb{Q}$-vector space (but a really big one).

One might ask, why does one require $F$ to be a field? We can define a similar structure for a ring.
\begin{defn}
    If $R$ is a commutative ring with unity, then an \textbf{$R$-module} $M$ is an abelian group $(M,+)$ with a scalar multiplication
    $R\times M\to M$ that satisfies the properties defined above for $F$-vector spaces.
\end{defn}
These are, however, more interesting for algebra in general, and not so much for our case, where we will extensively use the field properties.

%\subsection{Back to field extensions}

(March 4, 2013)

%Recall what we discussed about field extensions earlier. It should be clear that every element of $F(\alpha)$ is uniquely written
%as $c_0+c_1\alpha+\ldots+d_{d-1}\alpha^{d-1}$ where $d=\deg \irr(\alpha,F,x)$ and $c_i\in F$. This is because every coset
%in $F[x]/(\irr(\alpha,F,x))$ is uniquely $g(x)+(\irr(\alpha,F,x))$ with $\deg f(x)< d$ (or $g=0$). Take, for example,
%$\Q(\sqrt[3]{2})$. Every element here is uniquely written as $a+b\sqrt[3]{2}+c(\sqrt[3]{2})^2$, with $a,b,c\in\Q$. We know,
%of course, that $\sqrt[3]{2}^3=2$. Uniqueness here follows by assuming it can be written via $a_1,b_1,c_1$ and $a_2,b_2,c_2$,
%equating the two and subtracting one to the other side, and then noting that if at least one of the coefficients were not zero,
%then we would have a polynomial with degree $\leq 2$ that evaluated at $\sqrt[3]{2}$ would yield zero. However, we know that
%$\irr(\sqrt[3]{2},\Q,x)=x^3-2$, so we reach a contradiction.

What we wish to do with these ideas is to use linear algebraic ideas to understand more deeply field extensions.

Let's talk about a few basic notions.
\begin{defn}
    A \textbf{vector subspace} of an $F$-vector space $V$ is a subgroup $W$ of $(V,+)$ such that for all $a\in F$,$w\in W$, $aw\in W$.
    $W$ then becomes an $F$-vector space in its own right.
\end{defn}

\begin{defn}
    $F:V_1\to V_2$ is a \textbf{linear map} if
    \begin{enumerate}
        \item $f$ is a homomorphism of abelian groups
        \item $f$ preserves scalar multiplication: $a,b\in F, v\in V_1$, $f(av)=af(v)$
    \end{enumerate}
    A \textbf{linear isomorphism} is just a bijective linear map. Its inverse is linear as well.
\end{defn}

\begin{defn}
    Let $V$ be an $F$-vector space and let $v_1,\ldots v_n\in V$. Then a \textbf{linear combination} of these vectors is
    an element of v of the form $a_1v_1 + \ldots + a_nv_n$. We define the \textbf{span} of this set of vectors as the set of all such linear combinations.
    It should be clear that the span of a set of vectors is a vector space.
\end{defn}

\begin{defn}
    $V$ is \textbf{finite dimensional} if there exists a set of vectors in $V$ whose span equals $V$.
\end{defn}
Note, for example, that $F^n$ is finite dimensional (via the standard basis), but $F[x]$ is not. However, $F[x]$, does have many
interesting finite dimensional subspaces. If we define $P_n$ to be the set of polynomials in $F[x]$ with degree $n$ or less, it 
forms a vector subspace spanned by $1, x, x^2, \ldots, x^n$.

\begin{defn}
    A set of vectors $v_1, \ldots, v_n\in V$ are \textbf{linearly independent} if the only linear combination of them that yields
    zero is where the coefficients in the linear combination are all zero.
\end{defn}

\begin{thm}
    If $V$ is an $F$-vector space, and $v_1,\ldots, v_n\in V$ are linearly independent, and $w_1,\ldots, w_m$ span $V$, then
    $n\leq m$.
\end{thm}

\begin{defn}
    $v_1,\ldots v_n$ is a \textbf{basis} of $V$ if these vectors are linearly independent, and they span $V$.
\end{defn}

\begin{thm}
    If $v_1,\ldots, v_n$ and $w_1,\ldots, w_n$ are two bases of $V$, then $n=m$. In this case, we define
    this number $n=\dim_F V$.
\end{thm}
\begin{proof}
    By the counting theorem above, $n\leq m$ and $m\leq n$, so $n=m$.
\end{proof}

The main example that will be important for us to consider is as follows. Let $f(x)\in F[x]$ with $\deg f(x)=n$ and $f\notin F$.
Consider $F\leq F[x]/(f(x))$. In fact, we know that every element in the coset ring is uniquely of the form
$g(x)+(f(x))$ where $g$ is zero or $\deg g(x)<n$.  This says that the cosets $1+(f(x)), x+(f(x)), \ldots, x^{n-1}+(f(x))$ are an
$F$-basis for $F[x]/(f(x))$. This implies that $\dim_F F[x]/(f(x))=n$.

\begin{cor}
    Let $F\leq E$ and $\alpha$ algebraic over $F$. Let $\deg_F\alpha=\deg\irr(\alpha,F,x)$.
    Then $F(\alpha)$ is a finite dimensional $F$-vector space: $\dim_F F(\alpha)=\deg_F \alpha=\deg\irr(\alpha,F,x)$.
\end{cor}
We have already seen a few examples: $\dim_\R\C=2$ and $\dim_\Q\Q(\sqrt[3]{2})=3$.

Let's recall some more important linear algebra facts.

\begin{thm}
    Let $V$ be a finite-dimensional $F$-vector space. Then,
    \begin{enumerate}
        \item Any set $v_1,\ldots, v_n\in V$ that spans $V$ contains a subset which is a basis.
        \item Any set $v_1, \ldots, v_n\in V$ which is linearly independent can be completed to a basis.
        \item If $W$ is a vector subspace of $V$, then $\dim W\leq \dim V$. If $\dim W=\dim V$, then $W=V$.
        \item If $V_1$ and $V_2$ are two finite-dimensional vector spaces with bases given by $v_n$, $w_m$,
            and $f:V_1\to V_2$ is a linear map, then
            \[f(v_i)=\sum_{j=1}^ma_{ji}w_j\]
            where $A=(a_{ij})$ is an $m\times n$ matrix. $f$ determines and is determined by $A$.
    \end{enumerate}
\end{thm}

\begin{rem}
    Suppose $F$ is a finite field, with $q$ elements. Suppose $V$ is a finite-dimensional $F$-vector space of dimension $n$,
    i.e. there exists a basis $v_1,\ldots, v_n$ of $V$ where every vector can be written uniquely in terms of this basis.
    It should be clear that the number of elements in $V$ is $q^n$. In fact, any finite dimensional vector space of dimension
    $n$ is isomorphic to $F^n$.
    
    In particular, if $F$ itself is finite, its characteristic must be a prime $p$, and $\F_p\leq F$.
    Thus, any finite field is an $\F_p$-vector space. Since $F$ is also finite-dimensional, the number of elements in $F$
    is simply $p^k$.
\end{rem}

\begin{defn}
    Let $E$ be an extension field of $F$. Then $E$ is a \textbf{finite extension} of $F$ if $E$ is a finite-dimensional
    $F$-vector space. In this case, we define $\dim_F E=[E:F]$, the \textbf{degree of $E$ over $F$}.
\end{defn}
\begin{exmp}
    $\C$ is a finite extension of $\R$, as $[\C:\R]=2$. Similarly, $\Q(\sqrt{2})$ is a finite extension of $\Q$ with
    $[\Q(\sqrt{2}):\Q]=2$. Again, $[\Q(\sqrt[3]{2}):\Q]=3$. However, $\R$ is not a finite extension of $\Q$, as $\R$ is
    an infinite-dimensional $\Q$-vector space.

    Furthermore, consider $F$ any field, a subring of $F[x]$. Then, $F\leq F(x)$, but $F(x)$ is not a finite extension
    of $F$ since $F\subset F[x]\subset F(x)$.
\end{exmp}

\begin{thm}
    Let $E=F(\alpha)$ be a simple extension of $F$. Then, $E$ is a finite extension of $F$ if and only if
    $\alpha$ is algebraic over $F$. In this case, $[E:F]=\deg_F\alpha=\deg\irr(\alpha,F,x)$.
\end{thm}

Note that if $\alpha$ is transcendental, $F[\alpha]\cong F[x]$ (because there is no kernel), which is not a field,
but $F[\alpha]\subset F(\alpha)\cong F(x)$. In particular, $F(\alpha)$ (quotients) is the smallest subfield of $E$ containing $F$
and $\alpha$. Since $F(\alpha)\cong F(x)$, this is not a finite extension of $F$.

If $\alpha$ is algebraic over $F$, $\ker{\rm ev}_\alpha$ is a maximal ideal and thus $F[\alpha]={\rm Im}\;{\rm ev}_\alpha$
which is already a field, that we denote by $F[\alpha]=F(\alpha)$. Again, it is the smallest subfield of $E$ containing
$F$ and $\alpha$. In this case $F(\alpha)\cong F[x]/\ker{\rm ev}_\alpha\cong F[x]/(\irr(\alpha,F,x)).$ This forms a 
finite-dimensional $F$-vector space with basis $1,\alpha,\ldots,\alpha^{d-1}$, where $d=\deg\irr(\alpha,F,x)$. In other words,
every element of $F(\alpha)$ is uniquely written $c_0+c_1\alpha+\ldots+c_{d-1}\alpha^{d-1}$, $c_i\in F$.

One piece of notation: if $F\leq E$, we take $\alpha_1,\ldots, \alpha_n\in E$. We can define $F(\alpha_1,\ldots,\alpha_n)$
as the smallest subfield of $E$ containing $F$ and $\alpha_1,\ldots,\alpha_n$. It is easy to check that there is an inductive
definition $F(\alpha_1,\ldots, \alpha_n)=F(\alpha_1,\ldots \alpha_{n-1})(\alpha_n)$. For example, we could look at $\Q(\sqrt{2},\sqrt{3})$.
This is, in fact, the same thing is as $\Q(\sqrt{2})(\sqrt{3})$. If $\alpha=\sqrt{2}+\sqrt{3}$, $\Q(\alpha)\leq \Q(\sqrt{2},\sqrt{3})$.
On the other hand, by an explicit check, one can show that $\Q(\alpha)=\Q(\sqrt{2},\sqrt{3})$. It is enough to show that
$\sqrt{2}\in\Q(\alpha)$ and $\sqrt{3}\in\Q(\alpha)$.

\begin{lem}
    Let $E$ be a finite extension of $F$ and let $V$ be a finite dimensional $E$-vector space. Then $V$ is a finite-dimensional $F$-vector space
    and $\dim_F V=[E:F]\dim_E V$.
\end{lem}
\begin{proof}
    There exists a basis for $V$ as a vector space over $E$, say $w_1\cdots w_n$, with $n=\dim_E V$. Additionally, there exists
    a basis for $E$ as a vector space over $F$, say $\alpha_1\cdots\alpha_m$, where $m=\dim_F E$. Consider the collection
    $\alpha_i w_j$, which has $mn$ elements. We claim that $\alpha_i w_j$ is an $F$-basis for $V$. Once we prove the claim, we see
    that $V$ is a finite-dimensional $F$-vector space, and that $\dim_F V=mn=[E:F]\dim_E V$.

    Let us first show that $\alpha_i w_j$ spans $V$ over $F$. We know that $w_j$ span $V$ over $E$, i.e. for all $v\in V$,
    there exist $\beta_j\in E$ such that $v=\sum_{j=1}^n\beta_j w_j$. But we also know that $\alpha_i$ span $E$ over $F$, i.e.
    $\beta_j=\sum b_{ij}\alpha_i$, with $b_{ij}\in F$. Putting these together, we find 
    \[v=\sum_{j=1}^n\left( \sum_{i=1}^n b_{ij} \alpha_i \right)w_j=\sum_{i,j=1}^{m,n}b_{ij}(\alpha_iw_j),\]
    as desired.

    Next we show that $\alpha_i w_j$ are linearly independent. Suppose $\sum_{ij}b_{ij}\alpha_iw_j=0$ -- we wish to show that
    $b_{ij}=0$ for all $i,j$. We simply rewrite the sum as
    \begin{align*}
        \sum_{j=1}^n\left( \sum_{i=1}^mb_{ij}\alpha_i \right)w_j=0,
    \end{align*}
    but by linear independence of $w_j$, the inner sums must be zero, but if the inner sum is zero, by linear independence of
    $\alpha_i$, $b_{ij}=0$ for all $i,j$, and we are done.
\end{proof}

\begin{thm}
    Suppose $F\leq E\leq K$, all fields. Then $K$ is a finite extension of $F$ if and only if $K$ is a finite extension of $E$
    and $E$ is a finite extension of $F$. Furthermore, $[K:F]=[K:E][E:F]$.
\end{thm}
\begin{proof}
    Let us start with the forwards direction. If $K$ is a finite extension of $F$, then, $K$ is a finite dimensional $F$-vector space.
    In particular, $E$ is a vector subspace of $K$. But this implies that $\dim_F E$ is also finite, which in turn implies that
    $E$ is a finite extension of $F$. Also, $K$ is spanned over $F$ by a finite number of elements $\alpha_1,\ldots, \alpha_n\in K$.
    But notice that the span of the $\alpha_i$ with $E$-coefficients includes the span with $F$-coefficients. This implies that
    $\alpha_i$ span $K$ over $E$. In other words, for all $\alpha\in K$, $\alpha=\sum f_i\alpha_i$ with $f_i\in F$. Since $F\subset E$,
    $\alpha$ is in the $E$-span of $\alpha_1\ldots \alpha_n$, which implies that $K$ is spanned over $E$ by $\alpha_1, \ldots, \alpha_n$.

    The other direction is trivial via the above lemma; take $V=K$ -- then $\dim_FK=[K:F]$, and we are done.
\end{proof}

\begin{cor}
    Both $[K:E]$ and $[E:F]$ divide $[K:F]$.
\end{cor}
\begin{proof}
    Obvious.
\end{proof}

\begin{exmp}
    Is $\sqrt{2}\in\Q(\sqrt[3]{2})$? No. Why? Because if there were, we'd have
    $\Q\leq\Q(\sqrt{2})\leq\Q(\sqrt[3]{2})$. But $[\Q(\sqrt{2}):\Q]=2$ and $[\Q(\sqrt[3]{2}):\Q]=3$, but two does not divide three,
    so we are done by contradiction.
\end{exmp}

Note that the above proof shows that if  $\alpha_1\cdots\alpha_m$ is a basis for $E$ over $F$ and $\beta_1\cdots\beta_n$ is a basis 
for $K$ over $E$, $\alpha_i\beta_j$ is a basis for $K$ over $F$.

\begin{exmp}
    $\Q\leq \Q(\sqrt{2})\leq\Q(\sqrt{2},\sqrt{3})$ and $\sqrt{3}\neq \Q(\sqrt{2})$. We have
    $[\Q(\sqrt{2},\sqrt{3}):\Q]=[\Q(\sqrt{2},\sqrt{3}):\Q(\sqrt{2})][\Q(\sqrt{2}):\Q]$ where the second term is 2.
    But we have $\Q(\sqrt{2},\sqrt{3})=\Q(\sqrt{2})(\sqrt{3})$ and $[\Q(\sqrt{2},\sqrt{3}):\Q(\sqrt{2})]$ is the degree
    of $\irr(\sqrt{3},\Q(\sqrt{3}),x)$. What is this? It's $x^2-3$, but this is irreducible in $\Q(\sqrt{2})[x]$ since it has
    no root in $\Q(\sqrt{2})$. Thus this degree is 2, and $[\Q(\sqrt{2},\sqrt{3}):\Q]=4$.

    Note $1, \sqrt{2}$ is a basis for $\Q(\sqrt{2})$ over $\Q$ and the basis for $\Q(\sqrt{2})(\sqrt{3})$ over $\Q(\sqrt{2})$ is 
    $1,\sqrt{3}$ (because in general, $[F(\alpha):F]=1,\alpha,\cdots,\alpha^{d-1}$). Then the basis for $\Q(\sqrt{2},\sqrt{3})$
    over $\Q$ is $1,\sqrt{2},\sqrt{3},\sqrt{6}$.

    But note that we know if $\alpha=\sqrt{2}+\sqrt{3}$, $\Q(\sqrt{2},\sqrt{3})=\Q(\alpha)$. We know that $\alpha$ is a root of
    $x^4-10x^2+1$ and thus $\irr(\alpha,\Q,x)$ divides this polynomial. But $\deg\irr(\alpha,\Q,x)=[\Q(\alpha):\Q]=[\Q(\sqrt{2},\sqrt{3}):\Q]=4$.
    Consequently, $x^4-10x^2+1$ is irreducible in $\Q[x]$ and it is the $\irr(\alpha,\Q,x)$. Then, $1,\alpha,\alpha^2,\alpha^3$ is another $\Q$-basis
    for $\Q(\sqrt{2},\sqrt{3})$.
\end{exmp}

\begin{defn}
    Let $F\leq E$. We say that $E$ is an \textbf{algebraic extension} of $F$ if for all $\alpha\in E$, $\alpha$ is algebraic over $F$.
\end{defn}

\begin{exmp}
    $\R$ is not an algebraic extension of $\Q$.
\end{exmp}

\begin{thm}
    If $E$ is a finite extension of $F$, then $E$ is an algebraic extension of $F$.
\end{thm}
\begin{proof}
    If $\alpha\in E$ is not algebraic over $F$, then we have $F\leq F(\alpha)\leq E$. But we've seen that $F(\alpha)$ is not
    finite dimensional as an $F$-vector space. But this contradicts that $E$ is finite-dimensional. Thus every $\alpha$ is
    algebraic.
\end{proof}

\begin{thm}
    Suppose $F\leq E$ and $\alpha,\beta\in E$ are both algebraic over $F$. Then, $a\pm \beta, \alpha\beta, \alpha/\beta$
    are also algebraic over $F$ (where $\beta\neq 0$ for division).
\end{thm}
\begin{proof}
    Note that $F\leq F(\alpha)\leq F(\alpha,\beta)=F(\alpha)(\beta)$. We know that $\alpha$ is algebraic over $F$, which
    implies that $F(\alpha)$ is a finite extension of $F$. If $\beta$ is algebraic over $F$, it is clearly algebraic over $F(\alpha)$.
    This implies that $F(\alpha)(\beta)$ is a finite extension of $F(\alpha)$. Thus, $F(\alpha,\beta)$ is a finite extension of $F$.
    Consequently, every element of $F(\alpha,\beta)$ is algebraic over $F$. By closure of the field operations, then, and the previous theorem,
    we are done.
\end{proof}

\begin{cor}
    If $F\leq E$ then $\left\{ a\in E:\alpha\text{ is algebraic over }F \right\}$ is a subfield of $E$ called the \textbf{algebraic closure} of
    $F$ in $E$.
\end{cor}

\begin{exmp}
    The algebraic closure of $\Q$ in $\C$ is $\Q^{\rm alg}$, which is an algebraic extension of $\Q$, but is not finite.
    What is the algebraic closure of $F$ in $F(t)$? It is simply $F$.
\end{exmp}


(March 11, 2013)

Let's do a quick recap -- we've defined 3 types of extensions:
\begin{enumerate}
    \item $E$ is a simple extension of $F$ if $E=F(\alpha)$. If $\alpha$ is algebraic over $F$, then $F(\alpha)=F[\alpha]$.
    \item $E$ is a finite extension of $F$ if $\dim_FE=[E:F]<\infty$. $E$ is a finite-dimensional $F$-vector space.
    \item $E$ is an algebraic extension of $F$ if for all $\alpha\in E$, $\alpha$ is algebraic over $F$.
\end{enumerate}

What is the relationship between these three concepts? First of all, we know that if $\alpha$ is algebraic over $F$, then
the simple extension $F(\alpha)$ is a finite extension of $F$, and in fact, $[F(\alpha):F]=\deg_F\alpha=\deg\irr(\alpha,F,x)$. 
Second, we know that if $E$ is a finite extension of $F$, then it is an algebraic extension of $F$. What about the converses of
these two statements? First, note that there are algebraic extensions that are not finite. Take, for example, $\Q(\sqrt{2},\sqrt{3},\sqrt{5},\ldots)$.
This is clearly algebraic but not finite (we've joined all prime numbers). Second, finite doesn't always have to be simple, but in fact,
it is \textit{almost always}. We will talk about this later, but recall the example of $\Q(\sqrt{2},\sqrt{3})$ and $\Q(\sqrt{2}+\sqrt{3})$.
Of course, simple extensions can sometimes be harder to work with, so this is not always to be taken advantage of.

What else do we know? If $F\leq E$, with $\alpha,\beta\in E$ algebraic over $F$, then $\alpha\pm\beta,\alpha\beta,\alpha/\beta$ are all
algebraic over $F$ as well (if the quotient is defined). This is what motivated us to define above the algebraic closure of $F$.

\begin{lem}
    Let $E$ be an extension of $F$. Then $E$ is a finite extension of $F$ if and only if there exist $\alpha_1,\cdots,\alpha_n\in E$ with
    $\alpha_i$ algebraic over $F$ for all $i$ such that $E=F(\alpha_1,\cdots,\alpha_n)$.
\end{lem}
\begin{proof}
    Let us take the backwards direction. If $E=F(\alpha_1,\cdots,\alpha_n)$, where $\alpha_i$ are algebraic over $F$, we simply
    consider the sequence of extensions $F\leq F(\alpha_1)\leq F(\alpha_1,\alpha_2) \leq \cdots \leq F(\alpha_1,\cdots,\alpha_n)=E$.
    We claim by induction that $F(\alpha_1,\cdots,\alpha_i)$ is a finite extension of $F$ for $i=1,\cdots,n$. For the case $i=1$, we
    have a simple extension, and by hypothesis, $\alpha_1$ is algebraic over $F$, we know $F(\alpha)$ is a finite extension of $F$.
    Now assume that $F\leq F(\alpha_1,\cdots,\alpha_i)\leq F(\alpha_1,\cdots,\alpha_{i+1})$. We wish to show that this last extension
    is finite over $F$. We know that $F(\alpha_1,\cdots,\alpha_{i+1})=F(\alpha_1,\cdots,\alpha_i)(\alpha_{i+1})$, so this is a simple extension.
    In particular, $\alpha_{i+1}$ is algebraic over $F$, and hence algebraic over any bigger field. Then we clearly have a finite extension
    of $F(\alpha_1,\cdots,\alpha_i)$, and by the theorem proved last time, we have a finite extension of $F$.

    Now we show the forwards direction. Suppose $E$ be a finite extension of $F$. Then we know that $E$ is an algebraic extension of $F$,
    i.e. for all $\alpha\in E$, $\alpha$ is algebraic over $F$. Now we argue by complete induction on $[E:F]$. If $[E:F]=1$, $E=F$ and we are done.
    Suppose that the result is true for all finite extensions of fields of degree less than some $d>1$. Given $E$ an extension of $F$ with $[E:F]=d$.
    Clearly $E\neq F$ and so there exists some $\alpha_1\in E$, $\alpha_1\notin F$, which is algebraic over $F$ ($E$ is finite). Then, $F\leq F(\alpha_1)\leq E$
    and $[E:F]=[E:F(\alpha_1)][F(\alpha_1):F]$. The second term is bigger than 1, and so $[E:F(\alpha_1)]<d$. By induction, then, there
    exist $\alpha_2,\cdots,\alpha_n\in E$ that are algebraic over $F(\alpha_1)$ such that $E=F(\alpha_1,\alpha_2,\cdots,\alpha_n)$.
    In fact, $\alpha_i$ are algebraic over $F$, again since $E$ is a finite extension, and we are done.
\end{proof}
Thus, we have shown that every finite extension can be broken up into a sequence of simple extensions.

\begin{exmp}
    We looked at the case $\Q(\sqrt{2},\sqrt{3})=\sqrt{2}\sqrt{3}$ by direct computation because $\deg\irr(\sqrt{3},\Q(\sqrt{2}),x)=2$.
    In general, however, such a statement about polynomials is not always obvious.
\end{exmp}

\begin{lem}
    Take $F\leq E\leq K$. Suppose that $E$ is algebraic over $F$. Suppose $\alpha\in K$. Then $\alpha$ is algebraic over $F$
    if and only if $\alpha$ is algebraic over $E$.
\end{lem}
\begin{proof}
    The forward direction is trivial, as if $\alpha$ is the root of $f(x)\in F[x]$ non-zero, we just view that as a member
    of $E[x]$. The difficult direction is the backwards one. Suppose $\alpha$ is algebraic over $E$. This means that there exists an
    $f(x)=x^d+a_{d-1}x^{d-1}+\cdots+a_0\in E[x]$ such that $f(\alpha)=0$. Consider $F\leq F(a_0,\cdots,a_{d-1})\leq F(a_0,\cdots,a_{d-1},\alpha)$.
    By the above lemma, $F(a_0,\cdots,a_{d-1})$ is a finite extension of $F$. Additionally, $\alpha$ is algebraic over $F(a_0,\cdots,a_{d-1})$,
    and so $F(a_0,\cdots,a_{d-1},\alpha)$ is a finite extension of $F$. But this implies that it is an algebraic extension of $F$, and since
    $\alpha\in F(a_0,\cdots,a_{d-1},\alpha)$, $\alpha$ must be algebraic over $F$.
\end{proof}

\begin{cor}
    Suppose $F\leq E\leq K$. Then $K$ is algebraic over $F$ if and only if $K$ is algebraic over $E$ and $E$ is algebraic over $F$.
\end{cor}
\begin{proof}
    If $K$ is algebraic over $F$, then every element of $K$ is algebraic over $F$. In particular,
    given $\alpha\in E\leq K$, $\alpha$ is algebraic over $F$. And given $\alpha\in K$, $\alpha$ is algebraic over $F$, then $\alpha$ is algebraic
    over $E$.

    Now assume that $K$ is algebraic over $E$ and $E$ is algebraic over $F$. Then, we are done by the above lemma.
\end{proof}

\begin{defn}
    A field $K$ is \textbf{algebraically closed} if for all $f(x)\in K[x]$ with $\deg f(x)\geq 1$, then $f(x)$ has a root in $K$.
\end{defn}

\begin{thm}
    For a field $K$, the following are equivalent:
    \begin{enumerate}
        \item $K$ is algebraically closed.
        \item Every $f(x)\in K[x]$ non-constant is a product of linear factors, i.e. $f(x)$ is irreducible in $K[x]$ if and only if
            $\deg f(x)=1$
        \item If $E$ is an algebraic extension of $K$, then $E=K$.
    \end{enumerate}
\end{thm}
\begin{proof}
    These are fairly straightforward, so they are left as an exercise. [Hint: the second is via induction]
\end{proof}

\begin{defn}
    Let $F$ be a field. Then an \textbf{algebraic closure} of $F$ is an extension field $K$ such that $K$ is an algebraic extension
    of $F$ and $K$ is algebraically closed.
\end{defn}
\begin{rem}
    Notice that $\C$ is not an algebraic closure of $Q$, as $\C$ is not an algebraic extension (it contains transcendental elements over $\Q$).
\end{rem}

Let's pause and look at the definitions:
\begin{enumerate}
    \item If $F\leq E$, we've defined the algebraic closure of $F$ in $E$.
    \item $K$ is algebraically closed.
    \item $K$ is an algebraic closure of $F$.
\end{enumerate}

\begin{thm}
    Let $F\leq K$ with $K$ be algebraically closed. Then the algebraic closure of $F$ in $K$, $F^{\rm alg}$, is an algebraic closure of $F$.
\end{thm}

\begin{exmp}
    Before we prove this proposition, note the following intuitive example.
    We have some relationship of the form $\Q\leq \Q^{\rm alg}\leq \C$, and the proposition in this case states says that
    $\Q^{\rm alg}$ is an algebraic closure of $\Q$. Note, however, that $\C$ is not an algebraic closure of $\Q$, as $\C$
    has transcendental numbers, i.e. $i\in\Q^{\rm alg}$ but $\pi i\in\C\notin\Q^{\rm alg}$.
\end{exmp}

\begin{proof}
    We know, by definition, that $F^{\rm alg}$ is an algebraic extension of $F$, because it is the set of all things in $K$
    that are algebraic over $F$. We wish to show that $F^{\rm alg}$ is algebraically closed. In other words, given some 
    $f(x)\in F^{\rm alg}[x]$ non-constant, we must show that there exists an $\alpha$, a root of $f(x)$ in $F^{\rm alg}$.
    We know that $f(x)\in F^{\rm alg}\leq K[x]$. Since $K$ is algebraically closed, there exists an $\alpha$ in $K$ such that
    $f(\alpha)=0$. We want to show that $\alpha\in F^{\rm alg}$. But this is equivalent to saying that $\alpha$ is algebraic
    over $F$. But we know that $\alpha$ is algebraic over $F^{\rm alg}$ and that $F^{\rm alg}$ is algebraic over $F$. By the lemma
    we proved earlier, then, we have that $K$ must be algebraic over $F$, and thus $\alpha$ must be in $F^{\rm alg}$.
\end{proof}

\begin{rem}
    It is a fact that for any field $F$, there exists an algebraic closure of $F$ and any two of these algebraic closures of $F$ are isomorphic.
\end{rem}

\begin{rem}
Suppose one can construct a length $\alpha\in\R$. It can be shown that $\alpha$ can be constructed via compass/straightedge 
if and only if $[\Q(\alpha):\Q]=2^k$.
Then, one can show that trisecting a 60 degree angle allows one to construct $\alpha=\cos 20$, but $[\Q(\alpha):\Q]=3$, as one can show,
so an angle cannot be trisected.
Likewise, one can show that it is impossible to double the cube, i.e. construct the cube root of two. Finally, it can be show that it is
impossible to construct a square whose side length is $\sqrt{\pi}$, which is, of course, impossible, as $\pi$ and $\sqrt{\pi}$ are transcendental.
\end{rem}

\section{Multiple roots and derivatives}

Suppose $F$ is a field and $f(x)\in F[x]$, with $\deg f(x)\geq 1$. $\alpha$ is a root of $f(x)$ if and only if $(x-a)|f(x)$.
In general, given any $f(x)\in F[x]$ non-constant and any $\alpha\in F$, there exists an integer $m\geq0$ such that $(x-\alpha)^m|f(x)$
but $(x-\alpha)^{m+1}$ does not divide $f(x)$. Notice that $m=0$ if and only if $\alpha$ is not a root. This number $m$ is called the
\textbf{multiplicity} of the root $\alpha$.

\begin{rem}
    We might have $F\leq E$ and we might be looking at $\alpha\in E$ -- we talk about divisibility in $E[x]$, as
    $x-\alpha$ is not necessarily in the smaller field $F$.
\end{rem}

If the multiplicity of $\alpha$ is 1, we say that $\alpha$ is a simple root, and if the multiplicity of $\alpha$ is greater than 1, we say
that $\alpha$ is a \textbf{multiple} or $\textbf{repeated}$ root. To see if $\alpha$ is a repeated root, we write $f(x)=(x-\alpha)g(x)$
if and only if $f(\alpha)=0$. Notice also that $\alpha$ is a repeated root if and only if $f(\alpha)=f'(\alpha)=0$. But what is a derivative in
any field? Forget about limits -- we define derivatives purely formally.

\begin{defn}
    Given $f(x)=\sum_{i=0}^na_ix^i\in F[x]$, we define the \textbf{derivative} of $f$ in $F[x]$ purely formally as \[Df(x)=\sum_{i=1}^nia_ix^{i-1}.\]
    Note that $\deg Df(x)\leq \deg f(x)-1$. In particular, we may define the derivative generally as a function $D:F[x]\to F[x]$.
    In the context of an extension $F\leq E$ we can define compatibly $D:E[x]\to E[x]$ as the derivatve does not particularly care about where
    the coefficients live. $D$ is determined by $D(x^i)=ix^{i-1}$ and is $F$-linear. It should be obvious that $D$ is not a ring homomorphism,
    as it instead follows the product rule:
    \[D(x^ax^b)=D(x^{a+b})=(a+b)x^{a+b-1}=ax^{a-1}x^b+x^abx^{b-1}=(Dx^a)x^b+x^aDx^b\]
    and by linearity, this holds for all polynomials.
    
    There is a corollary of the product rule that states
    \[D( (f(x))^n)=nf(x)^{n-1}\cdot Df(x),\]
    proved via induction.

    We can, of course, inquire into the kernel of $D$. It is not just constant functions,
    however, as if $F$ has characteristic $p$, $Dx^p=px^{p-1}=0$! In fact, it is easy to see that if the characteristic of $F$ is zero,
    the $\ker D=F\leq F[x]$, but if the characteristic is $p$, then $\ker D=F[x^p]={\rm Im}\;{\rm Frob}\leq F[x]$, i.e. the polynomials in $x^p$.
\end{defn}

Now what is the connection of the derivative to multiple roots? Let us work in $F\leq E$ with $\alpha\in E$. Let $m$ be the multiplicity
of $\alpha$ in $f(x)$, i.e. $(x-\alpha)^m$ divides $f(x)$ but higher powers do not. In other words, we can write $f(x)=(x-\alpha)^mg(x)$
with $g(\alpha)\neq 0$. If $m=0$, $f(\alpha)\neq 0$. Let's assume that $m\geq 1$. Then we write $f(x)=(x-\alpha)^mg(x)$. Taking the derivative, we find
\begin{align*}
    Df(x)&=D\left( (x-a)^m \right)g(x)+(x-\alpha)^mDg(x)\\
    &= m(x-a)^{m-1}g(x)+(x-\alpha)^mDg(x)
\end{align*}
Now, if $m=1$, $Df(x)=f(x)+(x-a)Dg(x)$ and $Df(\alpha)=g(\alpha)\neq 0$. If $m\geq 2$, on the other hand, we have at least one overall factor of $x-\alpha$.
Here, unlike before, $Df(\alpha)=f(\alpha)=0$.

\begin{thm}
    Any $\alpha\in E$ is a multiple root of $f(x)$ if and only if $f(\alpha)=Df(\alpha)=0$.
\end{thm}

This is not exactly what we want, however, as we want a condition that $f(x)$ has a multiple root in some extension field without knowing
what the root or the extension field are.

\begin{lem}
    Let $F\leq E$ and $f(x),g(x)\in F[x]$ not both 0, $f(x)$ not constant.
    \begin{enumerate}
        \item $f(x)|g(x)$ in $F[x]$ if and only if $f(x)|g(x)$ in $E[x]$.
        \item $d(x)\in F[x]$ is a gcd of $f,g$ in $F[x]$ if and only if it is a gcd in $E[x]$.
        \item $f,g$ are relatively prime in $F[x]$ if and only if they are relatively prime in $E[x]$.
    \end{enumerate}
\end{lem}
\begin{proof}
    \begin{enumerate}
        \item The forward direction is obvious. Suppose $g(x)=f(x)h(x)$ with $h(x)\in E[x]$. We wish to show that $h(x)\in F[x]$.
            We can apply long division with remainder in $F[x]$: $g(x)=q(x)f(x)+r(x)$ with $q,r\in F[x]$ and either $r=0$ or $\deg r<\deg f$.
            We now have two different expression for $g$, but we know that in $E[x]$ long division with remainder is unique! Thus, they must be the same. 
            This implies that $h(x)=q(x)$ and $r(x)=0$, i.e. that $h(x)\in F[x]$.
        \item  Say $d(x)$ is a gcd of $f,g$ in $F[x]$. In particular, we know that $d|f, d|g$ and $d=af+bg$. To show that $d$ is a gcd of
            $f,g$ in $E[x]$, we need to show that anything that divides $f,g$ divides $d$. Suppose $e(x)$ in $E[x]$ divides $f,g$.
            Then $e|af+bg=d$ so $e|d$ and $d$ is a gcd in $E[x]$. Conversely suppose $d(x)\in F[x]$ is a gcd of $f,g$ in $E[x]$ so
            $d|f,d|g$ in $E[x]$. But by the first statement of this lemma, we know that $d|f,d|g$ in $F[x]$. Then if $e\in F[x]$
            with $e|f,e|g$, then $e|f,e|g$ in $E[x]$ implies $e|d$ in $E[x]$ and thus  $e|d$ in $F[x]$. This shows that $e$ is a gcd
            in $F[x]$.
            \item $f,g$ are relatively prime in $F[x]$ if and only if 1 is a gcd of $f,g$ in $F[x]$, which by statement 2 of the lemma,
            implies that 1 is a gcd of $f,g$ in $E[x]$. Then $f,g$ are relatively prime in $E[x]$.
    \end{enumerate}
\end{proof}

\begin{cor}
    Let $f(x)\in F[x]$ non-constant. Then there exists an extension field $E$ of $F$ and an $\alpha\in E$ which is a multiple root of $f(x)$
    if and only if $f(x)$ and $Df(x)$ are not relatively prime in $F[x]$.
\end{cor}
\begin{proof}
    Suppose that $f(x)$ has a multiple root $\alpha$ in some extension $E$. This implies that $x-\alpha|f(x)$ and $x-\alpha|Df(x)$.
    This implies that $f,Df$ are not relatively prime in $E[x]$. But if they are not relatively prime in $E[x]$, they are not relatively
    prime in $F[x]$.

    If $f(x),Df(x)$ are not relatively prime in $F[x]$, then there exists an irreducible $p(x)$ in $F[x]$ such that $p(x)|f(x)$ and
    $p(x)|Df(x)$. But we know that there exists an extension field $E$ of $F$ and a root $\alpha$ of $p(x)\in E$. In $E$, then,
    $\alpha$ is a root of $f(x)$ and $Df(x)$, which implies that $\alpha$ is a multiple root.
\end{proof}

\begin{cor}
    Let $f(x)$ be an irreducible polynomial in $F[x]$. $f(x)$ has a multiple root in some extension field $E$ of $F$ if (and only if)
    $Df(x)=0$.
\end{cor}
\begin{proof}
    If $f(x)$ has a multiple root in some extension field, then $f(x),Df(x)$ are not relatively prime, by the above corollary.
    But we know that $f(x)$, by hypothesis, is irreducible, and $Df(x)$, if non-zero, has smaller degree (not necessarily $n-1$ in
    positive characteristic fields). Since $f(x)$ is irreducible, we know that either the gcd of $f(x),Df(x)$ is one or $f(x)|Df(x)$.
    However, the first is not true due to the multiple root, and the second is not true due to the degrees of the polynomials.
    Thus we reach a contradiction, and $Df(x)=0$.
\end{proof}

\begin{cor}
    If the characteristic of $F$ is zero, an irreducible $f(x)\in F[x]$ never has a multiple root in an extension field.
\end{cor}
\begin{proof}
    $Df(x)\neq 0$.
\end{proof}

In fact, this does happen if the characteristic of $F$ is $p$, but not if $F$ is finite.
\begin{exmp}
    Let $F=\F_p(t)$, an infinite field of rational functions with characteristic $p$. Now consider $f(x)=x^p-t$.
    A zero $\alpha$ of $f(x)$ is equal to a $p$th root of $t$. Thus $f(x)$ has no root in $\F_p(t)$ just because
    the polynomial would have to have power $1/p$. Note that one can check that $f(x)$ is irreducible.

    Let $\alpha$ be a root of $f(x)$ in some extension field: $\alpha^p=t$. In $E[x]$, $f(x)=x^p-t=x^p-\alpha^p=(x-\alpha)^p$.
    Thus, $\alpha$ is a root of multiplicity $p$.
    Notice that in this case $D(x^p-t)=px^{p-1}=0$.
\end{exmp}

When we come back from break, we will move on to using this to classify finite fields.






\end{document}
