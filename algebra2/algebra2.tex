\documentclass{../mathnotes}

\title{Notes on Algebra II}
\author{Nilay Kumar}
\date{Last updated: \today}


\begin{document}

\maketitle

\section{Reducibility}

February 25, 2013

Let $F$ be a field, and $F[x]$ be the ring of polynomials over $F$. Recall we have already shown that every ideal
in $F[x]$ is principal, and that there exists a unique gcd of two non-zero polynomials. Additionally, we showed that
if $f$ and $g$ are two relatively prime polynomials, then $f|gh\implies f|h$.

\begin{defn}
    A polynomial $p(x)\in F[x]$ is \textbf{irreducible} if $\deg p(x)>0$, i.e. $p$ is not zero and not a unit,
    and if $p=fg$ implies that one of $f,g$ is a unit and the other is a unit times $p$. In words, $p(x)$ is
    irreducible if it does not factor into a product of two polynomials with strictly smaller (non-zero)
    degree. A polynomial is said to be $\textbf{reducible}$ if it is not irreducible.
\end{defn}

\begin{exmp}(Reducibility)
    \begin{enumerate}[(i)]
        \item Any linear polynomial $x+a$ is obviously irreducible.
        \item Any quadratic polynomial is clearly reducible if and only if it has two linear factors.
            This is equivalent to the polynomial having a root, as long division will yield the second factor.
        \item Similarly, a cubic polynomial is reducible if and only if it has a root.
        \item For higher degrees, the existence of a root is not equivalent to reducibility, as we will
            see in the next example.
    \end{enumerate}
\end{exmp}

\begin{exmp}(Simple examples)
    \begin{itemize}
        \item $x^2-2$ is irreducible in $\mathbb{Q}[x]$, as it has no roots in $\mathbb{Q}$. It is, however,
            reducible in $\mathbb{R}[x]$: $x^2-2=(x-\sqrt{2})(x+\sqrt{2})$.
        \item $x^2+1$ is irreducible in $\mathbb{R}[x]$ but reducible in $\mathbb{C}[x]$: $x^2+1=(x-i)(x+i)$.
        \item $x^3-2$ is irreducible in $\mathbb{Q}[x]$, but reducible in $\mathbb{R}[x]$, where we can write
            it as a product of $x-\sqrt[3]{2}$ and an irreducible quadratic.
        \item $x^4-4=(x^2-2)(x^2+2)$ is reducible in $\mathbb{Q}[x]$ but has no roots!
    \end{itemize}
\end{exmp}
In fact, it is generally a hard problem to determine whether an arbitrary polynomial $f(x)\in\mathbb{Q}[x]$ is
irreducible. Note, however, that we can think of irreducibility in analogy to that for natural numbers, as
the following dichotomy illustrates.
\begin{rem}
    If $p(x)\in F[x]$ is irreducible, then for any polynomial $f\in F[x]$, either $p|f$ or $p$ and $f$ are
    relatively prime.
\end{rem}
\begin{proof}
    Let $d=\gcd(p,f)$. By definition, $d$ divides $p$. However, as $p$ is irreducible, $d$ must either be
    a unit or $d$ must be $cp$ for $c$ a unit. In the first case, since the gcd of $p$ and $f$ is a unit,
    $p$ and $f$ must be relatively prime. In the second case, since $d=cp$ by construction divides $f$,
    $p$ must divide $f$.
\end{proof}

\begin{cor}
    If $p\in F[x]$ is irreducible and $p|fg$, then either $p|f$ or $p|g$.
\end{cor}
\begin{proof}
    By the above remark, either $p|f$ or $p$ and $f$ are relatively prime.
    If $p|f$, we are done. Otherwise, $p$ is relatively prime to $f$, and by what we showed last class,
    $p|g$.
\end{proof}

\begin{thm}[Unique factorization of polynomials]
    Let $f(x)\in F[x]$ with $\deg f(x)>0$. Then there exist $k$ irreducible polynomials in $F[x]$ such
    that
    \[f(x)=\prod_{i=1}^k p_i(x).\]
    Additionally, if it is also true that $f(x)=\prod_{i=1}^lq_i$, then $k=l$, and after some reordering,
    there exist nonzero constants such that $q_i=c_ip_i$.

    In other words, for any polynomial with degree greater than zero, there always exists a unique
    factorization into a product of irreducible polynomials.
\end{thm}
\begin{proof}
    Let us first show existence. We proceed by complete induction on the degree of $f$.
    If $\deg f=1$, $f$ is irreducible, and we are done. Otherwise, we assume that the theorem
    holds for all degrees less than $n$. Let $\deg f=n$. If $f$ is irreducible, we are done.
    Otherwise, $f=g_1g_2$ with $\deg g_1<n$ and $\deg g_2<n$. By the inductive hypothesis,
    $g_1$ and $g_2$ are products of irreducible polynomials, and thus $f$ must be as well, and we are done.

    The real muscle of this theorem comes in the form of uniqueness. Suppose $f=\prod_{i=1}^kp_i=\prod_{j=1}^lq_j$,
    with $p_i,q_j$ reducible. We proceed by induction on $k$. If $k=1$, $p_1=q_1\cdots q_l$. Clearly, then,
    $p_1|q_1\cdots q_l$, and thus (by induction over the statement at the beginning of lecture), $p_1$ must divide
    $q_i$ for some $i$. But the $q_i$ are irreducible and $p_1$ is not a constant, so $p_1=cq_i$ for some unit
    $c$. If we now reorder terms, we can assume that $i=1$ and we can cancel:
    \begin{align*}
        p_1=cq_1&=q_1q_2\cdots q_l\\
        c&=q_2\cdots q_l.
    \end{align*}
    But this is impossible, as the product of $q$'s has degree greater than zero. Consequently, $l$ must be 1, and
    thus $p_1=q_1$ and we have shown that $k=l$. The general case is similar; we write $p_1\cdots p_k=q_1\cdots q_l$. Then $p_1|q_1\cdots q_l$,
    and so for some $i$, $p_1=cq_i$. After reordering, we can write
    \begin{align*}
        cq_1p_2\cdots p_k&=q_1\cdots q_l\\
        cp_2\cdots p_k&=q_2 \cdots q_l,
    \end{align*}
    and by induction, we know that $k-1=l-1$. Reordering, we can write $p_=c_iq_i$ for $i=2\cdots k$,
    and we are done.
\end{proof}
Note that the irreducible factors need not be distinct.

\begin{thm}
    Let $F$ be a field. Let $I$ be an ideal in $F[x]$. Then the following are equivalent:
    \begin{enumerate}[(i)]
        \item $I$ is a maximal ideal.
        \item $I$ is a prime ideal and $I\neq \left\{ 0 \right\}$.
        \item $I=(p)$, where $p$ is a irreducible polynomial.
    \end{enumerate}
\end{thm}
\begin{proof}
    Let us first show that $(i)\implies(ii)$.
    Say $I$ is maximal. Then, $I$ must be prime. Additionally, $I$ cannot be the zero ideal, as it is not
    maximal, and so we are done.

    Showing $(ii)\implies(iii)$ is a little trickier. Suppose $I$ is a prime ideal with $I\neq \left\{ 0 \right\}$.
    We want to show that the ideal is generated by an irreducible element. Since every ideal in $F[x]$ is principal,
    $I=(p)$ for some $p\in F[x]$. Let us show that $p$ is irreducible. First note that $p$ cannot be a unit, because
    otherwise $1\in(p)$ which implies that $(p)=F[x]$, which is not possible for prime ideals. Furthermore,
    $p\neq0$, as $I$ is assumed not to be the zero ideal. To show that $p$ is irreducible, we need to show that if
    $p=fg$ then one of $f,g$ is a unit and the other is a unit times $p$. So take $p=fg$. Then, $fg\in (p)=I$.
    Since $I$ is prime, either $f\in I$ or $g\in I$. Take the first case, $f\in(p)$. Then, $f=hp$ for some $h\in F[x]$,
    and so $p=hpg\implies 1=hg$, i.e. $h,g$ are units, and thus $f$ is a unit times $p$.
    Thus, $p$ is irreducible.

    Finally, we show that $(iii)\implies(i)$. Let $I=(p)$, with $p$ irreducible. We wish to show that $I$ is maximal,
    i.e. $(p)\neq F[x]$ and if $(p)\subset J$ then either $J=(p)$ or $J=F[x]$. First note that $(p)\neq F[x]$ because
    $\deg p>1$ and so it can't generate constants. Next, since $J$ is necessarily a principal ideal, $J=(f)$, for some
    $f\in F[x]$. If $(p)\subset (f)$, then $p\in(f)$, so $p=fg$ for some $g\in F[x]$. But $p$ is irreducible, so either
    $f$ is a unit, in which case $J=(f)=F[x]$, or $f=cp$, for $c$ a unit, in which case $J=(f)=(p)$. Hence, $I$ is
    maximal.
\end{proof}
This theorem is quite handy in constructing interesting fields, as the following corollary shows.
\begin{cor}
    $F[x]/(f)$ is a field if and only if $f$ is irreducible.
\end{cor}
\begin{proof}
    This follows from above theorem and the fact that $F[x]/(f)$ is a field if and only if $(f)$ is a maximal ideal.
\end{proof}

This allows us to show that certain rings are, in fact, fields -- something that may not have been obvious -- or, in fact,
to find wholly new fields.
\begin{exmp}{\ \\} 
    \begin{itemize}
        \item $\mathbb{Q}[x]/(x^2-2)$ is a field, as $x^2-2$ is irreducible in $\mathbb{Q}[x]$, and its elements,
            by what we know about long division, are of the form $c+d\alpha$, where $\alpha=x+(x^2-2)$. In addition,
            $\alpha^2=2$.
        \item $\mathbb{R}[x]/(x^2+1)$ is a field, as $x^2+1$ is irreducible in $\mathbb{R}[x]$, and its elements are of
            the form $c+d\alpha$ where $\alpha=x+(x^2+1)$ satisfies $\alpha^2=-1$.
        \item $\mathbb{Q}[x]/(x^3-2)$ is a field, as $x^3-2$ is irreducible in $\mathbb{Q}[x]$, and its elements are of
            the form $c+d\alpha+e\alpha^2$, where $\alpha=x+(x^3-2)$ satisfies $\alpha^3=2$. We often rewrite the elements
            as $c+d\sqrt[3]{2}+e\sqrt[3]{2}^2$.
        \item Take the finite field $\mathbb{F}_2$ and the polynomial $x^2+x+1\in\mathbb{F}_2$. Since the only members of
            $\mathbb{F}_2$ are 0 and 1, it should be clear that this polynomial has no roots, and thus is irreducible in $\mathbb{F}_2[x]$.
            Consequently, $E=\mathbb{F}_2[x]/(x^2+x+1)$ is a field. Its elements are of the form $c+d\alpha$, where of course $c,d\in\mathbb{F}_2$
            and $\alpha=x+(x^2+x+1)$, which satisfies the property that $\alpha^2=-\alpha-1=\alpha+1$. $E$ has four elements
            (since $c$ and $d$ can each take 2 values).
    \end{itemize}
\end{exmp}

\end{document}
