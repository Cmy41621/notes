\documentclass{../mathnotes}

\title{Notes on Algebra II}
\author{Nilay Kumar}
\date{Last updated: \today}


\begin{document}

\maketitle

\section{Reducibility}

February 25, 2013

Let $F$ be a field, and $F[x]$ be the ring of polynomials over $F$. Recall we have already shown that every ideal
in $F[x]$ is principal, and that there exists a unique gcd of two non-zero polynomials. Additionally, we showed that
if $f$ and $g$ are two relatively prime polynomials, then $f|gh\implies f|h$.

\begin{defn}
    A polynomial $p(x)\in F[x]$ is \textbf{irreducible} if $\deg p(x)>0$, i.e. $p$ is not zero and not a unit,
    and if $p=fg$ implies that one of $f,g$ is a unit and the other is a unit times $p$. In words, $p(x)$ is
    irreducible if it does not factor into a product of two polynomials with strictly smaller (non-zero)
    degree. A polynomial is said to be $\textbf{reducible}$ if it is not irreducible.
\end{defn}

\begin{exmp}(Reducibility)
    \begin{enumerate}[(i)]
        \item Any linear polynomial $x+a$ is obviously irreducible.
        \item Any quadratic polynomial is clearly reducible if and only if it has two linear factors.
            This is equivalent to the polynomial having a root, as long division will yield the second factor.
        \item Similarly, a cubic polynomial is reducible if and only if it has a root.
        \item For higher degrees, the existence of a root is not equivalent to reducibility, as we will
            see in the next example.
    \end{enumerate}
\end{exmp}

\begin{exmp}(Simple examples)
    \begin{itemize}
        \item $x^2-2$ is irreducible in $\mathbb{Q}[x]$, as it has no roots in $\mathbb{Q}$. It is, however,
            reducible in $\mathbb{R}[x]$: $x^2-2=(x-\sqrt{2})(x+\sqrt{2})$.
        \item $x^2+1$ is irreducible in $\mathbb{R}[x]$ but reducible in $\mathbb{C}[x]$: $x^2+1=(x-i)(x+i)$.
        \item $x^3-2$ is irreducible in $\mathbb{Q}[x]$, but reducible in $\mathbb{R}[x]$, where we can write
            it as a product of $x-\sqrt[3]{2}$ and an irreducible quadratic.
        \item $x^4-4=(x^2-2)(x^2+2)$ is reducible in $\mathbb{Q}[x]$ but has no roots!
    \end{itemize}
\end{exmp}
In fact, it is generally a hard problem to determine whether an arbitrary polynomial $f(x)\in\mathbb{Q}[x]$ is
irreducible. Note, however, that we can think of irreducibility in analogy to that for natural numbers, as
the following dichotomy illustrates.
\begin{rem}
    If $p(x)\in F[x]$ is irreducible, then for any polynomial $f\in F[x]$, either $p|f$ or $p$ and $f$ are
    relatively prime.
\end{rem}
\begin{proof}
    Let $d=\gcd(p,f)$. By definition, $d$ divides $p$. However, as $p$ is irreducible, $d$ must either be
    a unit or $d$ must be $cp$ for $c$ a unit. In the first case, since the gcd of $p$ and $f$ is a unit,
    $p$ and $f$ must be relatively prime. In the second case, since $d=cp$ by construction divides $f$,
    $p$ must divide $f$.
\end{proof}

\begin{cor}
    If $p\in F[x]$ is irreducible and $p|fg$, then either $p|f$ or $p|g$.
\end{cor}
\begin{proof}
    By the above remark, either $p|f$ or $p$ and $f$ are relatively prime.
    If $p|f$, we are done. Otherwise, $p$ is relatively prime to $f$, and by what we showed last class,
    $p|g$.
\end{proof}

\begin{thm}[Unique factorization of polynomials]
    Let $f(x)\in F[x]$ with $\deg f(x)>0$. Then there exist $k$ irreducible polynomials in $F[x]$ such
    that
    \[f(x)=\prod_{i=1}^k p_i(x).\]
    Additionally, if it is also true that $f(x)=\prod_{i=1}^lq_i$, then $k=l$, and after some reordering,
    there exist nonzero constants such that $q_i=c_ip_i$.

    In other words, for any polynomial with degree greater than zero, there always exists a unique
    factorization into a product of irreducible polynomials.
\end{thm}
\begin{proof}
    Let us first show existence. We proceed by complete induction on the degree of $f$.
    If $\deg f=1$, $f$ is irreducible, and we are done. Otherwise, we assume that the theorem
    holds for all degrees less than $n$. Let $\deg f=n$. If $f$ is irreducible, we are done.
    Otherwise, $f=g_1g_2$ with $\deg g_1<n$ and $\deg g_2<n$. By the inductive hypothesis,
    $g_1$ and $g_2$ are products of irreducible polynomials, and thus $f$ must be as well, and we are done.

    The real muscle of this theorem comes in the form of uniqueness. Suppose $f=\prod_{i=1}^kp_i=\prod_{j=1}^lq_j$,
    with $p_i,q_j$ reducible. We proceed by induction on $k$. If $k=1$, $p_1=q_1\cdots q_l$. Clearly, then,
    $p_1|q_1\cdots q_l$, and thus (by induction over the statement at the beginning of lecture), $p_1$ must divide
    $q_i$ for some $i$. But the $q_i$ are irreducible and $p_1$ is not a constant, so $p_1=cq_i$ for some unit
    $c$. If we now reorder terms, we can assume that $i=1$ and we can cancel:
    \begin{align*}
        p_1=cq_1&=q_1q_2\cdots q_l\\
        c&=q_2\cdots q_l.
    \end{align*}
    But this is impossible, as the product of $q$'s has degree greater than zero. Consequently, $l$ must be 1, and
    thus $p_1=q_1$ and we have shown that $k=l$. The general case is similar; we write $p_1\cdots p_k=q_1\cdots q_l$. Then $p_1|q_1\cdots q_l$,
    and so for some $i$, $p_1=cq_i$. After reordering, we can write
    \begin{align*}
        cq_1p_2\cdots p_k&=q_1\cdots q_l\\
        cp_2\cdots p_k&=q_2 \cdots q_l,
    \end{align*}
    and by induction, we know that $k-1=l-1$. Reordering, we can write $p_=c_iq_i$ for $i=2\cdots k$,
    and we are done.
\end{proof}
Note that the irreducible factors need not be distinct.

\begin{thm}
    Let $F$ be a field. Let $I$ be an ideal in $F[x]$. Then the following are equivalent:
    \begin{enumerate}[(i)]
        \item $I$ is a maximal ideal.
        \item $I$ is a prime ideal and $I\neq \left\{ 0 \right\}$.
        \item $I=(p)$, where $p$ is a irreducible polynomial.
    \end{enumerate}
\end{thm}
\begin{proof}
    Let us first show that $(i)\implies(ii)$.
    Say $I$ is maximal. Then, $I$ must be prime. Additionally, $I$ cannot be the zero ideal, as it is not
    maximal, and so we are done.

    Showing $(ii)\implies(iii)$ is a little trickier. Suppose $I$ is a prime ideal with $I\neq \left\{ 0 \right\}$.
    We want to show that the ideal is generated by an irreducible element. Since every ideal in $F[x]$ is principal,
    $I=(p)$ for some $p\in F[x]$. Let us show that $p$ is irreducible. First note that $p$ cannot be a unit, because
    otherwise $1\in(p)$ which implies that $(p)=F[x]$, which is not possible for prime ideals. Furthermore,
    $p\neq0$, as $I$ is assumed not to be the zero ideal. To show that $p$ is irreducible, we need to show that if
    $p=fg$ then one of $f,g$ is a unit and the other is a unit times $p$. So take $p=fg$. Then, $fg\in (p)=I$.
    Since $I$ is prime, either $f\in I$ or $g\in I$. Take the first case, $f\in(p)$. Then, $f=hp$ for some $h\in F[x]$,
    and so $p=hpg\implies 1=hg$, i.e. $h,g$ are units, and thus $f$ is a unit times $p$.
    Thus, $p$ is irreducible.

    Finally, we show that $(iii)\implies(i)$. Let $I=(p)$, with $p$ irreducible. We wish to show that $I$ is maximal,
    i.e. $(p)\neq F[x]$ and if $(p)\subset J$ then either $J=(p)$ or $J=F[x]$. First note that $(p)\neq F[x]$ because
    $\deg p>1$ and so it can't generate constants. Next, since $J$ is necessarily a principal ideal, $J=(f)$, for some
    $f\in F[x]$. If $(p)\subset (f)$, then $p\in(f)$, so $p=fg$ for some $g\in F[x]$. But $p$ is irreducible, so either
    $f$ is a unit, in which case $J=(f)=F[x]$, or $f=cp$, for $c$ a unit, in which case $J=(f)=(p)$. Hence, $I$ is
    maximal.
\end{proof}
This theorem is quite handy in constructing interesting fields, as the following corollary shows.
\begin{cor}
    $F[x]/(f)$ is a field if and only if $f$ is irreducible.
\end{cor}
\begin{proof}
    This follows from above theorem and the fact that $F[x]/(f)$ is a field if and only if $(f)$ is a maximal ideal.
\end{proof}

This allows us to show that certain rings are, in fact, fields -- something that may not have been obvious -- or, in fact,
to find wholly new fields.
\begin{exmp}{\ \\} 
    \begin{itemize}
        \item $\mathbb{Q}[x]/(x^2-2)$ is a field, as $x^2-2$ is irreducible in $\mathbb{Q}[x]$, and its elements,
            by what we know about long division, are of the form $c+d\alpha$, where $\alpha=x+(x^2-2)$. In addition,
            $\alpha^2=2$.
        \item $\mathbb{R}[x]/(x^2+1)$ is a field, as $x^2+1$ is irreducible in $\mathbb{R}[x]$, and its elements are of
            the form $c+d\alpha$ where $\alpha=x+(x^2+1)$ satisfies $\alpha^2=-1$.
        \item $\mathbb{Q}[x]/(x^3-2)$ is a field, as $x^3-2$ is irreducible in $\mathbb{Q}[x]$, and its elements are of
            the form $c+d\alpha+e\alpha^2$, where $\alpha=x+(x^3-2)$ satisfies $\alpha^3=2$. We often rewrite the elements
            as $c+d\sqrt[3]{2}+e\sqrt[3]{2}^2$.
        \item Take the finite field $\mathbb{F}_2$ and the polynomial $x^2+x+1\in\mathbb{F}_2$. Since the only members of
            $\mathbb{F}_2$ are 0 and 1, it should be clear that this polynomial has no roots, and thus is irreducible in $\mathbb{F}_2[x]$.
            Consequently, $E=\mathbb{F}_2[x]/(x^2+x+1)$ is a field. Its elements are of the form $c+d\alpha$, where of course $c,d\in\mathbb{F}_2$
            and $\alpha=x+(x^2+x+1)$, which satisfies the property that $\alpha^2=-\alpha-1=\alpha+1$. $E$ has four elements
            (since $c$ and $d$ can each take 2 values).
    \end{itemize}
\end{exmp}


\section{Field extensions}
February 27, 2013

In general, a problem in algebra is to enlarge the domain of discourse, i.e. $\mathbb{R}\to\mathbb{C}$, so that one can solve
equations that were hitherto unsolvable. So given a polynomial $f(x)\in F[x]$, we wish to find a root of $f(x)$. Maybe there is
not root of $f(x)$ in $F$, so we wish to enlarge $F$.
A simple idea that we saw earlier was that if $R$ is any ring with $f(x)\in R[x]$, there was a way to enlarge $R$.
We consider $R[x]/(f(x))$, which always has a root $x+(f(x))=\alpha$. By construction, $f(\alpha)=0$.

There is a problem with this -- we don't know much about the algebraic structure of this new quotient ring, $R[x]/I$.
The solution is: if $R$ is a field, and $f(x)$ is irreducible, then $R[x]/I$ is good, i.e. it is a field, as we
saw last lecture. But really, even if $f(x)$ is reducible, we should think in analogy to the world of
$\mathbb{Z}/n\mathbb{Z}$, where $n>0$. The full details of this analogy are fleshed out in the file \texttt{analogy.pdf}.

\begin{thm}
    Let $f(x)\in F[x]$ and suppose $\deg f(x)>0$, i.e. $f$ is nonconstant. Then, there exists a field $E$ that
    contains (a subfield isomorphic to) $F$, and an element $\alpha\in E$ such that $f(\alpha)=0$.
\end{thm}
\begin{proof}
    Take $f(x)$ and find an irreducible factor (we know these exist from last time) $p(x)$. We consider $E=F[x]/(p(x))$;
    $E$ is a field. Additionally, there is a map $F\to E$ that sends $a\in F\mapsto a+(p(x))$. This map is injective (why?),
    and we identify $F$ with the image subfield. We know that if $\alpha=x+(p(x))$, then $p(\alpha)=0$. But
    $p(x)|f(x)$, so $f(\alpha)=0$ as well, and we are done.
\end{proof}

\begin{cor}
    Let $f(x)\in F[x]$, $\deg f(x)>0$. Then there exists a field $E$ such that, in $E[x]$, $f(x)$ is a product
    of linear factors.
\end{cor}
\begin{proof}
    We apply the above theorem to find $E_1$ and $\alpha_1\in E_1$ such that $f(\alpha_1)=0$ ($F\leq E_1$).
    In $E_1[x]$, $f(x)=(x-\alpha_1)g_1(x)$, where $\deg g_1(x)=\deg f(x)-1$. Informally speaking, now all we have to do
    is to keep going! We find $E_2$ with $E_1\leq E_2$ such that we can write $g_1(x)=(x-\alpha_2)g_2(x)$ with $\alpha_2\in E_2$ with
    $\deg g_2=\deg f(x)-2$. We continute until we run out of degrees, and clearly we have factored $f$ into linear factors.
\end{proof}

Let us now switch perspectives. Let's consider the situation where $E$ and $F$ are fields, and $F\leq E$, i.e. $F$ is a subfield
of $E$. We also say that $E$ is an \textbf{extension field} of $F$. Note that we used the machinery of prime and maximal ideals to
construct extensions. Now, given an extension, let us use these tools to analyze these fields.

Consider $E$ an extension field of $F$, and let $\alpha\in E$. Look at ${\rm ev}_\alpha: F[x]\to E$. Note that ${\rm Im\;} {\rm ev}_\alpha=F[\alpha]\leq E$.
At this point, all we know is that $F[\alpha]$ is an integral domain. We claim that there are exactly 2 possibilities.
\begin{enumerate}
    \item $\ker {\rm ev}_\alpha=\left\{ 0 \right\}$, i.e. that if $f(x)\in F[x]$, $f(x)\neq 0$, then $f(\alpha)\neq 0$.
        In this case, we say that $\alpha$ is \textbf{transcendental} over $F$. Additionally, ${\rm ev}_\alpha:F[x]\to E$
        is injective, which suggests that it extends to an injection $F(x)\to E$, whose image we call $F(\alpha)$. Elements
        of this image have the form $f(\alpha)/g(\alpha)$ where $f,g\in F[x]$ and $g\neq 0$. This is the smallest subfield of
        $E$ containing $F$ and $\alpha$.

        A famous example is that $\pi \in \mathbb{R}$ is transcendental over $\mathbb{Q}$ (Lindemann, 1880). The same holds
        for $e$. Note carefully, that $\pi\in \mathbb{R}$ is not transcendental over $\mathbb{R}$, as $\pi$ is a root of $x-\pi$.c w
    \item $\ker {\rm ev}_\alpha\neq \left\{ 0 \right\},$ i.e. there exists an $f(x)\in F[x]$, $f\neq 0$ such that $f(\alpha)=0$.
        In this case, we say that $\alpha$ is \textbf{algebraic} over $F$. What can we say here? An incredible amount, it turns out.

        First note that $\ker {\rm ev}_\alpha$ is a principal ideal in $F[x]$: $\ker {\rm ev}_\alpha=(p(x))$. Additionally,
        we know that $F[\alpha]={\rm Im\;}{\rm ev}_\alpha\cong F[x]/\ker{\rm ev}_\alpha$. But since $F[x]$ is an integral domain
        (subring of a field), $(p(x))$ is a prime ideal that is not $\left\{ 0 \right\}$. This means that $(p(x))$ is a maximal ideal
        and $p(x)$ is irreducible (by theorem proved last time). Consequently, $F[x]/(p(x))=F[x]/\ker{\rm ev}_\alpha$ is a field, and
        $F[\alpha]$ is a field. Recall the example of $\mathbb{Q}[\sqrt[3]{2}]$ being a field. Thus, we now write $F[\alpha]=F(\alpha)$,
        which is the smallest subfield of $E$ containing $F$ and $\alpha$.

        In particular, there is a unique monic generator of $(p(x))=\ker{\rm ev}_\alpha$. It is denoted $\irr(\alpha,F,x)$, which
        is read ``the irreducible polynomial for $\alpha$ over $F$.'' It satisfies $\irr(\alpha,F,\alpha)=0$. Let us do a few examples:
        \begin{exmp}
            \[\irr(\frac{1}{2},\mathbb{Q},x)=x-\frac{1}{2}\]
            \[\irr(\sqrt[3]{2},\mathbb{Q},x)=x^3-2\]
            \[\irr(\sqrt[3]{2},\mathbb{Q}(\sqrt[3]{2}),x)=x-\sqrt[3]{2}\]
            \[\irr(\sqrt[3]{2},\mathbb{Q}(\sqrt{2}),x)=?\]
        \end{exmp}

        \begin{rem}
        Note that if $f(x)\in F[x]$ is any polynomial such that $f(\alpha)=0$, then the $\irr(a,F,x)|f(x)$.
        \end{rem}
        \begin{proof}
            $f(\alpha)=0\iff f(x)\in\ker {\rm ev}_\alpha=(\irr(\alpha,F,x))$. By definition, then, $\irr(\alpha,F,x)|f(x)$.
        \end{proof}
        For example, take $f(x)\in \mathbb{R}[x]$. If $f(i)=0$, then $x^2+1|f(x)$.
\end{enumerate}

Mostly we will be working with the algebraic case, as the transcendental case belongs in a separate course.

\begin{defn}
    Given $F\leq E$ and $\alpha\in E$ algebraic over $F$, we define the \textbf{degree} of $\alpha$ over $F$ as $\deg\irr(\alpha,F,x)$.
\end{defn}
For example, $\deg_{\mathbb{Q}}\sqrt[3]{2}=3$ and $\deg_{\mathbb{R}}\sqrt[3]{2}=1$ and $\deg_{\mathbb{R}}i=2$.

\begin{defn}
    Let $F\leq E$. Then we say that $E$ is a \textbf{simple extension} of $F$ if $E=F(\alpha)$ for some $\alpha\in E$.
\end{defn}
Roughly speaking, this means that we can extend $F$ to $E$ by throwing in only one more element -- i.e. we can do this if $\alpha$ is transcendental.
Take, for example, $\mathbb{Q}(\sqrt{2})$. $\sqrt{3}\notin\mathbb{Q}(\sqrt{2})$. Then, $\mathbb{Q}(\sqrt{2})\leq \mathbb{Q}(\sqrt{2})(\sqrt{3})=\mathbb{Q}(\sqrt{2},\sqrt{3}).$
Then, $\mathbb{Q}\leq \mathbb{Q}(\sqrt{2}+\sqrt{3})$, one can find that $\mathbb{Q}(\sqrt{2}+\sqrt{3})=\mathbb{Q}(\sqrt{2},\sqrt{3})$. This shows
that simple extensions are not always obviously simple.

\section{$F$-vector spaces}
Let us take a detour through vector spaces.

Let $F$ be a field. An \textbf{$F$-vector space} is an abelian group $(V,+)$ and a function $F\times V\to V$ called \textbf{scalar multiplication}, which
we write as $av$ such that:
\begin{enumerate}
    \item $a(bv)=(ab)v$
    \item $a(v+w)=av+aw$
    \item $(a+b)v=av+bv$
    \item $1\cdot v=v$
\end{enumerate}

A very useful example is $V=F^n=\left\{ (a_1\cdots a_n): a_i\in F \right\}$, i.e. the Cartesian product of $F$ with itself $n$ times.
We define addition componentwise, and multiply the scalar through each component, as usual. It is easy to check that $F^n$ is a vector space.
The $n=0$ case is allowed, as it is the zero vector space with only 0.

Another example is the space of functions $X\to F$ on any set $X$, which we denote by $F^X$. Functions are added pointwise as usual,
and scalar multiplication is done pointwise as well.

The important example for us is actually a bit unexpected. Suppose $E$ is an extension field of $F$. Then $E$ is an $F$-vector space.
$E$ is already an abelian group and scalar multiplication is defined in the ordinary sense of multiplication. The rest of the axioms
follow straightforwardly. Now, this is not as strange as it might look. The complex numbers, for example, are an extension field of the reals,
and we are used to going back and forth between numbers/vectors: $a+bi\iff (a,b)$. Similarly, $\mathbb{Q}(\sqrt[3]{2})$ is a $\mathbb{Q}$-vector
space as $a+b\sqrt[3]{2}+c\sqrt[3]{2}^2\iff (a,b,c)$. Furthermore, $\mathbb{R}$ is a $\mathbb{Q}$-vector space (but a really big one).

One might ask, why does one require $F$ to be a field? We can define a similar structure for a ring.
\begin{defn}
    If $R$ is a commutative ring with unity, then an \textbf{$R$-module} $M$ is an abelian group $(M,+)$ with a scalar multiplication
    $R\times M\to M$ that satisfies the properties defined above for $F$-vector spaces.
\end{defn}
These are, however, more interesting for algebra in general, and not so much for our case, where we will extensively use the field properties.

\end{document}
