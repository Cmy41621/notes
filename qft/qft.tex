\documentclass{../mathnotes}

\title{Quantum Field Theory I}
\author{Nilay Kumar}
\date{Last updated: \today}


\begin{document}

\maketitle

\section{Free scalar field two-point correlation function}

(February 7, 2013)

Recall, last time we wrote down
\begin{align*}
    \langle 0|T\hat{\phi}(x_1)\hat{\phi}(2_2)|0\rangle = \frac{\int D\phi\; e^{iS}\phi(x_1)\phi(x_2)}{\int D\phi\;e^{iS}}
\end{align*}
Note carefully that we have operators on the LHS, but functions on the RHS. Additionally, the action above has no external source $J$
(we will use $S_J$ to denote the case with the source). For example, for a free theory, we have
\begin{align*}
    S=\int d^4 x\;-\frac{1}{2}(\partial \phi)^2-\frac{1}{2}m^2\phi^2
\end{align*}
whereas with a source we have:
\begin{align*}
    S=\int d^4 x\;-\frac{1}{2}(\partial \phi)^2-\frac{1}{2}m^2\phi^2+J\phi
\end{align*}
Additionally, we will often take the normalization of our path integration measure to be such that $\langle 0|0\rangle=1=\int D\phi\; e^{iS}$.
Finally, we will often use the shorthand:
\begin{align*}
    \langle 0|T\hat{\phi}(x_1)\hat{\phi}(2_2)|0\rangle=\langle \phi_1 \phi_2\rangle.
\end{align*}

We have already worked out what such a two-point correlation function looks like for free theory. First note that
\begin{align*}
    e^{iS}=\exp\left(-\frac{1}{2}\int d^4x\;i\phi\left( -\square+m^2 \right)\phi\right),
\end{align*}
which led to
\begin{align*}
    \langle \phi_1\phi_2\rangle=-i\Delta(x_1-x_2)=-i\Delta_{12},
\end{align*}
where $\Delta_{12}$ is the Green's function that satisfies the Klein-Gordon equation
\begin{align*}
    \left( -\square_{x_1}+m^2 \right)\Delta(x_1-x_2)=\delta^{(4)}(x_1-x_2)
\end{align*}

\begin{rem}
    Note that
    \begin{align*}
        \int d^nx e^{-\frac{1}{2}x^TMx}=(2\pi)^{n/2}\frac{1}{\sqrt{\left( \det M \right)}}=(2\pi)^{n/2}\sqrt{\det M^{-1}}
    \end{align*}
    and
    \begin{align*}
        \int D\phi e^{iS}=(2\pi)^{n/2} \sqrt{\left( \det -i\Delta \right)}
    \end{align*}
    We usually absorb $J$ into the definition of $D\phi$, as mentioned earlier (in regards to normalization).
\end{rem}

Let us actually compute this Green's function. First, we added an $-i\varepsilon$ (the sign is important) to the Klein-Gordon operator to
make the path integral converge (and to project onto the ground state of the system), and, then in Fourier space, we try
\begin{align*}
    \Delta(x_1-x_2)=\int \frac{d^4k}{(2\pi)^4}f(k)e^{ik\cdot (x_1-x_2)}
\end{align*}
Consequently, we have
\begin{align*}
    \left( \square_{x_1}+m^2+i\varepsilon \right)\Delta(x_1-x_2)=\int \frac{d^4k}{(2\pi)^4}(k^2+m^2-i\varepsilon)f(k)e^{ik\cdot (x_1-x_2)},
\end{align*}
which we want to be equal to $\delta^{(4)}(x_1-x_2)$. We conclude, then, that
\begin{align*}
    \Delta(x_1-x_2)=\int\frac{d^4k}{(2\pi)^4}\frac{e^{ik\cdot(x_1-x_2)}}{k^2+m^2-i\varepsilon}.
\end{align*}
In this integral, $k^0$ is integrated over; i.e. $k^0$ need not equal $\sqrt{\mathbf{k}^2+m^2}$. We immediately encounter problems if
we attempt to compute this integral - if the $i\varepsilon$ were not present, there'd be certain values of $k^0$ that would lead
to singularities as we integrate. Thankfully, the $i\varepsilon$ makes the integral well-defined. We factorize the denominator as:
$\left( k^0+\sqrt{\mathbf{k}^2+m^2-i\varepsilon} \right)\left( k^0-\sqrt{\mathbf{k}^2+m^2-i\varepsilon} \right)$. Since we consider $\varepsilon$
to be a small positive number, we can expand the square roots, making the denominator:
$\left( k^0+\sqrt{\mathbf{k}^2+m^2}-i\varepsilon' \right)\left( k^0-\sqrt{\mathbf{k}^2+m^2}+i\varepsilon' \right)$,
where $\varepsilon'$ is just another small positive number. In fact, we'll just be sloppy and keep following this number as $\varepsilon$.

What does this factorization tell us about the location of the integrand's poles? Let us focus on the $dk^0$ integral first. Let us consider $k^0$ as
a complex number (because of the $i\varepsilon$). The integral runs over the entire real line. Using the abbreviation $\omega_k=\sqrt{\mathbf{k}^2+m^2}$,
we find that we have poles at $\omega_k-i\varepsilon$ and $-\omega_k+i\varepsilon$ (i.e. second and fourth quadrants). In this sense, the
$i\varepsilon$ prescription has eased our integration, as there are no poles physically on our contour of integration. We perform the contour integral
over the southern hemisphere; hopefully the integral along the arc will vanish. Now, take a point on this contour that is a distance $r$ away from the origin.
This point is $r\cos\theta-ir\sin\theta$. At some point we will take $r\to\infty$ but let us not do so just yet.

Inserting this point into the exponential in the numerator yields
\begin{align*}
    e^{-i(r\cos\theta-ir\sin\theta)(t_1-t_2)+i\mathbf{k}\cdot (\mathbf{x}_1-\mathbf{x}_2)},
\end{align*}
which has a part that oscillates and one that doesn't. The non-oscillating part yields $e^{-r\sin\theta(t_1-t_2)}$. Consequently, as $r\to\infty$,
as long as $t_1>t_2$, we are guaranteed that the integral over the arc will go to zero. Therefore, if $t_1>t_2$ we must integrate over the southern
hemisphere, but if $t_1<t_2$, we must integrate over the northen hemisphere. This is nice, as we seem to be getting some sense of time-ordering from
this integral: a different pole is picked up based on which contour is actually taken. We now write
\begin{align*}
    \Delta(x_1-x_2)&=\Theta(t_1>t_2)(-2\pi i)\int\frac{d^3k}{(2\pi)^4}\frac{e^{-i\omega_k(t_1-t_2)+i\mathbf{k}\cdot(\mathbf{x}_1-\mathbf{x}_2)}}{-2\omega_k}\\
    &+\Theta(t_2>t_1)(2\pi i)\int\frac{d^3k}{(2\pi)^4}\frac{e^{i\omega_k(t_1-t_2)+i\mathbf{k}\cdot(\mathbf{x}_1-\mathbf{x}_2)}}{2\omega_k},
\end{align*}
which yields our Green's function. We rewrite, now, changing the sign of the integration variable in the second integral,
\begin{align*}
    \langle\phi_1\phi_2\rangle=-i\Delta(x_1-x_2)&=\Theta(t_1>t_2)\int\frac{d^3k}{(2\pi)^32\omega_k}e^{-i\omega_k(t_1-t_2)+i\mathbf{k}\cdot(\mathbf{x}_1-\mathbf{x}_2)}\\
    &+\Theta(t_2>t_1)\int\frac{d^3k}{(2\pi)^32\omega_k}e^{i\omega_k(t_1-t_2)-i\mathbf{k}\cdot(\mathbf{x}_1-\mathbf{x}_2)},
\end{align*}
which we can also write, now assuming that $k^0=\omega_k$, i.e. we are on-shell:
\begin{align*}
    \langle\phi_1\phi_2\rangle=-i\Delta(x_1-x_2)&=\Theta(t_1>t_2)\int\frac{d^3k}{(2\pi)^32\omega_k}e^{ik\cdot(x_1-x_2}\\
    &+\Theta(t_2>t_1)\int\frac{d^3k}{(2\pi)^32\omega_k}e^{ik\cdot(x_1-x_2)},
\end{align*}
and we have computed the two-point correlation function for our free scalar field theory.

The fact that we have the two step functions of time should be very gratifying, as the LHS has explicit time ordering, by definition of the two-point
correlation function. It is a non-trivial check to compare this against the canonical quantization method that uses $\hat{a}_k,\hat{a}_k^\dagger$.
One must insert $\phi$ as an integral over momentum of these two operators to get agreement. Notice that when $t_1=t_2$, the two integrands will
be equal, so we simply add the two.

Classically, one integrates over the product of Green's functions and the source to solve for the field \textit{after} the source appears.
Our Green's function is quite odd in that regardless of whether we are before or after the source, we have non-zero support. In this sense,
this is a half-retarded half-advanced solution; what is called a \textbf{Feynman Green's function}. This raises the question of causality, and we
will at some point show that causality is implied by $[\hat{\phi}_1,\hat{\phi}_2]=0$ for the points 1 and 2 spacelike separated.

Recall that 
\begin{align*}
    \langle \phi_1 \cdots \phi_m\rangle&=\frac{\int D\phi\;e^{iS}\phi_1\cdots\phi_m}{\int D\phi\;e^{iS}}.
\end{align*}
We can always use Wick pairings to simplify these integrals in free theory. For odd $m$, we had $\langle \phi_1\cdots\phi_m$, but for $m$ even,
we have, for example,
\begin{align*}
    \langle \phi_1\phi_2\phi_3\phi_4 \rangle=\langle\phi_1\phi_2\rangle\langle\phi_3\phi_4\rangle+\langle\phi_1\phi_3\rangle\langle\phi_2\phi_4\rangle+\langle\phi_1\phi_4\rangle\langle\phi_2\phi_3\rangle
\end{align*}
Physically speaking, the fact that this disappears for $m$ odd is because the path integral integrand is odd when sone flips signs (only true for
sourceless theory).

Let us now consider turning on $J$. In other words,
\begin{align*}
    S_J=\int d^4x -\frac{1}{2}(\partial\phi)^2-\frac{1}{2}m^2\phi^2+J\phi
\end{align*}
and the path integral will be
\begin{align*}
    Z[J]=\int D\phi\;e^{iS_J}
\end{align*}
with the normalization that $Z[0]=1=\langle 0|0\rangle$ with no source. This function is called the \textbf{generating function} of the n-point functions because
when we take functional derivatives we get
\begin{align*}
    \frac{\delta Z[J]}{\delta iJ(x_1)}=\int D\phi\;\phi(x_1)e^{iS_J}
\end{align*}
where we implicity assume $\frac{\delta J(x)}{\delta J(x_1)}=\delta(x-x_1)$. If we evaluate this derivative at $J=0$, we find
\begin{align*}
    \frac{\partial Z[J]}{\delta iJ(x_1)}|_{J=0}&=\int D\phi\;\phi(x_1)e^{iS}=\langle\phi(x_1)\rangle=0\\
    \frac{\delta^mZ[J]}{\delta iJ_m \cdots \delta iJ_1}|_{J=0}&=\langle \phi_1 \cdots \phi_m \rangle.
\end{align*}
Note that the order of the functional derivatives does not matter! Of course, for free theory, we don't need this machinery at all,
but instead, becomes useful for interacting theories. Let us write down this function regardless. First recall the Gaussian integral
\begin{align*}
    \frac{\int d^nx e^{-\frac{1}{2}x^TMx+J^Tx}}{\int d^nx e^{-\frac{1}{2}x^TMx}}=e^{\frac{1}{2}J^TM^{-1}J},
\end{align*}
which suggests
\begin{align*}
    Z[J]=\exp\left( \frac{i}{2}\int d^4x d^4y\;J_x \Delta_{xy}J_y\right)
\end{align*}
One can take functional derivatives to check that this does indeed yield the $n$-point correlation functions.

\section{Energy arising from a static source}

So far we have primarily focused on formalism -- let's now try to do some physics. Let's try to calculate the energy of a static source $J(\mathbf{x})$.
Let us try to solve the classical problem:
\begin{align*}
    J_x&=\left( -\square_x+m^2 \right)\phi_x\\
    \phi_x&=\int d^4x'\;J(\mathbf{x}') \int \frac{d^4k}{(2\pi)^4}\frac{e^{ik\cdot(x-x')}}{k^2+m^2}
\end{align*}
Note that we have dropped the $i\varepsilon$ and are using the Fenyman Green's function as the source is static, and we need not worry about changes
in time of the source. We can integrate over $t'$; we obtain a delta function that forces $\omega_k=0$: $2\pi\delta(k^0)$, and we get
\begin{align*}
    \int d^3x' J(\mathbf{x}') \int \frac{d^3k}{(2\pi)^3}\frac{e^{i\mathbf{k}\cdot(\mathbf{x}-\mathbf{x'})}}{\mathbf{k}^2+m^2}
\end{align*}
There is clearly no pole in the integrand, which confirms that we do not need to use the $i\varepsilon$ prescription. The computation of the inner integral
is left for a problem set:
\begin{align*}
    V(|\mathbf{x}-\mathbf{x}'|)=\int \frac{d^3k}{(2\pi)^3}\frac{e^{i\mathbf{k}\cdot(\mathbf{x}-\mathbf{x'})}}{\mathbf{k}^2+m^2}=\frac{1}{4\pi}\frac{e^{-m|\mathbf{x}-\mathbf{x}'|}}{|\mathbf{x}-\mathbf{x}'|},
\end{align*}
which, by the way, is known as the \textbf{Yukawa potential}. This potential drops off as $1/r$ when $r\ll 1/m$, but is exponentially supressed when $r\gg 1/m$.
Why we interpret this as a potential energy will be made clear in a moment. Let us now compute the energy of our $\phi_x$ sourced by a static $J$.
The Hamiltonian density is written
\begin{align*}
    \mathcal{H}&=\Pi\dot\phi-\mathcal{L}\\
    &=\frac{1}{2}(\partial_t\phi)^2+\frac{1}{2}(\nabla\phi)^2+\frac{1}{2}m^2\phi^2-J\phi
\end{align*}
Substituting $\phi(x)=\int d^3x' J(\mathbf{x}')V(|\mathbf{x}-\mathbf{x}'|)$ and compute the Hamiltonian (check this):
\begin{align*}
    H=-\frac{1}{2}\int d^3x d^3x'\;J(\mathbf{x})J(\mathbf{x}')V(|\mathbf{x}-\mathbf{x}'|).
\end{align*}
This is very satisfying, as this expression looks exactly analogous to the energy we typically find in electrostatics. This is telling us that once
we turn on a source, it's like turning on a electrostatic-esque potential. It is almost as if this $\phi$, our ``particle'', is what \textit{mediates}
a force! Indeed, we can see that this force is attractive, due to the minus sign out front. Finally, note that a mediating particle with any mass
will mediate a force that dies off exponentially.

\section*{Field theory with interactions}

(February 12, 2013)

Recall that the LSZ formula:
\begin{itemize}
    \item gives us the scattering amplitude that involves $n$ particles
    \item is related to $\langle\phi_1\cdots\phi_n\rangle$
\end{itemize}
Our task is to compute $n$-point functionss for general interacting theory; i.e. compute $\int D\phi\; e^{iS}\phi_1\cdots\phi_n$.
This integral was possible to do exactly for Gaussian $S$, where $S$ was dependent on $\phi^2$ but no higher powers. Unfortunately,
it is not so easy for a general $S$. Take, for example, 
\begin{align*}
    S=\int d^4x \left( \mathcal{L}_0+\mathcal{L}_1 \right)
\end{align*}
with
\begin{align*}
    \mathcal{L}_0&=-\frac{1}{2}(\partial\phi)^2-\frac{1}{2}m^2\phi^2\\
    \mathcal{L}_1&=\frac{1}{6}g\phi^3.
\end{align*}
We will see that we obtain some sort of non-trivial scattering due to the cubic term. In particular, the particle's (self-)coupling
strength is dependent on the coupling constant $g$. Essentially, we are going to have to compute $\langle \phi_1\cdots\phi_n\rangle$
perturbatively in $g$.
\begin{align*}
    \langle\phi_1\cdots\phi_n\rangle=\int D\phi\;e^{iS_{\rm free}}e^{i\int d^4x\;\frac{g}{6}\phi^3(x)}\phi(x_1)\cdots\phi(x_n)
\end{align*}
If $g$ is small, we expand the second exponential to get:
\begin{align*}
    \langle\phi_1\cdots\phi_n\rangle=\int D\phi\;e^{iS_{\rm free}}\left(1+i\int d^4x\;\frac{g}{6}\phi^3(x)+\cdots\right)\phi(x_1)\cdots\phi(x_n)
\end{align*}
The first term is, in fact, just the free-field theory $n$-point function: $\langle\phi_1\cdots\phi_n\rangle_{\rm free}$. The next term
is the first interesting one:
\begin{align*}
    \int D\phi\;e^{iS_{\rm free}}\left( \frac{ig}{6} \right)\int d^4x\;\phi^3(x)\phi(x_1)\cdots\phi(x_n).
\end{align*}
Note, however, that we can write this as:
\begin{align*}
    \frac{ig}{6}\int d^4x \langle\phi^3(x)\phi(x_1)\cdots\phi(x_n)\rangle_{\rm free},
\end{align*}
which we know! This generalizes straightforwardly for higher powers, and in some sense the perturbative expansion is almost trivial!
Of course, the computation itself is quite difficult, and we will spend quite trying to deal with these calculations.

Now suppose $n=4$ (2 to 2 scattering). Then the lowest-order non-trivial scattering contribution is the $g^2$ term, due to the 
the free $n$-point function vanishing for odd $n$ ($n+3$ in this case). What we have is:
\begin{align*}
    \langle \phi_1\phi_2\phi_3\phi_4\rangle&=\langle\phi_1\cdots\phi_4\rangle+\left( \frac{ig}{6} \right)^2\int d^4x d^4y\;\langle \phi^3(x)\phi^3(y)\phi(x_1)\cdots\phi(x_4)\rangle_{\rm free}.
\end{align*}
This is a highly non-trivial $n$-point function -- there are 945 terms in the Wick expansion!
One of the terms is written:
\begin{align*}
    \langle \phi_x\phi_1\rangle\langle\phi_x\phi_2\rangle\langle\phi_y\phi_3\rangle\langle \phi_y\phi_4\rangle\langle\phi_x\phi_y\rangle
\end{align*}
This is often thought of diagrammatically through \textbf{Feynman diagrams}, where the first 2 2-point functions represent the incoming vertex,
the next 2 represent the outgoing vertex, and the last one connects these two vertices through the ``exchange'' of a $\phi$ particle. At this point,
we are in some sense done -- we know how to compute all of these terms!

Let's now treat this a little more abstractly. Take
\begin{align*}
    S_J=\int d^4x\;\mathcal{L}_0+\mathcal{L}_1+J(x)\phi
\end{align*}
with
\begin{align*}
    \mathcal{L}_0+\mathcal{L}_1=-\frac{1}{2}Z_\phi\left( \partial\phi \right)^2-\frac{1}{2}Z_m m^2\phi^2+\frac{1}{6}Z_g g\phi^3 + Y\phi
\end{align*}
where $Z_\phi,Z_m,Z_g, Y$ are some arbitrary constants. Why do we need these additional coefficients? We will impose:
\begin{enumerate}
    \item $m$ is the physical mass of the $\phi$ particle. $Z_m$ is the \textbf{mass renormalization factor}
    \item $g$ is the physical ``coupling constant.'' $Z_g$ is the \textbf{coupling renormalization factor}
    \item $\langle 0|\hat{\phi}(x)|0\rangle=0$. This fixes the linear coupling $Y$
    \item $\langle k|\hat{\phi}(x)|0\rangle=e^{-ikx}$. This fixes $Z_\phi$ -- the \textbf{wavefunction renormalization factor}
\end{enumerate}
Note that the last two bullet points were implicitly assumed in our earlier LSZ derivation. The need for all of this renormalization will
arise naturally later. Additionally, what we call $\mathcal{L}_0$ is our usual $-\frac{1}{2}\left( \partial \phi \right)^2-\frac{1}{2}m^2\phi^2$,
and so we define $\mathcal{L}_1$ accordingly. Formally speaking, we now have many different interaction terms. But let us focus on the cubic term
for now.

We have the generating function
\begin{align*}
    Z_1(J)&\equiv\int D\phi\;e^{i\int d^4x\; \mathcal{L}_0+\frac{1}{6}Z_gg\phi^3+J\phi}=\left(e^{i\int d^4x\;\frac{1}{6}Z_gg \left(\frac{\delta}{\delta iJ(x)}\right)^3}\right)Z_0(J)\\
    &=\left(e^{i\int d^4x\;\frac{1}{6}Z_gg \left(\frac{\delta}{\delta iJ(x)}\right)^3}\right)e^{\frac{i}{2}\int d^4x d^4z\;J_y\Delta_{zy}J_z}
\end{align*}
where $Z_0(J)$ is the generating function for free theory. To see why the first equality holds, one can expand out this exponential and take functional
derivatives. We can rewrite this even further:
\begin{align*}
    Z_1(J)=\sum_{v=0}\frac{1}{v!}\left[ \frac{iZ_gg}{6}\int d^4x\;\left( \frac{\delta}{\delta iJ} \right)^3 \right]^v\sum_{P=0}\frac{1}{P!}\left[ \frac{1}{2}\int d^4y d^4z\; (iJ_y)\frac{1}{i}\Delta_{yz}(iJ_z)\right]^P.
\end{align*}
What do these derivatives do? We will organize the terms by the number of $J$'s left, which is a number that we will call $E$ (for external sources).
We have
\begin{align*}
    E=2P-3V,
\end{align*}
just by inspection of the above mess. Let's first look at the terms with $E=0$. Take, for example, the term where $V=2,P=3$:
\begin{align*}
    \frac{1}{2!}&\left[ \frac{iZ_gg}{6}\int d^4x\left( \frac{\delta}{\delta iJ_x} \right)^3 \right]\left[ \frac{iZ_gg}{6}\int d^4y\left( \frac{\delta}{\delta iJ_y} \right)^3 \right]\\
    \times &\left[ \frac{1}{2}\int d^4a d^4a'\; (iJ_a)\frac{1}{i}\Delta_{aa'}(iJ_a')\right]\\
    \times &\left[ \frac{1}{2}\int d^4b d^4b'\; (iJ_b)\frac{1}{i}\Delta_{bb'}(iJ_b')\right]\\
    \times &\left[ \frac{1}{2}\int d^4c d^4c'\; (iJ_c)\frac{1}{i}\Delta_{cc'}(iJ_c')\right]
\end{align*}
Writing it out in all the gory detail, it should be clear that each functional derivative now is paired up with one $J$, and then integrated over
(with the presence of delta functions). One should note that, for connected Feynman diagrams, each term in the above equation will look something like
\begin{align*}
    \frac{1}{S}\left( iZ_gg \right)^2\int d^4x d^4y\;\left( \frac{1}{i}\Delta_{xy} \right)^3
\end{align*}
where $S$ is some numerical, combinatoric symmetry factor.
The Feynman diagram is called will have 2 vertices and 3 propogators, with no sources left -- in general, this diagram can take a variety of different forms.
The diagram consisting of 2 loops connected by a propogator is in fact buried in the mess above, and it should be clear that (similar to the Wick expansions above)
it corresponds to a term that looks like:
\begin{align*}
    \frac{1}{S}\left( iZ_gg \right)^2\left( \frac{1}{i}\Delta(0) \right)^2\int d^4x d^4y\;\frac{1}{i}\Delta_{xy}.
\end{align*}

Let us try to compute the symmetry factor for the Feynman diagram that looks like a $\Theta$:
\begin{align*}
    \frac{1}{S}=\frac{1}{2!}\frac{1}{6^2}\frac{1}{3^!}\frac{1}{2^3}\cdot 6\cdot 4\cdot 2\cdot 3\cdot 2=\frac{1}{12}
\end{align*}
based off the coefficients in the mess above and the number of ways each of the functional  derivatives can act. This is the brute-force method.
A slightly more elegant method is to note that the symmetry factor is simply how much you overcount if you naively assume that each 3-vertex can
be permuted and swapped, and that each propogator can be permuted and flipped and that one should get $S=1$. Computing these factors is generally
a little confusing, and we will not trouble ourselves too much at the moment.

Diagrams with no external legs are generically known as \textbf{bubble diagrams}, and typically correspond to \textbf{virtual particles} -- particles that
pop out of the vacuum and then disappear. More precisely, these are additional contributions to the vacuum energy (the cosmological constant).


\section*{What \textit{is} the scalar field?}

(February 14, 2013)

Think about the surface of the ocean (or a drum). In the ground state the surface is, of course, flat. Any deviation from the ground state
can obviously be represented by a scalar field (as there is only a magnitude of deviation from flat). We know that one can describe
waves in such a field as 
\begin{align*}
    \left(-\frac{\partial^2}{\partial t^2}+c_s^2\frac{\partial^2}{\partial \mathbf{x}^2}\right)\phi(\mathbf{x},t)=0
\end{align*}
It should be clear that there exist solutions of the form $e^{-i\omega_kt +i\mathbf{k}\cdot\mathbf{x}}$ which yield the dispersion relation $\omega_k^2=c_s^2\mathbf{k}^2$.
We can even guess the action to be
\begin{align*}
    S=\int dt d^3x\;\left( \frac{1}{2}(\partial_t\phi)^2-\frac{c_s^2}{2}\left( \nabla \phi \right)^2 \right)
\end{align*}
In field theory language, we would say this field describes a massless particle. But where did this particle come from?

Well, quantum mechanically, it is impossible for the scalar field to be perfectly flat -- there will inevitably be fluctuations. Indeed, in computing
the $n$-point functions that we have been doing, we've been computing amplitudes of non-trivial phenomena in the ground state! This is clearly behavior that
is incompatible with classical mechanics. In fact, as we showed, Fourier decomposing the quantum fluctuations (i.e. Fourier transforming $\phi$) yields some
$a_k,a_k^\dagger$ operators. Of course, there is no physical meaning of these coefficients yet -- they are simply Fourier coefficients. It is imposing the
commutation relations in a way similar to that done for the harmonic oscillator,
\begin{align*}
    [a_k,a^\dagger_{k'}]=\delta(k-k')
\end{align*}
that provokes us to think of these operators as ``creation'' and ``annihilation'' operators. However, one must be careful to treat the scalar field as a field
not of a media, but in field theory, we imagine it residing in vacuum.

But $\phi$ is itself an operator. What are its eigenvalues and eigenstates? We take
\begin{align*}
    \hat{\phi}(x)|f(x)\rangle=f(x)|f(x)\rangle.
\end{align*}
What is this eigenstate? Roughly, it is a field configuration in the shape of $f(x)$, and thus any function $f(x)$ can be put in the above equation.
Now the question arises, how does the function $f(x)$ evolve? First note that $\hat{\phi}$ is a Heisenberg operator:
\begin{align*}
    \hat{\phi}(x,t)=e^{iHt}\hat{\phi}(x,0)e^{-iHt}.
\end{align*}
For the sake of clarity let us rewrite the eigen-equation as
\begin{align*}
    \hat{\phi}(x,t)|f(x),t\rangle=f(x)|f(x),t\rangle.
\end{align*}
Inserting the Heisenberg expression into the eigen-equation, we have that
\begin{align*}
    e^{iHt}\hat{\phi}(x,0)e^{-iHt}|f(x),t\rangle=f(x)|f(x),t\rangle.
\end{align*}
Additionally, at time 0, the Heisenberg and Schrodinger pictures line up:
\begin{align*}
    \hat{\phi}(x,0)|f(x),0\rangle=f(x)|f(x),0\rangle,
\end{align*}
Enforcing both of these equations, we find
\begin{align*}
    |f(x,t)\rangle=e^{iHt}|f(x),0\rangle,
\end{align*}
i.e. that the Heisenberg eigenstates evolve backwards in time, while the Heisenberg operators evolve forwards in time.

\section*{Bubble diagrams}

Let us now return to bubble diagrams. Recall that we have
\begin{align*}
    S=\int d^4x\left( \mathcal{L}_0 +\mathcal{L}_{\rm int}\right)
\end{align*}
where
\begin{align*}
    \mathcal{L}_{\rm int}=-\frac{1}{2}(Z_\phi - 1)(\partial \phi)^2-\frac{1}{2}(Z_m-1)m^2\phi^2+\frac{1}{3!}Z_gg\phi^3+Y\phi +\Lambda.
\end{align*}
It is still unclear why in adding the cubic term, we have been forced by quantum mechanics to add in many of these extra terms.
These additional terms are typically known as \textbf{counter terms}.
Let's see if we can determine where the cosmological constant $\Lambda$ come from.

We considered the generating function with only a cubic term (and an external source):
\begin{align*}
    Z_1[J]=\int D\phi e^{i\int d^4x\left( \mathcal{L}_0+\frac{1}{3!}Z_gg\phi^3+J\phi \right)}
\end{align*}
There are a huge number of terms in this path integral. We considered the terms with no $J$, that led to a myriad of bubble diagrams.
There are many others, with arbitrary high numbers of $J$'s.

\begin{thm}
    \begin{align*}
        Z_1[J]=\exp\left(\sum\nm{connected diagrams}\right)
    \end{align*}
    In other words, exponentiating the connected diagrams yields the full result (including the disconnected diagrams).
\end{thm}
\begin{proof}
    Take the sum over diagrams
    \begin{align*}
        \sum_{\left\{ n_I \right\}} D^{\left\{ n_I \right\}}
    \end{align*}
    where we $n_I$ labels each different type of diagram. This can be rewritten:
    \begin{align*}
        \sum\frac{1}{S_D}\prod_I(c_I)^{n_I}&=\sum\frac{1}{\prod_In_I!}\prod_Ic_I^{n_I}=\sum\prod_I\frac{c_I^{n_I}}{n_I!}\\
        &=\prod_I\sum_{n_I=0}^\infty \frac{c_I^{n_I}}{n_I!}=\prod_I e^{c_I}=e^{\sum_Ic_I},
    \end{align*}
    where the nontrivial step is the last one, which you should probably think a little bit about.
\end{proof}


\end{document}
