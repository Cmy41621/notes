\documentclass{../mathnotes}

\title{Quantum Field Theory I}
\author{Nilay Kumar}
\date{Last updated: \today}


\begin{document}

\maketitle

\section{Path integral formulation of field theory}

Recall, last time we wrote down
\begin{align*}
    \langle 0|T\hat{\phi}(x_1)\hat{\phi}(2_2)|0\rangle = \frac{\int D\phi\; e^{iS}\phi(x_1)\phi(x_2)}{\int D\phi\;e^{iS}}
\end{align*}
Note carefully that we have operators on the LHS, but functions on the RHS. Additionally, the action above has no external source $J$
(we will use $S_J$ to denote the case with the source). For example, for a free theory, we have
\begin{align*}
    S=\int d^4 x\;-\frac{1}{2}(\partial \phi)^2-\frac{1}{2}m^2\phi^2
\end{align*}
whereas with a source we have:
\begin{align*}
    S=\int d^4 x\;-\frac{1}{2}(\partial \phi)^2-\frac{1}{2}m^2\phi^2+J\phi
\end{align*}
Additionally, we will often take the normalization of our path integration measure to be such that $\langle 0|0\rangle=1=\int D\phi\; e^{iS}$.
Finally, we will often use the shorthand:
\begin{align*}
    \langle 0|T\hat{\phi}(x_1)\hat{\phi}(2_2)|0\rangle=\langle \phi_1 \phi_2\rangle.
\end{align*}

We have already worked out what such a two-point correlation function looks like for free theory. First note that
\begin{align*}
    e^{iS}=\exp\left(-\frac{1}{2}\int d^4x\;i\phi\left( -\square+m^2 \right)\phi\right),
\end{align*}
which led to
\begin{align*}
    \langle \phi_1\phi_2\rangle=-i\Delta(x_1-x_2)=-i\Delta_{12},
\end{align*}
where $\Delta_{12}$ is the Green's function that satisfies the Klein-Gordon equation
\begin{align*}
    \left( -\square_{x_1}+m^2 \right)\Delta(x_1-x_2)=\delta^{(4)}(x_1-x_2)
\end{align*}

\begin{rem}
    Note that
    \begin{align*}
        \int d^nx e^{-\frac{1}{2}x^TMx}=(2\pi)^{n/2}\frac{1}{\sqrt{\left( \det M \right)}}=(2\pi)^{n/2}\sqrt{\det M^{-1}}
    \end{align*}
    and
    \begin{align*}
        \int D\phi e^{iS}=(2\pi)^{n/2} \sqrt{\left( \det -i\Delta \right)}
    \end{align*}
    We usually absorb $J$ into the definition of $D\phi$, as mentioned earlier (in regards to normalization).
\end{rem}

Let us actually compute this Green's function. First, we added an $-i\varepsilon$ (the sign is important) to the Klein-Gordon operator to
make the path integral converge (and to project onto the ground state of the system), and, then in Fourier space, we try
\begin{align*}
    \Delta(x_1-x_2)=\int \frac{d^4k}{(2\pi)^4}f(k)e^{ik\cdot (x_1-x_2)}
\end{align*}
Consequently, we have
\begin{align*}
    \left( \square_{x_1}+m^2+i\varepsilon \right)\Delta(x_1-x_2)=\int \frac{d^4k}{(2\pi)^4}(k^2+m^2-i\varepsilon)f(k)e^{ik\cdot (x_1-x_2)},
\end{align*}
which we want to be equal to $\delta^{(4)}(x_1-x_2)$. We conclude, then, that
\begin{align*}
    \Delta(x_1-x_2)=\int\frac{d^4k}{(2\pi)^4}\frac{e^{ik\cdot(x_1-x_2)}}{k^2+m^2-i\varepsilon}.
\end{align*}
In this integral, $k^0$ is integrated over; i.e. $k^0$ need not equal $\sqrt{\mathbf{k}^2+m^2}$. We immediately encounter problems if
we attempt to compute this integral - if the $i\varepsilon$ were not present, there'd be certain values of $k^0$ that would lead
to singularities as we integrate. Thankfully, the $i\varepsilon$ makes the integral well-defined. We factorize the denominator as:
$\left( k^0+\sqrt{\mathbf{k}^2+m^2-i\varepsilon} \right)\left( k^0-\sqrt{\mathbf{k}^2+m^2-i\varepsilon} \right)$. Since we consider $\varepsilon$
to be a small positive number, we can expand the square roots, making the denominator:
$\left( k^0+\sqrt{\mathbf{k}^2+m^2}-i\varepsilon' \right)\left( k^0-\sqrt{\mathbf{k}^2+m^2}-i\varepsilon' \right)$,
where $\varepsilon'$ is just another small positive number. In fact, we'll just be sloppy and keep following this number as $\varepsilon$.

What does this factorization tell us about the location of the integrand's poles? Let us focus on the $dk^0$ integral first. Let us consider $k^0$ as
a complex number (because of the $i\varepsilon$). The integral runs over the entire real line. Using the abbreviation $\omega_k=\sqrt{\mathbf{k}^2+m^2}$,
we find that we have poles at $\omega_k-i\varepsilon$ and $-\omega_k+i\varepsilon$ (i.e. second and fourth quadrants). In this sense, the
$i\varepsilon$ prescription has eased our integration, as there are no poles physically on our contour of integration. We perform the contour integral
over the southern hemisphere; hopefully the integral along the arc will vanish. Now, take a point on this contour that is a distance $r$ away from the origin.
This point is $r\cos\theta-ir\sin\theta$. At some point we will take $r\to\infty$ but let us not do so just yet.

Inserting this point into the exponential in the numerator yields
\begin{align*}
    e^{-i(r\cos\theta-ir\sin\theta)(t_1-t_2)+i\mathbf{k}\cdot (\mathbf{x}_1-\mathbf{x}_2)},
\end{align*}
which has a part that oscillates and one that doesn't. The non-oscillating part yields $e^{-r\sin\theta(t_1-t_2)}$. Consequently, as $r\to\infty$,
as long as $t_1>t_2$, we are guaranteed that the integral over the arc will go to zero. Therefore, if $t_1>t_2$ we must integrate over the southern
hemisphere, but if $t_1<t_2$, we must integrate over the northen hemisphere. This is nice, as we seem to be getting some sense of time-ordering from
this integral: a different pole is picked up based on which contour is actually taken. We now write
\begin{align*}
    \Delta(x_1-x_2)&=\Theta(t_1>t_2)(-2\pi i)\int\frac{d^3k}{(2\pi)^4}\frac{e^{-i\omega_k(t_1-t_2)+i\mathbf{k}\cdot(\mathbf{x}_1-\mathbf{x}_2)}}{-2\omega_k}\\
    &+\Theta(t_2>t_1)(2\pi i)\int\frac{d^3k}{(2\pi)^4}\frac{e^{i\omega_k(t_1-t_2)+i\mathbf{k}\cdot(\mathbf{x}_1-\mathbf{x}_2)}}{2\omega_k},
\end{align*}
which yields our Green's function. We rewrite, now, changing the sign of the integration variable in the second integral,
\begin{align*}
    \langle\phi_1\phi_2\rangle=-i\Delta(x_1-x_2)&=\Theta(t_1>t_2)\int\frac{d^3k}{(2\pi)^32\omega_k}e^{-i\omega_k(t_1-t_2)+i\mathbf{k}\cdot(\mathbf{x}_1-\mathbf{x}_2)}\\
    &+\Theta(t_2>t_1)\int\frac{d^3k}{(2\pi)^32\omega_k}e^{i\omega_k(t_1-t_2)-i\mathbf{k}\cdot(\mathbf{x}_1-\mathbf{x}_2)},
\end{align*}
which we can also write, now assuming that $k^0=\omega_k$, i.e. we are on-shell:
\begin{align*}
    \langle\phi_1\phi_2\rangle=-i\Delta(x_1-x_2)&=\Theta(t_1>t_2)\int\frac{d^3k}{(2\pi)^32\omega_k}e^{ik\cdot(x_1-x_2}\\
    &+\Theta(t_2>t_1)\int\frac{d^3k}{(2\pi)^32\omega_k}e^{ik\cdot(x_1-x_2)},
\end{align*}
and we have computed the two-point correlation function for our free scalar field theory.

The fact that we have the two step functions of time should be very gratifying, as the LHS has explicit time ordering, by definition of the two-point
correlation function. It is a non-trivial check to compare this against the canonical quantization method that uses $\hat{a}_k,\hat{a}_k^\dagger$.
One must insert $\phi$ as an integral over momentum of these two operators to get agreement.


\end{document}
