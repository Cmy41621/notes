\documentclass{../mathnotes}

\title{Quantum Field Theory I}
\author{Nilay Kumar}
\date{Last updated: \today}


\begin{document}

\maketitle

\section{Free scalar field two-point correlation function}

Recall, last time we wrote down
\begin{align*}
    \langle 0|T\hat{\phi}(x_1)\hat{\phi}(2_2)|0\rangle = \frac{\int D\phi\; e^{iS}\phi(x_1)\phi(x_2)}{\int D\phi\;e^{iS}}
\end{align*}
Note carefully that we have operators on the LHS, but functions on the RHS. Additionally, the action above has no external source $J$
(we will use $S_J$ to denote the case with the source). For example, for a free theory, we have
\begin{align*}
    S=\int d^4 x\;-\frac{1}{2}(\partial \phi)^2-\frac{1}{2}m^2\phi^2
\end{align*}
whereas with a source we have:
\begin{align*}
    S=\int d^4 x\;-\frac{1}{2}(\partial \phi)^2-\frac{1}{2}m^2\phi^2+J\phi
\end{align*}
Additionally, we will often take the normalization of our path integration measure to be such that $\langle 0|0\rangle=1=\int D\phi\; e^{iS}$.
Finally, we will often use the shorthand:
\begin{align*}
    \langle 0|T\hat{\phi}(x_1)\hat{\phi}(2_2)|0\rangle=\langle \phi_1 \phi_2\rangle.
\end{align*}

We have already worked out what such a two-point correlation function looks like for free theory. First note that
\begin{align*}
    e^{iS}=\exp\left(-\frac{1}{2}\int d^4x\;i\phi\left( -\square+m^2 \right)\phi\right),
\end{align*}
which led to
\begin{align*}
    \langle \phi_1\phi_2\rangle=-i\Delta(x_1-x_2)=-i\Delta_{12},
\end{align*}
where $\Delta_{12}$ is the Green's function that satisfies the Klein-Gordon equation
\begin{align*}
    \left( -\square_{x_1}+m^2 \right)\Delta(x_1-x_2)=\delta^{(4)}(x_1-x_2)
\end{align*}

\begin{rem}
    Note that
    \begin{align*}
        \int d^nx e^{-\frac{1}{2}x^TMx}=(2\pi)^{n/2}\frac{1}{\sqrt{\left( \det M \right)}}=(2\pi)^{n/2}\sqrt{\det M^{-1}}
    \end{align*}
    and
    \begin{align*}
        \int D\phi e^{iS}=(2\pi)^{n/2} \sqrt{\left( \det -i\Delta \right)}
    \end{align*}
    We usually absorb $J$ into the definition of $D\phi$, as mentioned earlier (in regards to normalization).
\end{rem}

Let us actually compute this Green's function. First, we added an $-i\varepsilon$ (the sign is important) to the Klein-Gordon operator to
make the path integral converge (and to project onto the ground state of the system), and, then in Fourier space, we try
\begin{align*}
    \Delta(x_1-x_2)=\int \frac{d^4k}{(2\pi)^4}f(k)e^{ik\cdot (x_1-x_2)}
\end{align*}
Consequently, we have
\begin{align*}
    \left( \square_{x_1}+m^2+i\varepsilon \right)\Delta(x_1-x_2)=\int \frac{d^4k}{(2\pi)^4}(k^2+m^2-i\varepsilon)f(k)e^{ik\cdot (x_1-x_2)},
\end{align*}
which we want to be equal to $\delta^{(4)}(x_1-x_2)$. We conclude, then, that
\begin{align*}
    \Delta(x_1-x_2)=\int\frac{d^4k}{(2\pi)^4}\frac{e^{ik\cdot(x_1-x_2)}}{k^2+m^2-i\varepsilon}.
\end{align*}
In this integral, $k^0$ is integrated over; i.e. $k^0$ need not equal $\sqrt{\mathbf{k}^2+m^2}$. We immediately encounter problems if
we attempt to compute this integral - if the $i\varepsilon$ were not present, there'd be certain values of $k^0$ that would lead
to singularities as we integrate. Thankfully, the $i\varepsilon$ makes the integral well-defined. We factorize the denominator as:
$\left( k^0+\sqrt{\mathbf{k}^2+m^2-i\varepsilon} \right)\left( k^0-\sqrt{\mathbf{k}^2+m^2-i\varepsilon} \right)$. Since we consider $\varepsilon$
to be a small positive number, we can expand the square roots, making the denominator:
$\left( k^0+\sqrt{\mathbf{k}^2+m^2}-i\varepsilon' \right)\left( k^0-\sqrt{\mathbf{k}^2+m^2}-i\varepsilon' \right)$,
where $\varepsilon'$ is just another small positive number. In fact, we'll just be sloppy and keep following this number as $\varepsilon$.

What does this factorization tell us about the location of the integrand's poles? Let us focus on the $dk^0$ integral first. Let us consider $k^0$ as
a complex number (because of the $i\varepsilon$). The integral runs over the entire real line. Using the abbreviation $\omega_k=\sqrt{\mathbf{k}^2+m^2}$,
we find that we have poles at $\omega_k-i\varepsilon$ and $-\omega_k+i\varepsilon$ (i.e. second and fourth quadrants). In this sense, the
$i\varepsilon$ prescription has eased our integration, as there are no poles physically on our contour of integration. We perform the contour integral
over the southern hemisphere; hopefully the integral along the arc will vanish. Now, take a point on this contour that is a distance $r$ away from the origin.
This point is $r\cos\theta-ir\sin\theta$. At some point we will take $r\to\infty$ but let us not do so just yet.

Inserting this point into the exponential in the numerator yields
\begin{align*}
    e^{-i(r\cos\theta-ir\sin\theta)(t_1-t_2)+i\mathbf{k}\cdot (\mathbf{x}_1-\mathbf{x}_2)},
\end{align*}
which has a part that oscillates and one that doesn't. The non-oscillating part yields $e^{-r\sin\theta(t_1-t_2)}$. Consequently, as $r\to\infty$,
as long as $t_1>t_2$, we are guaranteed that the integral over the arc will go to zero. Therefore, if $t_1>t_2$ we must integrate over the southern
hemisphere, but if $t_1<t_2$, we must integrate over the northen hemisphere. This is nice, as we seem to be getting some sense of time-ordering from
this integral: a different pole is picked up based on which contour is actually taken. We now write
\begin{align*}
    \Delta(x_1-x_2)&=\Theta(t_1>t_2)(-2\pi i)\int\frac{d^3k}{(2\pi)^4}\frac{e^{-i\omega_k(t_1-t_2)+i\mathbf{k}\cdot(\mathbf{x}_1-\mathbf{x}_2)}}{-2\omega_k}\\
    &+\Theta(t_2>t_1)(2\pi i)\int\frac{d^3k}{(2\pi)^4}\frac{e^{i\omega_k(t_1-t_2)+i\mathbf{k}\cdot(\mathbf{x}_1-\mathbf{x}_2)}}{2\omega_k},
\end{align*}
which yields our Green's function. We rewrite, now, changing the sign of the integration variable in the second integral,
\begin{align*}
    \langle\phi_1\phi_2\rangle=-i\Delta(x_1-x_2)&=\Theta(t_1>t_2)\int\frac{d^3k}{(2\pi)^32\omega_k}e^{-i\omega_k(t_1-t_2)+i\mathbf{k}\cdot(\mathbf{x}_1-\mathbf{x}_2)}\\
    &+\Theta(t_2>t_1)\int\frac{d^3k}{(2\pi)^32\omega_k}e^{i\omega_k(t_1-t_2)-i\mathbf{k}\cdot(\mathbf{x}_1-\mathbf{x}_2)},
\end{align*}
which we can also write, now assuming that $k^0=\omega_k$, i.e. we are on-shell:
\begin{align*}
    \langle\phi_1\phi_2\rangle=-i\Delta(x_1-x_2)&=\Theta(t_1>t_2)\int\frac{d^3k}{(2\pi)^32\omega_k}e^{ik\cdot(x_1-x_2}\\
    &+\Theta(t_2>t_1)\int\frac{d^3k}{(2\pi)^32\omega_k}e^{ik\cdot(x_1-x_2)},
\end{align*}
and we have computed the two-point correlation function for our free scalar field theory.

The fact that we have the two step functions of time should be very gratifying, as the LHS has explicit time ordering, by definition of the two-point
correlation function. It is a non-trivial check to compare this against the canonical quantization method that uses $\hat{a}_k,\hat{a}_k^\dagger$.
One must insert $\phi$ as an integral over momentum of these two operators to get agreement. Notice that when $t_1=t_2$, the two integrands will
be equal, so we simply add the two.

Classically, one integrates over the product of Green's functions and the source to solve for the field \textit{after} the source appears.
Our Green's function is quite odd in that regardless of whether we are before or after the source, we have non-zero support. In this sense,
this is a half-retarded half-advanced solution; what is called a \textbf{Feynman Green's function}. This raises the question of causality, and we
will at some point show that causality is implied by $[\hat{\phi}_1,\hat{\phi}_2]=0$ for the points 1 and 2 spacelike separated.

Recall that 
\begin{align*}
    \langle \phi_1 \cdots \phi_m\rangle&=\frac{\int D\phi\;e^{iS}\phi_1\cdots\phi_m}{\int D\phi\;e^{iS}}.
\end{align*}
We can always use Wick pairings to simplify these integrals in free theory. For odd $m$, we had $\langle \phi_1\cdots\phi_m$, but for $m$ even,
we have, for example,
\begin{align*}
    \langle \phi_1\phi_2\phi_3\phi_4 \rangle=\langle\phi_1\phi_2\rangle\langle\phi_3\phi_4\rangle+\langle\phi_1\phi_3\rangle\langle\phi_2\phi_4\rangle+\langle\phi_1\phi_4\rangle\langle\phi_2\phi_3\rangle
\end{align*}
Physically speaking, the fact that this disappears for $m$ odd is because the path integral integrand is odd when sone flips signs (only true for
sourceless theory).

Let us now consider turning on $J$. In other words,
\begin{align*}
    S_J=\int d^4x -\frac{1}{2}(\partial\phi)^2-\frac{1}{2}m^2\phi^2+J\phi
\end{align*}
and the path integral will be
\begin{align*}
    Z[J]=\int D\phi\;e^{iS_J}
\end{align*}
with the normalization that $Z[0]=1=\langle 0|0\rangle$ with no source. This function is called the \textbf{generating function} of the n-point functions because
when we take functional derivatives we get
\begin{align*}
    \frac{\delta Z[J]}{\delta iJ(x_1)}=\int D\phi\;\phi(x_1)e^{iS_J}
\end{align*}
where we implicity assume $\frac{\delta J(x)}{\delta J(x_1)}=\delta(x-x_1)$. If we evaluate this derivative at $J=0$, we find
\begin{align*}
    \frac{\partial Z[J]}{\delta iJ(x_1)}|_{J=0}&=\int D\phi\;\phi(x_1)e^{iS}=\langle\phi(x_1)\rangle=0\\
    \frac{\delta^mZ[J]}{\delta iJ_m \cdots \delta iJ_1}|_{J=0}&=\langle \phi_1 \cdots \phi_m \rangle.
\end{align*}
Note that the order of the functional derivatives does not matter! Of course, for free theory, we don't need this machinery at all,
but instead, becomes useful for interacting theories. Let us write down this function regardless. First recall the Gaussian integral
\begin{align*}
    \frac{\int d^nx e^{-\frac{1}{2}x^TMx+J^Tx}}{\int d^nx e^{-\frac{1}{2}x^TMx}}=e^{\frac{1}{2}J^TM^{-1}J},
\end{align*}
which suggests
\begin{align*}
    Z[J]=\exp\left( \frac{i}{2}\int d^4x d^4y\;J_x \Delta_{xy}J_y\right)
\end{align*}
One can take functional derivatives to check that this does indeed yield the $n$-point correlation functions.

\section{Energy arising from a static source}

So far we have primarily focused on formalism -- let's now try to do some physics. Let's try to calculate the energy of a static source $J(\mathbf{x})$.
Let us try to solve the classical problem:
\begin{align*}
    J_x&=\left( -\square_x+m^2 \right)\phi_x\\
    \phi_x&=\int d^4x'\;J(\mathbf{x}') \int \frac{d^4k}{(2\pi)^4}\frac{e^{ik\cdot(x-x')}}{k^2+m^2}
\end{align*}
Note that we have dropped the $i\varepsilon$ and are using the Fenyman Green's function as the source is static, and we need not worry about changes
in time of the source. We can integrate over $t'$; we obtain a delta function that forces $\omega_k=0$: $2\pi\delta(k^0)$, and we get
\begin{align*}
    \int d^3x' J(\mathbf{x}') \int \frac{d^3k}{(2\pi)^3}\frac{e^{i\mathbf{k}\cdot(\mathbf{x}-\mathbf{x'})}}{\mathbf{k}^2+m^2}
\end{align*}
There is clearly no pole in the integrand, which confirms that we do not need to use the $i\varepsilon$ prescription. The computation of the inner integral
is left for a problem set:
\begin{align*}
    V(|\mathbf{x}-\mathbf{x}'|)=\int \frac{d^3k}{(2\pi)^3}\frac{e^{i\mathbf{k}\cdot(\mathbf{x}-\mathbf{x'})}}{\mathbf{k}^2+m^2}=\frac{1}{4\pi}\frac{e^{-m|\mathbf{x}-\mathbf{x}'|}}{|\mathbf{x}-\mathbf{x}'|},
\end{align*}
which, by the way, is known as the \textbf{Yukawa potential}. This potential drops off as $1/r$ when $r\ll 1/m$, but is exponentially supressed when $r\gg 1/m$.
Why we interpret this as a potential energy will be made clear in a moment. Let us now compute the energy of our $\phi_x$ sourced by a static $J$.
The Hamiltonian density is written
\begin{align*}
    \mathcal{H}&=\Pi\dot\phi-\mathcal{L}\\
    &=\frac{1}{2}(\partial_t\phi)^2+\frac{1}{2}(\nabla\phi)^2+\frac{1}{2}m^2\phi^2-J\phi
\end{align*}
Substituting $\phi(x)=\int d^3x' J(\mathbf{x}')V(|\mathbf{x}-\mathbf{x}'|)$ and compute the Hamiltonian (check this):
\begin{align*}
    H=-\frac{1}{2}\int d^3x d^3x'\;J(\mathbf{x})J(\mathbf{x}')V(|\mathbf{x}-\mathbf{x}'|).
\end{align*}
This is very satisfying, as this expression looks exactly analogous to the energy we typically find in electrostatics. This is telling us that once
we turn on a source, it's like turning on a electrostatic-esque potential. It is almost as if this $\phi$, our ``particle'', is what \textit{mediates}
a force! Indeed, we can see that this force is attractive, due to the minus sign out front. Finally, note that a mediating particle with any mass
will mediate a force that dies off exponentially.

\end{document}
