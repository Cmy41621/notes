\documentclass{../mathnotes}

\title{Quantum Field Theory I}
\author{Nilay Kumar}
\date{Last updated: \today}


\begin{document}

\maketitle

\section{Free scalar field two-point correlation function}

(February 7, 2013)

Recall, last time we wrote down
\begin{align*}
    \langle 0|T\hat{\phi}(x_1)\hat{\phi}(2_2)|0\rangle = \frac{\int D\phi\; e^{iS}\phi(x_1)\phi(x_2)}{\int D\phi\;e^{iS}}
\end{align*}
Note carefully that we have operators on the LHS, but functions on the RHS. Additionally, the action above has no external source $J$
(we will use $S_J$ to denote the case with the source). For example, for a free theory, we have
\begin{align*}
    S=\int d^4 x\;-\frac{1}{2}(\partial \phi)^2-\frac{1}{2}m^2\phi^2
\end{align*}
whereas with a source we have:
\begin{align*}
    S=\int d^4 x\;-\frac{1}{2}(\partial \phi)^2-\frac{1}{2}m^2\phi^2+J\phi
\end{align*}
Additionally, we will often take the normalization of our path integration measure to be such that $\langle 0|0\rangle=1=\int D\phi\; e^{iS}$.
Finally, we will often use the shorthand:
\begin{align*}
    \langle 0|T\hat{\phi}(x_1)\hat{\phi}(2_2)|0\rangle=\langle \phi_1 \phi_2\rangle.
\end{align*}

We have already worked out what such a two-point correlation function looks like for free theory. First note that
\begin{align*}
    e^{iS}=\exp\left(-\frac{1}{2}\int d^4x\;i\phi\left( -\square+m^2 \right)\phi\right),
\end{align*}
which led to
\begin{align*}
    \langle \phi_1\phi_2\rangle=-i\Delta(x_1-x_2)=-i\Delta_{12},
\end{align*}
where $\Delta_{12}$ is the Green's function that satisfies the Klein-Gordon equation
\begin{align*}
    \left( -\square_{x_1}+m^2 \right)\Delta(x_1-x_2)=\delta^{(4)}(x_1-x_2)
\end{align*}

\begin{rem}
    Note that
    \begin{align*}
        \int d^nx e^{-\frac{1}{2}x^TMx}=(2\pi)^{n/2}\frac{1}{\sqrt{\left( \det M \right)}}=(2\pi)^{n/2}\sqrt{\det M^{-1}}
    \end{align*}
    and
    \begin{align*}
        \int D\phi e^{iS}=(2\pi)^{n/2} \sqrt{\left( \det -i\Delta \right)}
    \end{align*}
    We usually absorb $J$ into the definition of $D\phi$, as mentioned earlier (in regards to normalization).
\end{rem}

Let us actually compute this Green's function. First, we added an $-i\varepsilon$ (the sign is important) to the Klein-Gordon operator to
make the path integral converge (and to project onto the ground state of the system), and, then in Fourier space, we try
\begin{align*}
    \Delta(x_1-x_2)=\int \frac{d^4k}{(2\pi)^4}f(k)e^{ik\cdot (x_1-x_2)}
\end{align*}
Consequently, we have
\begin{align*}
    \left( \square_{x_1}+m^2+i\varepsilon \right)\Delta(x_1-x_2)=\int \frac{d^4k}{(2\pi)^4}(k^2+m^2-i\varepsilon)f(k)e^{ik\cdot (x_1-x_2)},
\end{align*}
which we want to be equal to $\delta^{(4)}(x_1-x_2)$. We conclude, then, that
\begin{align*}
    \Delta(x_1-x_2)=\int\frac{d^4k}{(2\pi)^4}\frac{e^{ik\cdot(x_1-x_2)}}{k^2+m^2-i\varepsilon}.
\end{align*}
In this integral, $k^0$ is integrated over; i.e. $k^0$ need not equal $\sqrt{\mathbf{k}^2+m^2}$. We immediately encounter problems if
we attempt to compute this integral - if the $i\varepsilon$ were not present, there'd be certain values of $k^0$ that would lead
to singularities as we integrate. Thankfully, the $i\varepsilon$ makes the integral well-defined. We factorize the denominator as:
$\left( k^0+\sqrt{\mathbf{k}^2+m^2-i\varepsilon} \right)\left( k^0-\sqrt{\mathbf{k}^2+m^2-i\varepsilon} \right)$. Since we consider $\varepsilon$
to be a small positive number, we can expand the square roots, making the denominator:
$\left( k^0+\sqrt{\mathbf{k}^2+m^2}-i\varepsilon' \right)\left( k^0-\sqrt{\mathbf{k}^2+m^2}+i\varepsilon' \right)$,
where $\varepsilon'$ is just another small positive number. In fact, we'll just be sloppy and keep following this number as $\varepsilon$.

What does this factorization tell us about the location of the integrand's poles? Let us focus on the $dk^0$ integral first. Let us consider $k^0$ as
a complex number (because of the $i\varepsilon$). The integral runs over the entire real line. Using the abbreviation $\omega_k=\sqrt{\mathbf{k}^2+m^2}$,
we find that we have poles at $\omega_k-i\varepsilon$ and $-\omega_k+i\varepsilon$ (i.e. second and fourth quadrants). In this sense, the
$i\varepsilon$ prescription has eased our integration, as there are no poles physically on our contour of integration. We perform the contour integral
over the southern hemisphere; hopefully the integral along the arc will vanish. Now, take a point on this contour that is a distance $r$ away from the origin.
This point is $r\cos\theta-ir\sin\theta$. At some point we will take $r\to\infty$ but let us not do so just yet.

Inserting this point into the exponential in the numerator yields
\begin{align*}
    e^{-i(r\cos\theta-ir\sin\theta)(t_1-t_2)+i\mathbf{k}\cdot (\mathbf{x}_1-\mathbf{x}_2)},
\end{align*}
which has a part that oscillates and one that doesn't. The non-oscillating part yields $e^{-r\sin\theta(t_1-t_2)}$. Consequently, as $r\to\infty$,
as long as $t_1>t_2$, we are guaranteed that the integral over the arc will go to zero. Therefore, if $t_1>t_2$ we must integrate over the southern
hemisphere, but if $t_1<t_2$, we must integrate over the northen hemisphere. This is nice, as we seem to be getting some sense of time-ordering from
this integral: a different pole is picked up based on which contour is actually taken. We now write
\begin{align*}
    \Delta(x_1-x_2)&=\Theta(t_1>t_2)(-2\pi i)\int\frac{d^3k}{(2\pi)^4}\frac{e^{-i\omega_k(t_1-t_2)+i\mathbf{k}\cdot(\mathbf{x}_1-\mathbf{x}_2)}}{-2\omega_k}\\
    &+\Theta(t_2>t_1)(2\pi i)\int\frac{d^3k}{(2\pi)^4}\frac{e^{i\omega_k(t_1-t_2)+i\mathbf{k}\cdot(\mathbf{x}_1-\mathbf{x}_2)}}{2\omega_k},
\end{align*}
which yields our Green's function. We rewrite, now, changing the sign of the integration variable in the second integral,
\begin{align*}
    \langle\phi_1\phi_2\rangle=-i\Delta(x_1-x_2)&=\Theta(t_1>t_2)\int\frac{d^3k}{(2\pi)^32\omega_k}e^{-i\omega_k(t_1-t_2)+i\mathbf{k}\cdot(\mathbf{x}_1-\mathbf{x}_2)}\\
    &+\Theta(t_2>t_1)\int\frac{d^3k}{(2\pi)^32\omega_k}e^{i\omega_k(t_1-t_2)-i\mathbf{k}\cdot(\mathbf{x}_1-\mathbf{x}_2)},
\end{align*}
which we can also write, now assuming that $k^0=\omega_k$, i.e. we are on-shell:
\begin{align*}
    \langle\phi_1\phi_2\rangle=-i\Delta(x_1-x_2)&=\Theta(t_1>t_2)\int\frac{d^3k}{(2\pi)^32\omega_k}e^{ik\cdot(x_1-x_2}\\
    &+\Theta(t_2>t_1)\int\frac{d^3k}{(2\pi)^32\omega_k}e^{ik\cdot(x_1-x_2)},
\end{align*}
and we have computed the two-point correlation function for our free scalar field theory.

The fact that we have the two step functions of time should be very gratifying, as the LHS has explicit time ordering, by definition of the two-point
correlation function. It is a non-trivial check to compare this against the canonical quantization method that uses $\hat{a}_k,\hat{a}_k^\dagger$.
One must insert $\phi$ as an integral over momentum of these two operators to get agreement. Notice that when $t_1=t_2$, the two integrands will
be equal, so we simply add the two.

Classically, one integrates over the product of Green's functions and the source to solve for the field \textit{after} the source appears.
Our Green's function is quite odd in that regardless of whether we are before or after the source, we have non-zero support. In this sense,
this is a half-retarded half-advanced solution; what is called a \textbf{Feynman Green's function}. This raises the question of causality, and we
will at some point show that causality is implied by $[\hat{\phi}_1,\hat{\phi}_2]=0$ for the points 1 and 2 spacelike separated.

Recall that 
\begin{align*}
    \langle \phi_1 \cdots \phi_m\rangle&=\frac{\int D\phi\;e^{iS}\phi_1\cdots\phi_m}{\int D\phi\;e^{iS}}.
\end{align*}
We can always use Wick pairings to simplify these integrals in free theory. For odd $m$, we had $\langle \phi_1\cdots\phi_m$, but for $m$ even,
we have, for example,
\begin{align*}
    \langle \phi_1\phi_2\phi_3\phi_4 \rangle=\langle\phi_1\phi_2\rangle\langle\phi_3\phi_4\rangle+\langle\phi_1\phi_3\rangle\langle\phi_2\phi_4\rangle+\langle\phi_1\phi_4\rangle\langle\phi_2\phi_3\rangle
\end{align*}
Physically speaking, the fact that this disappears for $m$ odd is because the path integral integrand is odd when sone flips signs (only true for
sourceless theory).

Let us now consider turning on $J$. In other words,
\begin{align*}
    S_J=\int d^4x -\frac{1}{2}(\partial\phi)^2-\frac{1}{2}m^2\phi^2+J\phi
\end{align*}
and the path integral will be
\begin{align*}
    Z[J]=\int D\phi\;e^{iS_J}
\end{align*}
with the normalization that $Z[0]=1=\langle 0|0\rangle$ with no source. This function is called the \textbf{generating function} of the n-point functions because
when we take functional derivatives we get
\begin{align*}
    \frac{\delta Z[J]}{\delta iJ(x_1)}=\int D\phi\;\phi(x_1)e^{iS_J}
\end{align*}
where we implicity assume $\frac{\delta J(x)}{\delta J(x_1)}=\delta(x-x_1)$. If we evaluate this derivative at $J=0$, we find
\begin{align*}
    \frac{\partial Z[J]}{\delta iJ(x_1)}|_{J=0}&=\int D\phi\;\phi(x_1)e^{iS}=\langle\phi(x_1)\rangle=0\\
    \frac{\delta^mZ[J]}{\delta iJ_m \cdots \delta iJ_1}|_{J=0}&=\langle \phi_1 \cdots \phi_m \rangle.
\end{align*}
Note that the order of the functional derivatives does not matter! Of course, for free theory, we don't need this machinery at all,
but instead, becomes useful for interacting theories. Let us write down this function regardless. First recall the Gaussian integral
\begin{align*}
    \frac{\int d^nx e^{-\frac{1}{2}x^TMx+J^Tx}}{\int d^nx e^{-\frac{1}{2}x^TMx}}=e^{\frac{1}{2}J^TM^{-1}J},
\end{align*}
which suggests
\begin{align*}
    Z[J]=\exp\left( \frac{i}{2}\int d^4x d^4y\;J_x \Delta_{xy}J_y\right)
\end{align*}
One can take functional derivatives to check that this does indeed yield the $n$-point correlation functions.

\section{Energy arising from a static source}

So far we have primarily focused on formalism -- let's now try to do some physics. Let's try to calculate the energy of a static source $J(\mathbf{x})$.
Let us try to solve the classical problem:
\begin{align*}
    J_x&=\left( -\square_x+m^2 \right)\phi_x\\
    \phi_x&=\int d^4x'\;J(\mathbf{x}') \int \frac{d^4k}{(2\pi)^4}\frac{e^{ik\cdot(x-x')}}{k^2+m^2}
\end{align*}
Note that we have dropped the $i\varepsilon$ and are using the Fenyman Green's function as the source is static, and we need not worry about changes
in time of the source. We can integrate over $t'$; we obtain a delta function that forces $\omega_k=0$: $2\pi\delta(k^0)$, and we get
\begin{align*}
    \int d^3x' J(\mathbf{x}') \int \frac{d^3k}{(2\pi)^3}\frac{e^{i\mathbf{k}\cdot(\mathbf{x}-\mathbf{x'})}}{\mathbf{k}^2+m^2}
\end{align*}
There is clearly no pole in the integrand, which confirms that we do not need to use the $i\varepsilon$ prescription. The computation of the inner integral
is left for a problem set:
\begin{align*}
    V(|\mathbf{x}-\mathbf{x}'|)=\int \frac{d^3k}{(2\pi)^3}\frac{e^{i\mathbf{k}\cdot(\mathbf{x}-\mathbf{x'})}}{\mathbf{k}^2+m^2}=\frac{1}{4\pi}\frac{e^{-m|\mathbf{x}-\mathbf{x}'|}}{|\mathbf{x}-\mathbf{x}'|},
\end{align*}
which, by the way, is known as the \textbf{Yukawa potential}. This potential drops off as $1/r$ when $r\ll 1/m$, but is exponentially supressed when $r\gg 1/m$.
Why we interpret this as a potential energy will be made clear in a moment. Let us now compute the energy of our $\phi_x$ sourced by a static $J$.
The Hamiltonian density is written
\begin{align*}
    \mathcal{H}&=\Pi\dot\phi-\mathcal{L}\\
    &=\frac{1}{2}(\partial_t\phi)^2+\frac{1}{2}(\nabla\phi)^2+\frac{1}{2}m^2\phi^2-J\phi
\end{align*}
Substituting $\phi(x)=\int d^3x' J(\mathbf{x}')V(|\mathbf{x}-\mathbf{x}'|)$ and compute the Hamiltonian (check this):
\begin{align*}
    H=-\frac{1}{2}\int d^3x d^3x'\;J(\mathbf{x})J(\mathbf{x}')V(|\mathbf{x}-\mathbf{x}'|).
\end{align*}
This is very satisfying, as this expression looks exactly analogous to the energy we typically find in electrostatics. This is telling us that once
we turn on a source, it's like turning on a electrostatic-esque potential. It is almost as if this $\phi$, our ``particle'', is what \textit{mediates}
a force! Indeed, we can see that this force is attractive, due to the minus sign out front. Finally, note that a mediating particle with any mass
will mediate a force that dies off exponentially.

\section*{Field theory with interactions}

(February 12, 2013)

Recall that the LSZ formula:
\begin{itemize}
    \item gives us the scattering amplitude that involves $n$ particles
    \item is related to $\langle\phi_1\cdots\phi_n\rangle$
\end{itemize}
Our task is to compute $n$-point functionss for general interacting theory; i.e. compute $\int D\phi\; e^{iS}\phi_1\cdots\phi_n$.
This integral was possible to do exactly for Gaussian $S$, where $S$ was dependent on $\phi^2$ but no higher powers. Unfortunately,
it is not so easy for a general $S$. Take, for example, 
\begin{align*}
    S=\int d^4x \left( \mathcal{L}_0+\mathcal{L}_1 \right)
\end{align*}
with
\begin{align*}
    \mathcal{L}_0&=-\frac{1}{2}(\partial\phi)^2-\frac{1}{2}m^2\phi^2\\
    \mathcal{L}_1&=\frac{1}{6}g\phi^3.
\end{align*}
We will see that we obtain some sort of non-trivial scattering due to the cubic term. In particular, the particle's (self-)coupling
strength is dependent on the coupling constant $g$. Essentially, we are going to have to compute $\langle \phi_1\cdots\phi_n\rangle$
perturbatively in $g$.
\begin{align*}
    \langle\phi_1\cdots\phi_n\rangle=\int D\phi\;e^{iS_{\rm free}}e^{i\int d^4x\;\frac{g}{6}\phi^3(x)}\phi(x_1)\cdots\phi(x_n)
\end{align*}
If $g$ is small, we expand the second exponential to get:
\begin{align*}
    \langle\phi_1\cdots\phi_n\rangle=\int D\phi\;e^{iS_{\rm free}}\left(1+i\int d^4x\;\frac{g}{6}\phi^3(x)+\cdots\right)\phi(x_1)\cdots\phi(x_n)
\end{align*}
The first term is, in fact, just the free-field theory $n$-point function: $\langle\phi_1\cdots\phi_n\rangle_{\rm free}$. The next term
is the first interesting one:
\begin{align*}
    \int D\phi\;e^{iS_{\rm free}}\left( \frac{ig}{6} \right)\int d^4x\;\phi^3(x)\phi(x_1)\cdots\phi(x_n).
\end{align*}
Note, however, that we can write this as:
\begin{align*}
    \frac{ig}{6}\int d^4x \langle\phi^3(x)\phi(x_1)\cdots\phi(x_n)\rangle_{\rm free},
\end{align*}
which we know! This generalizes straightforwardly for higher powers, and in some sense the perturbative expansion is almost trivial!
Of course, the computation itself is quite difficult, and we will spend quite trying to deal with these calculations.

Now suppose $n=4$ (2 to 2 scattering). Then the lowest-order non-trivial scattering contribution is the $g^2$ term, due to the 
the free $n$-point function vanishing for odd $n$ ($n+3$ in this case). What we have is:
\begin{align*}
    \langle \phi_1\phi_2\phi_3\phi_4\rangle&=\langle\phi_1\cdots\phi_4\rangle+\left( \frac{ig}{6} \right)^2\int d^4x d^4y\;\langle \phi^3(x)\phi^3(y)\phi(x_1)\cdots\phi(x_4)\rangle_{\rm free}.
\end{align*}
This is a highly non-trivial $n$-point function -- there are 945 terms in the Wick expansion!
One of the terms is written:
\begin{align*}
    \langle \phi_x\phi_1\rangle\langle\phi_x\phi_2\rangle\langle\phi_y\phi_3\rangle\langle \phi_y\phi_4\rangle\langle\phi_x\phi_y\rangle
\end{align*}
This is often thought of diagrammatically through \textbf{Feynman diagrams}, where the first 2 2-point functions represent the incoming vertex,
the next 2 represent the outgoing vertex, and the last one connects these two vertices through the ``exchange'' of a $\phi$ particle. At this point,
we are in some sense done -- we know how to compute all of these terms!

Let's now treat this a little more abstractly. Take
\begin{align*}
    S_J=\int d^4x\;\mathcal{L}_0+\mathcal{L}_1+J(x)\phi
\end{align*}
with
\begin{align*}
    \mathcal{L}_0+\mathcal{L}_1=-\frac{1}{2}Z_\phi\left( \partial\phi \right)^2-\frac{1}{2}Z_m m^2\phi^2+\frac{1}{6}Z_g g\phi^3 + Y\phi
\end{align*}
where $Z_\phi,Z_m,Z_g, Y$ are some arbitrary constants. Why do we need these additional coefficients? We will impose:
\begin{enumerate}
    \item $m$ is the physical mass of the $\phi$ particle. $Z_m$ is the \textbf{mass renormalization factor}
    \item $g$ is the physical ``coupling constant.'' $Z_g$ is the \textbf{coupling renormalization factor}
    \item $\langle 0|\hat{\phi}(x)|0\rangle=0$. This fixes the linear coupling $Y$
    \item $\langle k|\hat{\phi}(x)|0\rangle=e^{-ikx}$. This fixes $Z_\phi$ -- the \textbf{wavefunction renormalization factor}
\end{enumerate}
Note that the last two bullet points were implicitly assumed in our earlier LSZ derivation. The need for all of this renormalization will
arise naturally later. Additionally, what we call $\mathcal{L}_0$ is our usual $-\frac{1}{2}\left( \partial \phi \right)^2-\frac{1}{2}m^2\phi^2$,
and so we define $\mathcal{L}_1$ accordingly. Formally speaking, we now have many different interaction terms. But let us focus on the cubic term
for now.

We have the generating function
\begin{align*}
    Z_1(J)&\equiv\int D\phi\;e^{i\int d^4x\; \mathcal{L}_0+\frac{1}{6}Z_gg\phi^3+J\phi}=\left(e^{i\int d^4x\;\frac{1}{6}Z_gg \left(\frac{\delta}{\delta iJ(x)}\right)^3}\right)Z_0(J)\\
    &=\left(e^{i\int d^4x\;\frac{1}{6}Z_gg \left(\frac{\delta}{\delta iJ(x)}\right)^3}\right)e^{\frac{i}{2}\int d^4x d^4z\;J_y\Delta_{zy}J_z}
\end{align*}
where $Z_0(J)$ is the generating function for free theory. To see why the first equality holds, one can expand out this exponential and take functional
derivatives. We can rewrite this even further:
\begin{align*}
    Z_1(J)=\sum_{v=0}\frac{1}{v!}\left[ \frac{iZ_gg}{6}\int d^4x\;\left( \frac{\delta}{\delta iJ} \right)^3 \right]^v\sum_{P=0}\frac{1}{P!}\left[ \frac{1}{2}\int d^4y d^4z\; (iJ_y)\frac{1}{i}\Delta_{yz}(iJ_z)\right]^P.
\end{align*}
What do these derivatives do? We will organize the terms by the number of $J$'s left, which is a number that we will call $E$ (for external sources).
We have
\begin{align*}
    E=2P-3V,
\end{align*}
just by inspection of the above mess. Let's first look at the terms with $E=0$. Take, for example, the term where $V=2,P=3$:
\begin{align*}
    \frac{1}{2!}&\left[ \frac{iZ_gg}{6}\int d^4x\left( \frac{\delta}{\delta iJ_x} \right)^3 \right]\left[ \frac{iZ_gg}{6}\int d^4y\left( \frac{\delta}{\delta iJ_y} \right)^3 \right]\\
    \times &\left[ \frac{1}{2}\int d^4a d^4a'\; (iJ_a)\frac{1}{i}\Delta_{aa'}(iJ_a')\right]\\
    \times &\left[ \frac{1}{2}\int d^4b d^4b'\; (iJ_b)\frac{1}{i}\Delta_{bb'}(iJ_b')\right]\\
    \times &\left[ \frac{1}{2}\int d^4c d^4c'\; (iJ_c)\frac{1}{i}\Delta_{cc'}(iJ_c')\right]
\end{align*}
Writing it out in all the gory detail, it should be clear that each functional derivative now is paired up with one $J$, and then integrated over
(with the presence of delta functions). One should note that, for connected Feynman diagrams, each term in the above equation will look something like
\begin{align*}
    \frac{1}{S}\left( iZ_gg \right)^2\int d^4x d^4y\;\left( \frac{1}{i}\Delta_{xy} \right)^3
\end{align*}
where $S$ is some numerical, combinatoric symmetry factor.
The Feynman diagram is called will have 2 vertices and 3 propogators, with no sources left -- in general, this diagram can take a variety of different forms.
The diagram consisting of 2 loops connected by a propogator is in fact buried in the mess above, and it should be clear that (similar to the Wick expansions above)
it corresponds to a term that looks like:
\begin{align*}
    \frac{1}{S}\left( iZ_gg \right)^2\left( \frac{1}{i}\Delta(0) \right)^2\int d^4x d^4y\;\frac{1}{i}\Delta_{xy}.
\end{align*}

Let us try to compute the symmetry factor for the Feynman diagram that looks like a $\Theta$:
\begin{align*}
    \frac{1}{S}=\frac{1}{2!}\frac{1}{6^2}\frac{1}{3^!}\frac{1}{2^3}\cdot 6\cdot 4\cdot 2\cdot 3\cdot 2=\frac{1}{12}
\end{align*}
based off the coefficients in the mess above and the number of ways each of the functional  derivatives can act. This is the brute-force method.
A slightly more elegant method is to note that the symmetry factor is simply how much you overcount if you naively assume that each 3-vertex can
be permuted and swapped, and that each propogator can be permuted and flipped and that one should get $S=1$. Computing these factors is generally
a little confusing, and we will not trouble ourselves too much at the moment.

Diagrams with no external legs are generically known as \textbf{bubble diagrams}, and typically correspond to \textbf{virtual particles} -- particles that
pop out of the vacuum and then disappear. More precisely, these are additional contributions to the vacuum energy (the cosmological constant).


\section*{What \textit{is} the scalar field?}

(February 14, 2013)

Think about the surface of the ocean (or a drum). In the ground state the surface is, of course, flat. Any deviation from the ground state
can obviously be represented by a scalar field (as there is only a magnitude of deviation from flat). We know that one can describe
waves in such a field as 
\begin{align*}
    \left(-\frac{\partial^2}{\partial t^2}+c_s^2\frac{\partial^2}{\partial \mathbf{x}^2}\right)\phi(\mathbf{x},t)=0
\end{align*}
It should be clear that there exist solutions of the form $e^{-i\omega_kt +i\mathbf{k}\cdot\mathbf{x}}$ which yield the dispersion relation $\omega_k^2=c_s^2\mathbf{k}^2$.
We can even guess the action to be
\begin{align*}
    S=\int dt d^3x\;\left( \frac{1}{2}(\partial_t\phi)^2-\frac{c_s^2}{2}\left( \nabla \phi \right)^2 \right)
\end{align*}
In field theory language, we would say this field describes a massless particle. But where did this particle come from?

Well, quantum mechanically, it is impossible for the scalar field to be perfectly flat -- there will inevitably be fluctuations. Indeed, in computing
the $n$-point functions that we have been doing, we've been computing amplitudes of non-trivial phenomena in the ground state! This is clearly behavior that
is incompatible with classical mechanics. In fact, as we showed, Fourier decomposing the quantum fluctuations (i.e. Fourier transforming $\phi$) yields some
$a_k,a_k^\dagger$ operators. Of course, there is no physical meaning of these coefficients yet -- they are simply Fourier coefficients. It is imposing the
commutation relations in a way similar to that done for the harmonic oscillator,
\begin{align*}
    [a_k,a^\dagger_{k'}]=\delta(k-k')
\end{align*}
that provokes us to think of these operators as ``creation'' and ``annihilation'' operators. However, one must be careful to treat the scalar field as a field
not of a media, but in field theory, we imagine it residing in vacuum.

But $\phi$ is itself an operator. What are its eigenvalues and eigenstates? We take
\begin{align*}
    \hat{\phi}(x)|f(x)\rangle=f(x)|f(x)\rangle.
\end{align*}
What is this eigenstate? Roughly, it is a field configuration in the shape of $f(x)$, and thus any function $f(x)$ can be put in the above equation.
Now the question arises, how does the function $f(x)$ evolve? First note that $\hat{\phi}$ is a Heisenberg operator:
\begin{align*}
    \hat{\phi}(x,t)=e^{iHt}\hat{\phi}(x,0)e^{-iHt}.
\end{align*}
For the sake of clarity let us rewrite the eigen-equation as
\begin{align*}
    \hat{\phi}(x,t)|f(x),t\rangle=f(x)|f(x),t\rangle.
\end{align*}
Inserting the Heisenberg expression into the eigen-equation, we have that
\begin{align*}
    e^{iHt}\hat{\phi}(x,0)e^{-iHt}|f(x),t\rangle=f(x)|f(x),t\rangle.
\end{align*}
Additionally, at time 0, the Heisenberg and Schrodinger pictures line up:
\begin{align*}
    \hat{\phi}(x,0)|f(x),0\rangle=f(x)|f(x),0\rangle,
\end{align*}
Enforcing both of these equations, we find
\begin{align*}
    |f(x,t)\rangle=e^{iHt}|f(x),0\rangle,
\end{align*}
i.e. that the Heisenberg eigenstates evolve backwards in time, while the Heisenberg operators evolve forwards in time.

\section*{Bubble diagrams}

Let us now return to bubble diagrams. Recall that we have
\begin{align*}
    S=\int d^4x\left( \mathcal{L}_0 +\mathcal{L}_{\rm int}\right)
\end{align*}
where
\begin{align*}
    \mathcal{L}_{\rm int}=-\frac{1}{2}(Z_\phi - 1)(\partial \phi)^2-\frac{1}{2}(Z_m-1)m^2\phi^2+\frac{1}{3!}Z_gg\phi^3+Y\phi +\Lambda.
\end{align*}
It is still unclear why in adding the cubic term, we have been forced by quantum mechanics to add in many of these extra terms.
These additional terms are typically known as \textbf{counter terms}.
Let's see if we can determine where the cosmological constant $\Lambda$ come from.

We considered the generating function with only a cubic term (and an external source):
\begin{align*}
    Z_1[J]=\int D\phi e^{i\int d^4x\left( \mathcal{L}_0+\frac{1}{3!}Z_gg\phi^3+J\phi \right)}
\end{align*}
There are a huge number of terms in this path integral. We considered the terms with no $J$, that led to a myriad of bubble diagrams.
There are many others, with arbitrary high numbers of $J$'s.

\begin{thm}
    \begin{align*}
        Z_1[J]=\exp\left(\sum\nm{connected diagrams}\right)
    \end{align*}
    In other words, exponentiating the connected diagrams yields the full result (including the disconnected diagrams).
\end{thm}
\begin{proof}
    Take the sum over diagrams
    \begin{align*}
        \sum_{\left\{ n_I \right\}} D^{\left\{ n_I \right\}}
    \end{align*}
    where we $n_I$ labels each different type of diagram. This can be rewritten:
    \begin{align*}
        \sum\frac{1}{S_D}\prod_I(c_I)^{n_I}&=\sum\frac{1}{\prod_In_I!}\prod_Ic_I^{n_I}=\sum\prod_I\frac{c_I^{n_I}}{n_I!}\\
        &=\prod_I\sum_{n_I=0}^\infty \frac{c_I^{n_I}}{n_I!}=\prod_I e^{c_I}=e^{\sum_Ic_I},
    \end{align*}
    where the nontrivial step is the last one, which you should probably think a little bit about.
\end{proof}

\section*{Back to perturbation theory}

(February 19, 2013)

Let us return to the interaction Lagrangian that we wrote down in the last lecture. Again, we will focus on the $\phi^3$ interaction:
\begin{align*}
    Z_1[J]=\int D\phi e^{i\int d^4x \mathcal{L}_0+\frac{1}{3!}Z_g g\phi^3+J\phi},
\end{align*}
which we formally expanded to get
 \begin{align*}
    Z_1(J)=\sum_{v=0}\frac{1}{v!}\left[ \frac{iZ_gg}{6}\int d^4x\;\left( \frac{\delta}{\delta iJ} \right)^3 \right]^v\sum_{P=0}\frac{1}{P!}\left[ \frac{1}{2}\int d^4y d^4z\; (iJ_y)\frac{1}{i}\Delta_{yz}(iJ_z)\right]^P,
\end{align*}
where the second sum is simply the free generating function. But now we can write this function as a sum of connected diagrams:
\begin{align*}
    Z_1[J]=e^{iW[J]},
\end{align*}
where $W[J]$ represents the sum. We separate the connected diagrams according to the number of $J$'s they contain, but let us for now
consider only terms with no external current $J$. Recall our convention for the free generating function that $Z_0[J=0]=1$. It should be
clear, however, that for our $\phi^3$ theory,
\begin{align*}
    Z_1[J=0]=e^{iW[J=0]}=\exp\left( \sum\rm{bubble\; diagrams} \right)
\end{align*}
is generically not unity. Let's introduce a counterterm $\Lambda$ to enforce $Z_1[J=0]=1$ for convenience:
\begin{align*}
    Z_1[J]=\int D\phi\; e^{i\int d^4x \mathcal{L}_0+\frac{1}{3!}Z_g g\phi^3+J\phi+\Lambda}.
\end{align*}
Let's choose:
\begin{align*}
    -i\int d^4x\;\Lambda=W[J=0]
\end{align*}
where of course, we are discussing the connected bubble diagrams. This is not quite obvious -- let us take the bubble diagram $\Theta$:
\begin{align*}
    \frac{(iZ_gg)^2}{12}\int d^4xd^4y\;(\frac{1}{i}\Delta_{xy})^3=\frac{(iZ_gg)^2}{12}\int\frac{d^4x d^4y}{i^3}\left( \int \frac{d^4k_1}{(2\pi)^4}\frac{e^{ik_1(x-y)}}{k_1^2+m^2-i\varepsilon} \right)\left( \cdots \right)\left( \cdots \right).
\end{align*}
Integrating over $y$ will yield a delta function $\delta(k_1+k_2+k_3$, and then integrating over $k_1$, will yield:
\begin{align*}
    \frac{(iZ_gg)^2}{12i^3}\int d^4x\int\frac{d^4k_2d^4k_3}{(2\pi)^3}\frac{1}{(-k_2-k_3)^2+m^2-i\varepsilon}\frac{1}{k_2^2+m^2-i\varepsilon}\frac{1}{k_3^2+m^2-i\varepsilon}.
\end{align*}
Consequently, we want $\Lambda$ to contain all of this term except the overall $d^4x$ integration. Note, however, that for very
large $k$, this integral goes as $k^8/k^6\sim k^2$ -- i.e. it diverges. Thus, if we wish to enforce our normalization, we have to introduce
a constant $\Lambda$ that is diverges. One might wave this off as simply a mathematical convenience tool, but the problem is much more severe
than it seems. Indeed, $\Lambda$ is in the Lagrangian, and thus in the Hamiltonian. $-\Lambda$ is then interpreted as the vacuum energy density,
i.e. the energy density when $\phi\equiv 0$.

So should we give up on this calculational convenience? It turns out that the normalization condition is more than just convenient;
we need
\begin{align*}
    \lim_{t\to\pm\infty}\langle 0|e^{-iH(t_f-t_i)}|0\rangle
\end{align*}
to give us the transition amplitude, so the presence of the vacuum energy will clearly contribute a phase shift.

So what do we do? Mumble a few words: we should not trust this theory at arbitrarily high momentum, i.e. past the ultraviolet cutoff.
We will choose to integrate only up until $k_{UV}$.
These are issues we will return to later in the course. But what's the takeaway for bubble diagrams? We can now throw them away, having
cancelled them out using $\Lambda$.

Let us now turn to the diagrams with 1 $J$. It turns out, in fact, that we want to cancel out these diagrams as well. Why is this?
First consider the one-point function:
\begin{align*}
    \langle 0 | \hat\phi(x) | 0 \rangle = \frac{\delta Z_1[J]}{\delta iJ}\bigg|_{J=0}=\frac{\delta e^{iW[J]}}{\delta iJ(x)}\bigg|_{J=0}
    =e^{iW[J]}\frac{\delta iW[J]}{\delta iJ(x)}\bigg|_{J=0}
\end{align*}
Note that the exponential prefactor is 1 at $J=0$, and we are left with the derivative w.r.t $iJ_x$ of the connected 1 J diagrams at $J=0$
So we now write in time explicitly:
\begin{align*}
    \langle 0|e^{iHt}\hat\phi(0,\vec{x})e^{-iHt}|0\rangle=0
\end{align*}
if we assume $H|0\rangle =0$, the exponentials become 1 and we expect the the 1 point function to be time-independent. 
Consequently, we add a $Y\phi$ to the Lagrangian to deal with this,
\begin{align*}
    Z_1[J]=\int D\phi\; e^{i\int d^4x \mathcal{L}_0+\frac{1}{3!}Z_g g\phi^3+J\phi+\Lambda+Y\phi},
\end{align*}
where $Y$ is chosen to cancel out the diagrams. We think of the $\phi$ term as another interaction; one which contributes to diagrams
as vertices with one leg sticking out that goes straight to a $J$ (there's no other options). 
\begin{align*}
    iY\int d^4x d^4y\;(iJ_y)\frac{1}{i}\Delta_{xy}.
\end{align*}
Let us now determine what $Y$ must be to cancel out
\begin{align*}
    \frac{(iZ_gg)}{2}\int d^4xd^4y\;(iJ_y)\frac{1}{i}\Delta_{xy}\frac{1}{i}\Delta(0),
\end{align*}
for example. Clearly we should choose $Y=-Z_gg/2\cdot \frac{1}{i}\Delta(0)$. Of course, there are other 1 $J$ diagrams that
have to be cancelled out. The argument is that one should write $Y$ as a power series in $g$, where each term's coefficients cancel out
appropriate 1 $J$ diagrams with arbitrary number of vertices, each of which come with that particular power of $g$. Note however,
that the order $g$ contribution to $Y$ is already fixed. Consequently, one can draw out diagrams with, say both types of interactions
that go as $gY^2$, $g^2Y$, $g^3$, etc. all of which can be cancelled by tuning the order $g^3$ component of $Y$. This holds for higher orders
as well, and while we do not prove it here, it should seem fairly plausible.

It turns out actually, that we cancel out more terms than we actually desired, as there are some diagrams, such as \textbf{tadpole diagrams},
for example, that go to zero due to how the 1 $J$ terms sum to zero. Indeed, this is how we characterize tadpole diagrams: any diagram
(not necessarily 1-$J$) such that when one leg is cut, the diagram has a part with no $J$. The claim is that tuning $Y$ to enforce
$\langle 0|\hat\phi|0\rangle=0$ cancels out all of the tadpoles.
We now write
\begin{align*}
    Z_1[J]=\exp\sum\left( \nm{connected}, -\nm{bubbles},-\nm{tadpoles}\right)
\end{align*}
since we have cancelled out the latter two.

Now, it is not so obvious why the mass and wavefunction renormalization terms must be present in the Lagrangian. For now, we shall assume that
they are there, and try to see what they do. We consider
\begin{align*}
    &\int d^4x -\frac{1}{2}(Z_\phi-1)(\partial\phi)^2-\frac{1}{2}(Z_m-1)m^2\phi^2\\
    &=\int d^4x +\frac{1}{2}\phi\square\phi-\frac{1}{2}(Z_m-1)m^2\phi^2\\
    &=\frac{1}{2}\int d^4x +\phi(x)\left[ (Z_\phi-1)\square_x-(Z_m-1)m^2 \right]\phi(x)
\end{align*}
We now write for the full generating function,
\begin{align*}
    Z[J]=\int D\phi\; e^{i\int d^4x \mathcal{L}_0+\mathcal{L}_{\rm int}+J\phi}
\end{align*}
and expand
\begin{align*}
    Z[J]=\exp\left( \frac{i}{2}\int d^4x\frac{\delta}{\delta iJ_x}\left[ (Z_\phi-1)\square_x-(Z_m-1)m^2 \right]\frac{\delta}{\delta iJ_x} \right)Z_1[J].
\end{align*}
This will give diagrams with vertice that have two legs sticking out. Let's do an example of a 2 $J$ diagram with the two $J$'s connected
to such a vertex. Such a diagram will yield
\begin{align*}
    \int d^4xd^4ad^4b\;(iJ_a)(iJ_b)\Delta_{ax}i\left[ (Z_\phi-1)\square_x-(Z_m-1)m^2 \right]\frac{1}{i}\Delta_{xb}
\end{align*}
where, since this vertex has an operator embedded in it, we have sandwiched it between the two propogators. Again, why we need these
vertices is not clear at the moment.

\section*{Scattering amplitudes}

(February 21, 2013)

So far, we have been working the generating function, whose derivatives yields $n$-point correlation functions, which we can use to, among other
things, compute scattering amplitudes via the LSZ-reduction formula. Recall our previous result that
\begin{align*}
    Z[J]=e^{iW[J]}
\end{align*}
where $W[J]$ is the sum of all connected diagrams, where the bubble and tadpole diagrams have been cancelled out.

Let's compute a two-point correlation function,
\begin{align*}
    \langle \phi_1\phi_2\rangle&=\frac{1}{i}\Delta_{\rm exact},
\end{align*}
where the propogator is to be distinguished from the free propogator. We compute
\begin{align*}
    \langle \phi_1 \phi_2 \rangle &= \frac{\delta Z[J]}{\delta iJ_2\delta iJ_1}\bigg|_{J=0}\\
    &= \frac{\delta}{\delta iJ_2}\left( e^{iW[J]}\frac{\delta iW}{\delta iJ_1} \right)\bigg|_{J=0}\\
    &= e^{iW[J]}\frac{\delta iW}{\delta iJ_2}\frac{\delta iW}{\delta iJ_1}\bigg|_{J=0}+e^{iW[J]}\frac{\delta iW}{\delta iJ_2\delta iJ_1}\bigg|_{J=0}\\
    &=\frac{\delta iW[J]}{\delta iJ_1\delta iJ_2}\bigg|_{J=0}
\end{align*}
where in the last step we have used the fact that 1-$J$ diagrams have been removed and that the exponential goes to 1 due to the removal of
bubble diagrams. We then have
\begin{align*}
    \frac{\delta^2}{\delta iJ_2\delta iJ_1}\left( \cdots \right)\bigg|_{J=0}
\end{align*}
where the ellipsis stands for an infinite number of diagrams in various orders of $g$. The lowest order diagram is that with 2 $J$'s connected
by a propogator and nothing else:
\begin{align*}
    \frac{\delta^2}{\delta iJ_2\delta iJ_1}\left(\frac{1}{2}\int d^4xd^4y\;(iJ_x)(iJ_y)\frac{1}{i}\Delta_{xy}\right)=\frac{1}{i}\Delta(x_1-x_2),
\end{align*}
where again the propogator is the free propogator. This is not surprising -- ignoring interactions completely, this is the only term that
contributes. However, what happens to the same diagram after we add a bubble (2 vertices now, order $g^2$). We have
\begin{align*}
    \frac{\delta^2}{\delta iJ_2\delta iJ_1}&\left( \frac{1}{4}\int d^4xd^4yd^4ad^4b\;(iZ_gg)^2\left(\frac{1}{i}\Delta_{xy}\right)^2\frac{1}{i}\Delta_{xa}\frac{1}{i}\Delta_{yb}(iJ_a)(iJ_b) \right)\bigg|_{J=0}\\
    =&\frac{(ig)^2}{2}\int d^4xd^4y\;\left( \frac{1}{i}\Delta_{xy} \right)^2\frac{1}{i}\Delta(x-x_1)\frac{1}{i}\Delta(y-x_2),
\end{align*}
where we have used the fact that $Z_g$ is of order $1+O(g^2)$, which we will see next week. Thus, we have calculated the first two terms in the 
expansion for the full two-point correlation function. We will see that this second integral will diverge, something that is due to the presence
of loops such as the bubble that we included.

Let's now look at the four-point function in order to talk about two-to-two scattering:
\begin{align*}
    \langle \phi_1 \phi_2 \phi_3 \phi_4\rangle=&\frac{\delta^4}{\delta iJ_1 \cdots \delta iJ_4}e^{iW[J]}\bigg|_{J=0}\\
    =&\frac{\delta^4 iW}{\delta iJ_4\cdots\delta iJ_1}+\frac{\delta iW}{\delta iJ_4\delta iJ_2}\frac{\delta^2iW}{\delta iJ_3iJ_1}+
    \frac{\delta iW}{\delta iJ_4\delta iJ_1}\frac{\delta^2iW}{\delta iJ_3iJ_2}\\
    +& \frac{\delta iW}{\delta iJ_4\delta iJ_3}\frac{\delta^2iW}{\delta iJ_2iJ_1}
\end{align*}
where we have anticipated that any term with only first and third derivatives of $W$ will vanish. Note that other than the first term,
we could have expected all of these via the usual Wick expansion for free theory. The first we call connected, and write
\begin{align*}
    \langle \phi_1 \phi_2 \phi_3 \phi_4\rangle=\langle \phi_1\phi_2\phi_3\phi_4\rangle_c+\langle\phi_4\phi_2\rangle+\langle\phi_3\phi_1\rangle
    +\langle \phi_4\phi_1\rangle\langle \phi_3\phi_2\rangle+\langle\phi_4\phi_3\rangle\langle\phi_2\phi_1\rangle
\end{align*}
It should be clear that the first term will require diagrams with at least 4 $J$'s and 2 vertices
Let us focus on the one that looks like pair annihilation and then creation.
Note that it has no loops (it is a tree diagram) and thus should not diverge:
\begin{align*}
    \frac{(ig)^2}{8}&\left(\int d^4xd^4y\int d^4ad^4bd^4cd^4d\;iJ_aiJ_biJ_ciJ_d\frac{1}{i}\Delta_{xy}\frac{1}{i}\Delta_{xa}\frac{1}{i}\Delta_{xb}\frac{1}{i}\Delta_{yc}\frac{1}{i}\Delta_{yd}\right)
\end{align*}
There are three distinct possibilities -- 2 and 1 are paired against 1' and 2', 2 and 2' are paired against 1 and 1', and 2 and 1' are paired against 1 and 2', which
we have to account for combinatorially. At the end of the day, one can check that we are left with
\begin{align*}
    (ig)^2\int d^4xd^4y\frac{1}{i}\Delta_{xy}\frac{1}{i}\Delta_{x1}\frac{1}{i}\Delta_{x2}\frac{1}{i}\Delta_{y1'}\frac{1}{i}\Delta_{y2'}
\end{align*}
for the first topology. To get the overall term, one would have to switch some of the indices and sum over the other topologies as well.

Let us now turn again to the LSZ formula, where the 1 stands for trivial scattering
\begin{align*}
    \langle k_2'k_1'|k_1k_2\rangle-1=i^4\int d^4x_1d^4x_2d^4x_1'd^4x_2'e^{i(k_1x_1+k_2x_2-k_1'x_1'-k_2'x_2')}\\
    (-\square_1+m^2)(-\square_2+m^2)(-\square_{1'}+m^2)(-\square_{2'}+m^2)\langle\phi_1\phi_2\phi_{1'}\phi_{2'}\rangle
\end{align*}
It turns out that the nonconnected terms in the Wick expansion contribute zero to the scattering amplitude -- this should make a lot of sense,
as they represent what is present already in the free-field theory, i.e. trivial scattering. Let's see that we get this. Integrating the boxes by
parts, and acting them on the exponentials, we get out terms that go as $k_1^2+m^2=-(k_1^0)^2+(\vec{k}_1)^2+m^2=0$. This will force such
contributions to vanish as long as the four-point function does not cancel these zeros out. For these terms, there are not enough poles to
cancel them out, while for interacting theory, there are. To see how these poles arise, note that the LSZ formula is almost a Fourier transform
of the four-point function:
\begin{align*}
    \langle k_2'k_1'|k_1k_2\rangle-1=i^4\int d^4x_1d^4x_2d^4x_1'd^4x_2'e^{i(k_1x_1+k_2x_2-k_1'x_1'-k_2'x_2')}\\
    (k_1^2+m^2)(k_2^2+m^2)(k_{1'}^2+m^2)(k_{2'}^2+m^2)\langle\phi_1\phi_2\phi_{1'}\phi_{2'}\rangle.
\end{align*}
For free theory, we have, ignoring the zeros,
\begin{align*}
    \int d^4x_1d^4x_2d^4x_{1}'d^4x_2'e^{i(k_1x_1+k_2x_2-k_1'x_1'-k_2'x_2')}\frac{1}{i^2}\left(\Delta_{12}\Delta_{1'2'}+\Delta_{11'}\Delta_{22'}+\Delta_{12'}\Delta_{21'}  \right),
\end{align*}
where each product of propogators goes as
\begin{align*}
    \int \frac{d^4q}{(2\pi)^4}\frac{e^{iq(x_1-x_2)}}{q^2+m^2-i\varepsilon}\int \frac{d^4p}{(2\pi)^4}\frac{e^{ip(x_1'-x_2')}}{p^2+m^2-i\varepsilon}
\end{align*}
and it should be clear that at most we will have two poles, and so will have no contributions from the two-point functions. Thus, let us now try
put the earlier diagram that we were considering into this context, seeing as it dominates over the others.

Consider the Fourier transform of the connected four-point correlation function, a fourty-four dimensional integral,
\begin{align*}
    \int d^4x_1d^4x_2d^4x_1'd^4x_2'e^{i(k_1x_1+k_2x_2-k_1'x_1'-k_2'x_2')}\\
    \frac{(ig)^2}{i^5}\int d^4xd^4y\int \frac{d^4p_a}{(2\pi)^4}\frac{e^{ip_a(x-y)}}{p_a^2+m^2}
    \int \frac{d^4p_b}{(2\pi)^4}\frac{e^{ip_b(x-x_1)}}{p_b^2+m^2}\int \frac{d^4p_c}{(2\pi)^4}\frac{e^{ip_c(x-x_2)}}{p_c^2+m^2}\\
    \int \frac{d^4p_d}{(2\pi)^4}\frac{e^{ip_d(y-x_1')}}{p_d^2+m^2}    \int \frac{d^4p_e}{(2\pi)^4}\frac{e^{ip_e(y-x_2')}}{p_e^2+m^2}
\end{align*}
Integrating over $x_1$ will use up the first part of the exponential to get rid of the $p_b$ integral and replace $p_b$ with $k_1$ (and
get rid of all other $x_1$'s); do this on your own, carefully. Doing this for $x_2,x_1',x_2'$ as well, simplifies what we have considerably.
Integrating over $x$ will set $p_a=-k_1-k_2$ and integrating over $y$ will set $p_a=-k_1'-k_2'$ and yields overall momentum conservation:
\begin{align*}
    \frac{(ig)^2}{i^5}(2\pi)^4\delta(k_1+k_2-k_1'-k_2')\frac{1}{(k_1+k_2)^2+m^2}\frac{1}{k_1^2+m}\frac{1}{k_2^2+m}\frac{1}{k_1'^2+m}\frac{1}{k_2'^2+m}.
\end{align*}
To summarize, at the end of the day, we have the LSZ formula equal to
\begin{align*}
    \frac{(ig)^2}{i}(2\pi)^4\delta(k_1+k_2-k_1'-k_2')\frac{1}{(k_1+k_2)^2+m^2}.
\end{align*}
So, at the end of the day we see that the only propogator that matters is the internal one, not those directly connected to $J$'s, as those are
cancelled by the zeros in the LSZ formula. Note, however, that $(k_1+k_2)^2\neq -m^2$, i.e. the internal particle is virtual, as it is not
on shell.
With this Feynman rule in mind, we can do the other, topologically distinct diagrams quite easily and write the LSZ formula as:
\begin{align*}
    \frac{(ig)^2}{i}&(2\pi)^4\delta(k_1+k_2-k_1'-k_2')\\
    &\left(\frac{1}{(k_1+k_2)^2+m^2-i\varepsilon}+\frac{1}{(k_1-k_1')^2+m^2-i\varepsilon}+\frac{1}{(k_1-k_2')^2+m^2-i\varepsilon}\right).
\end{align*}

Note, now, that when we talk about scattering, we say
\begin{align*}
    \langle k_2'k_1'|k_2k_1\rangle-1\equiv(2\pi)^4\delta(k_1+k_2-k_1'-k_2')i\mathcal{M},
\end{align*}
where by scattering amplitude, we are really just talking about $\mathcal{M}$.

\section*{Cross sections}

(February 26, 2013)

Now that we know how to compute scattering amplitudes, we can move on to computing cross-sections.
Let us first define the \textbf{Mandelstam} variables:
\begin{align*}
    s&\equiv -(k_1+k_2)^2,\\
    t&\equiv -(k_1-k_1')^2=-(k_2-k_2')^2,\\
    u&\equiv -(k_1-k_2')^2=-(k_2-k_1')^2,
\end{align*}
where the square is the usual four-dimensional contraction. These definitions are motivated by the three topologies of the
two-to-two processes that we worked with in the last section. The processes are called $s,t,u$-channel processes. Note that
the Mandelstam variables have the property that
\begin{align*}
    s+t+u=m_1^2+m_2^2+m_{1'}^2+m_{2'}^2
\end{align*}
where we are treating the general case of arbitrary masses. Additionally, in the center of momentum frame, where $\vec{k}_1+\vec{k}_2=0$,
we have
\begin{align*}
    s=-\left( (k_1+k_2)^0 \right)^2=(\omega_1+\omega_2)^2,
\end{align*}
and so $s$ is just the square of the total energy (in this frame).

Now, to compute the cross-section, first imagine that we have particle 2 shooting towards particle 1, which has a certain ``cross-section''.
The interaction rate should be given by $n_2|\vec{v}_1-\vec{v}_2|\sigma$ where $n_2$ is the number density of particle 2 and $\sigma$ is
the cross-section of particle 1. Before we try to calculate the total cross-section, let us for now play with a differential cross-section.
If we perform our measurements in a spatial volume $V$ and over a time-duration of $T$, we write
\[{|\vec{v}_1-\vec{v}_2|}{V}d\sigma,\]
which is the scattering rate times the number of outgoing state configurations. How do we isolate the scattering rate?

Well, let us first ask -- what are the dimensions of the scattering amplitude, $\mathcal{M}$? Recall that we are using units where
the action is dimensionless (since $\hbar=1$), and
\[S_{\rm free}=\int d^4x \left( -\frac{1}{2}(\partial\phi)^2-\frac{1}{2}m^2\phi^2 \right).\]
It is convenient to work in the units of mass. First note that length scales as the inverse of mass (by the uncertainty principle
$\Delta x\Delta p \geq 1$. Consequently, the differential $d^4x$ has units of $m^{-4}$, and so $\phi$ must have dimensions of mass
in order to make the action dimensionless. But since
\[S_{\rm int}=\int d^4x g\phi^3,\]
$g$ must also have dimensions of mass. Returning to our expression for the scattering amplitude $\mathcal{M}$ in the previous section, then, we find
that it is dimensionless. This is, in fact, due to the fact that we are only considering 2-to-2 scattering, i.e.
\[\langle k_1'|k_1\rangle = (2\pi)^32\omega_1\delta^{(3)}(\vec{k}_1'-\vec{k}_1).\]

Now the scattering rate goes as
\[\frac{1}{T}\frac{|(2\pi)^4\delta^{(4)}(k_1+k_2-k_1'-k_2')\mathcal{M}|^2}{\langle k_1'k_2'|k_1'k_2'\rangle\langle k_1k_2|k_1k_2\rangle}\]
where the first term in the denominator involves out states while the other involves in states. The denominator comes about
due to our peculiar choice for normalization for out states. Finally, we need to compute the number of outgoing states. 
The number of outgoing states in phase space is given as
\[\frac{d^3k_1'}{(2\pi)^3/V}\frac{d^3k_2'}{(2\pi)^3/V}.\]
This fact is explained thoroughly in the book.

At the end of the day, we obtain
\begin{align*}
    d\sigma=\frac{1}{4\omega_1\omega_2|\vec{v}_1-\vec{v}_2|}|\mathcal{M}|^2(2\pi)^2\delta^{(4)}(k_1+k_2-k_1'-k_2')\frac{d^3k_1'}{(2\pi)^3w\omega_{1'}}\frac{d^3k_1'}{(2\pi)^3w\omega_{2'}}.
\end{align*}
How? One of the delta functions disappears because the other takes it to $\delta(0)$, since they have the same argument.
What does this mean? Recall that we define $\delta^{(4)}(0)(2\pi)^4=\int d^4x=V\cdot T$, which is in fact, what leads (after some work)
to obtaining a cross-section that is well-defined and independent of $V$ and $T$. Again, this is derived more thoroughly in the book.
Notice that a cross-section has units of area, i.e. inverse mass squared -- one can check that this formula yields precisely these units.
Furthermore, this formula is more general than the same-mass model that we have been playing with, and can be generalized for more outgoing
particles by adding more differential terms.

Let us consider particles 1 and 2 approaching each other along the $z$-axis. One can show that the prefactor $1/4\omega_1\omega_2|v_1-v_2|$
is invariant under boosts along the $z$ direction. The rest of the cross-section formula is manifestly Lorentz invariant. So the cross-section
is only invariant in one direction; this is something that meshes well with the concept of cross-section as an area. If we look at these
two particles in the center of momentum frame ($\vec{k}_2=-\vec{k}_1$) and let $\theta$ be the angle between $k_1'$ and $k_2$, we will find that
\[d\sigma=\frac{1}{64\pi^2s}\frac{|\vec{k}_1'|}{|\vec{k}_1}|\mathcal{M}|^2d\Omega,\]
where $d\Omega$ is the solid angle of outgoing momenta with respect to the incoming momentum. Note that $\mathcal{M}$ is, in general,
angle dependent. With this in hand, we can compute the total cross section via
\[\sigma_{\rm total}=\frac{1}{2}\int \frac{d\sigma}{d\Omega}d\Omega,\]
where the constant in the denominator out front is more generall $\Pi_I n_I!$ where $n_I$ is the number of outgoing identical particles of
type $I$. Why is this factor necessary? Imagine $k_1'=(\sqrt{a^2+m^2},\vec{a})$ and $k_2'=(\sqrt{b^2+m^2},\vec{b})$. If these are identical particles,
one can swap $\vec{a}$ and $\vec{b}$ and obtain the same outgoing configuration.

Now, in the center of mass frame, with $m_1=m_2=m_1'=m_2'=m$, we have
\begin{align*}
    s&=(\omega_1+\omega_2)^2=4(|\vec{k}_1|^2+m^2)\geq 4m^2\\
    t&=-(k_1-k_1')^2=(\omega_1-\omega_1')^2-(\vec{k}_1-\vec{k}_1')^2=-2|\vec{k}_1|^2(1-\cos\theta)\\
    &=-2\left( \frac{s}{4}-m^2 \right)(1-\cos\theta)\\
    u&=-2\left( \frac{s}{4}-m^2 \right)\left( 1+\cos\theta \right)
\end{align*}
Using these, and working in the high energy limit where $s\gg m$, we can write
\begin{align*}
    \mathcal{M}&=\frac{(ig)^2}{i}\left( -\frac{1}{s}+\frac{1}{s/2(1-\cos\theta)}+\frac{1}{s/2(1+\cos\theta)} \right)\\
    &=\frac{(ig)^2}{is}\frac{3+\cos^2\theta}{\sin^2\theta}
\end{align*}
This formula tells us a number of interesting things. First, low-energy scattering has larger cross-section than high-energy scattering does.
In fact, in 4-dimensions, any cubic/quartic interaction acts like this, as we will see later, when we talk about renormalization.
Additionally, this scattering tends to favor angles that are either 0 or $\pi$, i.e. completely forwards or completely backwards. This formula
can then be used to compute $\sigma$.

\section*{Higher spacetime dimensions}

Let's consider
\[S=\int d^dx\left( -\frac{1}{2}(\partial\phi)^2-\frac{1}{2}m^2\phi^2+\frac{g}{3!}\phi^3 \right),\]
where $d$ is the number of space-time dimensions in which we are working. The differential then has dimensions of inverse mass
to the $d$ power. Consequently, $\phi$ must have dimensions of $[m]^{(d-2)/2}$ and $g$ must have dimensions of $[m]^{(6-d)/2}$.
Why do we care? Well, note that $g$ is dimensionless if $d=6$. This turns out to be nice because it will remove energy dependence from
the coupling constants when we think about renormalization.

Recall the Klein-Gordon equation, $(-\square+m^2)\phi=0$. Clearly, the form of the equation is independent of the number of dimensions.
The solution is given
\begin{align*}
    \langle 0 | T\hat\phi_1\hat\phi_2|0\rangle=\frac{1}{i}\int \frac{d^d}{(2\pi)^d}\frac{e^{ik\cdot(x-y)}}{k^2+m^2-i\varepsilon}=\frac{1}{i}\Delta_{\rm free}(x-y)
\end{align*}
where we are taking appropriate dimensional contractions. All in all, there is not much that changes, and thus we will from now on
often perform our computations, in general, in $d$ dimensions.

\section*{Lehmann-Kallen form of the exact propogator}

Let's find an explicit form for the two-point function. Let's first consider $x^0>y^0:$
\begin{align*}
    \langle 0|\hat\phi_x\hat\phi_y|0\rangle=\sum_{n}\langle 0 | \hat\phi_x|n\rangle\langle n|\hat\phi_y|0\rangle
\end{align*}
where $n$ represents any state, and 
\begin{align*}
    \hat\phi(x)=e^{-ip\cdot x}\phi(0)e^{ip\cdot x}
\end{align*}
we are considering the Heisenberg picture for position. Note that $e^{ip\cdot x}|0\rangle=|0\rangle$. If the momentum hits
any other state, on the other hand, we get out the total momentum $k_n$, so we write:
\begin{align*}
    \langle 0|\hat\phi_x\hat\phi_y|0\rangle&=\sum_ne^{ik_n\cdot(x-y)}|\langle0|\hat\phi_0|n\rangle|^2\\
    &=\int d^dp\; e^{ip\cdot (x-y)}\sum_n|\langle 0|\hat\phi(0)|n\rangle|^2\delta^{(d)}(p-k_n)\\
    &=\int d^dp\; e^{ip\cdot (x-y)} \frac{1}{(2\pi)^{d-1}}\Theta(p^0>0)\rho_{\rm total}(-p^2)\\
    &=\int d^dp\; e^{ip\cdot (x-y)} \frac{1}{(2\pi)^{d-1}}\Theta(p^0>0)\int_0^\infty d\mu^2\;\rho_{\rm tot}(\mu^2)\delta(p^2+\mu^2)\\
    &=\int_0^\infty d\mu^2\;\rho_{\rm tot}(\mu^2)\int\frac{d^dp}{(2\pi)^{d-1}}\Theta(p^0>0)e^{ip\cdot(x-y)}\delta(p^2+\mu^2)
\end{align*}
where we have simply rewritten quantities and defined the sum to be some mysterious function of $-p^2$.
The reverse time-ordered expression will be exactly the same, just with $e^{-ip\cdot(x-y)}$ instead. Note that the inner integral has been
written in a form that reminds us of a Green's function. Finally, we say
\begin{align*}
    \langle 0|T\hat\phi_x\hat\phi_y|0\rangle
    =\int_0^\infty d\mu^2\;\rho_{\rm tot}(\mu^2)\left[\Theta(x^0>y^0)\int \frac{d^{d-1}p}{(2\pi)^{d-1}2\omega_p}e^{ip\cdot(x-y)}\right.\\
        \left.+\Theta(y^0>x^0)\int \frac{d^{d-1}p}{(2\pi)^{d-1}2\omega_p}e^{-ip\cdot(x-y)}\right]\\
        =\int_0^\infty d\mu^2\rho_{\rm tot}(\mu^2)\frac{1}{i}\int \frac{d^dp}{(2\pi)^d}\frac{e^{ip\cdot (x-y)}}{p^2+\mu^2-i\varepsilon}.
\end{align*}
where $p^0=\omega_p=\sqrt{\vec{p}^2+\mu^2}$. What have we gained from rewriting in this way? Well $\rho_{\rm tot}$ is nice and positive-definite;
we expect it to be $\delta$-spiked at $m^2$, then $4m^2$, and then after that, a continuum that is nonzero. This is because in 2-particle states,
one can discuss the continuum of all possible momenta states. In fact, one can ever have isolated bound states between $m$ and $4m^2$. 
In fact, the final explicit expression for the propogator is given: (see Srednicki)



\end{document}
