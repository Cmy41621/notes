\documentclass{../../mathnotes}

\usepackage{tikz-cd}
\usepackage{todonotes}

\title{Representation Theory PSET 1}
\author{Nilay Kumar}
\date{Last updated: \today}


\begin{document}

\maketitle

\begin{prop}
    Consider the category of topological abelian groups with continuous homomorphisms
    between them. Show this category is additive but not abelian.
\end{prop}
\begin{proof}
    
\end{proof}



\begin{prop}
    Let $S\to R$ be a homomorphism of rings. It induces the restriction and induction
    functors between the corresponding categories of modules, $\text{Res}:\catname{R-Mod}\to\catname{S-Mod}$
    and $\text{Ind}:\catname{S-Mod}\to\catname{R-Mod}$, where $\text{Ind }A=R\otimes_S A$ and $\text{Res }B$
    is the module $B$ viewed as an $S$-module. Show that these functors are adjoint in the
    sense that $\Hom_R(\text{Ind } A,B)=\Hom_S(A,\text{Res }B).$
\end{prop}

\begin{prop}
    Show that every module in category $\mathcal{O}$ is finitely generated as a
    $\mathcal{U}\fr n_{-}$-module.
\end{prop}

\begin{prop}
    For $V_1,V_2\in\text{Obj }\mathcal{O}$, consider the tensor product $V_1\otimes V_2$.
    Show that it is in $\mathcal{O}$ is one of the factors is finite-dimensional, but
    not in general.
\end{prop}

\begin{prop}
    Consider the action of $GL(3)$ on polynomials of degree $d$ in $x_1,x_2,x_3$.
    Resolve this representation by Verma modules.
\end{prop}

\end{document}
