\documentclass{../../mathnotes}

\usepackage{tikz-cd}
\usepackage{todonotes}

\title{Representation Theory PSET 2}
\author{Nilay Kumar}
\date{Last updated: \today}


\begin{document}

\maketitle

\begin{exc}
    Describe the group $P/Q$ for classical simple Lie algebras and also for $G_2$
    and $F_4$.
\end{exc}
\begin{proof}
    \todo{finish}
\end{proof}

\begin{exc}
    Let $B$ be a nondegenerate bilinear form (symmetric or antisymmetric) on $\C^n$
    and let $G$ be its group of automorphisms. Describe the closed $G$-orbits on the
    variety of complete flags in $\C^n$.
\end{exc}
\begin{proof}
    Consider first the case of a nondegenerate symmetric bilinear form. The group
    $G$ is then the orthogonal group, and any flag in $\C^n$ is simply an ordered
    basis of vectors. The orthogonal group clearly acts transitively on such bases,
    and hence the flag variety decomposes into a single (closed) orbit under the action
    of the orthogonal group.

    For case of a nondegenerate antisymmetric bilinear form, \todo{finish}
\end{proof}

\begin{exc}
    Let $g\in GL_n$ be a unipotent operator. Describe the Zariski closure of the
    cyclic group generated by $g$. Conclude, in particular, that any algebraic
    group consisting of unipotent operators is connected.
\end{exc}
\begin{proof}
    We work over an algebraically closed field $k$ of characteristic zero. Then
    we note that any nilpotent matrix $N$ can be exponentiated to a unipotent
    matrix $\exp N$. Conversely, any unipotent matrix can be written as $U=1+A$
    where $A$ is nilpotent; more precisely, one can consider the logarithm of $U$
    to obtain $N=\log U$ such that $\exp N=U$. In particular, this correspondence
    gives us an isomorphism between the variety of nilpotent matrices and the
    variety of unipotent matrices, as the exponential and logarithm power series
    terminate to become polynomials. Hence, given any unipotent element $g\in GL_n$,
    we can write it as $g=\exp N$, and the cyclic subgroup generated by $g$ becomes
    the group $\exp(mN)$ for $m\in\Z$. Consider now the map $\phi:k\to GL_n$
    given by $x\mapsto g^x$, which is regular as the components of $g^x$ are simply
    polynomials for $g$ unipotent -- this is evident from the Taylor series for $x^c$
    for $c\in k$. The preimage of the closure $\phi^{-1}(\overline{\langle g\rangle})$
    must be a Zariski-closed set in $k$ containing $\Z$. Of course, the only closed
    set in $k$ containing an infinite number of points is $k$ itself. Hence
    the closure of the cyclic subgroup $\langle g\rangle$ must be of the form $g^x$
    for $x\in\C$. From this characterization, it is clear that any algebraic group
    consisting of unipotent operators is connected, as every operator is in the
    component of the identity (take $x=0$).
\end{proof}

\begin{exc}
    Let $G\subset GL_n$ be a solvable Lie subgroup, not necessarily connected. Is
    it true that $G$ is triangular in a suitable basis?
\end{exc}
\begin{proof}
    Consider the subgroup $S_2\subset GL_n$ embedded as a $2\times 2$ block in the
    upper-left corner with 1's down the diagonal. The group $S^2$ is of course solvable,
    but it is clear that there exists no basis in which it is triangular: if there were,
    it would no longer contain permutation matrices.
\end{proof}

\begin{exc}
    For $G$ as above, let $\bar G\subset GL_n$ be the Zariski closure of $G$. Is
    it true that $\bar G$ is solvable?
\end{exc}
\begin{proof}
    In this exercise, I reference Borel's \textit{Linear Algebraic Groups}. Note first
    that it is not \textit{a priori}
    obvious that $\bar G$ is a group. However, it can be shown that the Zariski closure
    $\bar G$ is precisely the intersection of the closed subgroups of $GL_n$ containing
    $G$ (c.f. Borel I.2.1a). Now, as $G$ is solvable, successive group commutators
    $\mathcal{D}^{n+1}=(\mathcal{D}^nG,\mathcal{D}^nG)$ (with $\mathcal{D}^0G=G$) must
    vanish after a finite $n$. To show that $\bar G$ is solvable, we must study the
    commutator $(\bar G,\bar G)$. We claim that the closure of $(\bar G,\bar G)$ is
    precisely the closure of $(G,G)$. To see this, consider the map $c:GL_n\times GL_n\to GL_n$
    given by $c(x,y)=xyx^{-1}y^{-1}$. Since $G\times G$ is dense in $\bar G\times\bar G$
    we find that $c(G\times G)$ must be dense in $c(\bar G\times \bar G)$. It follows
    then that $\overline{c(G\times G)}=\overline{c(\bar G\times\bar G)}$, but this is
    precisely the statement that $\overline{(G,G)}=\overline{(\bar G,\bar G)}$. Finally,
    we note that $(\bar G,\bar G)$ is in fact closed (c.f. Borel Proposition I.2.3),
    and hence $(\bar G,\bar G)=\overline{(G,G)}$. By a simple induction, then, we find
    that $\mathcal{D}^N\bar G=0$ for some large $N$, which shows that $\bar G$ is solvable.
\end{proof}


\end{document}
