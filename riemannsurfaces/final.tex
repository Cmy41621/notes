\documentclass{../mathnotes}

\usepackage{tikz-cd}
\usepackage{amsmath}
\usepackage{todonotes}


\title{Complex Analysis and Riemann Surfaces: Final}
\author{Nilay Kumar}
\date{Last updated: \today}


\begin{document}

\maketitle

\section*{Problem 1}

Let $\tau\in\C$ with $\text{Im }\tau>0$ and let $\Lambda=\left\{ m+n\tau; m,n\in\Z \right\}$ be the lattice generated by 1 and $\tau$.
\begin{enumerate}[(a)]
    \item Consider the functions defined by 
        \begin{align*}
            \theta(z|\tau)&=\sum_{n=-\infty}^\infty \exp\left( \pi i n^2\tau + 2\pi in\tau z \right)\\
            \theta_1(z|\tau)&=\exp\left( \frac{\pi i \tau}{4}+\pi i\left(z+\frac{1}{2}\right) \right)\theta\left(z+\frac{1+\tau}{2}|\tau\right)\\
            &=\sum_{n\in\Z}\exp\left( \pi i\left( n+\frac{1}{2} \right)^2\tau+2\pi i\left( n+\frac{1}{2} \right)\left( z+\frac{1}{2} \right) \right).
        \end{align*}
        $\theta_1$ is an odd holomorphic function, and in fact, its only zero is at $0\mod\Lambda$. Furthermore, it transforms in the following way 
        \begin{align*}
            \theta_1(z+1|\tau) &=-\theta_1(z|\tau)\\
            \theta_1(z+\tau|\tau)&=-e^{-\pi i\tau-2\pi i z}\theta_1(z|\tau).
        \end{align*}

    \item Let $P\neq Q$ be arbitrary points on the torus $\C/\Lambda$. Using $\theta_1$, we can construct a meromorphic form $\omega_{PQ}(z)$ on $\C/\Lambda$
        with simple poles at $P$ and $Q$ with residues $\pm 1$ respectively. Consider the form
        \[\omega_{PQ}(z)=\partial_z\left( \log \frac{\theta_1(z-P)}{\theta_1(z-Q)} \right)dz\]
        on $\C$.  Note first that, using the above transformation rules, $\omega_{PQ}(z)$ is doubly-periodic:
        \begin{align*}
            \omega_{PQ}(z+1)&=\omega_{PQ}(z)\\
            \omega_{PQ}(z+\tau)&=\partial_z\left( \log \frac{\exp\left( \pi i\tau -2\pi i(z-P) \right)\theta_1(z-P)}{\exp\left( -\pi i\tau -2\pi i(z-Q) \right)\theta_1(z-Q)} \right)\\
            &=\partial_z\left( \log e^{2\pi i(P+Q)}+\log \frac{\theta_1(z-P)}{\theta_1(z-Q)} \right)\\
            &=\omega_{PQ}(z)
        \end{align*}
        and hence well-defined on $\C/\Lambda$. Rewriting $\omega_{PQ}(z)$ as
        \[\omega_{PQ}(z)=\frac{\theta_1'(z-P)}{\theta_1(z-P)}-\frac{\theta_1'(z-Q)}{\theta_1(z-Q)},\]
        we see that $\omega_{PQ}(z)$ has poles at $P$ and $Q$ (as $\theta_1$ vanishes to order one at 0 and $\theta_1'$ does not because if it did, $\theta_1$ would have a double zero).
        
        %First note that by the transformation rules above, the logarithmic derivative
        %\[\left(\ln \theta_1(z|\tau)\right)'=\theta_1'(z|\tau)/\theta_1(z|\tau)\]
        %satisfies:
        %\begin{align*}
        %    \frac{\theta_1'(z+1|\tau)}{\theta_1(z+1|\tau)}&=\frac{\theta_1'(z|\tau)}{\theta_1(z|\tau)}\\
        %    \frac{\theta_1'(z+\tau|\tau)}{\theta_1(z+\tau|\tau)}&=\frac{\theta_1'(z|\tau)}{\theta_1(z|\tau)}-2\pi i.
        %\end{align*}
        %Then the form
        %\[\omega_{PQ}(z)=\left(\frac{\theta_1'(z-P|\tau)}{\theta_1(z-P|\tau)}-\frac{\theta_1'(z-Q|\tau)}{\theta_1(z-Q|\tau)}\right)dz\]
        %is doubly-periodic as the transformation factors cancel, and clearly has simple poles at $P$ and $Q$ with the appropriate residues
        %(as $\theta_1'$ has no poles since $\theta_1$ is holomorphic).

    \item Let $P$ be an arbitrary point on $\C/\Lambda$. Using $\theta_1$ we can construct a meromorphic form $\omega_{P}(z)$ on $\C/\Lambda$ with
        a double pole at $P$. In particular, consider
        \[\omega_{P}(z)=\partial_z^2\left( \log \frac{\theta_1(z-P)}{\theta_1'(0)} \right).\]
        First note that $\omega_{P}(z)$ is doubly-periodic:
        \begin{align*}
            \omega_{P}(z+1)&=\omega_{P}(z)\\
            \omega_{P}(z+\tau)&=\partial_z^2\left( \log\frac{\theta_1(z+\tau-P)}{\theta_1'(0)} \right)\\
            &=\partial_z^2\left( \log\frac{-e^{-\pi i \tau-2\pi i (z-P)}\theta_1(z-P)}{\theta_1'(0)} \right)\\
            &=\omega_{P}(z)
        \end{align*}
        as the derivatives annihilate the constant terms that pop out, and hence $\omega_{P}$ is well-defined on $\C/\Lambda$. Rewriting, we find that
        \[\omega_{P}(z)=\partial_z\frac{\theta_1'(z-P)}{\theta(z-P)}=\frac{\theta_1''(z-P)\theta_1(z-P)-(\theta_1'(z-P))^2}{\theta_1^2(z-P)}.\]
        When evaluated at $z=P$, since $\theta_1(0)=0$, we are left with $\theta_1'(0)^2/\theta_1^2(0)$ and hence $\omega_P(z)$ has a double pole at
        $z=P$ (as $\theta'(0)\neq 0$, as before).
\end{enumerate}

\section*{Problem 2}        

Let $L\to X$ be a holomorphic line bundle over a compact Riemann surface $X$.
\begin{enumerate}[(a)]
    \item Let $h(z)$ be a metric on $L$. Recall that a metric is a strictly positive section of $L^{-1}\otimes \bar L^{-1}$.
        Using $h$, we can differentiate sections of our line bundle $L$ as follows. 
        Recall that we defined the derivative
        \begin{align*}
            \nabla_{\bar z}=\bar\partial:\Gamma(X,L)&\to\Gamma(X,L\otimes\Lambda^{0,1})\\
            \phi=\{\phi_\alpha\}&\mapsto \nabla_{\bar z}\phi=\{\partial\phi_\alpha/\partial \bar z_\alpha\}
        \end{align*}
        as well as the derivative
        \begin{align*}
            \nabla_{z}:\Gamma(X,L)&\to\Gamma(X,L\otimes\Lambda^{0,1})\\
            \phi=\{\phi_\alpha\}&\mapsto h_\alpha^{-1}\frac{\partial}{\partial z_\alpha}(h_\alpha\phi_\alpha).
        \end{align*}
        We computed the failure of commutativity of these two derivatives, i.e. the curvature $F_{\bar zz}\in\Gamma(X,\Lambda^{1,1})$:
        \[ [\nabla_z,\nabla_{\bar z}]\phi=F_{\bar zz}\phi=-(\partial_{\bar z}\Gamma)\phi=-(\partial_{\bar z}\partial_z\log h)\phi. \]
        We then define the first Chern class $c_l(L)$ of the bundle to be
        \[ c_1(L)=\frac{i}{2\pi} \int_X F_{\bar z z} dz\wedge d\bar z.\]
        Recall that we showed in class that $c_1(L)$ is independent of the metric, as is suggested by the notation.
    \item We wish to show that if $c_1(L)<0$ then the line bundle $L$ does not admit any non-trivial holomorphic sections.
        To do this, we first show that given any section $\phi\in\Gamma(X,L)$ the number its zeros minus the number of its poles
        is precisely $c_1(L)$. Then it follows immediately that if $c_1(L)<0$, every section must have at least one pole. To show this,
        we work locally.  Locally we may write: 
        \[F_{\bar zz}dz\wedge d\bar z=-\partial_z\partial_{\bar z}\log h\; dz\wedge d\bar z=-d\left( (\partial_{\bar z}\log h)d\bar z \right).\]
        We cannot use Stoke's theorem here, as $\partial_{\bar z}\log h$ is not even globally well-defined. Instead, if we let $\{P_i\}_{i=1}^N$ be
        the zeroes and poles of a section $\phi$, outside $\varepsilon$-discs $D_i$ about $P_i$ (only containing one zero or pole), we can write
        \begin{align*}
            F_{\bar zz}dz\wedge d\bar z&=-\partial_z\partial_{\bar z}\log ||\phi||_h^2 dz\wedge d\bar z\\
            &=d\left( (\partial_{\bar z}\log ||\phi||_h^2) d\bar z \right)
        \end{align*}
        as $||\phi||_h^2=\phi\bar\phi h$, and so the $\log$ splits, but the extra terms involving $\log\phi$
        and $\log\bar\phi$ vanish under the $\partial_{\bar z}$ and the $\partial_{z}$ derivatives respectively. Then by Stokes' theorem
        we find that
        \begin{align*}
            \int_X F_{\bar zz}dz\wedge d\bar z&=\lim_{\varepsilon\to0}\int_{X\setminus\cup_i D_i}d\left( (\partial_{\bar z}\log ||\phi||_h^2) d\bar z \right)\\
            &=\sum_{i=1}^N\lim_{\varepsilon\to0}\oint_{\partial D_i}\partial_{\bar z}\log||\phi||_h^2 d\bar z\\
            &=\sum_{i=1}^N\lim_{\varepsilon\to 0}\oint_{\partial D_i}\partial_{\bar z}\log\bar\phi d\bar z\\
        \end{align*}
        where the $\log\phi$ vanishes under the $\partial_{\bar z}$ and the $\log h$ term vanishes when integrated against $d\bar z$, as it is smooth.
        As $\phi$ vanishes or blows up at each $P_i$, we can write $\phi(z)=(z-P_i)^{N_i} u(z)$ for some $u(z)$ holomorphic and non-zero at $P_i$. But then
        \[\partial_z\log\phi=\frac{\partial_z\phi}{\phi}=\frac{N_i}{z-P_i}+\frac{\partial_z u(z)}{u(z)}\]
        and so we have
        \begin{align*}
            \int_X F_{\bar zz} dz\wedge d\bar z&=\sum_{i=1}^N\lim_{\varepsilon\to 0}\oint_{\partial D_i}\frac{N_i}{\bar z-\bar P_i}+\overline{\left(\frac{\partial_z u(z)}{u(z)}\right)}d\bar z\\
            &=\sum_{i=1}^NN_i\lim_{\varepsilon\to 0}\oint_{\partial D_i}\frac{d\bar z}{\bar z-\bar P_i}\\
            &=-2\pi i\sum_{i=1}^{N}N_i,
        \end{align*}
        but then the first Chern class becomes:
        \[c_1(L)=\frac{1}{2\pi}\int_X F_{\bar zz}dz\wedge d\bar z=\sum_{i=1}^NN_i,\]
        but this is precisely the number of zeros minus the number of poles (counted with multiplicity). This proves the claim.
    \item The Riemann-Roch theorem states that
        \[\dim H^0(X,L)-\dim H^0(X,L^{-1}\otimes K_X)=c_1(L)+\frac{1}{2}c_1(K_X^{-1})\]
        where $K_X=\Lambda^{1,0}$ is the canonical bundle. Inserting $L=K_X^n$, we find that
        \begin{align*}
            \dim H^0(X,K_X^n)-\dim H^0(X,K_X^{1-n})&=c_1(K_X^n)+\frac{1}{2}c_1(K_X^{-1})\\
            &=\left( \frac{1}{2}-n \right)c_1(K_X^{-1})
        \end{align*}
        where we have used the homomorphism properties of the first Chern class.
    \item Let us use the above formula to deduce the value of $\dim H^0(X,K_X^n)$. There are three cases: $c_1(K^{-1}_X)>0$, $c_1(K^{-1}_X)=0,$ and $c_1(K^{-1}_X)<0$.
        
        Consider first the case $c_1(K^{-1}_X)=0$. It is clear that $c_1(K^{n}_X)=-nc_1(K^{-1}_X)=0$, and hence the holomorphic sections of $K^{n}_X$ must have
        no zeroes. This implies that the $\dim H^0(X,K_X^{n})=1$ for all $n$ - to see this, note that given any two linearly independent sections $f,g$ we can
        scale $g$ to equal $f$ at any one point. Taking the difference of $f$ and this scaled $g$ we obtain a section that is zero at that point, and hence this
        space must be one-dimensional.
        
        Next consider $c_1(K^{-1}_X)<0$. Then $c_1(K^{n}_X)=-nc_1(K^{-1}_X)$ is greater than zero for positive $n$ and less than zero for negative $n$.
        This implies that there are no holomorphic sections for negative $n$, i.e. $\dim H^0(X,K_X^{n})=0$ for $n<0$. For positive $n$, the formula above yields:
        \begin{align*}
            \dim H^0(X,K_X^n)-\dim H^0(X,K^{1-n}X)&=(1-2n)(1-g)\\            
            \dim H^0(X,K_X^n)&=(1-2n)(1-g).
        \end{align*}

        Finally consider $c_1(K^{-1}_X)>0$. Then $c_1(K^{n}_X)=-nc_1(K^{-1}_X)$ is less than zero for positive $n$ and greater than zero for negative $n$.
        This implies that there are no holomorphic sections for positive $n$, i.e. $\dim H^0(X,K_X^{n})=0$ for $n>0$. For negative $n$, the formula above yields:
        \begin{align*}
            \dim H^0(X,K_X^n)-\dim H^0(X,K^{1-n}X)&=(1-2n)\\
            \dim H^0(X,K_X^n)&=(1-2n).
        \end{align*}


        %Next consider $c_1(K^{-1}_X)<0$. Then $c_1(K^{n}_X)=-nc_1(K^{-1}_X)$ is greater than zero for positive $n$ and less than zero for negative $n$. This implies
        %that there are no holomorphic sections for negative $n$, i.e. $\dim H^0(X,K_X^{n})=0$ for $n<0$. For positive $n$, the formula above yields:\todo{missing $(1-g)$?}
        %\begin{align*}
        %    \dim H^0(X,K_X^n)-\dim H^0(X,K^{1-n}_X)&=1-2n\\
        %    \dim H^0(X,K_X^n)&=1-2n,
        %\end{align*}
        %where we have used the fact that the genus is zero and that $1-n<0$ if $n>0$.

        %Finally, consider $c_1(K^{-1}_X)>0$. Then $c_1(K^{n}_X)=-nc_1(K^{-1}_X)$ is less than zero for positive $n$ and greater than zero for negative $n$.
        %This implies that there are no holomorphic sections for positive $n$, i.e. $\dim H^0(X,K_X^{n})=0$. For negative $n$, the formula above yields:
        %\begin{align*}
        %    \dim H^0(X,K_X^n)-\dim H^0(X,K^{1-n}_X)&=(1-2n)(1-g)\\
        %    \dim H^0(X,K_X^n)&=(1-2n)(1-g),
        %\end{align*}
        %where we have used the genus and that $1-n>0$ if $n<0$.\todo{what's up with $n=0$?}

    \item Recall from class that we have a correspondence between complex structures and metrics, and hence it makes sense to attempt to construct
        the moduli space of Riemann surfaces by first considering the space of metrics. However, we have a natural action of both the Weyl group $W$ and
        the group of diffeomorphisms of the surface $\Diff X$ on the space of metrics, in the form of $W\rtimes\Diff X$. Hence we define the moduli space
        of a Riemann surface of genus $h$ to be
        \[\mathcal{M}_h=\left\{ \text{space of metrics} \right\}/W\rtimes\Diff X.\]
        To study this moduli space, we instead study the structure of its tangent space, in order to determine $\dim\mathcal{M}_h$. In particular,
        we claim, by functoriality of the tangent space functor, that
        \[T_{[g_{ij}]}\mathcal{M}_h=\{\delta g_{ij}\}/\{\delta\sigma g_{ij}\}+\{\nabla_i(\delta v)_j+\nabla_j(\delta v)_i\}\]
        where $g_{ij}$ is some equivalence class of metrics, the first is associated with the Weyl transform and the second term is associated
        with the tangent space of the group of diffeomorphisms (i.e. smooth vector fields).
        We will not prove this claim here, as it is
        rather computational and because it was done in class. To compute the dimension of this tangent space we note that
        \[T_{[g_{ij}]}\mathcal{M}_h=\left\{ \delta g_{\bar z\bar z}(d\bar z)^2 \right\}/\left\{ \partial_{\bar z}(\delta v^z) \right\}=\text{coker }\bar\partial|_{K_X^{-1}}\]
        where the $\bar\partial$ operator takes
        \[\Gamma(X,K_X^{-1})\ni\delta v^z\longrightarrow\partial_{\bar z}(\delta v^z)\in\Gamma(X,\overline{K_{X}}\otimes K_X^{-1}).\]
        Computing the dimension then simply reduces to computing $H^0(X,K_X^2)$ (as this gives us the ``orthogonal complement''), which can now easily
        be done via Riemann-Roch and our earlier work. In particular, if $h=0$, $c_1(K_X^2)=-2c_1(K_X^{-1})=-4<0$ and hence $\dim H^0(X,K_X^2)=0$,
        by the arguments of the previous part. Hence the tangent space (and thus the moduli space) is zero-dimensional.
        If $h=1$, we are in the case where $c_1(K_X^{-1})=0$ and hence the tangent space (and the moduli space) is one-dimensional.
        Finally, if $h\geq 2$ then $c_1(K_X^{-1})<0$ and by the above work we find $\dim H^0(X,K_X^2)=-3(1-h)=3h-3$.
\end{enumerate}

\section*{Problem 3}

Let $L\to X$ be a holomorphic line bundle over a compact Riemann surface $X$ and let $h(z)$ be a metric on $L$ and $g_{\bar zz}$ be a metric on $K_X^{-1}$.
\begin{enumerate}[(a)]
    \item Let us define on $\Gamma(X,L)$ an $L^2$ inner product as
        \[||\phi||^2=\int_X\phi\bar\phi hg_{\bar zz}.\]
        This is well-defined, as $\phi\bar\phi h$ yields a scalar, and $g_{\bar zz}$ is a $(1,1)$-form. Similarly, on $\Gamma(X,L\otimes \overline{K_{X}})$, we
        define the inner product as
        \[||\psi||^2=\int_X\psi\bar\psi h,\]
        which is well-defined as the components in $L$ cancel to yield, again, a $(1,1)$-form.

    \item Let the operator $\bar\partial:\Gamma(X,L)\to\Gamma(X,L\otimes\overline{K_X})$ be defined by $\bar\partial\phi=\partial_{\bar z}\phi$.
        In order to trace back to Riemann-Roch we must determine the formal adjoint of $\bar\partial$, i.e.
        \[\langle \bar\partial\phi,\psi\rangle=\langle\phi,\bar\partial^\dagger\psi\rangle,\]
        where we note that these inner products are taken in different spaces. More explicitly,
        \begin{align*}
            \int\partial_{\bar z}\phi\bar\psi h&=\int \phi\overline{\bar\partial^\dagger\psi}hg_{\bar zz}\\
            -\int\phi\overline{\partial_z(h\psi)}&=-\int \phi\overline{g_{\bar zz}^{-1}h^{-1}\partial_z(h\psi)}hg_{\bar zz},
        \end{align*}
        where we have integrated by parts. This shows that
        \[\bar\partial^\dagger\psi=-g^{z\bar z}h^{-1}\partial_z(h\psi)=-g^{\bar zz}\nabla_z\psi.\]
    \item Define the Laplacians $\Delta_+=\bar\partial^\dagger\bar\partial$ and $\Delta_-=\bar\partial\bar\partial^\dagger$. We assume that the eigenvalues
        $\lambda_n^{\pm}$ of $\Delta_{\pm}$ are discrete and tend to infinity polynomially. Note that if $\lambda\neq 0$ is an eigenvalue of $\Delta_+$,
        it is also an eigenvalue of $\Delta_-$: if $\Delta_+\phi=\bar\partial^\dagger\bar\partial\phi=\lambda\phi$, then
        $\bar\partial\bar\partial^\dagger\bar\partial\phi=\bar\partial\lambda\phi$, i.e. $\Delta_-\bar\partial\phi=\lambda\bar\partial\phi$. The converse
        holds very similarly. The zero eigenvalues, on the other hand, are precisely the kernels - in other words, we have paired everything
        but the kernel and cokernel. We can write
        \[\dim\ker\Delta_+-\dim\ker\Delta_-=\dim\ker\bar\partial-\dim\ker\bar\partial^\dagger.\]
        To see why, note that
        \begin{align*}
            \dim\ker\Delta_+&=\left\{ \phi\in\Gamma(X,L)\mid \bar\partial^\dagger\bar\partial\phi=0 \right\}\\
            &=\left\{ \phi\in\Gamma(X,L)\mid \langle\phi,\bar\partial^\dagger\bar\partial\phi\rangle=0 \right\}\\
            &=\left\{ \phi\in\Gamma(X,L)\mid \langle\bar\partial\phi,\bar\partial\phi\rangle=||\bar\partial\phi||^2=0 \right\}
        \end{align*}
        and similarly for $\Delta_-$. Now we claim that this is in fact also equal to $\tr e^{-t\Delta_+}-\tr e^{-t\Delta_-}$, where the exponential
        of an operator just acts in the obvious way on basis eigenfunctions as multiplication by the exponential evaluated at the eigenvalue (here
        we are using completeness). Note:
        \[\tr e^{-t\Delta_+}-\tr e^{-t\Delta_-}=\sum_{\lambda\text{ of }\Delta_+}e^{-t\lambda}-\sum_{\lambda\text{ of }\Delta_-}e^{-t\lambda}=\dim\ker\Delta_+-\dim\ker\Delta_-,\]
        as desired (we have implicitly used the pairing mentioned above). Hence we conclude that 
        \[\tr e^{-t\Delta_+}-\tr e^{-t\Delta_-}=\dim\ker\bar\partial-\dim\ker\bar\partial^\dagger.\]
    \item Note first that $H^0(X,L)=\ker\bar\partial|_{\Gamma(X,L)}$.
        Furthermore,
        \begin{align*}
            \ker\bar\partial^\dagger&=\left\{ \psi\in\Gamma(X,L\otimes \overline{K_X})\mid-g^{z\bar z}\nabla_z\psi=0 \right\}\\
            &=\left\{ \psi\in\Gamma(X,L\otimes \overline{K_X})\mid\partial_z(h\psi)=0 \right\}\\
            &=\left\{ \psi\in\Gamma(X,L\otimes \overline{K_X})\mid\partial_{\bar z}(h\bar\psi)=0 \right\},
        \end{align*}
        and thus we have (metric-dependent) isomorphism
        \begin{align*}
            \ker\bar\partial^\dagger\ni \psi \longleftrightarrow h\bar\psi\in\Gamma(X,K_X\otimes L^{-1})=\ker\bar\partial|_{\Gamma(X,K_X\otimes L^{-1})},
        \end{align*}
        which is precisely the second term of the left-hand side of the Riemann-Roch theorem:
        \[\dim\ker\bar\partial^\dagger=\dim H^0(X,K_X\otimes L^{-1}).\]
        Hence, from the part above, we find that
        \[\tr e^{-t\Delta_+}-\tr e^{-t\Delta_-}=\dim H^0(X,L)-\dim H^0(X,K_X\otimes L^{-1}).\]
\end{enumerate}


\section*{Problem 4}
Let $E\to X$ be a holomorphic vector bundle of rank $r$ over a complex manifold $X$ of dimension $n$. Let $H_{\bar\alpha\beta}(z)$ be a Hermitian metric on $E$.
Define the Chern unitary connection $\Delta$ on $E$ by
\[\nabla_{\bar j}\phi^{\alpha}=\partial_{\bar j}\phi^\alpha \text{ and } \nabla_j\phi^\alpha=H^{\alpha\bar\beta}\partial_j(H_{\bar\beta\gamma}\phi^{\gamma}).\]
\begin{enumerate}[(a)]
    \item First note that these formulas make sense because the indices are proprerly contracted. However, we must show that they are globally well-defined as operators
        from $\Gamma(X,E)\to\Gamma(X,E\otimes\Lambda^{0,1})$ and $\Gamma(X,E)\to\Gamma(X,E\otimes\Lambda^{1,0})$. For the first definition, note that given a section
        $\phi^{\alpha}$ (and transition matrices $\tau_{\mu\nu}$), the following quantity transforms as
        \[\frac{\partial\phi_\mu^\alpha}{\partial \bar z_\mu}=\frac{\partial(\tau_{\mu\nu\beta}^\alpha\phi_\nu^\beta)}{\partial\bar z_\mu} =\tau_{\mu\nu\beta}^\alpha\frac{\partial\phi_\nu^\beta}{\partial \bar z_\mu}=\tau_{\mu\nu\beta}^\alpha\frac{\partial\phi_\nu^\beta}{\partial\bar z_\nu}\frac{\partial\bar z_\nu}{\partial\bar z_\mu}\]
        where we have used the holomorphicity of $E$. In other words, $\partial_{\bar z}$ is a well-defined section of $\Gamma(X,E\otimes\Lambda^{0,1})$. Meanwhile, for
        the second definition, note first that $H_{\bar\beta\gamma}$ is a section of $\Gamma(X,\overline{E^{-1}}\otimes E^{-1})$, and hence $H_{\bar\beta\gamma}\phi^\gamma$
        is a well-defined section of $\Gamma(X,\overline{E^{-1}})$. Then $\partial_j(H_{\bar\beta\gamma}\phi^\gamma)$ is a well-defined section of $\Gamma(X,\overline{E^{-1}}\otimes \Lambda^{1,0})$
        (this is the same argument as above, just with $z$ instead of $\bar z$) and finally $H^{\alpha\bar\beta}\partial_j(H_{\bar\beta\gamma}\phi^{\gamma})$ is a well-defined
        section of $\Gamma(X,E\otimes \Lambda^{1,0})$, as desired.
    \item We can rewrite the second expression above as follows:
        \begin{align*}
            H^{\alpha\bar\beta}\partial_j(H_{\bar\beta\gamma}\phi^{\gamma})&=\delta^{\alpha}_\gamma\partial_j\phi^\gamma+(H^{\alpha\bar\beta}\partial_jH_{\bar\beta\gamma})\phi^\gamma\\
            &=\partial_j\phi^\alpha+A^{\alpha}_{j\gamma}\phi^\gamma
        \end{align*}
        where we have defined $A^\alpha_{j\gamma}=H^{\alpha\bar\beta}\partial_jH_{\bar\beta\gamma}$. If we now define the curvature tensor $F_{\bar kj\beta}^\alpha$ by
        \[ [\nabla_j,\nabla_{\bar k}]\phi^\alpha=F_{\bar kj\beta}^\alpha\phi^\beta, \]
        we find that
        \begin{align*}
            [\nabla_j,\nabla_{\bar k}]\phi^\alpha&=\nabla_j(\partial_{\bar k}\phi^\alpha)-\nabla_{\bar k}\left( H^{\alpha\bar\beta}\partial_j(H_{\bar\beta\gamma}\phi^\gamma) \right)\\
            &=H^{\alpha\bar\beta}\partial_j\left( H_{\bar\beta\gamma}\partial_{\bar k}\phi^\gamma \right)-\partial_{\bar k}\left(\partial_j\phi^\alpha+A^\alpha_{j\gamma}\phi^\gamma \right)\\
            &=A^\alpha_{j\gamma}\partial_{\bar k}\phi^\gamma+\partial_j\partial_{\bar k}\phi^\alpha-\partial_{\bar k}\partial_j\phi^{\alpha}-\partial_{\bar k}\left( A_{j\gamma}^\alpha\phi^\gamma \right)\\
            &=-\partial_{\bar k}A_{j\gamma}^\alpha\phi^\gamma,
        \end{align*}
        and hence $F_{\bar kj\beta}^\alpha=-\partial_{\bar k}A_{j\beta}^\alpha$.
    \item Next consider the following $\End E=E\otimes E^*$-valued forms:
        \[A^\alpha_\beta=A_{j\beta}^\alpha dz^j\text{ and }F^\alpha_\beta=F_{\bar kj\beta}^\alpha dz^j\wedge d\bar z^k.\]
        We claim that $F=dA+A\wedge A$. To see this, we first compute (suppressing indices)
        \begin{align*}
            dA&=d(A_jdz^j)=(\partial_kA_jdz^k+\partial_{\bar k}A_jd\bar z^k)\wedge dz^j\\
            &=\frac{1}{2}(\partial_kA_j-\partial_jA_k)dz^k\wedge dz^j-F_{\bar kj}d\bar z^k\wedge dz^j\\
            &=\frac{1}{2}(\partial_kA_j-\partial_jA_k)dz^k\wedge dz^j+F\\
            &=\frac{1}{2}\left(\partial_k(H^{-1}\partial_jH)-\partial_j(H^{-1}\partial_kH)\right)dz^k\wedge dz^j+F\\
            &=\frac{1}{2}\left((-H^{-1}\partial_kHH^{-1})\partial_jH+H^{-1}\partial_k\partial_jH-(j\leftrightarrow k)\right)dz^k\wedge dz^j+F\\
            &=\frac{1}{2}\left(-H^{-1}\partial_kHH^{-1}\partial_jH-(j\leftrightarrow k)\right)dz^k\wedge dz^j+F\\
            &=\frac{1}{2}\left(-A_kA_j+A_jA_k\right)dz^k\wedge dz^j+F\\
            &=-A\wedge A+F,
        \end{align*}
        where we have used the fact that $\partial_kH^{-1}=-H^{-1}\partial_kHH^{-1}$. Hence we obtain the desired identity. Finally, note that
        \begin{align*}
            dF+A\wedge F-F\wedge A&=d(dA+A\wedge A)+A\wedge (dA+A\wedge A)-(dA+A\wedge A)\wedge A\\
            &=dA\wedge A-A\wedge dA+A\wedge dA+A\wedge A\wedge A-dA\wedge A-A\wedge A\wedge A\\
            &=0,
        \end{align*}
        as desired.
    \item Take $\phi^\alpha\in\Gamma(X,E)$ and $\psi_\alpha\in\Gamma(X,E^*)$. We can compute the covariant derivative induced on the dual bundle (given a covariant
        derivative on $E$, i.e. a metric) by enforcing the Liebniz rule:
        \begin{align*}
            \psi_\alpha\partial_j\phi^\alpha+\phi^\alpha\partial_j\psi_\alpha&=(\partial_j\phi^\alpha+A_{j\beta}^\alpha\phi^\beta)\psi_\alpha+\phi^\alpha\nabla_j\psi_\alpha\\
            \phi^\alpha\nabla_j\psi_\alpha&=\phi^\alpha\partial_j\psi_\alpha-A_{j\beta}^\alpha\phi^\beta\psi_\alpha\\
            \phi^\alpha\nabla_j\psi_\alpha&=\phi^\alpha\partial_j\psi_\alpha-A_{j\alpha}^\gamma\phi^\alpha\psi_\gamma\\
            \nabla_j\psi_\alpha&=\partial_j\psi_\alpha-A_{j\alpha}^\gamma\psi_\gamma,
        \end{align*}
        where we have switched dummy indices in order to isolate the covariant derivative. Next, take $T\in\Gamma(X,\End E)$, and enforce the Liebniz rule again:
        \begin{align*}
            \nabla_j(T)\phi&=\nabla_j(T\phi)-T(\nabla_j\phi)\\
            &=\nabla_j(T_{\alpha}^\beta\phi^\alpha)-T^{\beta}_\alpha(\partial_j\phi^\alpha+A^{\alpha}_{j\gamma}\phi^\gamma)\\
            &=\partial_j(T_\alpha^\beta\phi^\alpha)+A_{j\gamma}^\beta T_\alpha^\gamma \phi^\alpha-T^\beta_\alpha\partial_j\phi^\alpha-T^\beta_\alpha A^\alpha_{j\gamma}\phi^\gamma\\
            &=\partial_jT^\beta_\alpha\phi^\alpha+A_{j\gamma}^\beta T_\alpha^\gamma \phi^\alpha-T^\beta_\alpha A^\alpha_{j\gamma}\phi^\gamma.
        \end{align*}
        Changing dummy indices, we find that
        \[\nabla_jT^\alpha_\beta=\partial_jT^\alpha_\beta+A_{j\gamma}^\alpha T_\beta^\gamma -T^\alpha_\gamma A^\gamma_{j\beta}.\]
    \item Finally, let us use the covariant derivative now defined on $\End E$ to construct an operator $d_A$
        that differentiates on $\End E$-valued forms. This is fairly simple - in the definition of the exterior derivative, instead of using the usual
        derivative, we now substitute the covariant derivative associated to the connection $A$ derived above. Let $\xi\in\Gamma(X,\End E\otimes\Lambda^p)$. Then
        \begin{align*}
            (d_A\xi)^\alpha_\beta&=\frac{1}{p!}\sum\nabla_j^A\xi^\alpha_{\beta,i_1,\ldots,i_p}dz^j\wedge dz^{i_1}\wedge\cdots\wedge dz^{i_p}\\
            &=\frac{1}{p!}\sum\left(\partial_j\xi^\alpha_{\beta}+A_{j\gamma}^\alpha\xi^\gamma_\beta-\xi^\alpha_\gamma A_{j\beta}^\gamma \right)_{i_1,\ldots,i_p}dz^j\wedge dz^{i_1}\wedge\cdots\wedge dz^{i_p}\\
            &=d\xi^\alpha_\beta+A^\alpha_{j\gamma}\wedge\xi^\gamma_\beta-\xi^\alpha_\gamma\wedge A^\gamma_{j\beta}.
        \end{align*}
        If $F$ is the curvature form of a metric $H_{\bar\alpha\beta}$, then by the identity $dF+A\wedge F-F\wedge A=0$, we find that
        \[d_AF=0.\]
\end{enumerate}



\end{document}
