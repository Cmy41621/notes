\documentclass{../../mathnotes}

\usepackage{tikz-cd}
\usepackage{todonotes}

\title{Riemann Surfaces PSET 1}
\author{Nilay Kumar}
\date{Last updated: \today}


\begin{document}

\maketitle

\section*{Problem 1}

As $D$ is a compact regular domain in $\R^2$, we can apply Stoke's theorem (equivalently, Green's theorem), $\int_{\partial D}\omega=\int_Dd\omega$.
We have $\omega=f(z)dz$ and thus
\begin{align*}
    d\omega=d(f(z)dz)=d(f(z))\wedge dz=\left( \frac{\partial f}{\partial z}dz+\frac{\partial f}{\partial \bar z}d\bar z \right)\wedge dz=\frac{\partial f}{\partial \bar z}d\bar z\wedge dz,
\end{align*}
which yields
\begin{align*}
    \int_{\partial D}f(z)dz=\int_{D}\frac{\partial f}{\partial \bar z}d\bar z\wedge dz,
\end{align*}
as desired.



\section*{Problem 2}

Define $g(z):\Omega\to\C$ as $0$ at $z=0$ and $z^2f(z)$ on $\Omega\setminus 0$. As $f(z)$ is holomorphic on $\Omega\setminus0$, it is clear that
$g(z)$ is holomorphic on $\Omega\setminus0$ as well. It is easy to see that $g(z)$ is in fact holomorphic at $z=0$ as well:
\begin{align*}
    g'(0)=\lim_{h\to 0}\frac{g(h)-g(0)}{h}=\lim_{h\to 0}\frac{g(h)}{h}=\lim_{h\to 0}hf(h)=0
\end{align*}
where in the last step we have used the boundedness of $f(z)$ in the neighborhood of $z=0$. We can now write out a power series for $g(z)$
on $\Omega$ (about $z=0$), $g(z)=a_0+a_1z+a_2z^2+\ldots$. Note, however, that $g(0)=0$ and hence $a_0=0$. In fact, $a_1=0$ as well; suppose it were
not: then $f(z)=g(z)/z^2=a_1/z+a_2+\ldots$ on $\Omega\setminus0$, which contradicts the boundedness of $f$ at 0. Hence we see that 
\begin{align*}
    g(z)&=a_2z^2+a_3z^3+\ldots\\
    f(z)&=a_2+a_3z+\ldots\text{ on } \Omega\setminus0,
\end{align*}
which implies that $f(z)$ can be holomorphically extended to $\Omega$ by simply defining it to take the value $a_2$ at $z=0$.


\section*{Problem 3}

Define $I(z)=\int_0^1 f(x)x^{z-1}dx$. 



\end{document}
