\documentclass{../mathnotes}

\usepackage{tikz-cd}
\usepackage{amsmath}
\usepackage{todonotes}


\title{Complex Geometry: Midterm}
\author{Nilay Kumar}
\date{Last updated: \today}


\begin{document}

\maketitle

\section*{Problem 1}

Let $(X,g_{\bar kj})$ be a compact complex manifold, equipped with a Hermitian metric $g_{\bar kj}$,
which is not necessarily K\"ahler. Define the torsion tensor
\[T^j_{lp}=g^{j\bar k}\partial_lg_{\bar kp}-g^{j\bar k}\partial_pg_{\bar kl}\]
and let $\omega=\frac{i}{2}g_{\bar kj}dz^j\wedge d\bar z^k$.
\begin{enumerate}[(a)]
    \item Now let $V^j$ be a smooth vector field. We may write, via the product rule,
        \begin{align*}
            \int_X \nabla_j V^j\omega^n&=\int_X\left( \partial_jV^j+\Gamma^j_{jk}V^k \right)\omega^n\\
            &=\int_X\left( \partial_jV^j+\Gamma_{jk}^jV^k \right)\det g\cdot\wedge^n_{l=1}dz^l\wedge^n_{m=1}d\bar z^m\\
            &=\int_X\left(\partial_j(V^j\det g)-V^j(\partial_j\det g)\right)\cdot\wedge_l dz^l\wedge_m d\bar z^m
            +\int_X \Gamma^j_{jk}V^k\omega^n.
        \end{align*}
        Next, note that
        \begin{align*}
            \partial_j\log\det g&=\sum_i \frac{\partial_j\lambda_i}{\lambda_i}=\tr\left( g^{-1}\partial_jg\right),
        \end{align*}
        where $\lambda_i$ are the eigenvalues of $g$ but the left side is simply $\partial_j\log\det g=\partial_j(\det g)/\det g$, which yields the formula
        \[\partial_j\det g=(\det g)\tr\left( g^{-1}\partial_jg \right)=(\det g)g^{\bar ab}\partial_jg_{b\bar a}=(\det g)\Gamma^{b}_{jb}.\]
        Inserting this into the integral above and noting that the first term of the first integral vanishes (as it is exact),
        we find that
        \begin{align*}
            \int_X\nabla_jV^j\omega^n&=\int_X\Gamma^j_{jk}V^k\omega^n-\int_X \Gamma_{jb}^bV^j\omega^n\\
            &=\int_X \left( \Gamma^{j}_{jk}-\Gamma^{j}_{kj} \right)V^k\omega^n\\
            &=\int_X T^{j}_{jk}V^k\omega^n,
        \end{align*}
        as desired.
    \item Let us derive the following integration by parts formula:
        \[\int_X(\nabla_j\phi_{\bar KI})\psi^{j\bar KI}\omega^n=-\int_X\phi_{\bar KI}(\nabla_j\psi^{j\bar KI})\omega^n+\int_X\phi_{\bar KI}\psi^{j\bar KI}T^p_{jp}\omega^n,\]
        where $\omega$ is any Hermitian metric. This follows immediately from the previous part: write
        \[(\nabla_j\phi_{\bar KI})\psi^{j\bar KI}=\nabla_j(\phi_{\bar KI}\psi^{j\bar KI})-\phi_{\bar KI}\nabla_j\psi^{j\bar KI},\]
        and if we denote $V^j=\phi_{\bar KI}\psi^{j\bar KI}$, the integral on the left becomes
        \begin{align*}
            \int_X(\nabla_j\phi_{\bar KI})\psi^{j\bar KI}\omega^n&=\int_X \nabla_jV^j\omega^n-\int_X \phi_{\bar KI}(\nabla_j\psi^{j\bar KI})\omega^n\\
            &=\int_X T^p_{jp}V^k\omega^n-\int_X \phi_{\bar KI}(\nabla_j\psi^{j\bar KI})\omega^n\\
            &=\int_X T^p_{jp}\phi_{\bar KI}\psi^{j\bar KI}\omega^n-\int_X \phi_{\bar KI}(\nabla_j\psi^{j\bar KI})\omega^n,
        \end{align*}
        as desired.
\end{enumerate}

\section*{Problem 2}

Let $X$ be a complex manifold, and define the operator $\bar\partial$ on forms of type $(p,q)$ by
\[\bar\partial\left( \frac{1}{p!q!}\sum_{\bar JI}\phi_{\bar JI}dz^I\wedge d\bar z^J \right)=\frac{1}{p!q!}\sum_{\bar JI}\frac{\partial\phi_{\bar JI}}{\partial\bar z^k}d\bar z^k\wedge dz^I\wedge d\bar z^J.\]
Note that, as written, $(\bar\partial\phi)_{\bar k\bar JI}=(q+1)\partial_{\bar k}\phi_{\bar JI}$. If we antisymmetrize, this becomes
\begin{align*}
    (\bar\partial\phi)_{\bar k\bar JI}&=(\bar\partial\phi)_{\bar k\bar j_q\cdots\bar j_1 I}\\
    &=\sum_{\sigma\in S_{q+1}}\text{sgn}(\sigma)\partial_{\sigma(\bar k)}\phi_{\sigma(j_q)\cdots\sigma(j_1)I}\\
    &=\sum_{\sigma\in S_{q+1}}\text{sgn}(\sigma)\partial_{\sigma(\bar k)}\phi_{\sigma(J)I}\\
    &=\sum_{\sigma\in S_{q+1}}\text{sgn}(\sigma)\left(\nabla_{\sigma(\bar k)}\phi_{\sigma(J)I}-\Gamma^{\bar L}_{\sigma(\bar k)\sigma(J)}\phi_{\bar LI}\right),
\end{align*}
where the multi-index $\bar L$ is being summed over. The K\"ahler condition on the metric $g$ gives us, in this case, that
$\Gamma^{\bar L}_{\sigma(\bar k)\sigma(J)}=\Gamma^{\bar L}_{\sigma(J)\sigma(\bar k)}$. In the summation above, then,
we can pair the terms to obtain $(q+1)!/2$ pairs in which the connection terms cancel by the K\"ahler condition. If we
then de-antisymmetrize, we find that, as desired,
\begin{align*}
    \bar\partial\left( \frac{1}{p!q!}\sum_{\bar JI}\phi_{\bar JI}dz^I\wedge d\bar z^J \right)&=\frac{1}{p!q!}\sum_{\bar JI}\nabla_{\bar k}\phi_{\bar JI}d\bar z^k\wedge dz^I\wedge d\bar z^J.
\end{align*}

If $g_{\bar k j}$ were not K\"ahler, then in the $(0,1)$ case, we would find that for $\phi=\sum\phi_kd\bar z^k$,
\begin{align*}
    (\bar\partial\phi)_{\bar i\bar j}&=\sum(\bar\partial\phi_{\bar k})\wedge d\bar z^k=\sum(\partial_{\bar l}\phi_{\bar k}d\bar z^l)\wedge d\bar z^k\\
    &=\frac{1}{2}\sum\left( \partial_{\bar l}\phi_{\bar k} -\partial_{\bar k}\phi_{\bar l}\right)d\bar z^l\wedge d\bar z^k\\
    &=\frac{1}{2}\sum\left( \nabla_{\bar l}\phi_{\bar k}-\nabla_{\bar k}\phi_{\bar l} +\Gamma^{\bar m}_{\bar l\bar k} \phi_{\bar m}-\Gamma^{\bar m}_{\bar k\bar l}\phi_{\bar m}\right)d\bar z^l\wedge d\bar z^k
\end{align*}
and hence the formula would become
\[\bar\partial\left( \sum_{\bar i}\phi_{\bar i} d\bar z^i \right)=\sum_{\bar i}(\nabla_{\bar l}\phi_{\bar i}+\Gamma^{\bar m}_{\bar l\bar i}\phi_{\bar m})d\bar z^l\wedge d\bar z^i.\]


\section*{Problem 3}

Let $(X,g_{\bar kj})$ be a compact K\"ahler manifold, and consider the operator $\bar\partial$ as defined on $(p,q)$-forms as in
the previous problem. Let $L\to X$ be a holomorphic line bundle over $X$, with metric $h$.
\begin{enumerate}[(a)]
    \item Suppose we have $\phi,\psi\in\Gamma(X,L\otimes \Lambda^{0,2})$. We can then define an inner product (with summations supressed)
        \[\langle\phi,\psi\rangle\equiv \frac{1}{4}\int_X\phi_{\bar j\bar i}\overline{\psi_{\bar k\bar l}}g^{k\bar j}g^{\bar i l}h\omega^n/n!.\]
        Note that this is clearly linear in the first argument and conjugate-symmetric (moreover, the indices match up appropriately).
        Positive-definiteness follows from the positive-definiteness of $h$ and $g$. Similarly, given $\phi,\psi\in\Gamma(X,L\otimes \Lambda^{0,3})$,
        we can define an inner product
        \[\langle\phi,\psi\rangle\equiv \frac{1}{36}\int_X\phi_{\bar i\bar j\bar k}\overline{\psi_{\bar l\bar m\bar p}}g^{\bar il}g^{\bar jm}g^{\bar kp}h\omega^n/n!.\]
    \item Now consider part of the the $\bar\partial$ complex:
        \begin{equation*}
            \begin{tikzcd}
                \Gamma(X,L\otimes\Lambda^{0,1})\arrow[bend right]{r}{\bar\partial}&\Gamma(X,L\otimes\Lambda^{0,2})\arrow[bend right]{r}{\bar\partial}\arrow[bend right]{l}{\bar \partial^\dagger}&\Gamma(X,L\otimes\Lambda^{0,3})\arrow[bend right]{l}{\bar \partial^\dagger}%\arrow[bend right]{r}{\bar\partial}&\Gamma(X,L\otimes\Lambda^{0,4})\arrow[bend right]{l}{\bar \partial^\dagger}
            \end{tikzcd}
        \end{equation*}
        Let us compute the formal adjoints $\bar\partial^\dagger:\Gamma(X,L\otimes\Lambda^{0,2})\to\Gamma(X,L\otimes\Lambda^{0,1})$ and
        $\bar\partial^\dagger:\Gamma(X,L\otimes\Lambda^{0,3})\to\Gamma(X,L\otimes\Lambda^{0,2})$.

        For the first, take $\phi\in\Gamma(X,L\otimes \Lambda^{0,1})$ and $\psi\in\Gamma(X,L\otimes\Lambda^{0,2})$. By definition,
        the formal adjoint is such that
        \[\langle \bar\partial \phi,\psi\rangle=\langle\phi,\bar\partial^\dagger\psi\rangle.\]
        We can write $\psi=\frac{1}{2}\sum\psi_{\bar l\bar m}d\bar z^l\wedge d\bar z^m$ and
        \begin{align*}
            \bar\partial\phi&=\sum\partial_{\bar k}\phi_{\bar j}d\bar z^k\wedge d\bar z^j\\
            &=\frac{1}{2}\sum\left( \partial_{\bar k}\phi_{\bar j}-\partial_{\bar j}\phi_{\bar k}\right)d\bar z^k\wedge d\bar z^j.
        \end{align*}
        Then the above requirement thus becomes
        \begin{align*}
            \int_X \frac{1}{2}\left( \partial_{\bar k}\phi_{\bar j}-\partial_{\bar j}\phi_{\bar k}  \right)\overline{\psi_{\bar l\bar m}}hg^{l\bar j}g^{m\bar k}\omega^n/n!=\int_X \phi_{\bar j}\overline{(\bar\partial^\dagger\psi)_{\bar k}}hg^{k\bar j}\omega^n/n!.
        \end{align*}
        Note now that
        \[\partial_{\bar k}\phi_{\bar j}-\partial_{\bar j}\phi_{\bar k}=\nabla_{\bar k}\phi_{\bar j}-\nabla_{\bar j}\phi_{\bar k}.\]
        Now we can simplify the left-hand side by de-antisymmetrizing and integrating by parts:
        \begin{align*}
            \text{LHS}&=\frac{1}{2}\int_X(\nabla_{\bar k}\phi_{\bar j}-\nabla_j\phi_{\bar k})\overline{\psi_{\bar l\bar m}}hg^{l\bar j}g^{m\bar k}\omega^n/n!\\
            &=\int_X(\nabla_{\bar k}\phi_{\bar j}) \overline{\psi_{\bar l\bar m}}hg^{l\bar j}g^{m\bar k}\omega^n/n!\\
            &=\int_X \phi_{\bar j}\overline{(-g^{k\bar m}\nabla_k\psi_{\bar l\bar m})}g^{l\bar j}\omega^n/n!
        \end{align*}
        Hence we can write the formal adjoint as
        \[(\bar\partial^\dagger\psi)_{\bar l}=-g^{k\bar m}\nabla_k\psi_{\bar l\bar m}.\]

        The computation for the next formal adjoint is similar. We take $\phi\in\Gamma(X,L\otimes\Lambda^{0,2})$ and
        $\psi\in\Gamma(X,L\otimes \Lambda^{0,3})$. We obtain the equality, as above
        \begin{align*}
            \int_X(\nabla_{\bar l}\phi_{\bar m\bar p})\overline{\psi_{\bar i\bar j\bar k}}hg^{\bar li}g^{\bar mj}g^{\bar pk}\omega^n/n!
            =\int_X\phi_{\bar m\bar p}\overline{(\bar\partial^\dagger\psi)_{\bar i\bar j}}hg^{\bar mi}g^{\bar pj}\omega^n/n!.
        \end{align*}
        We can now integrate by parts (note that we have switched to covariant derivatives on the left, just as above, by noting
        that the connection terms drop out in the antisymmetrized expression, vis a vis problem 2) to obtain
        \begin{align*}
            \text{LHS}&=-\int_X \phi_{\bar m\bar p}(\nabla_{\bar l}\overline{\psi_{\bar i\bar j\bar k}})hg^{\bar li}g^{\bar mj}g^{\bar pk}\omega^n/n!\\
            &=\int_X \phi_{\bar m\bar p}\overline{(-\nabla_l\psi_{\bar i\bar j\bar k}g^{l\bar i})}hg^{\bar mj}g^{\bar pk}\omega^n/n!,
        \end{align*}
        from which we conclude that the formal adjoint can be written
        \[(\bar\partial^\dagger\psi)_{\bar i\bar j}=-g^{l\bar k}\nabla_l\psi_{\bar k\bar i\bar j}.\]
    \item Define $\Delta\equiv\bar\partial\bar\partial^\dagger+\bar\partial^\dagger\bar\partial$. With the expressions
        for the formal adjoints as above, we can compute explicitly the action of $\Delta$. Take some $\phi\in\Gamma(X,L\otimes\Lambda^{0,2})$.
        The first term becomes
        \begin{align*}
            \left( \bar\partial\bar\partial^{\dagger}\phi \right)_{\bar i\bar j}&=\nabla_{\bar i}(\bar\partial^\dagger\phi)_{\bar j}-\nabla_{\bar j}(\bar\partial^\dagger\phi)_{\bar i}\\
            &=\nabla_{\bar i}(-g^{k\bar m}\nabla_k\phi_{\bar j\bar m})-\nabla_{\bar j}(-g^{k\bar m}\nabla_k\phi_{\bar i\bar m})\\
            &=-g^{k\bar m}\nabla_{\bar i}\nabla_{k}\phi_{\bar j\bar m}+g^{k\bar m}\nabla_{\bar j}\nabla_k\phi_{\bar i\bar m}\\
            &=-g^{l\bar k}\nabla_{\bar i}\nabla_{l}\phi_{\bar j\bar k}+g^{l\bar k}\nabla_{\bar j}\nabla_l\phi_{\bar i\bar k}.
        \end{align*}
        On the other hand, the second term is
        \begin{align*}
            \left( \bar\partial^\dagger\bar\partial\phi \right)_{\bar i\bar j}&=-g^{l\bar k}\nabla_l\left(\nabla_{\bar k}\phi_{\bar i\bar j}-\nabla_{\bar i}\phi_{\bar j\bar k}-\nabla_{\bar j}\phi_{\bar k\bar i} \right)\\
            &=-g^{l\bar k}\nabla_l\nabla_{\bar k}\phi_{\bar i\bar j}+g^{l\bar k}\nabla_l\nabla_{\bar i}\phi_{\bar j\bar k}+g^{l\bar k}\nabla_l\nabla_{\bar j}\phi_{\bar k\bar i}.
        \end{align*}
        Adding the two terms, we find a Bochner-Kodaira formula for $\Delta$:
        \begin{align*}
            (\Delta\phi)_{\bar i\bar j}=-g^{l\bar k}\nabla_l\nabla_{\bar k}\phi_{\bar i\bar j}+g^{l\bar k}[\nabla_l,\nabla_{\bar i}]\phi_{\bar j\bar k}+g^{l\bar k}[\nabla_l,\nabla_{\bar j}]\phi_{\bar k\bar i}.
        \end{align*}
        Recall from class that we can rewrite the commutator as a sum of curvatures
        \begin{align*}
            [\nabla_l,\nabla_{\bar i}]\phi_{\bar j\bar k}&=F_{\bar il}\phi_{\bar j\bar k}+R_{\bar il\bar j}^{\bar m}\phi_{\bar m\bar k}+R_{\bar il\bar k}^{\bar m}\phi_{\bar j\bar m}\\
            [\nabla_l,\nabla_{\bar j}]\phi_{\bar k\bar i}&=F_{\bar jl}\phi_{\bar i\bar k}+R_{\bar jl\bar i}^{\bar m}\phi_{\bar m\bar k}+R_{\bar jl\bar k}^{\bar m}\phi_{\bar i\bar m}
        \end{align*}
        Inserting these into the formula for the Laplacian, we find that
        \begin{align}
            (\Delta\phi)_{\bar i\bar j}&=-g^{l\bar k}\nabla_l\nabla_{\bar k}\phi_{\bar i\bar j}+g^{l\bar k}F_{\bar il}\phi_{\bar j\bar k}+g^{l\bar k}R^{\bar m}_{\bar il\bar j}\phi_{\bar m\bar k}+g^{l\bar k}R^{\bar m}_{\bar il\bar k}\phi_{\bar j\bar m}\notag\\
            &+g^{l\bar k}F_{\bar jl}\phi_{\bar i\bar k}+g^{l\bar k}R^{\bar m}_{\bar jl\bar i}\phi_{\bar m\bar k}+g^{l\bar k}R^{\bar m}_{\bar jl\bar k}\phi_{\bar i\bar m}
            \label{eq:bk}
        \end{align}
    \item Given the above expression for the Laplacian in terms of the appropriate curvatures, we now reach a vanishing result.
        In particular, let $L$ be a positive line bundle (i.e. $h$ has positive curvature). Given a K\"ahler metric $\omega$ on $X$,
        we consider the $\Delta$ operator on $\Gamma(X,L^m\otimes\Lambda^{0,2})$. If we consider the inner product
        \begin{align*}
            \langle \Delta\phi,\phi\rangle=\int_X (\Delta\phi)_{\bar i\bar j}\overline{\phi_{\bar k\bar l}}g^{k\bar j}g^{l\bar i}h\frac{\omega^n}{n!}.
        \end{align*}
        Looking at equation (\ref{eq:bk}), we see that the first term in the integrand will (after integrating by parts and thus losing the minus
        sign) become $||\bar \nabla\phi||^2$, just as in the simpler case done in class. The rest of the integral,
        \begin{align*}
            \int_X g^{l\bar k}\left( mF_{\bar il}\phi_{\bar j\bar k}+R^{\bar m}_{\bar il\bar j}\phi_{\bar m\bar k}+R^{\bar m}_{\bar il\bar k}\phi_{\bar j\bar m}+mF_{\bar jl}\phi_{\bar i\bar k}+R^{\bar m}_{\bar jl\bar i}\phi_{\bar m\bar k}+R^{\bar m}_{\bar jl\bar k}\phi_{\bar i\bar m}\right)\overline{\phi_{\bar p\bar q}}g^{\bar i\bar p}g^{\bar j\bar q}h\frac{\omega^n}{n!},
        \end{align*}
        can be forced positive by choosing $m\gg 1$ sufficiently large, by the positivity of the line bundle (and hence $F$).
        This forces the inner product $\langle\Delta\phi,\phi\rangle$ to be positive, and hence we find that $\Delta\phi$ has trivial
        kernel, as the inner product is \textit{strictly} positive.
\end{enumerate}


\section*{Problem 4}

Let $f(z)$ be a holomorphic function in a neighborhood of $0$ in $\C^n$. Consider the plurisubharmonic function
$\phi(z)=(n+5)\log |z|^2$ (we showed in class that functions of this form are plurisubharmonic). Now suppose
that
\[\int_{B_\rho(0)}|f(z)|^2e^{-\phi(z)}=\int_{B_\rho(0)}|f(z)|^2|z|^{-2(n+5)}<\infty,\]
where we have chosen $\rho$ such that $B_{\rho}(0)\subset U$. Holomorphicity of $f(z)$ allows us to express it as
\[f(z)=\sum_{|\alpha|\leq n+5}a_{\alpha}z^{\alpha}+E(z),\]
where $|E(z)|\leq C|z|^{n+6}$. Applying the triangle inequality, we find that
\begin{align*}
    \int \bigg|\sum_{|\alpha|\leq n+5}a_\alpha z^\alpha\bigg|^2|z|^{-2(n+5)}\leq \int \left( |f|^2+|E(z)|^2 \right)|z|^{-2(n+5)}<\infty.
\end{align*}
The left-hand side can be computed in polar coordinates to be
\begin{align*}
    \int^\rho_0\int_{S^{2n-1}} \bigg|\sum_{|a|\leq \gamma}a_\alpha r^{|\alpha|}\omega^\alpha\bigg|^2 r^{-2(n+5)}r^{2n-1}dr d\sigma(\omega)&<\infty\\
    \int^\rho_0\sum_{|\alpha|\leq n+5}|a_\alpha|^2r^{2|\alpha|}r^{2n-1-2(n+5)}\left( \int_{S^{2n-1}}|\omega^\alpha|^2d\sigma(\omega) \right)dr&<\infty
\end{align*}
Note that the integral diverges unless $a_\alpha=0$ for $2|\alpha|+2n-1-2(n+5)\leq -1$, or equivalently $|\alpha|\leq 5$. Hence we find
that $f(z)$ vanishes of order 5 at the origin, as $a_\alpha=0$ for $|\alpha|\leq 5$.




%\setcounter{section}{-1}
\end{document}
