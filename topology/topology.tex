\documentclass{mathnotes}

\title{Notes on Topological and Differentiable Manifolds}
\author{Nilay Kumar}
\date {}

\begin{document}

\maketitle

\section{Elementary Topology}

Let us begin with the definition of a topology:
\begin{defn}
    A \textbf{topology} on a set $X$ is a collection $\mathcal{T}$ of subsets of $X$, called \textbf{open sets},
    satisfying the following properties:
    \begin{enumerate}
        \item $X$ and $\varnothing$ are elements of $\mathcal{T}$.
        \item $\mathcal{T}$ is closed under finite intersections: If $U_1\dots U_n\in\mathcal{T}$, then their intersection
            $U_1\cap\dots\cap U_n$ is in $\mathcal{T}$.
        \item $\mathcal{T}$ is closed under arbitrary unions: If $U_1\dots U_n\dots$ is any (finite or infinite) collection
            of elements of $\mathcal{T}$, then their union $\cup_\alpha U_\alpha$ is in $\mathcal{T}$.
    \end{enumerate}
    A pair $(X,\mathcal{T}$) consisting of a set $X$ and a topology $\mathcal{T}$ on $X$ is called a \textbf{topological space}.
    The elements of a topological space are usually called its \textbf{points}.
\end{defn}

\begin{defn}
    If $X$ is a topological space and $q\in X$, a \textbf{neighborhood} of $q$ is just an open set containing $q$. More generally, a neigborhood
    of a subset $K\subset X$ is an open set containing $K$.
\end{defn}

\begin{defn}
    If $X$ is a topological space and $\left\{ q_i \right\}$ is any sequence of points in $X$, we say that the sequence \textbf{converges} to $q\in X$, and
    $q$ is the \textbf{limit} of the sequence, if for every neighborhood $U$ of $q$ there exists $N$ such that $q_i\in U$ for all $i\geq N$. We denote this as
    $q_i\ra q$ or $\lim_{i\ra\infty}q_i=q$.
\end{defn}

\begin{exmp}
    Let $Y$ be a trivial topological space (i.e. the only open sets are $X$ and $\varnothing$). Each point has only 1
    neighborhood: $X$ itself. Thus, any sequence can be entirely contained in the neighborhood $X$, and consequently, any sequence
    converges to any point in $X$.
\end{exmp}

\begin{exmp}
    Let $X$ be a discrete topological space (i.e. all every subset of $X$ is open). Take any sequence of points $\left\{ q_i \right\}$.
    If the sequence converges to $q$, every open set containing $q$ must contain all but a finite elements of the sequence. By virtue of
    the discrete topology, there exists an open set that contains only $q$. Obviously, then, there must exist an $N$ such that
    $q_i=q$ for all $i\geq N$. Consequently, the only convergent sequences in $X$ are the ones that are ``eventually constant.''
\end{exmp}

\begin{defn}
    If $X$ and $Y$ are topological spaces, a map $f:X\ra Y$ is said to be \textbf{continuous} if for every open set
    $U\subset Y$, $f^{-1}(U)$ is open in $X$.
\end{defn}

\begin{lem}
    Let $X, Y, Z$ be topological spaces.
    \begin{enumerate}
        \item Any constant map $f:X\ra Y$ is continuous.
        \item The identity map $\id: X\ra X$ is continuous.
        \item If $f: X\ra Y$ is continuous, so is the restriction of $f$ to any open subset of $X$.
        \item If $f: X\ra Y$ and $g:Y \ra Z$ are continuous, so is their composition $g \circ f: X\ra Z$.
    \end{enumerate}
\end{lem}
\begin{proof}
    Let us begin with the constant map. Suppose $f$ maps $X$ to the constant $\lambda\in Y$. We wish to show
    that the preimage of $f$ of every open set $U$ in $Y$ is open. There are two cases: $U$ either does or does not contain
    $\lambda$. If it does, $f^{-1}(U)=X$; otherwise, $f^{-1}(U)=\varnothing.$ As both $X$ and $\varnothing$ are open sets,
    $f$ is continuous.

    The continuity of the identity map follows trivially from the fact that $\id$ maps any open set back to the same open set.

    To prove the third statement, take any open set $U$ in $Y$. $U$ can be written as a union of points in and outside
    $f(V)\subset Y$: $U=U_i\cup U_o$. We want to show that $g^{-1}(U)$ is open in $V$. Since $g^{-1}(U_o)=\varnothing$, which is open,
    and $g^{-1}(U_i)\subset V$ and is open in $X$ by the continuity of $f$, $g^{-1}(U_o\cup U_i)=g^{-1}(U_o)\cup g^{-1}(U_i)$ is open in $V$.

    To prove the fourth statement, it suffices to show that $(g\circ f)^{-1}(U)$, with $U\subset Z$ open, is open in $X$.
    First note that $(g\circ f)^{-1}(U)=f^{-1}(g^{-1}(U))$. Since $g$ is continuous, $g^{-1}(U)$ is an open set in $Y$.
    Similarly, $f^{-1}$ of an open set in $Y$ is open in $X$ as $f$ is continuous, and we are done.
\end{proof}

\begin{lem}[Local Criterion for Continuity]
    A map $f:X\ra Y$ between topological spaces is continuous if and only if each point of $X$ has a neighborhood on which
    (the restriction of) $f$ is continuous.
\end{lem}
\begin{proof}
    If $f$ is continuous, each point of $X$ a neighborhood on which $f$ is continuous; namely, $X$ itself.
    
    To prove the converse, suppose that each point of $X$ has a neighborhood on which $f$ is continuous - we wish to show that for any
    open set $U\subset Y,$ $f^{-1}(U)$ is open in $X$. By continuity at each point, we know that any point $x\in f^{-1}(U)$ has a neighborhood
    $V_x$ on which $f$ is continuous. In other words, $(f|_{V_x})^{-1}(U)=f^{-1}(U)\cap V_x$ is open in $X$ and is contained in $f^{-1}(U)$.
    As $f^{-1}(U)$ is the union of such sets for all $V_x$, and these sets are open, it follows that $f^{-1}(U)$ is open, and we are done.
\end{proof}

\begin{defn}
    If $X$ and $Y$ are topological spaces, a \textbf{homeomorphism} from $X$ to $Y$ is defined to be a continuous bijective map
    $\phi:X\ra Y$ with continuous inverse. If there exists a homeomorphism betwee $X$ and $Y$, we say that $X$ and $Y$ are \textbf{homeomorphic}
    or \textbf{topologically equivalent}. Sometimes this is abbreviated $X\approx Y$.
\end{defn}

\begin{exc}
    Show that homeomorphisms are an equivalence relation.
\end{exc}
\begin{proof}
    To show that homeomorphisms are an equivalence relation, we show
    \begin{itemize}
        \item $X\approx X$: The identity map $\id$ is a homeomorphism from $X$ to $X$.
        \item $X\approx Y \implies Y\approx X$: There exists a homeomorphism from $X$ to $Y$. Its inverse is clearly
            a homeomorphism from $Y$ to $X$.
        \item $X\approx Y \nm{and} Y\approx Z \implies X\approx Z$: As the composition of two homeomorphisms is also a homeomorphism
            (from elementary set theory), the homeomorphism from $X$ to $Z$ is simply the composition of the homeomorphisms
            from $Y$ to $Z$ and from $X$ to $Y$, respectively.
    \end{itemize}
\end{proof}

\begin{exmp}
    Any open ball in $\mathbb{R}^n$ is homeomorphic to any other open ball. The homeomorphism can be constructed simply by
    composition translations $x\mapsto x+x_0$ and dilations $x\mapsto cx$. This shows that size is not a topological property.
\end{exmp}

\begin{exmp}
    If $\mathbb{B}^n$ is the open unit ball, we can define $F:\mathbb{B}^n\ra \mathbb{R}^n$ by
    \[y=F(x)=\frac{x}{1-|x|^2}.\]
    The inverse is given by
    \[x=F^{-1}(y)=\frac{2y}{1+\sqrt{1+4|y|^2}}.\]
    As both are continuous and bijective, $F$ is a homeomorphism, so $\mathbb{R}^n$ is homeomorphic to $\mathbb{B}^n$. This
    shows that boundedness is not a topological property.
\end{exmp}

\begin{exmp}
    Take the surface of the unit sphere in $\mathbb{R}^3$ and the surface of the cube of side 2, centered at the origin. There exists
    a homeomorphism between these two surfaces,
    \[\phi(x,y,z)=\frac{(x,y,z)}{\sqrt{x^2+y^2+z^2}}\]
    whose inverse is given by
    \[\phi^{-1}(x,y,z)=\frac{(x,y,z)}{\max(|x|,|y|,|z|)}.\]
    Thus, corners are not a topological property either.
\end{exmp}

\begin{exmp}
    Now for an example of a continuous bijection that is not a homeomorphism by failure of its inverse to be continuous.
    Let $X$ be the interval $[0,1)\subset \mathbb{R}$, and let $\mathbb{S}^1$ denote the unit circle in $\mathbb{R}^2$
    (both with the Euclidean metric topologies). Define a map $a:X\ra \mathbb{S}^1$ by $a(t)=(\cos 2\pi t, \sin 2\pi t)$.
    It is clear that the map is continuous and bijective. The inverse, however, is not continuous. To see why, take any
    neighborhood of the point $(1,0)$ - the inverse of $a$ on this neighborhood will inevitably contain the point $0\in[0,1).$
    Thus the preimage of open sets is not necessarily open, and thus $a^{-1}$ is not continuous.
\end{exmp}

\begin{defn}
    A map $f:X\ra Y$ is said to be an \textbf{open map} if for any open set $U\subset X$, the image set $f(U)$ is open in $Y$.
    A map can be open but not continuous, continuous but not open, both, or neither.
\end{defn}

\begin{defn}
    We say that a continuous map $f:X\ra Y$ between topological spaces is a \textbf{local homeomorphism} if every point
    $x\in X$ has a neighborhood $U\subset X$ such that $f(U)$ is an open subset of $Y$ and $f|_U:U\ra f(U)$ is a homeomorphism.
\end{defn}

\begin{exc}
    Show that:
    \begin{enumerate}
        \item every local homeomorphism is an open map.
        \item every homeomorphism is a local homeomorphism.
        \item a bijective continuous open map is a homeomorphism.
        \item a bijective local homeomorphism is a homeomorphism.
    \end{enumerate}
\end{exc}
\begin{proof}
    \hspace{1mm} 
    \begin{enumerate}
        \item Take any open set $U\in X$. For every $x\in U$, there exists some neighborhood $V=U_x\cap U$
            of $x$ for which the local homeomorphism $f(V)$ is open. Note that $U$ is the union of all such $V$ (for various $x$)
            and thus $f(U)$ is the union of all $f(V)$, and as the union of open sets must be an open set, $f(U)$ must be open.
            Consequently, $f$ is an open map.
        \item Take $f$ to be a homeomorphism from $X$ to $Y$. $f$ is trivially a local homeomorphism; the neighborhood needed by
            the definition is simply $X$ itself.
        \item Let $f$ be a bijective, continuous, open map. To show that $f$ is a homeomorphism, we must show that $f^{-1}$
            is a continuous map. In other words, we must show that the preimage of $f$ on open sets of $Y$ are open
            sets in $X$. This is true as $f$ is open and bijective: open sets in $X$ are taken to open sets in $Y$ (open), and
            all open sets in $Y$ are images of open sets in $X$ (bijective).
        \item Let us first show that any local homeomorphism $f$ from $X$ to $Y$ is continuous. Take any open set $V\subset Y$
            whose preimage under $f$ we call $U$. We wish to show that $U$ is open. Take any point $y\in V$.
            For any $x\in f^{-1}(y)$, there is a neighborhood $M_x\subset X$ of $x$ that is homeomorphic to a
            neighborhood $N_y$ of $y$ in $Y$.  This implies that $U_x=M_x\cap U$ is homeomorphic to $V_y=N_y\cap V$.
            Take the union of all the sets $U_x$ for every $x$ in the preimage of $y$ and call this $W_y$.
            Note that the preimage of $V$ is the union of all such $W_y$ for $y\in V$. This union is open, and we are done.\\
            Now it remains to show that the inverse of $f$ (that exists via bijectivity) is continuous.
            Let $U\subset X$ be open and $V=(f^{-1})^{-1}(U)=f(U)$.  We wish to show that $V$ is open.
            This follows trivially from the fact that $f$ is an open map, and we are done.
    \end{enumerate}
\end{proof}

\pagebreak

Up until now, we have worked with topological spaces defined through open sets. There is a complementary notion that is just
as important.

\begin{defn}
    A subset $F$ of a topological space $X$ is said to be \textbf{closed} if its complement $X\setminus F$ is open.
    It follows immediately from the definition of topological spaces that
    \begin{enumerate}
        \item $X$ and $\varnothing$ are closed.
        \item Finite unions of closed sets are closed.
        \item Arbitrary intersections of closed sets are closed.
    \end{enumerate}
    A topology on a set $X$ can be defined by describing the collection of closed sets, as long as they satisfy these three
    properties; the open sets are then just those sets whose complements are closed.
\end{defn}

\begin{exmp}[Closed Sets]
    \hspace{1mm}
    \begin{itemize}
        \item Any closed interval $[a, b]\subset \mathbb{R}$ is a closed set, as are the half-infinite closed intervals
            $[a,\infty)$ and $(-\infty, b]$.
        \item Any closed ball in a metric space is a closed set.
        \item Every subset of a discrete space is closed.
    \end{itemize}
\end{exmp}

It is important to not that closed is \textit{not} the same as ``not open,'' as sets can be both open and closed
(such as $X, \varnothing$), or neither open nor closed, such as $[0,1)\subset\mathbb{R}$.

\begin{lem}
    A map between topological spaces is continuous if and only if the inverse image of every closed set is closed.
\end{lem}
\begin{proof}
    Assume $f:X\ra Y$ is continuous. Given $V\subset Y$ is closed, and its preimage under $f$ is $U\subset X$, we want to show
    that $U$ is closed. By definition, $Y\setminus V$ is open, and as by continuity of $f$, we have that its preimage $X\setminus U$
    is open. Thus, again by definition of closed sets, $U$ must be closed.

    Now assume that $f:X\ra Y$ is such that the preimage of any closed set $Y$ is a closed set in $X$. Take an open set $V\subset Y$ whose
    preimage under $f$ is $U\subset X$. The preimage of $Y\setminus V$ is, of course $X\setminus U$, which are both closed. Since
    $X\setminus U$ is closed, $U$ must be open, and we are done.
\end{proof}

\begin{defn}
    Given any set $A\subset X$, we define several related sets as follows. The \textbf{closure} of $A$ in $X$, denoted by $\bar{A}$,
    is the set
    \[\bar{A}=\bigcap\left\{ B\subset X:B\supset A \nm{and} B \nm{is closed in} X \right\}.\]
    The \textbf{interior} of $A$, written $\Int A$ is
    \[\Int A=\bigcup \left\{ C\subset X:C\subset A \nm{and} C \nm{is open in} X \right\}.\]
    It is obvious that $\bar{A}$ is closed and $\Int A$ is open. In words, $\bar{A}$ is the ``smallest closed set containing $A$,''
    and $\Int A$ is ``the largest open set contained in $A$.''
    We also define the \textbf{exterior} of $A$, written $\Ext A$, as
    \[\Ext A=X\setminus \bar{A},\]
    and the \textbf{boundary} of $A$, written $\partial A$, as
    \[\partial A=X\setminus\left( \Int A \cup \Ext A \right)\]
\end{defn}

It follows from the definitions that for any subset $A\subset X$, the whole space $X$ is equal to the disjoint union of $\Int A, \Ext A,$
and $\partial A$. The set $A$ always contains all of its interior points and none of its exterior points, and may contain all, some, or none
of its boundary points.

\begin{lem}
    Let $X$ be a topological space and $A\subset X$ any subset.
    \begin{enumerate}
        \item A point $q$ is in the interior of $A$ if and only if $q$ has a neighborhood contained in $A$.
        \item A point $q$ is in the exterior of $A$ if and only if $q$ has a neighborhood contained in $X\setminus A$.
        \item A point $q$ is in the boundary of $A$ if and only if $q$ every neighborhood of $q$ contains both a point of
            $A$ and a point of $X\setminus A$.
        \item $\Int A$ and $\Ext A$ are open in $X$, while $\partial A$ is closed in $X$.
        \item $A$ is open if and only if $A=\Int A$.
        \item $A$ is closed if and only if it contains all its boundary points, which is true if and only if $A=\Int U\cup\partial A$.
        \item $A=\bar{A}\cup\partial A=\Int A\cup\partial A$.
    \end{enumerate}
\end{lem}
\begin{proof}
    Left to the reader.
\end{proof}

\begin{defn}
    Given a topological space $X$ and a set $A\subset X$, we say that a point $q\in X$ is a \textbf{limit point} of $A$ if every neighborhood
    of $q$ contains a point of $A$ other than $q$ (which may or may not itself be in $A$). If we let $X=\mathbb{R}$ and $A=\left\{ 1/n \right\}_{n=1}^\infty$,
    for example, then 0 is the only limit point of $A$.
\end{defn}

\begin{exc}
    Show that a set $A$ in a topological space is closed if and only if contains all of its limit points.
\end{exc}
\begin{proof}
    It is clear from the previous lemma that every boundary point is also a limit point. In addition, it is clear that no limit point
    can also be outside of $A$.  If $A$ is closed, it must contain its boundary, and thus, since it contains its interior and its
    boundary, it contains all of its limit points. Conversely, if $A$ contains all of its limit points, it contains its boundary,
    and is closed.
\end{proof}

\begin{defn}
    A subset $A$ of a topological space $X$ is said to be \textbf{dense} in $X$ if $\bar{A}=X$.
\end{defn}

\begin{exc}
    Show that a subset $A\subset X$ is dense if and only if every nonempty open set in $X$ contains a point of $A$.
\end{exc}
\begin{proof}
    Suppose every nonempty open set in $X$ contains a point of $A$. Every neighborhood of every point of $X$ must then contain a point
    of $A$, and thus every point in $X$ must be a limit point of $A$. $\bar{A}=\Int{A}\cup\partial A$ contains all of its limit points,
    and thus $X=\bar{A}$ and $A$ is dense in $X$.

    Conversely, suppose that $A$ is dense in $X$. Then, $\bar{A}=X$, i.e. $X$ is equal to the union of the interior and the boundary
    of $A$. Any point in the interior or the boundary of a set has the property that all of it neighborhoods contain a point in the interior.
    Thus, since any open set in $X$ is a neighborhood of any point in $X$, any open set in $X$ must contain a point of $A$.
\end{proof}

\begin{defn}
    A map $f:X\ra Y$ is said to be \textbf{closed} if it takes closed sets in $X$ to closed sets in $Y$.
\end{defn}

\begin{exc}
    Show that a bijective continuous map is a homeomorphism if and only if it is open if and only if it is closed.
\end{exc}
\begin{proof}
    Suppose $f$ is a bijective continuous open map. We want to show that the inverse $f^{-1}$ is continuous: the preimage of open sets in $X$ under
    $f^{-1}$ (i.e. the image of $f$) must be open. Since $f$ is open, we are done. Conversely, if $f$ is a homeomorphism, $f$ must be bijective
    and continuous. It remains to show that $f$ is an open map. This follows from the fact that $f^{-1}$ is continuous, similar to above.

    Suppose $f$ is a bijective continuous closed map. We want to show that the inverse $f^{-1}$ is continuous: the preimage of closed sets in $X$ under
    $f^{-1}$ (i.e. the image of $f$) must be closed (by the definition of continuity in terms of closed sets). Since $f$ is closed, we are done.
    Conversely, if $f$ is a homeomorphism, $f$ must be bijective and continuous. It remains to show that $f$ is an closed map. This follows from
    the fact that $f^{-1}$ is continuous, similar to above, using the definition in terms of closed sets.
\end{proof}


\begin{defn}
    Suppose $X$ is any set. A \textbf{basis} in $X$ is a collection $\mathcal{B}$ of subsets of $X$ satisfying the following conditions:
    \begin{enumerate}
        \item Every element of $X$ is in some element of $\mathcal{B}$; in other words, $X=\bigcup_{B\in \mathcal{B}}B$.
        \item If $B_1, B_2\in\mathcal{B}$ and $x\in B_1\cap B_2$, there exists an element $B_3\in \mathcal{B}$ such that
            $x\in B_3\subset B_1\cap B_2$.
    \end{enumerate}
\end{defn}

\begin{thm}
    Let $\mathcal{B}$ be a basis in a set $X$ and let $\mathcal{T}$ be the collection of all unions of elements of $\mathcal{B}$.
    Then $\mathcal{T}$ is a topology on $X$. This topology $\mathcal{T}$ is called the \textbf{topology generated by $\mathcal{B}$}.
\end{thm}

Before we prove this theorem, let us develop a parallel definition to that of an open set: given $X$ and a collection $\mathcal{B}$ of subsets of $X$,
we say that a subset $U\subset X$ satisfies the \textbf{basis criterion} with respect to $\mathcal{B}$ if for every $x\in U$,
there exists $B\in\mathcal{B}$ such that $x\in B\subset U.$

\begin{lem}
    Suppose $\mathcal{B}$ is a basis in $X$. Then $\mathcal{T}$, defined as above, is precisely the set of all subsets of $X$ that
    satisfy the basis criterion with respect to $\mathcal{B}$.
\end{lem}
\begin{proof}
    Let $U\subset X$, and suppose first that $U$ satisfies the basis criterion. Let
    \[V=\bigcup\left\{ B\in\mathcal{B}:B\subset U \right\}.\]
    $V\in\mathcal{T}$ as it is a union of basis sets. If we can show that $U=V$, we will have $U\in\mathcal{T}$ and we will be done.
    Clearly, $V\subset U$, as $V$ is a union of subsets of $U$. We want to show that $U\subset V$. For any point $x\in U$, since $U$
    satifsfies the basis criterion, there must exist a basis set $B\in\mathcal{B}$ such that $x\in B\subset U$. It follows that $x\in V$,
    and we are done.

    Conversely, suppose that $U\in\mathcal{T}$. Consequently, $U$ is a union of elements of $\mathcal{B}$. $U$ satisfies the basis
    criterion, as each $x\in U$ satisfies $x\in B\subset U$ for some $B\in\mathcal{B}$.
\end{proof}

\begin{proof}
    We now prove the earlier theorem. We want to show that the collection $\mathcal{T}$ satisfies the conditions for a topology.
    Since $X=\bigcup_{B\in\mathcal{B}}B$, $X\in\mathcal{T}$. The empty set is as well, as it is the ``union of no elements'' of
    $\mathcal{B}$. A union of elements of $\mathcal{T}$ is a union of unions of elements of $\mathcal{B}$, and therefore is
    a union of elements of $\mathcal{B}$, and thus $\mathcal{T}$ is closed uner arbitrary unions. To show that $\mathcal{T}$ is closed
    under finite intersections, suppose first that $U_1, U_2\in\mathcal{T}$. Then, for any $x\in U_1\cap U_2$, the basis criterion
    says that there exist basis elements $B_1, B_2\in\mathcal{B}$ such that $x\in B_1\subset U_1$ and $x\in B_2\subset U_2$.
    By the definition of the basis, however, we know that there exists $B_3\in\mathcal{B}$ such that
    $x\in B_3\subset B_1\cap B_2\subset U_1 \cap U_2$. Thus $U_1\cap U_2$ satisfies the basis criterion, so it is again in $\mathcal{T}$.
    This shows that $\mathcal{T}$ is closed under pairwise intersections, and closure under finite intersections follows via induction.
\end{proof}

\begin{lem}
    Suppose $X$ is a topological space, and $\mathcal{B}$ is a collections of open subsets of $X$. If every open subset of $X$ satisfies
    the basis criterion with respect to $\mathcal{B}$, then $\mathcal{B}$ is a basis for the topology of $X$.
\end{lem}

\begin{proof}
    If every open subset satisfies the basis criterion, the previous lemma tells us that the collection $\mathcal{T}$ is the collection
    of all open subsets of $X$, which does indeed form a topology. All that remains is to show that $\mathcal{B}$ is, in fact, a basis.
    The first requirement is that every point in $X$ must be in a basis set. As $X$ itself is an open set, and thus a basis set, every
    point is indeed in a basis set. The second requirement asserts that given basis (open) sets $B_1$ and $B_2$ and $x\in B_1\cap B_2$,
    $x$ must be in $B_3\subset B_1\cap B_2$. This follows from the definition of topology, which requires the intersection of two open
    sets to be an open set.
\end{proof}

\begin{exc}
    In each of the following cases, prove that the given set $\mathcal{B}$ is a basis for the given topology.
    \begin{itemize}
        \item $M$ is a metric space with the metric topology, and $\mathcal{B}$ is the collection of all open balls in $M$.
        \item $X$ is a set with the discrete topology, and $\mathcal{B}$ is the collection of all one-point subsets of $X$.
        \item $X$ is a set with the trivial topology, and $\mathcal{B}=\left\{ X \right\}$.
    \end{itemize}
\end{exc}

\begin{proof}
    By the previous lemma, for each case, we must show that the every open subset of the topology satisfies the basis criterion with
    respect to $\mathcal{B}$.
    \begin{itemize}
        \item We wish to show that for each point in any open set $U$ in the metric space $M$, there exists an open ball in $U$
            that contains the point. Any ball with radius small enough such that $B\subset U$ will do the trick, and we are done.
        \item The collection of open sets forming the discrete topology $X$ is the power set $\mathcal{P}(X)$. It is clear that for every
            such open set $U$, for all $x\in U$, there is a basis set in $U$ containing $x$: namely, $x$'s one-point basis set.
        \item The collection of open sets forming the trivial topology $X$ is simply $\left\{ X,\varnothing \right\}$. The given $\mathcal{B}$
            has one basis set $X$. Obviously for each $x\in X$, said basis set contains $x$ and is a subset of $X$. The same holds
            vacuously for the null set.
    \end{itemize}
\end{proof}


\begin{lem}
    Let $X$ and $Y$ be topological spaces and let $\mathcal{B}$ be a basis for $Y$. A map $f:X\ra Y$ is continuous if and only if for every
    basis open set $B\in\mathcal{B}$, $f^{-1}(B)$ is open in $X$.
\end{lem}

\begin{proof}
    If $f$ is continuous, by definition, the preimage of every basis open set is open in $X$. Conversely, suppose $f^{-1}(B)$ is open for every
    $B\in\mathcal{B}$. If $V$ is an open set in $Y$, and $x\in U=f^{-1}(V)$, by the basis criterion, we know that there exists a basis set $B$
    such that $f(x)\in B\subset V$. Thus, $x\in f^{-1}(B)\subset U$, so every $x$ has a neighborhood contained in $U$. The union of all such
    neighborhoods, then, is $U$, and consequently, $U$ is open.
\end{proof}

\begin{defn}
    Let $X$ be a topological space. A topological space $M$ is said to be \textbf{locally Euclidean of dimension $n$} if every point $q\in M$
    has a neighborhood that is homeomorphic to an open subset of $\mathbb{R}^n$. Such a neighborhood is called a \textbf{Euclidean neighborhood}
    of $q$.
\end{defn}

The following lemma shows that the ``open subset'' in the definition can be replaced by open ball, or $\mathbb{R}^n$.

\begin{lem}
    A topological space $M$ is locally Euclidean of dimension $n$ if and only if either of the following properties holds:
    \begin{itemize}
        \item Every point of $M$ has a neighborhood homeomorphic to an open ball in $\mathbb{R}^n$.
        \item Every point of $M$ has a neighborhood homeomorphic to $\mathbb{R}^n$.
    \end{itemize}
\end{lem}

\begin{proof}
    As $\mathbb{R}^n$ is a Euclidean space, it immediately follows that if $M$ is locally Euclidean of dimension $n$, one of the
    above two conditions must hold. For the converse proof, first note that the above two conditions are equivalent, as we
    showed earlier that any open ball in $\mathbb{R}^n$ is homeomorphic to $\mathbb{R}^n$. Thus, we need only prove the first
    condition.

    Take any point $q\in M$ and let $U$ be a neighborhood of $q$ that admits a homeomorphism $\phi: U\ra V$, where $V$ is an open
    subset of $\mathbb{R}^n$. Since $V$ is open, there must be some open ball $B$ around $\phi(q)$ that is contained in $V$.
    Therefore, $\phi^{-1}(B)$ is a neighborhood of $q$ homeomorphic to an open ball in $\mathbb{R}^n$, and we are done.
\end{proof}

\begin{defn}
    If $M$ is locally Euclidean of dimension $n$, a homeomorphism from an open subset $U\subset M$ to an open subset of $\mathbb{R}^n$
    is called a \textbf{coordinate chart} on $U$. We will call any open subset of $M$ that is homeomorphic to a ball in $\mathbb{R}^n$
    a \textbf{Euclidean ball} in $M$. The previous lemma shows that every point in a locally Euclidean space has a Euclidean ball
    neighborhood.
\end{defn}

Note that the definition of locally Euclidean spaces makes sense even if $n=0$. Since $\mathbb{R}^0$ is by convention a single point,
the second condition of the previous lemma (that $M$ is locally homeomorphic to the whole $\mathbb{R}^n$) implies that a space
can be locally Euclidean of dimension 0 if and only if each point has a neighborhood that is homeomorphic to a one-point space. In
other words: if and only if the space is discrete.

\begin{defn}
    A topological space $X$ is said to be a \textbf{Hausdorff space} if given any pair of distinct points $q_1, q_2\in X$, there
    exist neighborhoods $U_1$ of $q_1$ and $U_2$ of $q_2$ with $U_1\cap U_2=\varnothing$.
\end{defn}

Note that any open subset of a Hausdorff space is Hausdorff.

\begin{lem}
    Let $X$ be a Hausdorff space.
    \begin{enumerate}
        \item Every one-point set in $X$ is closed.
        \item If a sequence $\left\{ x_i \right\}$ in $X$ converges to a limit $x\in X$, the limit is unique.
    \end{enumerate}
\end{lem}

\begin{proof}
    For the first part, take any one-point set $\left\{ q \right\}\in X$. For any $p\neq q$, we are assured that there exist
    disjoint neighborhoods $U_p$ of $q$ and $V_p$ of $p$. The complement of the one-point set is $X\setminus \left\{q\right\}$, which can be
    expressed as the union of the open sets $V_p$ for every $p\in X\setminus\left\{ q \right\}$.

    To prove that the limits are unique, first assume for the sake of contradiction that the sequence converges to both $x$ and $x'$.
    By the Hausdorff property, there exist disjoint neighborhoods $U$ of $x$ and $U'$ of $x'$. By definition of convergence, there
    exist $N, N'$ such that $i\geq N$ implies $x_i\in U$ and $i\geq N'$ implies $x_i\in U'$. But since $U$ and $U'$ must be disjoint,
    we reach a contradiction for when $i$ is greater than both $N$ and $N'$.
\end{proof}

\begin{exc}
    Show that the only Hausdorff topology on a finite set is the discrete topology.
\end{exc}

\begin{proof}
    Suppose we have a finite, discrete topology $X$. It is clear that $X$ is Hausdorff, as each point's one-point subset satisfies
    the Hausdorff condition of disjoint subsets.

    Suppose we have a finite, Hausdorff topology $X$. We wish to show that $X$ is the discrete topology. By the Hausdorff property
    and the above lemma, we know that every one-point set in $X$ is closed. Consequently, for each point $q\in X$, the set
    $X\setminus \left\{ q \right\}$ must be open. Label the points in $X$ as $\left\{ x_1\cdots x_n \right\}$ and take the open subset
    $U=X\setminus \left\{ x_1 \right\}=\left\{ x_2\cdots x_n \right\}$. For each point in $U$, $q$, define $ _q=X\setminus \left\{ q \right\}$,
    which are all open. Note that the point $x_1$ is a member of $V_q$ for all $q\in U$. Since $V_q$ are open, it must be that $\bigcap_q V_q$
    is open as well, and contains \textit{only} $x_1$ (by virtue of how $V_q$ was defined in terms of complements; drawing a picture of a
    set with 3 elements is a good way to visualize this). Thus, since $x_1$ was arbitrarily chosen in $X$, every one-point set in $X$ is open;
    clearly, then, $X$ is a discrete topology.
\end{proof}

\begin{defn}
    We say that a topological space is \textbf{second countable} if it admits a countable basis.
\end{defn}

\begin{defn}
    If $X$ is a topological space and $q\in X$, a collection $\mathcal{B}_q$ of neighborhoods of $q$ is called a \textbf{neighborhood basis}
    at $q$ if every neighborhood of $q$ contains some $B\in \mathcal{B}_q$. $X$ is said to be \textbf{first countable} if there exists a 
    countable neighborhood basis at each point.
\end{defn}

\begin{cor}
    Second countability implies first countability.
\end{cor}

\begin{proof}
    Second countability states that there exists a countable basis for $X$. The collection of basis open sets containing any point $q\in X$
    is, of course, a countable neighborhood basis for $q$.
\end{proof}

\begin{defn}
    If $X$ is any topological space, a collection $\mathcal{U}$ of subsets of $X$ is said to \textbf{cover} $X$, or be a cover of $X$, if every point
    in $X$ is in one of the sets of $\mathcal{U}$. An \textbf{open cover} is a collection of open sets that covers $X$. Given any cover $\mathcal{U}$, a
    \textbf{subcover} of $\mathcal{U}$ is a subset of $\mathcal{U}$ that is still a cover.
\end{defn}

\begin{lem}
    If $X$ is a second countable space, every open cover of $X$ has a countable subcover.
\end{lem}

\begin{proof}
    Let $\mathcal{B}$ be a countable basis for $X$, and let $\mathcal{U}$ be an arbitrary open cover of $X$. Let $\mathcal{B}'$ denote the subset
    of $\mathcal{B}$ consisting of those basis sets that are entirly contained in some element of $\mathcal{U}$. As a subset of a countable set
    is countable, $\mathcal{B}'$ must be a countable set. For each element $B\in \mathcal{B}'$, choose an element $U_b\in \mathcal{U}$ such that
    $B\subset U_B$. The collection $\left\{ U_B:B\in\mathcal{B}' \right\}$ is a countable subset of $\mathcal{U}$. Now we want to show that it
    covers $X$, and we will be done.

    Choose $x\in X$ arbitrary. Then $x\in U_0$ for some open $U_0\in\mathcal{U}$, as $\mathcal{U}$ is an open cover. By the basis criterion, we
    know that there is some $B\in\mathcal{B}$ such that $x\in B\subset U_0$. Thus $B\in\mathcal{B}'$, and there exists a set $U_B\in\mathcal{U}'$
    such that $x\in B\subset U_B$. This shows that $\mathcal{U}'$ is a cover.
\end{proof}

Any open subset $U$ of a second countable space $X$ is second countable: starting with a countable basis for $X$, simply throw away all the
elements of the basis that do not lie in $U$; then it is easy to check that the remaining basis sets form a countable basis for the topology
of $U$.

\begin{defn}
    An \textbf{$n$-dimensional topological manifold} is a second countable Hausdorff space that is locally Euclidean of dimension $n$.
\end{defn}

The most obvious example of an $n$-manifold is $\mathbb{R}^n$ itself. More generally, and open subset of $\mathbb{R}^n$ - or in fact of
any $n$-manifold - is again an $n$-manifold, as the next lemma shows.

\begin{lem}
    Any open subset of an $n$-manifold is an $n$-manifold.
\end{lem}

\begin{proof}
    Let $M$ be an $n$-manifold, and let $V$ be an open subset of $M$. Any $q\in V$ has a neighborhood (in $M$) that is homeomorphic to an
    open subset of $\mathbb{R}^n$; the intersection of that neighborhood with $V$ is still open, still homeomorphic to an open subset of $\mathbb{R}^n$,
    and lies in $V$, so $V$ is locally Euclidean. Since any open subset of a Hausdorff space is Hausdorff and any open subset of a second countable
    space is second countable, $V$ is an $n$-manifold.
\end{proof}

\begin{defn}
    An $n$-dimensional \textbf{manifold with boundary} is a second countable Hausdorff space in which every point has a neighborhood
    homeomorphic to an open subset of the $n$-dimensional upper half space
    $\mathbb{H}^n=\left\{ \left( x_1,\cdots,x_n \right)\in\mathbb{R}^n:x_n\geq 0\right\}$. Just as in the case of manifolds, we will
    call any homeomorphism from an open subset $U$ of $M$ to an open subset of $\mathbb{H}^n$ a \textbf{chart} on $U$.
\end{defn}

The upper half space $\mathbb{H}^n$ is of course a manifold with boundary, as is any closed interval in $\mathbb{R}$, any closed
disk in $\mathbb{R}^2$, or in fact a closed ball in any Euclidean space.

\begin{defn}
    The boundary of $\mathbb{H}^n$ in $\mathbb{R}^n$ is the set of points where $x_n=0.$ If $M$ is a manifold with boundary, a point that is
    in the inverse image $\partial\mathbb{H}^n$ under some chart is called a \textbf{boundary point} of $M$, and a point that is in the
    inverse image of $\Int \mathbb{H}^n$ is called an \textbf{interior point}. The \textbf{boundary} of $M$ (the set of all of its boundary
    points) is denoted by $\partial M$; similarly, its \textbf{interior} is denoted by $\Int M$.
\end{defn}

Since any open ball in $\mathbb{R}^n$ is homeomorphic to an open subset of $\mathbb{H}^n$, an $n$-manifold is automatically an $n$-manifold with
boundary (with empty boundary), but the converse is not true: a manifold with boundary is a manifold if and only if its boundary is empty.






































\end{document}
