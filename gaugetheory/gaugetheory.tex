\documentclass{book}
\usepackage[utf8]{inputenc}

\usepackage{todonotes}
\usepackage{tikz-cd}

\usepackage{amsmath}
\usepackage{amssymb}
\usepackage{amsthm}
\usepackage{enumerate}

\newcommand{\nm}[1]{\;\textnormal{#1}\;}
\newcommand{\ra}[0]{\rightarrow}
\newcommand{\fa}[0]{\;\forall}
\newcommand{\R}{\mathbb{R}}
\newcommand{\Q}{\mathbb{Q}}
\newcommand{\Z}{\mathbb{Z}}
\newcommand{\F}{\mathbb{F}}
\newcommand{\C}{\mathbb{C}}
\newcommand{\CP}{\mathbb{C}\mathbb{P}}
\newcommand{\RP}{\mathbb{R}\mathbb{P}}
\newcommand{\Proj}{\mathbb{P}}
\newcommand{\N}{\mathbb{N}}
\newcommand{\HH}{\mathbb{H}}
\newcommand{\p}{\partial}
\newcommand{\fr}{\mathfrak}


\DeclareMathOperator{\Hom}{Hom}
\DeclareMathOperator{\length}{length}
\DeclareMathOperator{\res}{Res}
\DeclareMathOperator{\Int}{Int}
\DeclareMathOperator{\Ext}{Ext}
\DeclareMathOperator{\Aut}{Aut}
\DeclareMathOperator{\Gal}{Gal}
\DeclareMathOperator{\Sym}{Sym}
\DeclareMathOperator{\Lie}{Lie}
\DeclareMathOperator{\Stab}{Stab}
\DeclareMathOperator{\id}{Id}
\DeclareMathOperator{\tr}{tr}
\DeclareMathOperator{\irr}{irr}
\DeclareMathOperator{\supp}{supp}
\DeclareMathOperator{\sgn}{sgn}
\DeclareMathOperator{\GL}{GL}
\DeclareMathOperator{\SL}{SL}
\DeclareMathOperator{\SO}{SO}
\DeclareMathOperator{\OO}{OO}
\DeclareMathOperator{\ad}{ad}
\DeclareMathOperator{\Ad}{Ad}

\theoremstyle{plain}
\newtheorem{thm}{Theorem}
\newtheorem{lem}[thm]{Lemma}
\newtheorem{cor}[thm]{Corollary}
\newtheorem{prop}[thm]{Proposition}
\newtheorem{exc}{Exercise}
\theoremstyle{definition}
\newtheorem{defn}{Definition}
\newtheorem{exmp}{Example}
\theoremstyle{remark}
\newtheorem*{rmk}{Remark}

\title{An Introduction to Gauge Theory and its Applications}
\author{Matei Ionita and Nilay Kumar}
\date{Fall 2013}

\begin{document}

\maketitle

\tableofcontents

\newpage

\chapter{Introduction}

% Prerequisite knowledge:
% - Working knowledge of differential topology (manifolds, vector bundles, fields and forms), rudiments of geometry (metrics, etc.)
% Basic algebraic topology, i.e. fundamental group
% - Basic familiarity with Lie groups, lie algebras, group actions, and representations.
% - Physics knowledge: Maxwell's equations, quantum mechanics, and basics of quantum field theory

\chapter{Principal Bundles}

\section{Definition and examples}

Let $G$ be a Lie group with identity $e$. A \textbf{left $G$-space} is a smooth manifold $X$ equipped with a smooth left action $G\times X\to X$. 
A \textbf{right $G$-space} is defined similarly.  We will typically denote the action of $G$ on a $G$-space by a ``$\cdot$'', even when there may
be multiple $G$-spaces present, as the particular action should be clear from the context.
If $X$ and $Y$ are two (right) $G$-spaces, a map $\phi:X\to Y$ is said to be \textbf{$G$-equivariant} if the diagram
\[\begin{tikzcd}
        X\arrow[swap]{d}{g}\arrow{r}{\phi}&Y\arrow{d}{g}\\
        X\arrow{r}{\phi}&Y
\end{tikzcd}\]
commutes, i.e. $\phi(x\cdot g)=\phi(x)\cdot g$ in the case of left $G$-spaces.

\begin{defn}
    Let $X$ and $P$ be right $G$-spaces and $\pi: P\to X$ be a smooth $G$-equivariant projection onto $X$.
    We say that $(P,\pi)$ is a \textbf{principal $G$-bundle over the base space} $X$ if the following two conditions hold:
    \begin{enumerate}[(i)]
        \item $G$ acts trivially on $X$, i.e. $x\cdot g=x$ for all $x\in X$ and $g\in G$;
        \item $X$ has an open cover $\{U_\alpha\}_{\alpha\in A}$ for which there exist $G$-equivariant diffeomorphisms $\phi_\alpha:\pi^{-1}(U_\alpha)\to U_\alpha\times G$ called \textbf{local trivializations} such that the diagram
            \[\begin{tikzcd}[column sep=small]
                \pi^{-1}(U_\alpha)\arrow[swap]{dr}{\pi}\arrow{rr}{\phi_\alpha}&&U\times G\arrow{dl}{\pi_\alpha}\\
                &U_\alpha&
            \end{tikzcd}\]
            commutes, where $\pi_\alpha$ is projection onto the first component. Here, $U\times G$ has the structure of a right $G$-space such that $(u,g)\cdot h=(u,gh)$.
    \end{enumerate}
    We will also often denote the bundle $(P,\pi)$ as $G\to P\to X$.
\end{defn}

Note that the first condition above is equivalent to requiring the action of $G$ to preserve the fibers of $\pi$,
\[\pi(p\cdot g)=\pi(p)\cdot g=\pi(p).\]
In fact, we can say more, as the following lemma shows.

\begin{lem}
    Given an $x\in X$ and some $p\in\pi^{-1}(x)$ above $x$, the orbit of $p$ under the action of $G$ is the whole fiber:
    \[\pi^{-1}(x)=\{p\cdot g: g\in G\}=p\cdot G.\]
\end{lem}
\begin{proof}
    The inclusion $p\cdot G\subseteq \pi^{-1}(x)$ holds due to condition (i). To show the reverse inclusion, $\pi^{-1}(x)\subseteq p\cdot G$, it suffices to show that for any $q\in\pi^{-1}(x)$, there exists an $h\in G$ such that $p\cdot h=q$. Using condition (ii) we can take advantage of the equivariant local trivialization and work instead in local coordinates, $\phi(p)=(x, g_1)$ and $\phi(q)=(x,g_2)$, for some $g_1,g_2\in G$. Then $\phi(p)\cdot (g_1^{-1}g_2)=\phi(q)$, which implies (by injectivity of $\phi$) that $p\cdot h=q$ for $h=g_1^{-1}g_2$.
\end{proof}

Hence we may identify the fibers of $\pi$ with the group $G$.
Informally, then, one can think of a principal bundle as a space that is locally just the product of a base manifold $X$ with a Lie group $G$.
Globally, however, there may exist topological complications; consider, for example, the (infinite) M\"obius strip $M$.
Locally $M$ is just $S^1\times \R$, but globally there is some kind of un-productlike twisting. Those familiar with vector bundles
may have found the M\"obius strip a good example to keep in mind - in fact, as we shall show shortly, $M$ can also be
treated as a principal bundle. 

Before we move on to some examples of principal bundles, let us prove the following useful lemma.

\begin{lem}
    $G$ acts freely on $P$.
\end{lem}
\begin{proof}
    Let us compute the stabilizer of some $p\in P$ where $\phi(p)=(x,g)$.
    In coordinates, the requirement that $h\in\Stab_p$ is
    $\phi(p\cdot h)=(x,gh)=(x,g)=\phi(p).$
    This, of course, implies that $h=e$.
\end{proof}

To be more explicit, let us now consider some examples.

\begin{exmp}
    The obvious example of a principal bundle is, of course, a bundle that is globally just a product. In other words, given
    a Lie group $G$ and a smooth manifold $X$, we can construct the \textbf{trivial $G$-bundle over $X$} to be
    $G\to X\times G\overset{\pi}{\to} X$ where $\pi$ is simply the projection onto the first factor and $X\times G$ is treated as
    a $G$-space under the action $(x,g)\cdot h=(x,gh)$.
\end{exmp}

\begin{exmp}
    The reader familiar with projective spaces will recall that the sphere $S^n$ is a double cover of the real projective space $\RP^n$.
    Thus we may consider the natural action of $O(1)=\Z_2=\left\{ 1,-1 \right\}$ (given the discrete topology) on $S^n$ identifying antipodal points,
    i.e. for any $p\in S^n$,
    \[p\cdot (\pm 1)=(x^1,\ldots, x^{n+1})\cdot (\pm 1)=\pm (x^1,\ldots,x^{n+1}).\]
    We can thus construct the principal bundle $O(1)\to S^{n+1}\overset{\pi}{\to} \RP^{n+1}$. Let us check that the required properties are satisfied.
    First note that $O(1)$ does indeed act trivially on the base space, as multiplication by a scalar preserves, by construction, points in $\RP^{n+1}$.
    Next let us construct local trivializations, i.e. $O(1)$-equivariant diffeomorphisms $\phi_\alpha:\pi^{-1}(U_\alpha)\to U_\alpha\times G$ where
    $U_\alpha$ for $\alpha\in\{1,\ldots, n+1\}$ are the usual charts on $\RP^n$ given by the non-vanishing of the $\alpha$th homogeneous coordinate.
    If we now consider the graphical charts for the sphere given by 
    \begin{align*}
        V_\beta^+&=\left\{ (x^1,\ldots,x^{n+1})\in S^n\mid x^\beta>0 \right\}\\
        V_\beta^-&=\left\{ (x^1,\ldots,x^{n+1})\in S^n\mid x^\beta<0 \right\},
    \end{align*}
    we see that
    \[\pi^{-1}(U_\alpha)=V_\alpha^+\cup V_\alpha^-.\]
    Hence we define the map $\phi_\alpha:\pi^{-1}(U_\alpha)\to U_\alpha\times O(1)$ as
    \[\phi_\alpha\left( (x^1,\ldots,x^{n+1}) \right)=\left( [x^1:\ldots:x^{n+1}], \sgn x^{\alpha} \right).\]
    Note first that $\phi_\alpha$ is indeed equivariant,
    \[\phi_\alpha(x\cdot \pm 1)=\phi_\alpha(\pm x)=([x],\sgn \pm x^\alpha)=([x],\pm \sgn x^\alpha)=\phi_\alpha(x)\cdot (\pm 1),\]
    and smooth as the first component is simply a smooth projection and the second component is constant and hence smooth in the disjoint $V_\beta^+$
    and $V_\beta^-$ each. Moreover, $\phi_\alpha$ has an inverse given simply by treating a point in projective space as a point in the sphere
    and multiplying by the given sign. This is certainly smooth - hence $\phi_\alpha$ is indeed a local trivializtion, and $S^n$ forms a principal
    $O(1)$-bundle over $\RP^n$.

    We leave it as an exercise at the end of the chapter to construct similar bundles over the complex and quaternionic projective spaces $\CP^n$ and $\HH\Proj^n$.
\end{exmp}

\begin{exmp}
    Let $G$ be a Lie group, with $H$ a closed subgroup. Let us show that $G$ is a principal $H$-bundle over the coset space $G/H$.
    Let us treat $G$ and $G/H$ as right $H$-spaces with the obvious right-multiplication. Then the action of $H$ on $G/H$ is trivial:
    \[gH\cdot h = gH\]
    for all $g\in G$ and $h\in H$. Showing local triviality is a little trickier. The interested reader may find a proof in \todo{Brocker-Dieck, thm 4.3}.
\end{exmp}

\begin{exmp}
    Let $E\to X$ be a real vector bundle of rank $k$ over a smooth manifold $X$.
    Define a \textbf{frame} at a point $x\in X$, denoted by $F_x$, to be an ordered basis for the fiber above $x$. In the interests of producing a fiber bundle,
    we note that choosing a frame for $\pi^{-1}(x)$ is equivalent to choosing a linear isomorphism from $\R^k$ to $\pi^{-1}(x)$, i.e. choosing an
    element of $\GL(k)$ that takes the usual orthonormal basis of $\R^k$ to the chosen frame. Of course, $F_x$ is equipped with a natural right action by
    $\GL(k)$ (given by right-multiplication), which is free and transitive. In this sense, the set of frames $F_x$ is, when viewed as a Lie group, diffeomorphic to $\GL(k)$.
    \todo{finish frame bundles, see DuPont}
\end{exmp}

Now that we have worked through a few concrete examples of principal bundles, it should be quite obvious that trivial bundles are as pleasant as can be,
while non-trivial bundles are quite the opposite. With this in mind, let us approach the theory of principal bundles by asking exactly how a
given bundle fails to be trivial. This can be done via \textbf{transition functions}, in analogy to those used in the defining smooth manifolds.
In particular, we investigate how the local coordinates of a point in the principal bundle depend on the choice of local trivialization (assuming, of
course, that the point is in a region of overlap $U_\alpha\cap U_\beta$).

\begin{defn}
    Let $P$ be a principal $G$-bundle over $X$ with local trivializations $\Phi_\alpha$ defined over an open cover $X=\{U_\alpha\}$. \todo{finish}
\end{defn}

\todo{tr. func. example}

\todo{reconstruction theorem}


\section{Morphisms and sections}

We may define morphisms between princpal $G$-bundles, thus defining the category of principal $G$-bundles.
\begin{defn}
    Let $G\overset{\pi_1}{\to}P_1\to X_1$ and $G\overset{\pi_2}{\to}P_2\to X_2$ be two principal $G$-bundles. We define a \textbf{bundle
    map} between $P_1$ and $P_2$ to be a smooth $G$-equivariant map $\Phi:P_1\to P_2$:
    \[\Phi(p\cdot g)=\Phi(p)\cdot g,\]
    for all $p\in P_1, g\in G$, with the group action on both $P_1,P_2$ denoted by ``$\cdot$''. Note that bundle maps preserve fibers.
\end{defn}

We will be concerned primarily with maps between bundles with the same base space. This is quite a special case, as the following theorem shows.
\begin{thm}
    A bundle map $f$ between two principal $G$-bundles $P_1$ and $P_2$ over the same base $X$ is an isomorphism.
\end{thm}
\begin{proof}
    Let us first suppose that $P_1,P_2$ are both trivial, i.e. $P_1=P_2=X\times G$. Then, since $f$ must preserve fibers, we see that
    \[f(x,g)=(x,\sigma(x)g)\]
    for some function $\sigma:X\to G$.
    \todo{finish}
\end{proof}
Indeed, this result suggests how strong the condition of being a principal bundle is.


% Bundle maps

Just as with vector bundle, we can construct ``sections'' of a principal bundle, i.e. globally twisted functions on our base manifold.

\begin{defn}
    Let $G\to P\overset{\pi}{\to} X$ be a principal $G$-bundle and $U$ be an open neighborhood of $X$. A \textbf{local section} (sometimes \textbf{cross-section}) of $\pi$ is a continuous map $\sigma: U\to P$ such that $\pi(\sigma(x))$. If $U=X$, we say that $\sigma$ is a \textbf{global section}.
\end{defn}

\begin{exmp}
    Given a local trivialization $\Psi:\pi^{-1}(U)\to U\times G$, we can define a local section $\sigma:U\to P$ as $\sigma(x)=\Psi^{-1}(x,e)$. In words, we simply take the identity section in our local trivialization (that assigns to each point $x$ on the manifold the identity element in the copy of $G$ above $x$) and then pull it back onto the bundle. As the section is continuous in the local trivialization and because $\Psi$ is continuous and takes fiber to fiber, we see that $\sigma$ is indeed a local section. We will refer to this section as the \textbf{canonical section} associated to the trivialization $\Psi$.
\end{exmp}

\begin{thm}
    A principal bundle $G\to P\overset{\pi}{\to} X$ is trivial iff it has a global section.
\end{thm}
\begin{proof}
    Let $\sigma:X\to P$ be a global section. Consider now the trivial bundle $G\to X\times G\overset{\rho}{\to}X$; it suffices to find a bundle map $\Theta: X\times G\to P$ making the diagram 
    \begin{equation*}
        \begin{tikzcd}
            X\times G\arrow{rr}{\Theta}\arrow[swap]{rd}{\rho}&&P\arrow[bend left]{ld}{\pi}\\
            &X\arrow[bend left]{ru}{\sigma}&
        \end{tikzcd}
    \end{equation*}
    commute. Define the map $\Theta:X\times G\to P$ by
    \[\Theta(x,g)=\sigma(x)\cdot g.\]
    This map clearly takes fiber to fiber and is $G$-equivariant, as
    \[\Theta(x,gh)=\sigma(x)\cdot (gh)=(\sigma(x)\cdot g)\cdot g.\]
    Hence $\Theta$ defines a bundle equivalence between $X\times G$ and $P$, and we are done. Intuitively, one thinks of the given global
    section $\sigma$ as a (non-canonical) choice of a reference ``identity'' section.

    Conversely, suppose $P\cong X\times G$. We can easily construct the global identity section $\sigma: X\to X\times G$ given by $\sigma(x)=(x,e)$.
\end{proof}

\section{Exercises}


\chapter{Connections on principal bundles}

\section{Background and definitions}

It is intuitive to think of a principal bundle as having ``vertical''  and ``horizontal'' components, namely the directions along the fiber and the base manifold,
respectively. The theory of connections and curvature that we detail below formalizes these notions. Though their basic idea may seem rather
innocuous, connections as we shall see are surprisingly sophisticated and have a large number of important applications. \todo{put in applications}

Before we discuss connections, however, we will define a few preliminary concepts that the reader may find familiar from Lie theory.
We shall omit the proofs of some of the statements below. For a more thorough treatment, we suggest \todo{Lee}.

\begin{defn}
    Let $G$ be a Lie group. Define the left and right-translation maps as
    \[
        \begin{array}{ccc}
            L_g(h)=gh & \text{and} & R_g(h)=hg.
        \end{array}
    \]
    Both of these maps are diffeomorphisms (multiplication and inversion are smooth) and hence the pushforwards $(L_g)_*$ and $(R_g)_*$
    acting on smooth vector fields are vector space isomorphisms. We say that a vector field $X\in\Gamma(TG)$ is \textbf{left} or \textbf{right-invariant} if
    \[
        \begin{array}{ccc}
            (L_g)_*X|_h=X|_{gh} & \text{or} & (R_g)_*X|_h=X|_{hg},
        \end{array}
    \]
    respectively, for all $g,h\in G$.
    Similarly, we say that a one-form $\omega\in\Gamma(T^*G)$ is \textbf{left} or \textbf{right-invariant} if\todo{is this correct?}
    \[
        \begin{array}{ccc}
            \omega(g)=\left( L_{hg^{-1}} \right)^*\omega(h) & \text{or} & \omega(g)=\left( R_{g^{-1}h} \right)^*\omega(h),
        \end{array}
    \]
    respectively, for all $g,h\in G$.
    %Invariance for one-forms is a little harder to visualize than for vectors fields, so let us justify
    %this definition. Evaluating $\omega$ on a vector $v\in T_gG$,
    %\begin{align*}
    %    \omega(g)(v)=\left( (L_{hg^{-1}})^*\omega(h) \right)(v)=\omega(h)\left( (L_{hg^{-1}})_*v \right)
    %\end{align*}
\end{defn}

\begin{lem}
    Let $G$ be a Lie group and $X,Y$ be left-invariant vector fields on $G$. Then their Lie bracket $[X,Y]$ is also left-invariant.
\end{lem}
\begin{proof}
    This follows straightforwardly from the naturality of the Lie bracket.\todo{refer to Lee, or do this proof}
\end{proof}

\begin{thm}
    The space of left-invariant vector fields of a Lie group $G$ forms a Lie algebra, and is in fact isomorphic (as a Lie algebra) to the Lie algebra
    $\Lie G=\fr g$ of $G$.
\end{thm}
\begin{proof}
    Omitted.\todo{make this into an exercise in the back}
\end{proof}

\todo{Adjoint and adjoint actions}

\begin{defn}
    Let $P$ be a principal $G$-bundle over $M$. Let $p\in P$ be a point in the fiber above $x\in M$. Let $T_pP$ denote the tangent space to $P$ to $p$
    and denote by $V_p$ the tangent space to the fiber $\pi^{-1}(x)$ treated as a subspace of $T_pP$. We call $V_p$ the \textbf{vertical subspace} of
    $T_pP$. A \textbf{connection} $\Gamma$ on $P$ is an assignment of a \textbf{horizontal subspace} $H_p$ to each $p\in P$ such that
    \begin{enumerate}[(i)]
        \item there is a direct sum decomposition $T_pP=V_p\oplus H_p;$
        \item $H_p$ depends smoothly on $p$, as defined below.
        \item $H_{pg}=(R_g)_*H_p$ for every $g\in G$;
    \end{enumerate}
    A tangent vector $X\in T_pP$ is called \textbf{vertical} (resp. \textbf{horizontal}) if it lies in $V_p$ (resp. $H_p$). By 
    $(i)$ we see that any vector can uniquely be decomposed into vertical and horizontal components: $X=X_v+X_h$ with
    $X_v\in V_p$ and $X_h\in H_p$. This definition extends naturally to the idea of vertical and horizontal vector fields.
    Indeed, by saying that $H_p$ depends smoothly on $p$ in $(ii)$, we mean that for any smooth vector field $X\in \Gamma(TP)$, the vertical
    and horizontal components $X_v$ and $X_h$ are smooth as well. Finally, $(iii)$ requires that the choice of horizontal subspace is equivariant along the fiber direction.
\end{defn}

The above definition of a connection is reasonable, as given a vector space without an inner product defined on it, there is no canonical choice
of a horizontal subspace given an arbitrary vertical subspace. If, by chance, we are given a Riemannian metric $\eta$ on $P$, we may choose the horizontal
subspaces according to the orthogonal complement given by $\eta$. Hence the choice of metric induces a choice of connection.
We will return to this case later, for now working in full generality.

Unfortunately, thinking of a connection as a choice of horizontal subspaces,
while intuitive, is rather difficult to compute with. Hence we provide an alternate definition in terms of Lie algebra-valued one-forms, and we
prove that the two definitions are equivalent. Before we do so, let us familiarize ourselves a little bit with vector-valued forms.

\begin{defn}
    Let $V$ be a $d$-dimensional vector space and $M$ a smooth manifold. Denote by $E=M\times V$ the trivial bundle over $M$ and by $\Lambda^pT^*M$ the
    $p$th exterior power of the cotangent bundle of $M$.
    We define a \textbf{$V$-valued $p$-form} on $M$ to be a section of the tensor product bundle of $\Lambda^pT^*M$ and $E$.
    More explicitly, fixing a basis $\left\{ e_1,\ldots,e_d \right\}$ for $V$ (equivalently a global frame for $E$) we can write any $V$-valued $p$ form as:
    \[\mu_x(v_1,\ldots,v_p)=\mu_x^1(v_1,\ldots,v_p)e_1+\ldots+\mu_x^d(v_1,\ldots,v_p)e_d\]
    for all $x\in M$ and $v_i\in T_xM$, where the $\mu^i\in\Gamma(\Lambda^pT^*M)$ are the \textbf{components} of $\mu$. We say that a $V$-valued
    form is smooth if and only if its components are smooth in the usual sense of forms (we leave it as an exercise to the reader to show
    that this definition is independent of basis).\todo{exercise}
    We denote the space of smooth such sections on $M$ by $\Omega^p(M;V).$

    For $p=0$, we define by convention $\Omega^0(M;V)$ to be the space of smooth vector-valued functions on $M$.
    For $p=1$, we get smooth $V$-valued one-forms: if $\mu$ is such a form then for $x\in M$, $\mu_x$
    is a linear transformation from $T_xM$ to $V$, i.e. $\mu_x\in\Hom(T_xM;V).$ In this sense, ordinary one-forms on $M$
    may be seen as $\R$-valued one-forms.
\end{defn}

Just as with usual forms, we may pullback vector-valued forms.
\begin{defn}
    Let $M,N$ be smooth manifolds and $f:M\to N$ a smooth map. If $\mu$ is a $V$-valued $p$-form on $N$ we define the \textbf{pullback of $\mu$ by $f$}
    to be the $V$-valued $p$-form $f^*\mu$ on $M$ defined by
    \[(f^*\mu)_x(v_1,\ldots,v_p)=\mu_{f(x)}(f_*v_1,\ldots,f_*v_p)\]
    with $v_i\in T_xM$, for all $x\in M$. Note that if $\mu=\mu^ie_i$ then\todo{prove this} $f^*\mu=(f^*\mu^i)e_i$; hence if $\mu$ is smooth, so is $f^*\mu$.
\end{defn}

We now introduce the Maurer-Cartan form, a Lie algebra-valued form on a Lie group, which will be useful for redefining connections on principal bundles.
\begin{defn}
    The \textbf{Maurer-Cartan form} $\mu$ on a Lie group $G$ is the unique Lie algebra-valued one-form $\mu \in \Omega^1(G; \fr g)$
    that is left-invariant and acts as the identity map on $T_eG$. Uniqueness is clear: if two such forms exist they agree, by definiton, at
    the identity; subsequently, left-invariance forces the two forms to agree everywhere. We can in fact write out the action of $\mu$
    completely explicitly using just the definition. For some $v\in T_gG$,
    \[\mu(g)(v)=\left( (L_{g^{-1}})^*\mu(e) \right)(v)=(L_{g^{-1}})_{*g}(v).\]
\end{defn}

The following lemma relates a given basis of the Lie algebra to the Maurer-Cartan form.
\begin{lem}
    \label{LMC}
    Fix a basis $\{e^i\}$ for $\fr g=T_eG$ and hence a dual basis $\left\{ \mu^i(e) \right\}$ for $\fr g^*=T_e^*G$.
    By construction, $\mu^i(e)(e_j)=\delta^i_j$. If we denote by $\mu^i$ the left-invariant vector fields generated by $\mu^i(e)$, then 
    the Maurer-Cartan form is written
    \[\mu(g)=\sum_i\mu^i(g)e_i.\]
\end{lem}
\begin{proof}
    Let us start with the right-hand side of the identity and show that it evaluates everywhere to the left-hand side.
    Take any $v\in T_gG$ and denote by $w$ the pushforward $(L_{g^{-1}})_*v$. Then
    \begin{align*}
        \left(\sum_i\mu^ie_i\right)(v)&=\sum_i\mu^i(g)(v)e_i=\sum_i\left((L_{g^{-1}})^*\mu^i(e)\right)(v)e_i\\
        &=\sum_i \mu^i(e)(w)e_i=\sum_i\mu^i(e)( \sum_j w^j e_j )\\
        &=\sum_{i,j}w_j\mu^i(e)(e_j)e_i=\sum_iw^ie_i=w\\
        &=(L_{g^{-1}})_{*}v=\mu(g)(v),
    \end{align*}
    as desired.
\end{proof}

\begin{exmp}
    We now follow [Naber] in computing the Maurer-Cartan form $\mu$ for $G=\GL_n\R$ as an example. 

    We denote by $x^{ij}$ the coordinate functions of $\R^{n^2}$ such that $x^{ij}(g)=g^{ij}$ for $g\in\GL_n\R$. Recall that $\fr{gl}_n\R$ is the space of
    all $n\times n$ matrices. Fix a basis for $\fr {gl}_n\R$ to be $\left\{ \partial/\partial x^{ij}|_{e} \right\}_{i,j=1,\ldots n}$.
    The corresponding dual basis is $\left\{ dx^{ij}(e) \right\}_{i,j=1,\ldots n}$; to find (the components of) $\mu$, we wish to
    find left-invariant $\R$-valued one-forms $\mu^{ij}$ satisfying $\mu^{ij}(e)=dx^{ij}(e)$. Recall that left-invariance of $\mu^{ij}$ requires that
    $\mu^{ij}(g)=\left( L_{g^{-1}} \right)^*(\mu^{ij}(e))$. Let $v\in T_g\GL_n\R$ be the tangent vector to a curve $\gamma:(-\varepsilon,\varepsilon)\to\GL_n\R$,
    i.e. $\gamma'(0)=v$. Then we have
    \begin{align*}
        \mu^{ij}(g)(v)&=\mu^{ij}(e)\left( (L_{g^{-1}})_*v \right)=dx^{ij}(e)\left( (L_{g^{-1}})_*v \right)\\
        &=dx^{ij}(e)\left( (L_{g^{-1}}\circ\gamma)'(0) \right)=dx^{ij}(e)\left(\frac{d}{dt}\bigg|_{t=0}(g^{-1}\gamma(t))  \right)\\
        &=\sum_k\left(g^{-1}\right)^{ik}v^{kj}
    \end{align*}
    Rewriting this as
    \begin{align*}
        \mu^{ij}(g)(v)=\sum_k\left(g^{-1}\right)^{ik}v^{kj}=\sum_k(g^{-1})^{ik}dx^{kj}(g)(v),
    \end{align*}
    we find that
    \[\mu^{ij}(g)=\sum_k (g^{-1})^{ik}dx^{kj}(g).\]
    Using Lemma \ref{LMC} we conclude
    \[\mu(g)=\sum_{i,j=1}^n\mu^{ij}(g)\frac{\partial}{\partial x^{ij}}\bigg|_{e}=\sum_{i,j,k=1}^nx^{ik}(g^{-1})dx^{kj}(g)\frac{\partial}{\partial x^{ij}}\bigg|_{e}\]
    This is often rather slickly abbreviated as $\mu(g)=g^{-1}dg$. Let us now apply this to the simple case of $\GL_2\R$. Writing
    \[g=\begin{pmatrix}
            a&b\\c&d
        \end{pmatrix}\text{ and }g^{-1}=\frac{1}{ad-bc}\begin{pmatrix}
            d&-b\\-c&a
        \end{pmatrix},
    \]
    we can express the Maurer-Cartan form as
    \begin{align*}
        \mu(g)&=g^{-1}dg=\frac{1}{ad-bc}\begin{pmatrix}
            d&-b\\-c&a
        \end{pmatrix}\begin{pmatrix}
            dx^{11}&dx^{12}\\dx^{21}&dx^{22}
        \end{pmatrix}\\
        &=\frac{1}{ad-bc}\begin{pmatrix}
            d\;dx^{11}-b\;dx^{21}&d\;dx^{12}-b\;dx^{22}\\-c\;dx^{11}+a\;dx^{21}&-c\;dx^{12}+a\;dx^{22}.
        \end{pmatrix}
    \end{align*}
\end{exmp}

\begin{defn}
    Let $G$ be a Lie group and $\fr g$ be its Lie algebra. We say that a $\fr g$-valued one-form $\mu$ is right equivariant if
    \[(R_g)^*\mu=\ad(g^{-1})\circ \mu\]
    for all $g\in G$. More generally, let $P$ be a manifold equipped with smooth right action $\sigma(p,g)=p\cdot g$ of $G$ and let $\mu$ be a $\fr g$-valued one-form
    on $P$. Then we say that $\mu$ is \textbf{right-equivariant under $\sigma$} if
    \[(\sigma_g)^*\mu=\ad(g^{-1})\circ \mu\]
    for all $g\in G$.
\end{defn}

\begin{lem}
    The Maurer-Cartan form is right-invariant.
\end{lem}
\begin{proof}
    Take $g,h\in G$ and a vector $v\in T_{hg^{-1}}G$. It suffices to show right-equivariance for this case:
    \begin{align*}
        \mu(h)\left( (R_g)_*v \right)&=(L_{h^{-1}})_*\left( (R_g)_*v \right)\\
        &=(L_{g^{-1}}\circ L_{gh^{-1}})_*\left( (R_g)_*v \right)\\
        &=(L_{g^{-1}})_*\left( (L_{gh^{-1}}\circ R_g)_*v \right)\\
        &=(L_{g^{-1}})_*\left( (R_g\circ L_{gh^{-1}})_*v\right)\\
        &=\ad(g^{-1})\circ (L_{gh^{-1}})_*v\\
        &=\ad(g^{-1})\circ \mu(hg^{-1})(v),
    \end{align*}
    as desired. Note that we have used the covariance of the pushforward together with the fact that the left and right actions
    commute.
\end{proof}

Let us finally return to connections.
\begin{defn}
    Given a connection $\Gamma$ on a principal $G$-bundle $P$ over $M$, we define the corresponding \textbf{connection one-form} $\omega\in \Omega^1(P;\fr g)$
    as follows. Recall that the elements of $\fr g$ are in one-to-one correspondence with left-invariant vector fields on $G$. Given a Lie algebra
    element $X$, denote by $\tilde X$ its corresponding left-invariant vector field. Note that when $\tilde X$ is considered
    as a vector field on $P$ it is a purely vertical vector field. Thus, for some vector $Y\in T_pP$ with vertical component $Y_v$ we define
    $\omega(p)(Y)$ to be the (unique) Lie algebra element $X$ such that $\tilde X(p)=Y_v$.
\end{defn}











We have seen how the tangent space at a point $p$ of a principal bundle $P$ contains vectors that can be naturally called vertical: those that are in the kernel of the projection pushforward $\pi_*$, or that ``point in the direction of the fiber''. However, principal bundles are not a priori equipped with horizontal vectors. In other words, there is no canonical way of moving from a fiber to the next, and no way to tell where exactly on that fiber one would reach by moving. In order to talk about a horizontal direction, we need to define the concept of a connection. We will see that connections on principal bundles give rise to tools that are crucial in differential geometry and mathematical physics. Among such, we can pullback a connection to the base space in order to obtain a generalization of the electromagnetic vector potential. We will also use connections to define parallel transport and covariant differentiation. Before we define connections, it's useful to define a 1-form on the tangent spaces of the group $G$:

\begin{defn}
    The \emph{Maurer-Cartan form} on a Lie group $G$ is the one-form $\omega_{MC} \in \Omega^1(G, \fr g)$ which is invariant under left multiplication by group elements and which acts as the identity map on $T_eG$, the tangent space at the identity.
\end{defn}

In the formulation of this definition we assume that there exists a unique 1-form with the required properties. The proof of this claim is easy. The left-invariance of $\omega_{MC}$ means that:
\[    \omega_{MC} (X) =  (L_g^* \omega_{MC}) (X) = \omega_{MC} (L_{g*} X)     \]
This gives an intuitive picture of the way the Maurer-Cartan form acts: it left-translates vectors to $T_eG$. This is the same as constructing a left-invariant vector field from the vector $X$, and then evaluating this field at the identity. Since the tangent space at the identity is isomorphic as a Lie algebra to the space of left-invariant vector fields (you can see a proof of this in John Lee \todo{reference}), the Maurer-Cartan form is unique.
\\
\\
Now we are ready to define connections, following Spivak \todo{reference}.

\begin{defn}
    A \emph{connection} in a principal bundle $\pi: P \to M$ with group $G$ is a $\C^{\infty}$ $\fr g$-valued 1-form $\omega$ on $P$ such that:
    \begin{enumerate}[(1)]
        \item $R_p^*(\omega) = \omega_{MC}$, where $R_p$ is an embedding of $G$ in $P$, given by $R_p(g) = p \cdot g$;
        \item $\omega(R_{g*} Y) = \text{Ad}(g^{-1})\omega(Y)$ for all $g\in G$ and all tangent vectors on $P$.
    \end{enumerate}
\end{defn}
Before we move on, we should unravel the meaning behind this cryptic definition, which puzzled the authors for a long time before they finally understood it. Condition $(1)$ simply gives the way in which the connection acts on vertical vectors. Indeed, pulling back by $R_p$ gives a form on the fiber that the vertical vector belongs to, and $R_p^*(\omega) = \omega_{MC}$ is saying that $\omega$ will left-translate the vector to the tangent space at the identity of the fiber. An alternate way to think about this condition is in a local trivialization: here the fiber is explicitly identified, and left-translation should be intuitive. Condition $(2)$ expresses the way in which $\omega$ changes under right actions. Pushing forward by $R_g$ moves the vector along the fiber, and the value of $\omega$ at the new vector is simply the old value acted on by the adjoint action.
\\
\\
We should also explain in what sense $\omega$ gives a canonical way of moving from one fiber to another. If the base manifold $M$ is $m$-dimensional and the group $G$ is $n$-dimensional (as a manifold), then $\omega$ acts on the $m+n$-dimensional tangent space at $p\in P$, and it takes values in the $n$-dimensional Lie algebra $\fr g$. By the rank-nullity theorem of linear algebra, the kernel of $\omega$ is $m$-dimensional, and therefore can be thought of as ``parallel'' to the base manifold, or horizontal. In fact, as the following proposition shows, a connection on a principal bundle is equivalent to a collection of horizontal spaces. For this reason, some authors define connections as a choice of horizontal subspace in every tangent space to $P$. In the next proposition and throughout the section, if $u\in P$, then we denote by $V_u$ and $H_u$ the space of vertical and horizontal vectors at $u$ respectively.

\begin{prop}
    If $H_u$ is the horizontal subspace at $u\in P$ determined by the connection $\omega$, then:
    \begin{enumerate} [(a)]
        \item $T_uP = V_u \oplus H_u$
        \item $H_{u\cdot g} = (R_g)_* H_u$
    \end{enumerate}
    Conversely, a collection of subspaces $H_u$ at every point $u\in P$ determines a unique connection $\omega$ on $P$.
\end{prop}

\section{Lifts}
\section{The space of connections}




\end{document}








