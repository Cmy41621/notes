\documentclass{mathnotes}

\title{Notes on Differentiable Manifolds}
\author{Nilay Kumar}
\date{Last updated: \today}


\begin{document}

\maketitle

\setcounter{section}{-1}

\section{Administrativia}

Recommended textbooks:
\begin{itemize}
    \item John Lee: \textit{Introduction to Smooth Manifolds} (2012)
    \item Spivak: \textit{Differential Geometry: A Comprehensive Introduction}
    \item L. Tu: \textit{An Introduction to Manifolds} (2008 E-book available)
\end{itemize}

Problem sets will be assigned every one/two weeks through email. Some homework problems will be taken from Lee (second edition).
There will most likely be two midterms and a final, all in-class.

\section{Introduction}

\begin{defn}
    A function $f$ defined on $\mathbb{R}^n$ is $C^k$ for a positive integer $k$ if $\frac{\partial^k f}{\partial x_{i_1}\cdots\partial x_{i_l}}$ 
    exists and is continuous for any positive integer $l\leq k$, where $1\leq i_1\cdots i_l\leq n$. $f$ is $C^\infty$ if it is $C^k$ for any positive
    integer $k$.
\end{defn}

\begin{exmp}
    $f(x)=x^{1/3}$ for $x\in\mathbb{R}$ is $C^0$ but not $C^1$.
\end{exmp}

\begin{exmp}
    $f(x)=x^{1/3}+k$ for $x\in\mathbb{R}$ is $C^k$ but not $C^{k+1}$, for $k\geq 1$.
\end{exmp}

\begin{defn}
    A \textbf{coordinate chart} $(U,\phi)$ on a topological space $X$ is an open set $U\subset X$ together with a map $\phi:U\ra \mathbb{R}^n$
    such that $\phi$ is a homeomorphism onto $\phi(U)$, an open set in $\mathbb{R}^n$. In other words, $(U,\phi)$ gives each $p\in U$ a
    coordinate.
\end{defn}

\begin{exmp}
    Let $S^2=\left\{ (x,y,z) | x^2+y^2+z^2=1 \right\}\subset \mathbb{R}^3$. Let $U=\left\{ z>0 \right\}\cap S^2$ be the upper hemisphere.
    $x, y, z$ are not good coordinates in that they are not free - they are constrained to the surface. Note that if we define $\phi:U\ra\mathbb{R}^2$
    such that it takes $(x,y,z)\ra(x,y)$, we have a projection map and we now have free coordinates ($z$ can be computed). This is a
    \textbf{graphical coordinate chart}. We can work similarly with the lower hemisphere. But what about the equator? We can build similar charts
    for the equator by projecting onto different planes. It is clear, however, that to cover every point, we will need a total of 6 graphical charts to cover $S^2$.

    This is nice because we can now do calculus on the open sets that $\phi$ maps us to in Euclidean space.
\end{exmp}

\begin{exmp}[Stereographic projection of $S^2$]
    Use a different model for $S^2$. Consider $\left\{ (x,y,z) | x^2+y^2+(z-\frac{1}{2})^2=\frac{1}{4} \right\}\subset\mathbb{R}^3$, the sphere of radius $1/2$
    centered at $(0,0,\frac{1}{2})$. Note that the south pole's coordinates are $(0,0,0)$ and the north pole's are $(0,0,1)$. Imagine that there is a light source
    at the north pole, which projects through the sphere onto the $xy$-plane. If the line hits the point $(x,y,z)$ on the sphere, we can solve for the point at
    which it hits the $xy$-plane. The line is given by $(0,0,1)+t(x,y,z-1)$ for $t\in\mathbb{R}$. Solving this for where $z=0$ yields $t=\frac{1}{1-z}$.
    The point is then $(0,0,1)+\frac{1}{1-z}(x,y,z-1)=(\frac{x}{1-z},\frac{y}{1-z},0)$. This gives a coordinate chart $(U,\phi)$ with $U=S-\left\{ (0,0,1) \right\}$
    (as the chart is undefined there) and $\phi:U\ra \mathbb{R}^2$ that maps $(x,y,z)\ra(\frac{x}{1-z},\frac{y}{1-z})$.
    
    In order to cover the south pole as well, we can perform stereographic projection from the south pole onto $z=1$ plane. The corresponding line is now
    given by $(0,0,0)+t(x,y,z)$ which has $z=1$ for $t=\frac{1}{z}$, and the corresponding point is $(\frac{x}{z},\frac{y}{z},1)$. This gives us a $(V,\psi)$
    where $V=S-\left\{ (0,0,0) \right\}$ with $\psi:V\ra\mathbb{R}^2$ mapping $(x,y,z)\ra(\frac{x}{z},\frac{y}{z})$.
\end{exmp}

\begin{exmp}
    Let $X$ be the set of all lines on $\mathbb{R}^2$. $X$ is a topological space (check this!). Take the set $U=\left\{ \nm{lines of the form} y=mx+c \right\}$,
    which is the collection of all non-vertical lines. To cover the vertical lines, we can have $V=\left\{ \nm{lines of the form}x=\bar{m}y+\bar{c} \right\}$,
    the collection of all non-horizontal lines. We now define $\phi:U\ra\mathbb{R}^2$ that maps $y=mx+c\ra(m,c)$ and $\psi:V\ra\mathbb{R}^2$ that maps
    $x=\bar{m}y+\bar{c}\ra(\bar{m},\bar{c})$.
\end{exmp}

Notice that for this example and the stereographic projection example, there are instances where the charts overlap. For the sake of consistency, we want
the coordinate charts to be compatible with one another. For example, a function that is differentiable in one chart should be differentiable in the other
as well.

\begin{defn}
    Given a coordinate chart $(U,\phi)$ and a function $f$ defined on $U$, we can consider $f\circ \phi^{-1}$ as a function on $\phi(U)$ and differentiate
    $f\circ \phi^{-1}$. Suppose $(V,\psi)$ is another coordinate chart and that $U\cap V\neq \varnothing$. Now we can consider $f\circ\psi^{-1}$ as a function
    to do calculus with. Now the question is: is $f\circ \phi^{-1}$ differentiable the same as $f\circ\psi^{-1}$ differentiable? This should be the case!
    We are considering a function on an abstract space, and the coordinates should respect properties such as differentiablity. So let us write
    $f\circ\psi^{-1}\circ(\psi\circ\phi^{-1})$, which is differentiable if the term in the parentheses is differentiable. We can do the same, but with
    $\phi$ and $\psi$ switched. Thus, we want to make sure that both $\psi\circ\phi^{-1}$ and $\phi\circ\psi^{-1}$ are differentiable. These are called
    \textbf{transition maps} of these coordinate charts. Two coordinate charts are \textbf{smoothly compatible} if their transition maps are diffeomorphisms
    (or if, trivially, they don't intersect).
\end{defn}

Returning to the previous manifold of lines, suppose we have a line labeled by $(m,c)$ with $m\neq 0$. This can be expressed in the other chart as
$(m^{-1},-m^{-1}c)$. It turns out, that for $m\neq 0$, these transition maps are differentiable. For the stereographic projection of $S^2$, it is the same,
just a little trickier.

\begin{defn}
    An atlas $\mathcal{A}$ for a topological space $X$ is a collection of coordinate charts that cover $X$ such that any two charts in $\mathcal{A}$
    are smoothly compatible.
\end{defn}


\end{document}
