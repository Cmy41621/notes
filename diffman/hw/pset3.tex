\documentclass{../../mathnotes}


\title{Differentiable Manifolds Problem Set 3}
\author{Nilay Kumar}
\date{Last updated: \today}


\begin{document}

\maketitle

\subsection*{Problem 1}

Let us take two copies of $\R^2$, call them $M$ and $N$. On $M$, we define the standard coordinates $(x,y)$
given by the global identity coordinate chart $\id_1:M\to\R^2$. On $N$, let us try to define a different global smooth
coordinate chart, $\phi:N\to\R^2$ that maps $\id_1^{-1}(x,y)\mapsto \left( \tilde x=x,\tilde y=f(x,y) \right)$,
where $f$ is a smooth function from $\R^2\to\R$. For $\phi$ to be a global coordinate chart, it must be a global
homeomorphism. It is clear that $\phi$ is continuous (smooth, in fact) by the smoothness of $f$, but it is not obvious
that its inverse is continuous as well. We use the inverse function theorem on $\phi$ (which acts
from $N=\R^2\to\R^2$):
\begin{align*}
    D\phi=\left(
    \begin{array}[]{cc}
        1 & 0\\
        \frac{\partial f}{\partial x} & \frac{\partial f}{\partial y}
    \end{array}
    \right),
\end{align*}
which is clearly nonsingular as long as $\partial f/\partial y\neq 0$. Consequently, as long as this condition holds,
$\phi$ is locally a diffeomorphism, i.e. its inverse is locally continuous (and smooth).
It will follow that $\phi$ is a global diffeomorphism if we can show that the inverse of $\phi$ exists.
In other words, we need to make sure that for a given $(x,z)$, there exists a unique point $(x,y)$ such that
$\phi(x,y)=(x,z)$ -- i.e. that $z$ is not the same for two different values of $y$. But since $z=f(x,y)$
is smooth, by the mean value theorem, if this did happen, $\partial f/\partial y$ would have to be zero
somewhere. As this is never true under the condition above, $\phi$ is injective. Furthermore, $\phi$ is
surjective, as the inverse function theorem guarantees that we can find a $y$ that is mapped by $f$ to a given
$z$. Consequently, $\phi$ is a global homeomorphism (diffeomorphism) and is thus a global chart for $\R^2$.

Let us define the diffeomorphism $\id_2:M\to N$. Derivatives on $M$ take the form of the usual derivatives on
$\R^2$, as we are using the identity chart. Derivatives on $N$, however, need not be of the same form, as we
are working in the nontrivial coordinates given by $\phi$. If we take some smooth function $g:N\to\R$, and its
pullback $h=g\circ\id_2:M\to\R$, and compare the derivatives of these two functions, we will find that they are
in general different. On $M$, the derivative is expressed in coordinates as
\begin{align*}
    \frac{\partial}{\partial x}\bigg|_{(x,y)}h&=(\id_1^{-1})_*\frac{\partial}{\partial x}\bigg|_{\id_1(x,y)}h\\
    &=\frac{\partial}{\partial x}\bigg|_{\id_1(x,y)}\left( h\circ \id_1^{-1} \right)\\
    &=\frac{\partial h}{\partial x}\bigg|_{(x,y)}=\frac{\partial g}{\partial x}\bigg|_{\phi(x,y)},
\end{align*}
as expected. Let us now look at
\begin{align*}
    \frac{\partial}{\partial \tilde x}\bigg|_{(x,y)}g&=(\phi^{-1})_*\frac{\partial}{\partial \tilde x}\bigg|_{\phi(x,y)}g\\
    &=\frac{\partial}{\partial \tilde x}\bigg|_{\phi(x,y)}(g\circ\phi^{-1})\\
    &=\frac{\partial g}{\partial (\phi^{-1})^j}\bigg|_{\phi(x,y)}\frac{\partial (\phi^{-1})^j}{\partial \tilde x}\bigg|_{(x,y)}\\
    &=\frac{\partial g}{\partial x}\bigg|_{\phi(x,y)}\frac{\partial x}{\partial \tilde x}\bigg|_{(x,y)}
    +\frac{\partial g}{\partial y}\bigg|_{\phi(x,y)}\frac{\partial y}{\partial \tilde x}\bigg|_{(x,y)}\\
    &=\frac{\partial g}{\partial x}\bigg|_{\phi(x,y)}
    +\frac{\partial g}{\partial y}\bigg|_{\phi(x,y)}\frac{\partial y}{\partial \tilde x}\bigg|_{(x,y)}.
\end{align*}
This is clearly not the same as the above derivative, $\p y/\p\tilde x$ is not necessarily zero everywhere on the manifold.

\subsection*{Problem 2}

Recall one of the transition functions for the stereographic atlas on $S^2$,
\begin{align*}
    \psi\circ\phi^{-1}(u,v)=\left( \frac{4u}{u^2+v^2},\frac{4v}{u^2+v^2} \right).
\end{align*}
Here, let $\phi$ be the chart that excludes the north pole and let $\psi$ be the chart that excludes the
south pole.
Now, given the vector field $X=u\p/\p u+v\p/\p v$ (smooth, as the components are smooth functions on $\R^2$)
defined on the copy of $\R^2$ that $\phi$ sends points on the manifold to (what we'll call $M$), we can push it forward to a 
vector field on the copy of $\R^2$ that $\psi$ sends points on the manifold to (what we'll call $N$). For some $p\in N$ and
some $f\in C^\infty(N)$,
\begin{align*}
    X'=\left( \psi \circ \phi^{-1} \right)_*&\left( u\frac{\p}{\p u} + v\frac{\p}{\p v} \right)(f)(p)
    =\left(u\frac{\p}{\p u}+v\frac{\p}{\p v}\right)\left(f\circ \psi \circ \phi^{-1}\right)\left(\phi\circ\psi^{-1}(p)\right)\\
    &=u\frac{\p}{\p u}f\left( \psi\circ\phi^{-1} \right)+v\frac{\p}{\p v}f\left( \psi \circ \phi^{-1} \right)\\
    &=u\frac{\p(\psi\circ\phi^{-1})^j}{\p u}\bigg|_{\phi\circ\psi^{-1}(p)}\frac{\p f}{\p(\psi\circ\phi^{-1})^j}\bigg|_p
    +v\frac{\p(\psi\circ\phi^{-1})^j}{\p v}\bigg|_{\phi\circ\psi^{-1}(p)}\frac{\p f}{\p(\psi\circ\phi^{-1})^j}\bigg|_p
\end{align*}
Note that we have used the Einstein summation convention. Taking the necessary derivatives of the transition
maps, we find
\begin{align*}
    u\frac{\p}{\p u}\left( \frac{4u}{u^2+v^2} \right)&=u\frac{4(v^2-u^2)}{(u^2+v^2)^2}\\
    u\frac{\p}{\p u}\left( \frac{4v}{u^2+v^2} \right)&=-\frac{8 u^2 v}{\left(u^2+v^2\right)^2}\\
    v\frac{\p}{\p v}\left( \frac{4u}{u^2+v^2} \right)&=-\frac{8 u v^2}{\left(u^2+v^2\right)^2}\\
    v\frac{\p}{\p v}\left( \frac{4v}{u^2+v^2} \right)&=\frac{4 v \left(u^2-v^2\right)}{\left(u^2+v^2\right)^2}.
\end{align*}
Let us first note that the pushforward of the vector field is not defined at $(0,0)\in N$, as that corresponds
to the north pole. However, we can smoothly define it to be 0 there; the pushforward of the vector field
goes smoothly to zero as $(0,0)$ because $\lim_{q\to\infty}\psi\circ\phi^{-1}(q)=(0,0)$ and because the above
derivatives similarly go smoothly to zero, where $q$ is a point on $M$,
by the transition function above. Furthermore, the pushforward of the field is smooth everywhere else as well,
as seen by inserting the above derivatives into the expression for the pushforward at the top of the page.

Informally speaking, we now have smooth vector fields in coordinate charts that will cover the manifold when
each is pushforwarded back onto the manifold (in their respective domains of definitions). In other words, if
we push forward our original vector field from $M$ to $S^2$ we'll get a smooth vector field defined everywhere
but at the north pole, and if we push forward $X'$ from $N$ to $S^2$ we'll get a smooth vector field defined
everywhere but at the south pole. If we then define a vector field $X$ that is on the southern hemisphere
equal to the pushfoward by $\phi^{-1}$ of $X$ and on the northern hemisphere equal to the pushforward by
$\psi^{-1}$ of $X$, we have a globally smooth manifold on $S^2$, as the two are equal everywhere but the poles,
and on the poles they smoothly go to zero. This is easily seen from the fact that pushforwarding $X'$ via
$\psi^{-1}$ will yield the same eexpression, as desired (we have already shown that they go smoothly to zero).

Suppose now that we were working with the vector field $X=v\p/\p u+-\p/\p v$. Then, all of the above statements
still hold -- although the derivatives will be slightly different, as below,
\begin{align*}
    v\frac{\p}{\p u}\left( \frac{4u}{u^2+v^2} \right)&=v\frac{4(v^2-u^2)}{(u^2+v^2)^2}\\
    v\frac{\p}{\p u}\left( \frac{4v}{u^2+v^2} \right)&=-\frac{8 u v^2}{\left(u^2+v^2\right)^2}\\
    u\frac{\p}{\p v}\left( \frac{4u}{u^2+v^2} \right)&=-\frac{8 u^2 v}{\left(u^2+v^2\right)^2}\\
    u\frac{\p}{\p v}\left( \frac{4v}{u^2+v^2} \right)&=\frac{4 u \left(u^2-v^2\right)}{\left(u^2+v^2\right)^2}.
\end{align*}
The pushforward should will still smoothly go to zero, as these derivatives go to zero with $u,v\to\infty$.
Thus we will again end up with a global vector field on $S^2$.


\subsection*{Problem 3}

Consider the torus $\mathbb{T}=S^1\times S^1$. It should be clear that $\mathbb{T}$ is a manifold via its
product structure; $S^1\times S^1$ forms a manifold with charts of the form $(\prod U_i, \prod\psi_i)$, where $U_i$ are
the the (graphical) charts for $S^1$ (smoothness of transition maps follows trivially).

To show that $\p/\p x,\p/\p y$ are smooth on all of $\mathbb{T}$, we compute in one coordinate chart -- by analogy,
it is easily extended to the other charts by symmetry. First let us parameterize the torus as $\left\{ x^2+y^2=1,w^2+z^2=1 \right\}$.
Take then, for example, the vector field $\p/\p x$ and the graphical chart defined on $x,w>0$, given by
$\psi(x,y,z,w)=(x,z)$ and $\psi^{-1}(x,z)=(x,\sqrt{1-x^2},z,\sqrt{1-z^2})$.
Pushing forward the vector field, from the chart to $\mathbb{T}$, we find for some $f\in C^\infty(\mathbb{T})$,
\begin{align*}
    \phi^{-1}_*\left( \frac{\p}{\p x} \right)(f)\bigg|_p&=\frac{\p}{\p x}\left( f\circ\phi^{-1} \right)\bigg|_{\phi(p)}\\
    &=\frac{\p f}{\p x}\bigg|_p-\frac{x}{\sqrt{1-x^2}}\bigg|_{\phi(p)}\frac{\p f}{\p w}
\end{align*}
Since $f$ is smooth and because $x>0$ in this chart, the pushforward of the vector field onto the torus is smooth.
This procedure can be carried out for any chart, and thus we can construct a smooth global vector field that covers
$\mathbb{T}$ similar to how we did in the last problem (as the vector fields are equal where they overlap). The same
holds for $\p/\p y$.

\subsection*{Problem 4}

Let $x,y$ be the standard coordinates on $\R^2$, and $X$ be a vector field given by $X(x,y)=y\p/\p x-x\p/\p y$.
We want to solve for $\phi(t,x,y)$ where $\frac{\p\phi}{\p t}(t,x,y)=X(\phi(t,x,y))$ with the initial
conditions that $\phi(0,x,y)=(x,y)$. Since $\phi$ sends points on the manifold to other points on the manifold
as a function of time, we write $\phi(t,x,y)=(x(t),y(t))$. Using the fact that $X(\phi(t,x,y))=\phi_*(\p/\p t)$,
we can compute
\begin{align*}
    \frac{\p x}{\p t}&=\left( y\frac{\p}{\p x}-x\frac{\p}{\p y} \right)x=y\\
    \frac{\p y}{\p t}&=\left( y\frac{\p}{\p x}-x\frac{\p}{\p y} \right)y=-x.
\end{align*}
This is a system of linear first order equations that can be easily decoupled by taking
$\p^2x/\p t^2=dy/dt$ and $dy/dt=-x$, which yields the differential equation
\[\frac{d^2x}{dt^2}+x=0,\]
which results in the solutions
\begin{align*}
    x(t)&=A\sin t+B\cos t\\
    y(t)&=A\cos t-B\sin t.
\end{align*}
Using the initial conditions, we find that
\begin{align*}
    \phi(t,x,y)=\left( y\sin t+x\cos t,y\cos t-x\sin t \right),
\end{align*}
where $x,y$ on the right hand side are the initial conditions.

\subsection*{Problem 5}

Let $X$ be the smooth vector field on $S^2$ which is the extension of $u\p/\p u+v\p/\p v$ from
$U=S^2-\left\{ \left( 0,0,1 \right) \right\}$ to the whole $S^2$. We wish to find the one parameter
group of diffeomorphisms $\phi_t$ generated by $X$. If we take any integral curve $\gamma(t)$ on $S^2$,
we have the requirement that $\gamma'(t)=X_{\gamma(t)}$. To solve for $\gamma(t)$ let us work in coordinates,
i.e. the chart $\psi$ without the north pole, and solve for $\psi(\gamma)$. By pushing forward, we find that
\begin{align*}
    \frac{\p}{\p t}\left( u(t),v(t) \right)=\left( u\frac{\p}{\p v}+v\frac{\p}{\p v} \right)\left( u(t),v(t) \right),
\end{align*}
which yields the differential equations,
\begin{align*}
    u(t)=u_0 e^t\\
    v(t)=v_0 e^t
\end{align*}
where $u_0,v_0$ are the coordinates of the initial point, $\gamma(0)$. Note carefully that $\gamma(t)$
is an arbitrary integral curve. Now, $\phi_t(x,y)$, in some sense, is simply the set of all integral curves, but
parameterized by the initial point $x,y$. 
For any given $x,y$, then, we can write
\begin{align*}
    \psi\circ\phi_t\circ\psi^{-1}(x,y)=(x e^t, y e^t).
\end{align*}
Using the expressions for the stereographic chart,
\begin{align*}
    \psi(x,y,z)&=\left( \frac{2x}{1-z},\frac{2y}{1-z} \right)\\
    \psi^{-1}(u,v)&=\left( \frac{4u}{u^2+v^2+4}, \frac{4v}{u^2+v^2+4},\frac{u^2+v^2-4}{u^2+v^2+4} \right),
\end{align*}
we can solve
\begin{align*}
    \phi_t(x,y,z)=\left( \frac{2x e^t}{(1+z)e^{2t}+(1-z)},\frac{2y e^t}{(1+z)e^{2t}+(1-z)},\frac{(1+z)e^{2t}-(1-z)}{(1+z)e^{2t}+(1-z)} \right).
\end{align*}
Intuitively, these equations tell us that the vector field induces a flow from the south pole up to the north pole
that in coordinates equates to following a ray exponentially to infinity.
As we approach the north pole, we have $z\to 1$, and so the above function approaches $(xe^{-t},ye^{-t},1)$. In other
words, as we get closer to the north pole, we flow towards it exponentially slower and slower. 
On the contrary, near the south pole, the flow is exponentially fast, as can be seen by inserting $z=-1$.
At the north/south poles themselves, of course, due to $\phi_t$ having to be bijective, there can be no flow.

\subsection*{Problem 6}

Let $A\subset\mathcal{X}(\R^3)$ be the subspace spanned by $\left\{ X,Y,Z \right\}$, where
\begin{align*}
    X=y\frac{\p}{\p z}-z\frac{\p}{\p y}, Y=z\frac{\p}{\p x}-x\frac{\p}{\p z}, Z=x\frac{\p}{\p y}-y\frac{\p}{\p x}.
\end{align*}
We wish to show that $A$ is a Lie subalgebra of $\mathcal{X}(\R^3)$, which is isomorphic to $\R^3$ with the
cross product. To show the first part, we must simply show that it is closed under the Lie brackets:
\begin{align*}
    [X,Y]&=\left( y\frac{\p}{\p z}-z\frac{\p}{\p y} \right)\left( z\frac{\p}{\p x}-x\frac{\p}{\p z} \right)
    -\left( z\frac{\p}{\p x}-x\frac{\p}{\p z} \right)\left( y\frac{\p}{\p z}-z\frac{\p}{\p y} \right)\\
    &=y\p_x+yz\p_{zx}-xy\p_{zz}-z^2\p_{yx}+xz\p_{yz}
    -zy\p_{xz}+z^2\p_{xy}+xy\p_z^2-xz\p_{zy}-x\p_y\\
    &=y\p_x-x\p_y=-Z\\
    [Y,Z]&=\left( z\frac{\p}{\p x}-x\frac{\p}{\p z} \right)\left( x\frac{\p}{\p y}-y\frac{\p}{\p x} \right)
    -\left( x\frac{\p}{\p y}-y\frac{\p}{\p x} \right)\left( z\frac{\p}{\p x}-x\frac{\p}{\p z} \right)\\
    &=z\p_y+xz\p_{xy}-zy\p_{xx}-x^2\p_{zy}+xy\p_{zx}-xz\p_{yx}+x^2\p_{yz}+yz\p_{xx}-yx\p_{xz}-y\p_z\\
    &=z\p_y-y\p_z=-X\\
    [X,Z]&=\left( y\frac{\p}{\p z}-z\frac{\p}{\p y} \right)\left( x\frac{\p}{\p y}-y\frac{\p}{\p x} \right)
    -\left( x\frac{\p}{\p y}-y\frac{\p}{\p x} \right)\left( y\frac{\p}{\p z}-z\frac{\p}{\p y} \right)\\
    &=xy\p_{zy}-y^2\p_{zx}-zx\p_{yy}+zy\p_{yx}+z\p_x-x\p_z-xy\p_{yz}+xz\p_{yy}+y^2\p_{xz}-yz\p_{xy}\\
    &=z\p_x-x\p_z=Y.
\end{align*}
Now we take $\hat x\in\R^3$ to be the unit vector in the $x$-direction, $\hat y\in\R^3$ to be the unit
vector in the $y$-direction, $\hat z\in\R^3$ to be unit vector in the $z$-direction. Then, defining the usual
cross-product (with a left-handed convention), we find 
\begin{align*}
    \hat x \times \hat y &= -\hat z\\
    \hat y \times \hat z &= -\hat x\\
    \hat x \times \hat z &= \hat y.
\end{align*}
If we identifyeach of these unit vectors with the $X,Y,Z$ above, we see that there exists a Lie algebra
isomorphism $\varphi(aX+bY+cY)=a\hat x+b\hat y+c\hat z$, as $\varphi$ would preserve the action of the Lie bracket,
as we see in the cross products above.

\subsection*{Problem 7}

Let $\pi_1:M\to S^1$ be the Mobius bundle given by
\begin{align*}
    \pi_1: M &\to S^1\\
    [(x,t)]&\mapsto \left( \cos(2\pi x),\sin(2\pi x) \right)
\end{align*}
We define $U=S^1\setminus\left\{ 1 \right\}$ and $V=S^1\setminus\left\{ -1 \right\}$.
Let us define the following maps:
\begin{align*}
    \phi_1: \pi_1^{-1}(U)&\to U\times \R\\
    [(x,t)]&\mapsto \left( (\cos(2\pi x), \sin(2\pi x)), t \right)
\end{align*}
and
\begin{align*}
    \phi_2: \pi_1^{-1}(V)&\to V\times \R\\
    [(x,t)]&\mapsto
    \left\{ 
        \begin{array}[]{cc}
            ( (\cos(2\pi x),\sin(2\pi x)), t) & 0\leq x < 1/2\\
            ( (\cos(2\pi x),\sin(2\pi x)), -t) & 1/2< x \leq 1\\
        \end{array}
        \right..
\end{align*}
It should be clear that these are local trivializations, as they are smooth, and the fibers are
isomorphically mapped to fibers (multiplication by either plus or minus).
Note that the transition function $\phi_1\circ \phi_2^{-1}$ will yield precisely what we are given
for the problem - i.e. $\tau$ is 1 on the top half of the circle and -1 on the bottom half of the
circle.

Note that for the given $\pi_2:F\to S^2$, we know that there exist local trivializations 
\begin{align*}
    \psi_1: & \pi^{-1}(U)\to U\times \R\\
    \psi_2: & \pi^{-1}(V)\to V\times \R.
\end{align*}
These are, by definition, diffeomorphisms, and thus if we compose $\psi_1^{-1}\circ\phi_1:\pi_1^{-1}(U)\to\pi_2^{-1}(U)$,
and $\psi_2^{-1}\circ\phi_2:\pi_1^{-1}(V)\to\pi_2^{-1}(V)$, we now have diffeomorphisms between subsets of the two bundles.
To find a smooth isomorphism between the two vector bundles, we wish to show that these two maps are equal where they overlap,
i.e. we want $\psi_2^{-1}\circ\phi_2=\psi_1^{-1}\circ\phi_1$ on $\pi_1^{-1}(U\cap V)\cap\pi_2^{-1}(U\cap V)$.
This can be done by realizing that the transition functions for the local trivializations on each vector bundle
are exactly the same, as we remarked earlier (the $\tau$'s are the same). Thus we have that
\begin{align*}
    \phi_1\circ\phi_2^{-1}&=\psi_1\circ\psi_2^{-1}\\
    \psi_2^{-1}\circ\phi_2&=\psi_1^{-1}\circ\phi_1,
\end{align*}
as desired. Consequently, there is a smooth isomorphism between $F$ and $E$.




\end{document}
