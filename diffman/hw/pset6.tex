\documentclass{../../mathnotes}


\title{Differentiable Manifolds Problem Set 6}
\author{Nilay Kumar}
\date{Last updated: \today}


\begin{document}

\maketitle

\subsection*{Problem 1}

Let $V_1,\ldots, V_k$ and $W$ be finite-dimensional real vector spaces. We wish to show that there is a canonical isomorphism
$V_1^* \otimes \cdots \otimes V_k^*\otimes W\cong L(V_1,\ldots,V_k;W)$. First consider the map
$\Phi:V_1^*\times \cdots \times V_k^*\times W\to L(V_1,\ldots,V_k;W)$ given by
\[\Phi(\xi^1,\ldots,\xi^k,w)(v_1,\ldots,v_k)=\xi^1(v_1)\cdots\xi^k(v_k)w.\]
This right-hand side is clearly linear in each of the arguments and takes values in $W$, so it is indeed a member of
$L(V_1,\ldots,V_k;W)$. Note that $\Phi$ is multilinear in $\xi^1,\ldots\xi^k,w$, as the right-hand side is simply
multiplication. By the characteristic property of tensor product spaces, then, $\Phi$ descends uniquely to a linear map
$\tilde\Phi$ from $V_1^* \otimes \cdots \otimes V_k^*\otimes W$ to $L(V_1,\ldots,V_k;W)$:
\[\tilde\Phi(\xi^1 \otimes \cdots \otimes \xi^k \otimes w)(v_1,\ldots,v_k)=\xi^1(v_1)\cdots\xi^k(v_k)w.\]
Furthermore, $\tilde\Phi$ takes the basis of $V_1^* \otimes \cdots \otimes V_k^*\otimes W$ to the
basis of $L(V_1,\ldots,V_k;W)$:
\[\tilde\Phi(e^1\otimes\cdots\otimes e^k \otimes E_1)(v_1,\ldots,v_k,w)=e^1(v_1)\cdots e^k(v_k)E_1,\]
where we take linear combinations of tensors.
Thus $\tilde\Phi$ must be an isomorphism (independent of the specific bases chosen), and we are done.

\subsection*{Problem 2}

Let $(e^1,e^2,e^3)$ be the standard dual basis for $(\R^3)^*$. Suppose that $\omega=e^1\otimes e^2\otimes e^3$ can be written
as the sum of an alternating tensor and a symmetric tensor,
\begin{align*}
    \omega=e^1\otimes e^2\otimes e^3=\left(\eta+\theta\right)_{ijk}e^i\otimes e^j\otimes e^k
\end{align*}
with $\eta$ antisymmetric, $\theta$ symmetric. By linear independence of the basis vectors of $T^3( (\R^3)^*)$,
the only non-vanishing term on the right-hand side must be that which contains a $e^1\otimes e^2\otimes e^3$,
i.e. $(\eta+\theta)_{123}=1$. Note, however, that $(\eta+\theta)_{312}=(\eta+\theta)_{123}$ must be zero. This
contradicts the previous statment, and thus this tensor cannot be written as a sum of an alternating and a symmetric
tensor.

\subsection*{Problem 3}

We wish to show that the covectors $\omega^1,\ldots,\omega^k$ are linearly dependent on a finite-dimensional space
if and only if $\omega^1\wedge\cdots\wedge\omega^k=0$. Let us first assume that we have the set of covectors is linearly
dependent. Then, since
\[\omega^1\wedge\cdots\wedge\omega^k(v_1,\ldots,v_k)=\det(\omega^j(v_i)),\]
and since $\omega^j(v_i)$ is not full rank by assumption, the determinant is zero. Consequently the wedge product must
be zero.

Conversely, let us assume that the wedge product is zero. It follows that one of the arguments must be a multiple of another.
Clearly, then the set of covectors is linearly dependent.

\subsection*{Problem 4}

\begin{enumerate}[(a)]
    \item 
        Given an ordered $k$-tuple $(v_1,\ldots,v_k)$ of elements of $V$, we wish to show that it's linearly dependent if and
        only if $v_1\wedge\cdots\wedge v_k=0$. Note that everything that we have shown for differential forms built up using
        alternating covariant $k$-tensors can be shown for differential forms built up using alternating contravariant $k$-tensors,
        simply by the canonical isomorphism between $V$ and its double dual $V^{**}$. Then exactly the same logic holds over from
        the last problem.
    \item
        Given two linearly independent ordered $k$-tuples $(v_1,\ldots,v_k)$ and $(w_1,\ldots,w_k)$, we wish to show that they
        have the same span if and only if $v_1\wedge\cdots\wedge v_k=cw_1\wedge\cdots\wedge w_k$ for some nonzero real number $c$.
        Suppose that these $k$-tuples have the same span. Then it should be clear that these are two different bases for the space
        and are related by an invertible linear map. Consequently, we can solve for $w_i$ in terms of $v_j$ and insert these expressions
        into the right hand wedge product. Note that there will be many terms with repeated indices, which will vanish, and thus, after
        rearranging forms at the cost of negative signs, we will be left with one term that looks like $v_1\wedge\cdots\wedge v_k$
        with some constant term out front that is a linear combination of certain entries of the change of basis matrix that relates
        the two bases.

        Conversely, let us assume that the wedge products differ by a multiplicative constant. Let us take
        \[w_i\wedge v_1\wedge\cdots\wedge v_k=0\]
        due to the repeated vectors on the right-hand side. This implies, by the previous part of the problem, that $(w_i,v_1,\ldots, v_k)$
        are linearly dependent, i.e. $w_i$ can be written as a linear combination of the $v_j$. We can do precisely the same trick and take the
        wedge product with $v_i$ on both sides to show that $v_i$ can be written as $w_j$. This shows that the $k$-tuples have the same
        span.
\end{enumerate}

\subsection*{Problem 5}

\begin{enumerate}[(a)]
    \item Take the 2-form $\omega=x dy\wedge dz+ydz\wedge dx+zdx\wedge dy$. We can convert to spherical coordinates,
        \[(x,y,z)\mapsto(\rho\sin\phi\cos\theta,\rho\sin\phi\sin\theta,\rho\cos\phi)\]
        which is rather messy:
        \begin{align*}
            \omega=\rho\cos\phi(&-\rho\cos^2\phi\sin^2\theta d\theta\wedge d\rho -\rho\sin^2\phi\sin^2\theta d\theta\wedge d\rho \\
            &-\rho^2\cos^2\phi\cos\theta\sin\theta d\theta\wedge d\phi-\rho^2\cos\phi\sin^2\theta\sin\phi d\theta\wedge d\phi)\\
            +\rho\sin\theta\sin\phi(&-\rho\cos\phi\sin\theta\sin\phi d\theta\wedge d\rho+\rho^2\sin\theta\sin^2\phi d\theta\wedge d\phi\\
            &-\rho\cos\theta\cos^2\phi d\rho\wedge d\phi-\rho\cos\theta\sin^2\phi d\rho\wedge d\phi)\\
            +\rho\cos\theta\sin\theta(&\rho\cos\theta\cos\phi\sin\phi d\theta\wedge d\rho-\rho^2\cos\theta\sin^2\phi d\theta\wedge d\phi\\
            &-\rho\cos^2\phi-\rho\cos^2\phi\sin\theta d\rho\wedge d\phi-\rho\sin\theta\sin^2\phi d\rho\wedge d\phi)
        \end{align*}
        which can be simplified by using the alternating property of differential forms and trig identities to get
        \[\omega=-\rho^3 \sin\phi d\theta\wedge d\phi.\]
    \item We can now compute the exterior derivative:
        \begin{align*}
            d\omega=-d(\rho^3\sin\phi d\theta \wedge d\phi)=- 3\rho^2\sin\phi d\rho\wedge d\theta \wedge d\phi
        \end{align*}
        and in Cartesian coordinates:
        \begin{align*}
            d\omega&=d(x dy\wedge dz+ydz\wedge dx+zdx\wedge dy)\\
            &=dx\wedge dy \wedge dz+dy\wedge dz\wedge dx+dz\wedge dx \wedge dy\\
            &=3dx\wedge dy \wedge dz
        \end{align*}
        These two, are in fact, the same, as if we convert the Cartesian version to spherical, we find
        \begin{align*}
            d\omega&=3d(\rho\sin\phi\cos\theta)\wedge d(\rho\sin\phi\sin\theta)\wedge d(\rho\cos\phi)\\
            &=3(\rho^2\cos^2\theta\cos^2\phi\sin\phi d\theta\wedge d\rho\wedge d\phi+\rho^2\cos^2\phi\sin^2\theta\sin\phi d\theta\wedge d\rho\wedge d\phi\\
            &+\rho^2\cos^2\theta\sin^3\phi d\theta\wedge d\rho \wedge d\phi+\rho^2\sin^2\theta\sin^3\phi d\theta\wedge d\rho\wedge d\phi)\\
            &=-3\rho^2\sin\phi d\theta\wedge d\rho \wedge d\phi
        \end{align*}
        precisely as desired. In other words, if we view the coordinate change as a pullback onto the same manifold, we simply have
        that the pullback commutes with the exterior derivative.
    \item Let us compute the pullback $\iota^*_{\mathbb{S}^2}\omega$ to $\mathbb{S}^2$ using coordinates $(\phi,\theta)$ on the
        open subset where these coordinates are defined. First note that
        \[\iota(\phi,\theta)=(\sin\phi\cos\theta,\sin\phi\sin\theta,\cos\phi).\]
        The pullback of $\omega=x dy\wedge dz+ydz\wedge dx+zdx\wedge dy$ is then simply:
        \begin{align*}
            \iota^*\omega&=-\cos^2\theta\sin^3\theta d\theta\wedge d\phi-\sin^2\theta\sin^3\phi d\theta\wedge d\phi\\
            &+\cos\phi\left( -\cos^2\theta\cos\phi\sin\phi d\theta\wedge d\phi-\cos\phi\sin^2\theta\sin\phi d\theta\wedge d\phi \right)\\
            &=\sin\phi d\phi\wedge d\theta
        \end{align*}
    \item 
        The above pullback 2-form is not zero anywhere where the coordinates are defined, as $\phi=0,\pi$ are not in the coordinate system.
        As the spherical coordinates are defined on the open set that is the complement of half any great circle (including
        the north and south poles), it now remains to show that the 2-form is zero on this set. It should be clear, however,
        that we have simply chosen one choice of coordinates, and since there is rotational symmetry, by simply choosing a different
        spherical coordinate system, we would again arive at the fact that the 2-form is non-zero.
\end{enumerate}


\subsection*{Problem 6}

\begin{enumerate}[(a)]
    \item Take $F(s,t)=(st,e^t)$ from $M=\R^2\to N=\R^2$. Let $\omega=x dy$. Then we can compute
        \begin{align*}
            d\omega=d(x dy)=dx\wedge dy
        \end{align*}
        If we now pullback, we have
        \begin{align*}
            F^*(dx\wedge dy)=d(st)\wedge d(e^t)=(t ds+s dt)\wedge(e^t dt)=te^t ds\wedge dt.
        \end{align*}
        On the other hand, we compute
        \begin{align*}
            F^*(x dy)=st d(e^t)=ste^t dt
        \end{align*}
        and then take
        \begin{align*}
            d(ste^t dt)=te^t ds\wedge dt
        \end{align*}
        as the product rule terms with $t$ will contribute $dt$'s, which will ultimately vanish due to the wedge product.
        Consequently, we see that here $F^*(d\omega)=d(F^*\omega)$.
    \item Note that since $\omega$ is a 2-form, $d\omega$ must be a 3-form. Since pulling back a 3-form to $\R^2$
        will yield a 3-form on a 2-dimensional space, there must be a repeated index and thus the pullback must be zero.
    \item Here, again, $\omega$ is a 2-form and thus $d\omega$ must be a 3-form. Precisely the same argument as in part (b) holds:
        pulling back will yield a 3-form on a 2-dimensional space, which will always be zero.
\end{enumerate}


\end{document}
