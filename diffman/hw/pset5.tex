\documentclass{../../mathnotes}


\title{Differentiable Manifolds Problem Set 5}
\author{Nilay Kumar}
\date{Last updated: \today}


\begin{document}

\maketitle

\subsection*{Problem 1}

Let $\mathring{g}$ denote the round metric on $\mathbb{S}^n$. We wish to compute the coordinate representation of
$\mathring{g}$ in stereographic coordinates. First note that the inverse of the stereographic projection for $\mathbb{S}^n$
is given by
\[\sigma^{-1}(u^1,\ldots,u^n)=\frac{(2u^1,\ldots,2u^n,|u|^2-1)}{|u|^2+1}.\]
The coordinate representation of $\mathring{g}$ is then given by the pullback
\begin{align*}
    (\sigma^{-1})^*\bar g&=(\sigma^{-1})^*\left( (dx^1)^2+\ldots+(dx^{n+1})^2\right)\\
\end{align*}
First note that
\begin{align*}
    d\left( \frac{2u^i}{|u|^2+1} \right)^2=&\left(  \frac{2du^i(|u|^2+1)-2u^i\cdot2\sum_ju^jdu^j}{(|u|^2+1)^2}\right)^2\\
    =\frac{1}{(|u|^2+1)^4}&\left(4(du^i)^2(|u|^2+1)^2+4(u^i)^2\sum_{j,k}u^ju^kdu^jdu^k-8u^idu^i(|u|^2+1)\sum_ju^jdu^j \right).
\end{align*}
We also have that
\begin{align*}
    d\left( \frac{|u|^2-1}{|u|^2+1} \right)^2=\frac{16\sum_{j,k}u^ju^kdu^jdu^k}{(|u|^2+1)^4}.
\end{align*}
Then,
\begin{align*}
    (\sigma^{-1})^*\bar g&=d\left( \frac{|u|^2-1}{|u|^2+1} \right)^2+\sum_i d\left( \frac{2u^i}{|u|^2+1} \right)^2\\
    &=\left( \frac{2}{1+|u|^2} \right)^2\sum_i (du^i)^2
\end{align*}
where we brought in the sum over $i$ to cancel out sums and to cancel out an overall factor $|u|^2+1$.


\subsection*{Problem 2}

Suppose $(M,g)$ is a Riemannian manifold. A smooth curve $\gamma:J\to M$ is said to be a unit-speed curve if
$|\gamma'(t)|_g\equiv1$. We wish to show that every smooth curve with nowhere-vanishing velocity has a unit-speed
reparametrization. Let us define a reparametrization of $\gamma$, $\lambda:K\to M$ defined by $\lambda=\gamma\circ\phi$
where $\phi:K\to J$ with $K$ some other interval.
We compute in local coordinates
\begin{align*}
    |\lambda'(s)|_g&=\sqrt{g_{\lambda(s)}(\lambda'(s),\lambda'(s))}\\
    &=\sqrt{g_{\gamma(\phi(t))}(\dot\phi(t)\dot\gamma(\phi(t))\p/\p x,\dot\phi(t)\dot\gamma(\phi(t))\p/\p x)}\\
    &=\dot\phi(t)\sqrt{g_{\gamma(\phi(t))}(\dot\gamma(\phi(t))\p/\p x,\dot\gamma(\phi(t))\p/\p x)}\\
    &=\dot\phi(t)\sqrt{g_{\gamma(t)}(\gamma'(t),\gamma'(t))}\\
    &=\dot\phi(t)|\gamma'(t)|_g
\end{align*}
Consequently, if we reparametrize our curve by $\phi(t)=\int_0^t 1/|\gamma'(\tau)|_gd\tau$, we will have a unit-speed parametrization.
We know that this is a diffeomorphism (and thus a good reparametrization), as the velocity is nowhere-vanishing and
thus $1/|\gamma'(\tau)|_g$ is smooth (as the curve is smooth).

\subsection*{Problem 3}

We wish to show that the shortest path between two points in Euclidean space is a straight line segment.
Let us consider points $x,y\in\R^n$ along the $x$-axis; we can do this without loss of generality, as one can always
choose suitable coordinates. Now take a straight-line path $\gamma$ from $x$ to $y$ along the axis with length $L_1$ as well as some
other path $\lambda$ from $x$ to $y$, not necessarily along the axis, with length $L_2$:
\begin{align*}
    L_1&=\int_a^b\sqrt{\bar g(\gamma'(t),\gamma'(t))} dt\\
    &=\int_a^b\dot\gamma(t) dt=\gamma(b)-\gamma(a)
\end{align*}
On the other hand, we have that
\begin{align*}
    L_2&=\int_a^b\sqrt{\bar g(\lambda'(t),\lambda'(t))} dt\\
    &\geq\int_a^b\dot\lambda(t) dt=\lambda(b)-\lambda(a)
\end{align*}
where the inequality comes from dropping all terms under the square root except for the contribution to the dot product
from the $x$ components of $\lambda'(t)$. Note, however, that $\gamma(b)=\lambda(b)=y,\gamma(a)=\lambda(a)=x$ by construction,
and thus we have that $L_1\leq L_2$. Equality occurs only when the terms that we have dropped were zero in the first place,
i.e. $\lambda$ is the same curve as $\gamma$ up to reparametrization. Of course, the length will be the same, due to the
parametrization-invariance of length. Thus, $L_1<L_2$ except when $\lambda=\gamma$ (up to parametrization).

\subsection*{Problem 4}

Let $M=\R^2\setminus\left\{0 \right\}$, and let $g$ be the restriction to $M$ of the Euclidean metric $\bar g$. We wish
to show that there are points $p,q\in M$ for which there is no piecewise smooth curve segment $\gamma$ from $p$ to $q$
in $M$ with $L_g(\gamma)=d_g(p,q)$. Take $p=(0,1)$ and $q=(0,1)$. This distance in $\R^2$ would simply be 2. On $\R^2\setminus\left\{ 0 \right\}$,
however, it is not so obvious. Note first that that $d_g(p,q)\leq d_{\bar g}(p,q)=2$, as it could not possibly be greater -- the induced metric
is simply a restriction of $\bar g$. Let us now assume for the sake of contradiction that $d_g(p,q)<d_{\bar g}(p,q)$.
Then there must exist a piecewise smooth curve $\gamma$ in $M$ such that $L_g(\gamma)\leq(d_g(p,q)+d_{\bar g}(p,q))/2$.
But we know that $L_g(\gamma)=L_{\bar g}(\gamma)$. Using now the assumed inequality, we find $L_{g}(\gamma)<d_{\bar g}(p,q)$,
and reach a contradiction. Thus $d_g(p,q)=d_{\bar g}(p,q)=2$.

Now we wish to show that there is no piecewise smooth curve $\gamma$ from $p$ to $q$ that has a length of 2. It should be clear
that the curve must intersect the $x$-axis at some point in order to get from $p$ to $q$ (as defined above). Roughly 
speaking, since we know that the shortest curve between two points (where the straight line connecting them does not hit
the origin) is a straight line, we can replace the pieces of $\gamma$ with straight lines to get a $\tilde \gamma$ with
smaller length. However, now we may simply use the triangle inequality to show that the length of $\tilde \gamma$ will
always be greater than 2 (simply because the point at which the lines intersect the $x$-axis will never be able to hit the
origin). Consequently, there is no piecewise smooth curve on $M$ that achieves the desired length.

\subsection*{Problem 5}

Suppose $g=f(t)dt^2$ is a Riemannian metric on $\R$. First note that we must determine what we mean by distance in this metric:
\begin{align*}
    d(x,y)&=\inf_\gamma\int_a^b|\gamma'(t)|_g dt\\
    &=\inf_\gamma\int_a^b\sqrt{g_{\gamma(t)}dt^2(\gamma'(t),\gamma'(t))} dt\\
    &=\inf_\gamma\int_a^b\sqrt{f(\gamma(t))}\gamma'(t) dt\\
    &=\inf_\gamma\int_{\gamma(a)}^{\gamma(b)}\sqrt{f(s)} ds\\
    &=\int_{\gamma(a)}^{\gamma(b)}\sqrt{f(s)} ds\\
\end{align*}
where in the last step we have dropped the infimum, as the integral is insensitive to everything but the endpoints of $\gamma$
(this should be intuitively clear as we are constrained on $\R$ and thus the shortest path should be a straight-line path).

Let us assume that the integral $\int_0^\infty\sqrt{f(t)}dt$ converges.
We wish to show that there exists some Cauchy sequence on $\R$ (under this metric) that does not converge, i.e. the space
is not complete. Take for now some the sequence $\left\{ x_i=i \right\}$ with $i\in\N$. It should be clear that this sequence
diverges. Hence, it suffices to show that this sequence is Cauchy. We do so for $n,m\in\N$ with $n>m$:
\begin{align*}
    d(x_m,x_n)=\int_m^n\sqrt{f(s)} ds.
\end{align*}
If we take $n,m\to\infty$, this integral must go to zero. This can be shown rigorously by applying the fact that
if $\lim_{i\to\infty}a_i$ converges, then $\lim_{i\to\infty}a_{i+1}$ converges as well to this integral (after splitting it up)
and making use of the our assumption that the integral from 0 to $\infty$ converges (and thus ``smaller'' integrals converge as well).
Consequently, this sequence is Cauchy, and we are done (note that an analogous argument would have followed if we had assumed that
the other integral converged).

The converse is a bit easier. Let us assume that we have a Cauchy sequence $\left\{ x_i \right\}$ that does not converge. We
wish to show that one of the integrals converges. 
Again, take the example of the sequence of integers we used earlier. Since the sequence is Cauchy, we know that there exists an $N>0$
such that for $m,n>N$, the distance $d(x_m,x_n)<\varepsilon$, for arbitrary positive $\varepsilon=17$. Using the triangle
inequality, we find that
\begin{align*}
    d(0,x_n)&\leq d(0,x_m)+17\\
    17&\geq d(0,x_n)-d(x_m,0)
\end{align*}
Rewriting this in terms of integrals and taking $m\to-\infty,n\to\infty$ we find that
\begin{align*}
    \int_0^{x_n}\sqrt{f(s)} ds-\int_{x_m}^0\sqrt{f(s)} ds\leq 17\\
    \int_0^{\infty}\sqrt{f(s)} ds-\int_{-\infty}^0\sqrt{f(s)} ds\leq 17.
\end{align*}
Consequently, each of the integrals must converge, and we are done.



\subsection*{Problem 6}

Let $(M,g)$ be a Riemannian manifold, let $f\in C^\infty(M)$, and let $p\in M$ be a regular point of $f$.
\begin{enumerate}[(a)]
    \item Let us compute the angle between the gradient of $f$ at $p$ and any (unit) directional derivative, $X|_p$ at $p$:
        \begin{align*}
            \cos\theta=\frac{\langle \text{grad }f,X\rangle_g}{|\text{grad }f|_g|X|_g}&=\frac{Xf|_p}{|\text{grad }f|_g}\\
            |\text{grad }f|_g\cos\theta&=Xf|_p
        \end{align*}
        The left-hand side of the equation is maximized when $\theta=0$. But the right hand side, the directional derivative of
        $f$ in the direction of $X$, which represents the change of $f$, must also be maximized at $\theta=0$. Thus,
        the gradient of $f$ at $p$ points in the direction of greatest change at $p$ and has magnitude, seen above, to
        be equal to this change (the directional derivative).
    \item The level set of $f$ through $p$ forms a submanifold of $M$, since $p$ is a regular point of $f$. Simply by definition
        of a level set, $f$ must be constant over this submanifold. Now we find $|\text{grad }f|_g\cos\theta=Xf|_p=0$ for some
        directional derivative $X$ in the tangent space to the level set at $p$.
        But since the gradient cannot be zero (as $f$ is arbitrary), it must be that $\cos\theta=0$, and thus the gradient must
        be orthogonal to the level set of $f$ through $p$.
\end{enumerate}


\end{document}
