\documentclass{../../mathnotes}


\title{Differentiable Manifolds Problem Set 8}
\author{Nilay Kumar}
\date{Last updated: \today}


\begin{document}

\maketitle

\subsection*{Problem 1}

Let $m,n,k$ be positive integers with $k\leq\min(m,n)$. We wish to show that an $m\times n$ matrix $M$ has rank greater
than or equal to $k$ if and only if there exists a $k\times k$ submatrix with nonzero determinant.
First suppose that there exists a $k\times k$ submatrix with nonzero determinant. Then it is clear that
there are either (at least, as there could be others) $k$ linearly independent rows or columns of the matrix, as adding $m-k$ or $n-k$ components
to a linearly independent vector keeps them linearly independent. Thus $M$ has either at least $k$ linearly independent rows
or columns and thus has rank greater than or equal to $k$.
Conversely, suppose $M$ has rank greater than or equal to $k$, say $l\leq \min(m,n)$. Then, we can always rewrite $M$ in its canonical form
by changing bases, which as we showed in class, results in something that looks like (in block form):
\begin{align*}
    \left(\begin{array}[]{cc}
        I_l & 0 \\
        0 & 0
    \end{array}\right)
\end{align*}
where $I_l$ is the $l\times l$ identity. Of course, if $l=m$ or if $l=n$, some of the zero blocks will not be present.
Clearly this matrix has a $k\times k$ submatrix that has nonzero determinant, as the top $k\times k$ submatrix of the
$I_l$ block has determinant 1. 


\subsection*{Problem 2}

Let $M(m\times n)$ be the space of real $m\times n$ matrices as a smooth manifold and $M_k(m\times n)$ be the set
consisting of matrices with rank $k$. We wish to show that $M_k(m\times n)$ is an embedded submanifold of $M(m\times n)$
with codimension $(m-k)\times (n-k)$.

We proceed almost exactly as we did in class for the $k=1$ case.
Given a matrix of rank $k$, we know we can find a $k\times k$ submatrix with non-zero determinant.
Thus we can cover $M_k(m\times n)$ by $(m-k)(n-k)$ open sets $U_i$ (seen by counting). It suffices to show that $M_k(m\times n)\cap U_1$
is an embedded submanifold. Take some matrix $E$ in $M_k(m\times n)\cap U_1$ of the form 
\begin{align*}
    \left(\begin{array}[]{c|c}
        A & B \\ \hline
        C & D
    \end{array}\right)
\end{align*}
where $A$ is the $k\times k$ submatrix. We can apply row-reduction to obtain in the lower-right block a
term that must vanish for the rank of $E$ to be $k$:
\[D-CA^{-1}B=0_{(m-k)\times(n-k)}.\]
Hence, in order to realize matrices in $M_k(m\times n)\cap U_1$ as a level set of a smooth map with full rank, we define
a function $F:M_k(m\times n)\cap U_1\to M( (m-k)\times(n-k))$ that takes some matrix of the form $E$ to its $D-CA^{-1}B$
and then we have that $M_k(m\times n)\cap U_1=F^{-1}(0_{(m-k)\times(n-k)})$. To prove that this is an embedded
submanifold, we must show that $dF$ is full rank on $F^{-1}(0)$. To do this, consider the set of matrices in $F^{-1}(0)$
where the lower right block in of the form $D+tX$ for some arbitrary $X\in T_{F(p)}M( (m-k)\times (n-k))$. It's clear that applying
$F$ to such matrices will yield $D+tX-CA^{-1}B$, the differential of which, yields $X$. But since this hold for arbitrary $X$,
$dF$ must be surjective, and we are done.

We can compute the codimension as follows. Consider any $(k+1)\times(k+1)$ minor $A_{ij}$ of an element of $M_k(m\times n)$,
where the minor is labelled by its top-left position. Any such minor must have zero determinant: $\det A_{ij}=0$, again for
$1\leq i\leq m-k, 1\leq j\leq n-k$. Thus we have $(m-k)(n-k)$ linear equations in $mn$ unknowns. Hence, the codimension is
$(m-k)(n-k)$.


\subsection*{Problem 3}

We wish to show that any closed subset of a compact space is compact. Take a compact space $X$ and some open cover
$\mathcal{U}$ of a compact subset $A$. It is clear that $\mathcal{U}\cup (X-A)$ is an open cover of $X$ (as
$A$ is closed). But by the compactness of $X$, we find that this cover must have a finite subcover $\left\{U_1,\ldots,U_n\right\}\cup (X-A)$.
It follows that $\left\{ U_1,\ldots,U_n \right\}$ is an open over for $A$; thus, $A$ must be compact, as this cover is finite.

Next, we wish to show that any compact subset $A$ of a Hausdorff space $X$ is closed. Take some sequence $\left\{ a_i \right\}$ in $A$ that
converges to some $a\in X$. If we can show that $a\in A$, we are done, as this means $A$ contains all of its limit points. By compactness,
we know that $\left\{ a_i \right\}$ has a subsequence $\left\{ a_{i_k} \right\}$ that converges to $b\in A$. Suppose $a\neq b$. Then, by the
Hausdorff property, we can find disjoint open sets $U,V$ such that $a\in U$ and $b\in V$. By the definition of convergence, we know that
far enough out in $a_i$ we will be guaranteed to be in $U$, and far enough out in $a_{i_k}$ we will be guaranteed to be in $V$. But
since $a_{i_k}$ is a subsequence of $a_i$, this means after some point, we will be both in $U$ and $V$. This is a contradiction as $U$ and $V$
are disjoint. Hence $a=b$ and $A$ must be closed.


\subsection*{Problem 4}

We wish to show that the figure-eight curve is not an embedded submanifold of $\R^2$. Note that in order to be an
embedded submanifold, the figure-eight curve on its own would have to be at least a topological manifold of its own right
with the subspace topology.
However, with this topology, the curve is not locally homeomorphic to $\R$ at the self-intersection point;
take some open ball in $\R^2$ about the origin (an open set in the subspace topology). We can see that the part of the curve contained 
in this ball is not homeomorphic to $\R$ here (although it is away from the origin) as follows.
If we remove the origin we are left with four disconnected components of the curve, whereas in $\R$,
removing a point from an open set yields only 2 disconnected components. Since the number of components is a topological invariant, there
cannot be a homeomorphism, and thus the curve is not a topological manifold.


\subsection*{Problem 5}

We wish to show that the dense curve on the torus from the previous homework is not an embedded submanifold of $\R^2$.
Similar to the previous problem, we can show that, with the subspace topology, the curve $\gamma$ cannot possibly
be a topological manifold, and thus not a embedded submanifold. If we work with the flat torus, we can show that
it will not be locally homeomorphic to a Euclidean space as follows.
First note that the intersection of the curve and some open set $B$ will look like a subset of $\R\times\Q$ as there are countably many ``lines'' inside $B$
(this can be seen by examining the space pre-quotient and noting that we will have a countable number of squares in the space, and the
number of squares is the total number of lines in the flat torus); hence, this ball is homeomorphic neither to $\R$ nor $\R^2$. 


\subsection*{Problem 6}
Let $\Phi:\R^2\to\R$ be defined by $\Phi(x,y)=x^2-y^2$.
\begin{enumerate}[(a)]
    \item We wish to show that $\Phi^{-1}(0)$ is not an embedded submanifold of $\R^2$. But note that this is
        very similar to the case of problem 4 above -- we have two lines that intersect at the origin. In the subspace
        topology, the intersection of $\Phi^{-1}(0)$ and any open set containing the origin is not homeomorphic to $\R^1$ (which we want because other parts of the
        space are), and thus the level set is not locally Euclidean.
    \item Let us give $\Phi^{-1}(0)$ a different topology in order to make it an immersed submanifold of $\R^2$. On
        one branch of the set, call it $X$, we can take the open sets to be intervals as usual, but on the other branch, call it $Y$,
        we take the open sets to be all intervals that don't contain the origin. We can now very easily give this topology
        a smooth structure by treating it as a disjoint union of $\R$ and $\R\setminus \left\{ 0 \right\}$, where the coordinates
        on each branch are simply given by projections to the $x$-axis (we of course need 2 coordinate charts, one for each branch).
        It remains now to check that the inclusion map is
        a smooth immersion. The coordinate representation for the chart $\phi$ is just given: $\id\circ\iota\circ\phi^{-1}(x)=(x,x)$ (this holds for the second
        chart as well). Clearly this is smooth
        and the differential has full rank everywhere, and thus we have a smooth immersion, and we are done.
    \item Here we follow Lee's Example 5.45.
        It is easy to check that $\psi^{-1}(0)\setminus\left\{ (0,0) \right\}$ is an embedded 1-dimensional submanifold of $\R^2$,
        so if $\psi^{-1}(0)$ is to be a smooth submanifold at all, it must be 1-dimensional. Suppose there were some smooth manifold
        structure on $\psi^{-1}(0)$ making it into an immersed submanifold. Then $T_{(0,0)}\psi^{-1}(0)$ would be a 1-dimensional
        subspace of $T_{(0,0)}\R^2$, so by Proposition 5.35 of Lee, there would be a smooth curve $\gamma:(-\varepsilon,\varepsilon)\to\R^2$
        whose image lies in $\psi^{-1}(0)$, and that satisfies $\gamma(0)=(0,0)$ and $\gamma'(0)\neq0$. Writing $\gamma(t)=(x(t),y(t))$,
        we see that $y(t)$ takes a global minimum at $t=0$, so $\dot y(0)=0$ (evident from the plot). On the other hand, because every
        point $(x,y)\in\psi^{-1}(0)$ satisfies $x^2=y^3$, taking derivatives, we conclude that $\dot x(0)=\dot y(0)=0$, which is a contradiction, as
        we see that the differential is not full rank. Consequently, there can be no such manifold structure.
        Note that this implies that this cannot be an embedded submanifold, either, as that is a stronger condition.
\end{enumerate}


\end{document}
