\documentclass{../../mathnotes}


\title{Differentiable Manifolds Problem Set 4}
\author{Nilay Kumar}
\date{Last updated: \today}


\begin{document}

\maketitle

\subsection*{Problem 1}

Let $\pi: E\to M$ be a smooth vector bundle of rank $k$ over a smooth manifold $M$. Suppose that $\left\{ U_\alpha \right\}_{\alpha\in A}$
is an open cover of $M$, and for each $\alpha\in A$ we are given a smooth local trivialization $\Phi_\alpha:\pi^{-1}(U_\alpha)\to U_\alpha\times\R^k$
of $E$. For each $\alpha,\beta\in A$ such that $U_\alpha\cap U_\beta\neq\varnothing$, let $\tau_{\alpha\beta}:U_\alpha\cap U_\beta\to GL(k,\R)$ be
the transition function defined by $\Phi_\alpha\circ\Phi_\beta^{-1}(p,v)=(p,\tau_{\alpha\beta}(p)v)$.

Let $U_\alpha,U_\beta,U_\gamma$ be elements of the open cover. We know that $\phi_\alpha\circ\phi^{-1}_\beta(p,v)=(p,\tau_{\alpha\beta}(p)v)$
and $\phi_\beta\circ\phi^{-1}_\gamma(p,v)=(p,\tau_{\beta\gamma}(p)v)$ and $\phi_\alpha\circ\phi^{-1}_\gamma(p,v)=(p,\tau_{\alpha\gamma}(p)v)$.
In $U_\alpha\cap U_\beta\cap U_\gamma$, we can compose the first two maps to obtain a map $\phi_\alpha\circ\phi^{-1}_\gamma(p,v)=(p,\tau_{\alpha\beta}(p)\tau_{\beta\gamma}(p)v)$.
Consequently
\begin{align*}
    \phi_\alpha\circ\phi^{-1}_\gamma(p,v)=(p,\tau_{\alpha\beta}(p)\tau_{\beta\gamma}(p)v)=(p,\tau_{\alpha\gamma}(p),v),
\end{align*}
and thus
\begin{align*}
    \tau_{\alpha\beta}(p)\tau_{\beta\gamma}(p)=\tau_{\alpha\gamma}(p)
\end{align*}

\subsection*{Problem 2}

Let $V$ be a finite-dimensional real vector space, and let $G_k(V)$ be the Grassmanian of $k$-dimensional subspaces of $V$. Let $T$
be the subset of $G_k(V)\times V$ defined by
\begin{align*}
    T=\left\{ (S,v)\in G_k(V)\times V:v\in S \right\}.
\end{align*}

\subsection*{Problem 3}

Let $E$ be the tautological vector bundle over $G_1(\R^2)$.

\subsection*{Problem 4}

Let $M$ be a smooth manifold and $p$ be a point of $M$. Let $\ell_p$ denote the subspace of $C^\infty(M)$ consisting of smooth functions
that vanish at $p$, and let $\ell_p^2$ be the subspace of $\ell_p$ spanned by functions of the form $fg$ for some $f,g\in\ell_p$.
\begin{enumerate}[(a)]
    \item We wish to show that $h\in\ell^2_p$ if and only if in any smooth local coordinates, its first-order Taylor
        polynomial at $p$ is zero. First note that the coordinate representation in any coordinate chart $\phi$ is $\hat h=h\circ\phi^{-1}$.
        The backwards implication is easy to see -- if $h\in\ell^2_p$, we know that $h=fg$ for some $f,g$ vanishing at $p$, then
        \begin{align*}
            \nabla\hat h=\nabla\left( (fg)\circ\phi^{-1} \right)=\nabla(f\circ\phi^{-1})\cdot (g\circ\phi^{-1})+\nabla(g\circ\phi^{-1})\cdot (f\circ\phi^{-1}).
        \end{align*}
        Both terms vanish at $\phi(p)$, as $f,g$ vanish at $p$. $\hat h$ vanishes at $p$ as well, for the same reason.
        Consequently, the first-order Taylor polynomial vanishes at $p$ in any coordinate chart.

        We can show the converse as follows. If the first order Taylor polynomial for $h$ is zero in any coordinate chart $\phi$, by Taylor's theorem
        with remainder, we can write (in some open ball about $\phi(p)$)
        \begin{align*}
            \hat h(\vec x)=\sum_i g_i(\vec x)(x^i-\phi(p)^i),
        \end{align*}
        where $g_i(\phi(p))=0$. But this is simply the form of a dot product of two vector-valued functions, each of is in $\ell_p$.
        Thus $\hat h\in\ell_p^2$.
    \item We define a map $\Phi:\ell_p\to T_p^*M$ by setting $\Phi(f)=df_p$. If we restrict $\Phi$ to $\ell_p^2$,
        \begin{align*}
            \Phi(f)=\Phi(gh)=d(gh)_p=g(p)dh_p+h(p)dg_p
        \end{align*}
        which is simply zero, as $g,h$ are $\ell_p$ functions, i.e. they vanish at $p$. To show that $\Phi$ descends to a vector
        space isomorphism from $\ell_p/\ell_p^2$ to $T^*M$, we first show that $\Phi$ is surjective, and then use the first isomorphism theorem.
        $\Phi$ is surjective because the functions $x^i$ in any chart centered at $p$ are zero at $p$ and thus in $\ell_p$ (by centering).
        They get sent by $\Phi$ to $dx^i$, the basis for $T^*M$ and thus, by linearity of $\Phi$, $\Phi$ is surjective.
        Furthermore, it should be clear that $\ker\Phi=\ell_p^2$ -- the above computation showed that $\ell_p^2\subset\ker\Phi$ and the following
        computation shows that $\ker\Phi\subset\ell_p^2$:
        \begin{align*}
            \Phi(f)=df_p=\frac{\partial f}{\partial x^i}dx^i=0
        \end{align*}
        because by linear independence of $dx^i$, the gradient of $f$ must be zero at $p$, and by part a, then, $f\in\ell^2_p$.
\end{enumerate}

\subsection*{Problem 5}

Let $f:\R^3\to\R$ be the function $f(x,y,z)=x^2+y^2+z^2$, and let $F:\R^2\to\R^3$ be the following map
\begin{align*}
    F(u,v)=\left( \frac{2u}{u^2+v^2+1},\frac{2v}{u^2+v^2+1}, \frac{u^2+v^2-1}{u^2+v^2-1} \right).
\end{align*}
First note that $df=2xdx+2ydy+2zdz$.
We then compute
\begin{align*}
    F^*df&=2\frac{2u}{u^2+v^2+1}d(\frac{2u}{u^2+v^2+1})+2\frac{2v}{u^2+v^2+1}d(\frac{2v}{u^2+v^2+1})\\
    &+2\frac{u^2+v^2-1}{u^2+v^2-1}d(\frac{u^2+v^2-1}{u^2+v^2-1})\\
    &=\frac{4u}{u^2+v^2+1}\left( \frac{2(u^2+v^2+1)-4u^2}{(u^2+v^2+1)^2}du-\frac{4uv}{(u^2+v^2+1)^2}dv \right)\\
    &+\frac{4v}{u^2+v^2+1}\left( \frac{2(u^2+v^2+1)-4v^2}{(u^2+v^2+1)^2}dv-\frac{4uv}{(u^2+v^2+1)^2}du \right)\\
    &+\frac{2(u^2+v^2-1)}{u^2+v^2+1}\left(\frac{4u}{(u^2+v^2+1)^2}du-\frac{4v}{(u^2+v^2+1)^2}dv \right)\\
    &=0,
\end{align*}
Alternatively, we can compute
\begin{align*}
    d(f\circ F)&=d\left(\frac{4u^2}{(u^2+v^2+1)^2}+\frac{4v^2}{(u^2+v^2+1)^2}+\frac{(u^2+v^2-1)^2}{(u^2+v^2+1)^2}\right)\\
    &=d(1)=0,
\end{align*}
which yields the same result, as expected.

\subsection*{Problem 6}

\begin{enumerate}[(a)]
    \item
        In this case, we compute
        \begin{align*}
            df=d\left( \frac{x}{x^2+y^2} \right)=\frac{x^2+y^2-2x^2}{(x^2+y^2)^2}dx+\frac{-2xy}{(x^2+y^2)^2}dy
        \end{align*}
        Note that for $df$ to be zero, the second term must be zero, but this only happens when $x$ or $y$ is zero.
        The numerator of the first term is zero when $y^2-x^2=0$, i.e. $y=\pm x$. These two conditions together hold
        only at the origin, which is not a point in our manifold, and thus $df$ is nowhere zero on $M$.
    \item In polar coordinates we have
        \begin{align*}
            df=d\left( \frac{r\cos\theta}{r^2} \right)=\frac{-\cos\theta}{r}dr-\frac{\sin\theta}{r}d\theta
        \end{align*}
        Note that, again, $df$ is zero nowhere on the manifold, as $\cos\theta=\sin\theta=0$ holds nowhere.
    \item If we take $f(p)=z(p)$ on $M=S^2\subset\R^3$,
        \begin{align*}
            d\left(\frac{u^2+v^2-4}{u^2+v^2+4}\right)&=\frac{2u(u^2+v^2+4)-2u(u^2+v^2-4)}{(u^2+v^2+4)^2}du\\
            &+\frac{2v(u^2+v^2+4)-2v(u^2+v^2-4)}{(u^2+v^2+4)^2}dv\\
            &=\frac{16(udu+vdv)}{(u^2+v^2+4)^2}
        \end{align*}
        This is zero only when $u=v=0$, i.e. at the north and south poles (depending on the stereographic charts one has chosen).
    \item If $M=\R^n$ and $f(x)=|x|^2=\sum_i (x^i)^2$, we have
        \begin{align*}
            df=\sum_i^n2x^i dx^i,
        \end{align*}
        which is clearly zero only at the origin.
\end{enumerate}

\subsection*{Problem 7}

\begin{enumerate}[(a)]
    \item Using the hint, if we define $\omega_x(v)=X_x\cdot v$, we can compute the components of $\omega$ in the case that $X=\text{grad} f$:
        \[\omega_i=X\cdot \frac{d}{dx^i}=\frac{\partial f}{\partial x^i}\]
        and thus $\omega=\frac{\partial f}{\partial x^i}dx^i=df$. Using the hint to rewrite the integral, we have
        \[\int_a^bX_{\gamma(t)}\cdot \gamma'(t)dt=\int_a^b\omega_{\gamma(t)}(\gamma'(t))dt=\int_\gamma\omega=\int_\gamma df\]
        and by what we know about integrals of exact forms, this must be zero on piecewise smooth closed curves. This is exactly
        what conservative is defined for vector fields, and thus $X=\text{grad} f$ is conservative.

        Conversely, if we know that $X$ is a conservative vector field, the integral above of $\omega$ on a piecewise smooth closed curve (in $U$) always vanishes.
        This implies that $\omega$ is conservative, which we know implies that $\omega$ is exact: $\omega=\frac{\p f}{\p x^i}dx^i$ for some $f$. Consequently, 
        since $\omega$ is defined as the dot product of $X$ with the argument, $X$ is the gradient of this $f$.
    \item Suppose $n=3$. If $X$ is conservative, we know that it can be written as $X=\text{grad} f$. Then $\omega$ is exact and closed. Consequently
        $\p \omega_i/\p x^j=\p\omega_j/\p x^i$ and thus $\p X^i/\p x^j=\p X^j/\p x^i$ and thus $\text{curl\;} X=0$ by the formula given.
    \item If $\text{curl\;} X=0$, the associated $\omega$ must be closed. Since every closed covector field is exact on some star-shaped $U$, by
        the above computations, $X$ must be conservative on this star-shaped region. Conversely, if $U$ is star-shaped and $X$ is conservative
        on $U$, by the previous part of the problem we know that $\text{curl\;} X=0$.
\end{enumerate}

\subsection*{Problem 8}

Let $V=W=\R^2$ anything of the form $v\otimes w$ can be written as
\begin{align*}
    (ae_1+be_2)\otimes (ce_1+de_2)=ace_1\otimes e_1+bde_2\otimes e_2+ade_1\otimes e_2+bce_2\otimes e_1
\end{align*}
Now take $x=e_1\otimes e_2+e_2\otimes e_1\in V\otimes W$. It is clear that this cannot be written as the above form,
as either $a$ or $c$ and either $b$ or $d$ would have to be zero, reducing the above expression to zero instead of $x$.

\subsection*{Problem 9}

\subsection*{Problem 10}

Let $M$ be a smooth $n$-manifold, and let $A$ be a smooth covariant $k$-tensor field on $M$. If $(U,(x^i))$ and $(\tilde U,(\tilde x^j))$
are overlapping smooth charts on $M$, we can write
\begin{align*}
    A=A_{i_1\ldots i_k}dx^{i_1}\otimes \cdots \otimes dx^{i_k}=\tilde A_{j_1\ldots j_k}d\tilde x^{j_1}\otimes \cdots \otimes d\tilde x^{j_k}
\end{align*}
Using the transformation rule for covectors, we can write
\begin{align*}
    \omega_i=\frac{\p \tilde x^j}{\p x^i}(p) \tilde \omega_j
\end{align*}
and thus each factor in the tensor product pulls out such a derivative by linearity of the tensor product, and we can relate
the tensor components as
\begin{align*}
    A_{i_1\ldots i_k}=\frac{\p \tilde x^{j_1}}{\p x_{i_1}}\cdots\frac{\p \tilde x^{j_k}}{\p x_{i_k}} \tilde A_{j_1\ldots j_k}
\end{align*}

\end{document}
