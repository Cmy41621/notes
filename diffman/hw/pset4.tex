\documentclass{../../mathnotes}


\title{Differentiable Manifolds Problem Set 4}
\author{Nilay Kumar}
\date{Last updated: \today}


\begin{document}

\maketitle

\subsection*{Problem 1}

Let $\pi: E\to M$ be a smooth vector bundle of rank $k$ over a smooth manifold $M$. Suppose that $\left\{ U_\alpha \right\}_{\alpha\in A}$
is an open cover of $M$, and for each $\alpha\in A$ we are given a smooth local trivialization $\Phi_\alpha:\pi^{-1}(U_\alpha)\to U_\alpha\times\R^k$
of $E$. For each $\alpha,\beta\in A$ such that $U_\alpha\cap U_\beta\neq\varnothing$, let $\tau_{\alpha\beta}:U_\alpha\cap U_\beta\to GL(k,\R)$ be
the transition function defined by $\Phi_\alpha\circ\Phi_\beta^{-1}(p,v)=(p,\tau_{\alpha\beta}(p)v)$.

Let $U_\alpha,U_\beta,U_\gamma$ be elements of the open cover. We know that $\phi_\alpha\circ\phi^{-1}_\beta(p,v)=(p,\tau_{\alpha\beta}(p)v)$
and $\phi_\beta\circ\phi^{-1}_\gamma(p,v)=(p,\tau_{\beta\gamma}(p)v)$ and $\phi_\alpha\circ\phi^{-1}_\gamma(p,v)=(p,\tau_{\alpha\gamma}(p)v)$.
In $U_\alpha\cap U_\beta\cap U_\gamma$, we can compose the first two maps to obtain a map $\phi_\alpha\circ\phi^{-1}_\gamma(p,v)=(p,\tau_{\alpha\beta}(p)\tau_{\beta\gamma}(p)v)$.
Consequently
\begin{align*}
    \phi_\alpha\circ\phi^{-1}_\gamma(p,v)=(p,\tau_{\alpha\beta}(p)\tau_{\beta\gamma}(p)v)=(p,\tau_{\alpha\gamma}(p),v),
\end{align*}
and thus
\begin{align*}
    \tau_{\alpha\beta}(p)\tau_{\beta\gamma}(p)=\tau_{\alpha\gamma}(p)
\end{align*}

\subsection*{Problem 2}

Let $V$ be a finite-dimensional real vector space, and let $G_k(V)$ be the Grassmanian of $k$-dimensional subspaces of $V$. Let $T$
be the subset of $G_k(V)\times V$ defined by
\begin{align*}
    T=\left\{ (S,v)\in G_k(V)\times V:v\in S \right\}.
\end{align*}
We wish to show that $\pi:T\to G_k(V)$ is a smooth rank-$k$ subbundle of the product bundle $G_k(V)\times V\to G_k(V)$

Let us define an open cover consisting of $\binom{n}{k}$ open sets $\{U_i\}$ of $G_k(V)$ to be the $\binom{n}{k}$ $k$-dimensional vector subspaces of $V$
such that the intersection of $U_i$ with any $n-k$ dimensional vector space is $\{0\}$. Let us choose a basis for $S$ to be $(e_{i_1}, \dots, e_{i_l})$.
Next consider the projection map $\pi : G_k \times V \to G_k(V)$ and its restriction $\pi|_T = \tilde{\pi}$.
We construct a local trivialization $\Phi : \tilde{\pi}^{-1}(U_i) \to U_i \times \R^k$ defined by $\Phi(S,v) = (S, v^{i_1}, \dots, v^{i_k})$, which
is clearly a diffeomorphism.
We now wish to check that these local trivializations are smoothly compatible. Choose any $U_j$ such that $U_i \cap U_j \neq 0$. Let $S$ be in this intersection, and define a similar $\Psi : \tilde{\pi}^{-1}(U_j) \to U_j \times \R^k$.
Let the basis here to be $(x^{i_1}, \dots, x^{i_k})$. The transition map $\Psi \circ \Phi^{-1}(U_i \cap U_j)$ is defined by
\[ \Psi \circ \Phi^{-1}(S, v^{i_1}, \dots, v^{i_k}) = (S, \tilde{v}^{i_1}, \dots, \tilde{v}^{i_k}), \] where $(\tilde{v}^{i_1}, \dots, \tilde{v}^{i_k})$
is the coordinate representation of $v$ in the $x^{i_j}$ basis.
Since this is a change of basis, the restriction of $\Phi$ to $\tilde{\pi}^{-1}(S)$ to an element $S \in U$ is a vector space isomorphism.
Thus $T$ is a bundle and thus is a subbundle of $G_k(v) \times V$. 

\subsection*{Problem 3}

Let $T$ be the tautological vector bundle over $G_1(\R^2)=\R P^1$. We wish to show $T$ is smoothly isomorphic
to the Mobius bundle.

First note that $\R P^1$ is the space of lines going through the origin of $\R^2$. This topology
is in fact homeomorphic to that of a circle, as one can think of the projective line in terms of unit vectors in the plane, where
pairs of antiparallel vectors are quotiented out; i.e. the right half of the unit circle, with the north pole identified with the south
pole, which is clearly homeomorphic to the circle. Since the same smooth structure can be constructed on these two homeomorphic topologies,
it should be clear that $\R P^1$ is diffeomorphic to $S^1$, and thus that the base manifolds of our vector bundles are diffeomorphic.

Now, since the fibers of each bundle are both 1-dimensional, if we can show that there are local trivializations
(defined on the same charts on the base manifolds) that have identical transition functions. The charts on our base manifold $S^1$
(or $\R P^1$) that we will use for constructing local trivializations of both $T$ and the Mobius bundle are as follows. Given
the unit circle, take open sets $U$ and $V$ such that $U$ covers the bottom half of the circle, $V$ covers the top half, and $U$
intersects $V$ in small neighborhoods at the east and west poles.

%Note that the west pole in the case of $\R P^1$ is actually the
%simultaneously the north and south pole of the right-half-circle constructed in the previous paragraph; this distinction does
%not change anything in the construction of the trivializations, of course, due to the diffeomorphism.

For the Mobius bundle it should be clear that if we take local trivializations $\Phi',\Psi'$ to $U\times\R^1,V\times\R^1$,
the transition function will be -1 at one of the poles, say the west pole, and +1 at the other (by construction of the Mobius
bundle, as we saw on the last homework). Now it remains to show that we can construct similar local trivializations
of $T$: $\Phi,\Psi$. For both charts $(U,\Phi),(V,\Psi)$, we take the unit vector in the direction of the line in $\R P^1$
to to parameterize points in the base manifold. The coordinate along the line attached can be given by some real number, which,
when multiplied by the unit vector gives you the suitable point on the line. Note that where the two charts overlap on the
east pole, the unit vectors that represent a point on the base manifold are the same, i.e. they point in the same direction, and
thus the transition function is uniformly +1 on this overlap. However, on the west pole, due to how the north pole and the south
pole have been quotiented out, the same unit vector is represented in different coordinate charts with a transition factor of -1;
in one chart the unit vector is pointing upwards, but in the other it points downwards, and thus the bundle has a ``twist'' in it.
But this is exactly the transition function (the twisting) that is used when constructing the Mobius bundle, and thus $T$ must
be smoothly isomorphic to the Mobius bundle.

\subsection*{Problem 4}

Let $M$ be a smooth manifold and $p$ be a point of $M$. Let $\ell_p$ denote the subspace of $C^\infty(M)$ consisting of smooth functions
that vanish at $p$, and let $\ell_p^2$ be the subspace of $\ell_p$ spanned by functions of the form $fg$ for some $f,g\in\ell_p$.
\begin{enumerate}[(a)]
    \item We wish to show that $h\in\ell^2_p$ if and only if in any smooth local coordinates, its first-order Taylor
        polynomial at $p$ is zero. First note that the coordinate representation in any coordinate chart $\phi$ is $\hat h=h\circ\phi^{-1}$.
        The backwards implication is easy to see -- if $h\in\ell^2_p$, we know that $h=fg$ for some $f,g$ vanishing at $p$, then
        \begin{align*}
            \nabla\hat h=\nabla\left( (fg)\circ\phi^{-1} \right)=\nabla(f\circ\phi^{-1})\cdot (g\circ\phi^{-1})+\nabla(g\circ\phi^{-1})\cdot (f\circ\phi^{-1}).
        \end{align*}
        Both terms vanish at $\phi(p)$, as $f,g$ vanish at $p$. $\hat h$ vanishes at $p$ as well, for the same reason.
        Consequently, the first-order Taylor polynomial vanishes at $p$ in any coordinate chart.

        We can show the converse as follows. If the first order Taylor polynomial for $h$ is zero in any coordinate chart $\phi$, by Taylor's theorem
        with remainder, we can write (in some open ball about $\phi(p)$)
        \begin{align*}
            \hat h(\vec x)=\sum_i g_i(\vec x)(x^i-\phi(p)^i),
        \end{align*}
        where $g_i(\phi(p))=0$. But this is simply the form of a dot product of two vector-valued functions, each of is in $\ell_p$.
        Thus $\hat h\in\ell_p^2$.
    \item We define a map $\Phi:\ell_p\to T_p^*M$ by setting $\Phi(f)=df_p$. If we restrict $\Phi$ to $\ell_p^2$,
        \begin{align*}
            \Phi(f)=\Phi(gh)=d(gh)_p=g(p)dh_p+h(p)dg_p
        \end{align*}
        which is simply zero, as $g,h$ are $\ell_p$ functions, i.e. they vanish at $p$. To show that $\Phi$ descends to a vector
        space isomorphism from $\ell_p/\ell_p^2$ to $T^*M$, we first show that $\Phi$ is surjective, and then use the first isomorphism theorem.
        $\Phi$ is surjective because the functions $x^i$ in any chart centered at $p$ are zero at $p$ and thus in $\ell_p$ (by centering).
        They get sent by $\Phi$ to $dx^i$, the basis for $T^*M$ and thus, by linearity of $\Phi$, $\Phi$ is surjective.
        Furthermore, it should be clear that $\ker\Phi=\ell_p^2$ -- the above computation showed that $\ell_p^2\subset\ker\Phi$ and the following
        computation shows that $\ker\Phi\subset\ell_p^2$:
        \begin{align*}
            \Phi(f)=df_p=\frac{\partial f}{\partial x^i}dx^i=0
        \end{align*}
        because by linear independence of $dx^i$, the gradient of $f$ must be zero at $p$, and by part a, then, $f\in\ell^2_p$.
        We now use the first isomorphism theorem, and we are done.
\end{enumerate}

\subsection*{Problem 5}

Let $f:\R^3\to\R$ be the function $f(x,y,z)=x^2+y^2+z^2$, and let $F:\R^2\to\R^3$ be the following map
\begin{align*}
    F(u,v)=\left( \frac{2u}{u^2+v^2+1},\frac{2v}{u^2+v^2+1}, \frac{u^2+v^2-1}{u^2+v^2-1} \right).
\end{align*}
First note that $df=2xdx+2ydy+2zdz$.
We then compute
\begin{align*}
    F^*df&=2\frac{2u}{u^2+v^2+1}d(\frac{2u}{u^2+v^2+1})+2\frac{2v}{u^2+v^2+1}d(\frac{2v}{u^2+v^2+1})\\
    &+2\frac{u^2+v^2-1}{u^2+v^2-1}d(\frac{u^2+v^2-1}{u^2+v^2-1})\\
    &=\frac{4u}{u^2+v^2+1}\left( \frac{2(u^2+v^2+1)-4u^2}{(u^2+v^2+1)^2}du-\frac{4uv}{(u^2+v^2+1)^2}dv \right)\\
    &+\frac{4v}{u^2+v^2+1}\left( \frac{2(u^2+v^2+1)-4v^2}{(u^2+v^2+1)^2}dv-\frac{4uv}{(u^2+v^2+1)^2}du \right)\\
    &+\frac{2(u^2+v^2-1)}{u^2+v^2+1}\left(\frac{4u}{(u^2+v^2+1)^2}du-\frac{4v}{(u^2+v^2+1)^2}dv \right)\\
    &=0,
\end{align*}
Alternatively, we can compute
\begin{align*}
    d(f\circ F)&=d\left(\frac{4u^2}{(u^2+v^2+1)^2}+\frac{4v^2}{(u^2+v^2+1)^2}+\frac{(u^2+v^2-1)^2}{(u^2+v^2+1)^2}\right)\\
    &=d(1)=0,
\end{align*}
which yields the same result, as expected.

\subsection*{Problem 6}

\begin{enumerate}[(a)]
    \item
        In this case, we compute
        \begin{align*}
            df=d\left( \frac{x}{x^2+y^2} \right)=\frac{x^2+y^2-2x^2}{(x^2+y^2)^2}dx+\frac{-2xy}{(x^2+y^2)^2}dy
        \end{align*}
        Note that for $df$ to be zero, the second term must be zero, but this only happens when $x$ or $y$ is zero.
        The numerator of the first term is zero when $y^2-x^2=0$, i.e. $y=\pm x$. These two conditions together hold
        only at the origin, which is not a point in our manifold, and thus $df$ is nowhere zero on $M$.
    \item In polar coordinates we have
        \begin{align*}
            df=d\left( \frac{r\cos\theta}{r^2} \right)=\frac{-\cos\theta}{r}dr-\frac{\sin\theta}{r}d\theta
        \end{align*}
        Note that, again, $df$ is zero nowhere on the manifold, as $\cos\theta=\sin\theta=0$ holds nowhere.
    \item If we take $f(p)=z(p)$ on $M=S^2\subset\R^3$,
        \begin{align*}
            d\left(\frac{u^2+v^2-4}{u^2+v^2+4}\right)&=\frac{2u(u^2+v^2+4)-2u(u^2+v^2-4)}{(u^2+v^2+4)^2}du\\
            &+\frac{2v(u^2+v^2+4)-2v(u^2+v^2-4)}{(u^2+v^2+4)^2}dv\\
            &=\frac{16(udu+vdv)}{(u^2+v^2+4)^2}
        \end{align*}
        This is zero only when $u=v=0$, i.e. at the north and south poles (depending on the stereographic charts one has chosen).
    \item If $M=\R^n$ and $f(x)=|x|^2=\sum_i (x^i)^2$, we have
        \begin{align*}
            df=\sum_i^n2x^i dx^i,
        \end{align*}
        which is clearly zero only at the origin.
\end{enumerate}

\subsection*{Problem 7}

\begin{enumerate}[(a)]
    \item Using the hint, if we define $\omega_x(v)=X_x\cdot v$, we can compute the components of $\omega$ in the case that $X=\text{grad} f$:
        \[\omega_i=X\cdot \frac{d}{dx^i}=\frac{\partial f}{\partial x^i}\]
        and thus $\omega=\frac{\partial f}{\partial x^i}dx^i=df$. Using the hint to rewrite the integral, we have
        \[\int_a^bX_{\gamma(t)}\cdot \gamma'(t)dt=\int_a^b\omega_{\gamma(t)}(\gamma'(t))dt=\int_\gamma\omega=\int_\gamma df\]
        and by what we know about integrals of exact forms, this must be zero on piecewise smooth closed curves. This is exactly
        what conservative is defined for vector fields, and thus $X=\text{grad} f$ is conservative.

        Conversely, if we know that $X$ is a conservative vector field, the integral above of $\omega$ on a piecewise smooth closed curve (in $U$) always vanishes.
        This implies that $\omega$ is conservative, which we know implies that $\omega$ is exact: $\omega=\frac{\p f}{\p x^i}dx^i$ for some $f$. Consequently, 
        since $\omega$ is defined as the dot product of $X$ with the argument, $X$ is the gradient of this $f$.
    \item Suppose $n=3$. If $X$ is conservative, we know that it can be written as $X=\text{grad} f$. Then $\omega$ is exact and closed. Consequently
        $\p \omega_i/\p x^j=\p\omega_j/\p x^i$ and thus $\p X^i/\p x^j=\p X^j/\p x^i$ and thus $\text{curl\;} X=0$ by the formula given.
    \item If $\text{curl\;} X=0$, the associated $\omega$ must be closed. Since every closed covector field is exact on some star-shaped $U$, by
        the above computations, $X$ must be conservative on this star-shaped region. Conversely, if $U$ is star-shaped and $X$ is conservative
        on $U$, by the previous part of the problem we know that $\text{curl\;} X=0$.
\end{enumerate}

\subsection*{Problem 8}

Let $V=W=\R^2$ anything of the form $v\otimes w$ can be written as
\begin{align*}
    (ae_1+be_2)\otimes (ce_1+de_2)=ace_1\otimes e_1+bde_2\otimes e_2+ade_1\otimes e_2+bce_2\otimes e_1
\end{align*}
Now take $x=e_1\otimes e_2+e_2\otimes e_1\in V\otimes W$. It is clear that this cannot be written as the above form,
as either $a$ or $c$ and either $b$ or $d$ would have to be zero, reducing the above expression to zero instead of $x$.

\subsection*{Problem 9}

\begin{enumerate}[(a)]
    \item Let $\alpha$ be a covariant $k$-tensor on a finite-dimensional real vector space $V$. We wish to show that
        $\text{Sym\;}\alpha$ is the unique symmetric $k$-tensor satisfying
        \begin{align*}
            (\text{Sym\;}\alpha)(v,\ldots,v)=\alpha(v,\ldots, v).
        \end{align*}
        First note that 
        \[\text{Sym\;}\alpha(v,\ldots,v)=\frac{1}{k!}\sum_{\sigma\in S_k}\alpha(v,\ldots, v)=k!\cdot\frac{1}{k!}\alpha(v,\ldots,v)=\alpha(v,\ldots, v)\]
        as the permutations are all the same. We must now show that this is unique, i.e. that given some symmetric $k$-tensor $\beta$, if
        \begin{align*}
            \beta(v,\ldots,v)=\alpha(v,\ldots, v),
        \end{align*}
        then $\beta=\text{Sym\;}\alpha$. Let us prove this for $k=2$, and then proceed by induction.
        It is clear that
        \begin{align*}
            \beta(v+w,v+w)-\beta(v-w,v-w)&=\alpha(v+w,v+w)-\alpha(v-w,v-w)\\
            \beta(v,w)=\frac{1}{2}\left( \alpha(v,w)+\alpha(w,v) \right)&=\text{Sym }\alpha(v,w)
        \end{align*}
        Now suppose this holds for $k$. Then we can take
        \begin{align*}
            \beta(v_1,\ldots,v_k,v_{k+1})=\frac{1}{k!}\sum_{\sigma\in S_k}\alpha(v_{\sigma(1)},\ldots, v_{\sigma(k)},v_{k+1}).
        \end{align*}
        Let's permute now $v_{k+1}$:
        \begin{align*}
            (k+1)\beta(v_1,\ldots,v_{k+1})=\frac{1}{k!}\sum_{\sigma \in S_k}\sum_{\tau}\alpha(v_{\tau\sigma(1)},\ldots,v_{\tau\sigma(k)},v_{\tau(k+1)})
        \end{align*}
        where the $\tau$'s are cyclic permutations. We can then write
        \begin{align*}
            \beta(v_1,\ldots,v_{k+1})=\frac{1}{(k+1)!}\sum_{\sigma\in S_k}\alpha(v_{\sigma(1)},\ldots, v_{\sigma(k+1)}),
        \end{align*}
        and we're done, by induction.
    \item To see that the symmetric product is associative, we use the last part (because $\alpha,\beta,\gamma$ are symmetric and thus equal to their $\text{Sym}$) to compute:
        \begin{align*}
            (\alpha\beta)\gamma\underbrace{(v, \dots, v)}_{\text{$k + l + m$ times}}& = \bigg( \text{Sym }\left( \big(\text{Sym }\left( \alpha \otimes \beta \right)\big) \otimes \gamma \right) \bigg) \underbrace{(v, \dots, v)}_{\text{$k + l + m$}} \\
            & = \big( \big( \text{Sym }(\alpha \otimes \beta) \big)\underbrace{(v, \dots, v)}_{\text{$k + l$}} \big)  \cdot \big( \gamma \underbrace{(v, \dots, v)}_{\text{$m$ }} \big) \\
            & = \big( \big(\alpha(\otimes)\beta\big)\underbrace{(v, \dots, v)}_{\text{$k + l$ }} \big) \cdot \big( \gamma\underbrace{(v, \dots, v)}_{\text{$m$ }} \big) \\
            & = \big( \alpha\underbrace{(v, \dots, v)}_{\text{$k$ }} \cdot \beta \underbrace{(v, \dots, v)}_{\text{$l$ }} \big)  \cdot \gamma \underbrace{(v, \dots, v)}_{\text{$m$ }} \\
            & =  \alpha\underbrace{(v, \dots, v)}_{\text{$k$ }} \cdot \big( \beta \underbrace{(v, \dots, v)}_{\text{$l$ }}   \cdot \gamma \underbrace{(v, \dots, v)}_{\text{$m$ }} \big)\\
            & = \alpha(\beta\gamma)\underbrace{(v, \dots, v)}_{\text{$k + l + m$ times}},
        \end{align*}
        where in the last step, we worked backwards to reexpress the product as a symmetric product.
    \item Since any covector $\omega$ has only one argument, it is clearly symmetric. Consequently, the $k$ given $\omega^i$ are
        symmetric 1-tensors. It follows directly from the definition of a symmetric product, then, that
        \begin{align*}
            \omega^1\omega^2=\text{Sym\;}(\omega^1\otimes\omega^2)=\frac{1}{2}\left( \omega^1\otimes\omega^2+\omega^2\otimes\omega^1 \right).
        \end{align*}
        Let us assume that the formula for the product of $k-1$ such covectors is given by
        \begin{align*}
            \omega^1\cdots\omega^{k-1}=\frac{1}{(k-1)!}\sum_{\sigma\in S_{k-1}}\omega^{\sigma(1)}\otimes \cdots\otimes\omega^{\sigma(k-1)}.
        \end{align*}
        Then, the product of $k$ covectors can be written (using the associativity of the symmetric product)
        \begin{align*}
            \omega^1\cdots\omega^k&=\frac{1}{2} \left(\frac{1}{(k-1)!}\sum_{\sigma\in S_{k-1}}\omega^{\sigma(1)}\otimes \cdots\otimes\omega^{\sigma(k-1)}\right)\otimes\omega^k\\
            &+\frac{1}{2}\omega^k\otimes\left(\frac{1}{(k-1)!}\sum_{\sigma\in S_{k-1}}\omega^{\sigma(1)}\otimes \cdots\otimes\omega^{\sigma(k-1)}\right).
        \end{align*}
        We can now perform the following trick:
        \begin{align*}
            \frac{k}{k}\omega^1\cdots\omega^k&=\frac{1}{k}\sum_{\text{cyclic }\sigma}\omega^{\sigma(1)}\cdots\omega^{\sigma(k)}\\
            &=\frac{1}{2k}\sum_{\text{cyclic }\sigma}\left( \omega^{(\sigma(1)}\cdots\omega^{\sigma(k-1)}\otimes \omega^{\sigma(k)}+\omega^{\sigma(k)}\otimes\omega^{(\sigma(1)}\cdots\omega^{\sigma(k-1)}\right)\\
            &=\frac{1}{2k!}\sum_{\text{cyclic }\sigma}\sum_{\rho\in S_{k-1}}\omega^{\rho(\sigma(1))}\otimes \cdots\otimes\omega^{\rho(\sigma(k-1))}\otimes\omega^{\sigma(k)}+\cdots,
        \end{align*}
        where the ellipsis on the last line represents the second term with $\omega^{\sigma(k)}$ at the beginning instead of the end. Note that the
        combination of cyclic permutations and $S_{k-1}$ permutations gives us every combination for both terms (which cancels the one-half) and thus we arrive at
        \begin{align*}
            \omega^1\cdots\omega^k=\frac{1}{k!}\sum_{\tau\in S_{k}}\omega^{\tau(1)}\otimes\cdots\otimes\omega^{\tau(k)},
        \end{align*}
        and we are done by induction.
\end{enumerate}

\subsection*{Problem 10}

Let $M$ be a smooth $n$-manifold, and let $A$ be a smooth covariant $k$-tensor field on $M$. If $(U,(x^i))$ and $(\tilde U,(\tilde x^j))$
are overlapping smooth charts on $M$, we can write
\begin{align*}
    A=A_{i_1\ldots i_k}dx^{i_1}\otimes \cdots \otimes dx^{i_k}=\tilde A_{j_1\ldots j_k}d\tilde x^{j_1}\otimes \cdots \otimes d\tilde x^{j_k}
\end{align*}
Using the transformation rule for covectors, we can write
\begin{align*}
    \omega_i=\frac{\p \tilde x^j}{\p x^i}(p) \tilde \omega_j
\end{align*}
and thus each factor in the tensor product pulls out such a derivative by linearity of the tensor product, and we can relate
the tensor components as
\begin{align*}
    A_{i_1\ldots i_k}=\frac{\p \tilde x^{j_1}}{\p x_{i_1}}\cdots\frac{\p \tilde x^{j_k}}{\p x_{i_k}} \tilde A_{j_1\ldots j_k}
\end{align*}

\end{document}
