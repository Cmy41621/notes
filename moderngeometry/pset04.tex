\documentclass{../mathnotes}

\usepackage{tikz-cd}
\usepackage{todonotes}

\newgeometry{margin=1.75in}

\title{Modern Geometry I: PSET 4}
\author{Nilay Kumar\footnote{Collaborated with Matei Ionita and Leonardo Abbrescia.}}
\date{Last updated: \today}


\begin{document}

\maketitle

\subsection*{Problem 1}
Let $X,Y$ be smooth vector fields on a smooth manifold $M$. We wish to show
that
\[L_XL_Y-L_YL_X = L_{[X,Y]}\]
where the Lie derivatives are taken for smooth $(r,s)$-tensors $T\in C^\infty(M,T^r_sM)$.
In other words, we wish to show first that the left-hand side acts as a Lie derivative,
and second that the Lie derivative agrees with $L_{[x,y]}$.

We first show that for any tensors $S,T$ the left-hand side satisfies the product rule
\begin{align*}
    \text{LHS}(S\otimes T) &= L_X(L_Y(S\otimes T))-L_Y(L_X(S\otimes T))\\
    &= L_X(L_YS\otimes T+S\otimes L_YT)-L_Y(L_XS\otimes T+S\otimes L_XT)\\
    &= L_X(L_YS)\otimes T+L_YS\otimes L_XT +L_XS\otimes L_YT+S\otimes L_X(L_YT)\\
    &- L_Y(L_XS)\otimes T-L_XS\otimes L_YT-L_YS\otimes L_XT-S\otimes L_Y(L_XT)\\
    %&= L_{[X,Y]}S\otimes T+S\otimes L_{[X,Y]}T
    &= L_X(L_YS)\otimes T+S\otimes L_X(L_YT)- L_Y(L_XS)\otimes T-S\otimes L_Y(L_XT),
\end{align*}
as desired. To show that the left-hand side is a Lie derivative, it now suffices to
show that it respects contractions (c.f. GHL p.38):
\begin{align*}
    \text{LHS}(c(T)) &= L_X(L_Yc(T))-L_Y(L_Xc(T))\\
    &= L_Xc(L_YT)-L_Yc(L_XT)\\
    &= c(L_X(L_YT))-c(L_Y(L_XT))\\
    &= c(L_X(L_YT)-L_Y(L_XT)).
\end{align*}
This proves that the left-hand side is a Lie derivative. Now note that it is
clear that the above equality holds for $f\in C^\infty(M,T^0_0M)$, as
\begin{align*}
    L_XL_Yf-L_YL_Xf &= L_X(Y(f))-L_Y(X(f))\\
    &= (XY-YX)f=[X,Y]f\\
    &= L_{[X,Y]}f.
\end{align*}
Similarly, for $Z\in C^\infty(M,TM)$, we find
\begin{align*}
    L_XL_YZ-L_YL_XZ &= L_X[Y,Z]-L_Y[X,Z]\\
    &= [X,[Y,Z]]-[Y,[X,Z]]\\
    &= [ [X,Y], Z]\\
    &= L_{[X,Y]}Z,
\end{align*}
by the Jacobi identity for the Lie derivative. It now suffices to show that the equality
holds for tensors of higher rank. For $(r,0)$-tensors this follows from the previous
computation via the Leibniz rule above. If we can prove the equality for $(0,1)$-tensors,
the Leibniz rule also enforces this equality for $(0,s)$-tensors. Let $S$ be a $(1,0)$-tensor
and $T$ be a $(0,1)$-tensor with $c(S\otimes T)=f$ some function. Then
\begin{align*}
    c(\text{LHS}(S\otimes T)) &= c\left( ((L_XL_Y-L_YL_X)S)\otimes T+S\otimes ((L_XL_Y-L_YL_X)T)\right).
\end{align*}
Note that the left hand side in this equation is simply the Lie derivative of a function;
this implicitly defines the action of the $L_{[X,Y]}$ as the other terms in the equation
are already well-defined as shown above.

\subsection*{Problem 2}
\begin{enumerate}[(a)]
    \item We first show the identity for $0$- and $1$-forms. For $f\in C^\infty(M)$ we find
        \begin{align*}
            d(i_Xf)+i_X(df) &= 0+df(X)\\
            &= L_Xf.
        \end{align*}
        Now suppose $\omega\in\Omega^1(M)$.
        Let us prove the identity in coordinates; fix $X,Y$
        vector fields given locally by $X=a_i\partial_i,Y=b_j\partial_j$ and take $\omega=\omega_kdx^k$.
        Then
        \begin{align*}
            (L_X\omega)(Y) &= X(\omega(Y))-\omega([X,Y])\\
            &= a_i\partial_i(\omega_jb_j)-\omega_j(a_i\partial_ib_j-b_i\partial_ia_j)\\
            &= a_i\omega_j\partial_ib_j+a_ib_j\partial_i\omega_j-\omega_ja_i\partial_ib_j+\omega_jb_i\partial_ia_j\\
            &= a_ib_j\partial_i\omega_j+\omega_jb_i\partial_ia_j
        \end{align*}
        Meanwhile, the right hand side yields
        \begin{align*}
            (d(i_X\omega)+i_X(d\omega))(Y)&=Y\omega(X)+d\omega(X,Y)\\
            &=b_j\partial_j(\omega_ia_i) + \partial_i\omega_ka_ib_k-\partial_i\omega_ka_kb_i\\
            &=b_ja_i\partial_j\omega_i+b_j\omega_i\partial_ja_i+\partial_i\omega_ka_ib_k-\partial_i\omega_ka_kb_i\\
            &=b_j\omega_i\partial_ja_i+\partial_i\omega_ka_ib_k,
        \end{align*}
        which agrees with the above.
        Let us now show that $P_X=d\circ i_X+i_X\circ d$ is a derivation; this will allow
        us to induct to show the identity for $n$-forms. Take $\omega\in\Omega^r(M),\eta\in\Omega^s(M)$.
        Then, using the fact that $d$ and $i_X$ both act as (up to sign) derivations,
        \begin{align*}
            (d\circ i_X+i_X\circ d)(\omega\wedge\eta) &=d(i_X\omega\wedge\eta+(-1)^r\omega\wedge i_X\eta)+i_X(d\omega\wedge\eta+(-1)^r\omega\wedge d\eta)\\
            &=\left( d(i_X\omega)\wedge\eta+(-1)^{r-1}i_X\omega\wedge d\eta+(-1)^rd\omega\wedge i_X\eta+\omega\wedge d(i_X\eta) \right)\\
            &+\left( i_X(d\omega)\wedge\eta+(-1)^{r+1}d\omega\wedge i_X\eta+(-1)^ri_X\omega\wedge d\eta+\omega\wedge i_X(d\eta) \right)\\
            &=((d\circ i_X+i_X\circ d)\omega)\wedge\eta+\omega\wedge ( (d\circ i_X+i_X\circ d)\eta ).
        \end{align*}
        Using this, it is clear that the identity holds for $n$-forms, as any $n$-form can be
        written as a (sum of) wedge products of $1$-forms. The derivation property of $P_X$ applies,
        thus yielding a sum of Lie derivatives. Of course, since Lie derivatives also satisfy the Leibniz rule,
        we obtain the Lie derivative of the original $n$-form.
    \item The identity is clear for $f\in\Omega^0(M)$, as both the left and the right hand side are zero.
        Now suppose we have $\omega\in\Omega^1(M)$. By definition, we know that
        \begin{align*}
            L_X\omega(Y) &= L_X(\omega(Y)) -\omega([X,Y])\\
            i_Y(L_X\omega) &= L_X(i_Y\omega)-i_{[X,Y]}\omega,
        \end{align*}
        which is precisely the identity.
        We now show that the left-hand side acts as a (kind-of) derivation over wedge products. Let
        $\omega\in\Omega^r(M),\eta\in\Omega^s(M)$ with $r,s\geq 1$. Then
        \begin{align*}
            L_X(i_Y(\omega\wedge\eta))&-i_Y(L_X(\omega\wedge\eta))\\
            &=L_X(i_Y\omega\wedge\eta+(-1)^r\omega\wedge i_X\eta)-i_Y(L_X\omega\wedge i_Y\eta+(-1)^r\omega\wedge(L_Xi_Y\eta)\\
            &-(i_YL_X\omega)\eta-(-1)^r(L_X\omega)\wedge(i_Y\eta)-(i_Y\omega)\wedge L_X\eta-(-1)^r\omega\wedge(i_YL_X\eta)\\
            &=(L_xi_Y\omega-i_YL_X\omega)\wedge\eta+(-1)^r\omega\wedge(L_xi_Y\eta-i_yL_X\eta),
        \end{align*}
        as desired. This, together with and inductive argument, completes the proof.
    \item It's clear that since $\omega$ is multilinear $i_{fX}\omega=fi_X\omega$. Then by $(a)$,
        \begin{align*}
            L_{fX}\omega &= d(fi_X\omega)+fi_X(d\omega)\\
            &= df\wedge i_X\omega+fdi_X\omega+fi_Xd\omega\\
            &= df\wedge i_X\omega+fL_X\omega,
        \end{align*}
        as desired.
\end{enumerate}

\subsection*{Problem 3}
We prove first a useful lemma where $S\in C^\infty(T^0_qM)$ and $X,X_1,\ldots,X_q$ be $(q+1)$ vector fields.
Then by the Leibniz rule on Lie derivatives, we find that
\begin{align*}
    L_Xc(S\otimes X_1\otimes\cdots \otimes X_q) &= c(L_X(S\otimes X_1\otimes \cdots X_1))\\
    &= c(L_XS\otimes X_1\otimes \cdots \otimes X_q) + c(S\otimes \sum_{i=1}^q X_1\otimes \cdots\otimes L_XX_i\otimes \cdots \otimes X_q)\\
    &=L_XS(X_1,\ldots,X_q) + \sum_{i=1}^q S(X_1,\ldots, X_{i-1},[X,X_i], X_{i+1}, \ldots ,X_q).
\end{align*}
Now consider the formula we wish to prove. For $f\in\Omega^0(M)$ we find $df(X_0)=X_0f,$ which is true
by definition. Now suppose the formula holds for $\omega\in\Omega^{s-1}(M)$; let us show it for $s$.
Using the formula $(a)$ from Problem 2, we find that
\begin{align*}
    i_{X_0}(d\omega)(X_1,\ldots, X_s) &= (L_{X_0}\omega)(X_1,\ldots,X_s)-d(i_{X_0}\omega)(X_1,\ldots,X_s)\\
    &= L_{X_0}(\omega(X_1,\ldots,X_s))-\sum_{i=1}^s\omega(X_1,\ldots,X_{i-1},[X_0,X_i],X_{i+1},\ldots,X_s)\\
    &- \sum_{i=1}^s(-1)^{i+1}L_{X_i}(\omega(X_0,X_1,\ldots,\hat X_i,\ldots,X_s))\\
    &-\sum_{1\leq i\leq j\leq s}(-1)^{i+j+1}(i_{X_0}\omega)([X_i,X_j],X_1,\ldots,\hat X_i,\ldots,\hat X_j,\ldots,X_s)\\
    &=\sum_{i=0}^s(-1)^iL_{X_i}(\omega(X_0,\ldots,\hat X_i,\ldots,X_s))\\
    &+\sum_{0\leq i\leq j\leq s}(-1)^{i+j}\omega([X_i,X_j],X_0,\ldots,\hat X_i,\ldots,\hat X_j,\ldots,X_s),
\end{align*}
where in the last term we have done the following: we combine the first and third terms to change the
sign of the third term to positive and its initial index from 1 to 0 to obtain the first desired term;
we then change the index of the second term to $j$, move the commutator term to the far left gaining
powers of $(-1)^j$, and then combining these terms with the last term to obtain the second desired term.

\subsection*{Problem 4}
\begin{enumerate}[(a)]
    \item The normal to the sphere at any point $(x,y,z)$ is given by $(2x,2y,2z)$; we check
        that $\tilde X$ and $\tilde Y$ lie tangent to $S^2$ at every point in $S^2$ by
        computing the inner products
        \begin{align*}
            (2x,2y,2z)\cdot (-zx,-zy,x^2+y^2) &= 0\\
            (2x,2y,2z)\cdot (-y,x,0) &= 0.
        \end{align*}
    \item Note that if $\phi_t$ is the flow induced by $X$ then
        \begin{align*}
            L_X\omega &= \frac{d}{dt}\bigg|_{t=0}(\phi_t^*\omega)\\
            &= \frac{d}{dt}\bigg|_{t=0}(i\circ\phi_t)^*\tilde\omega\\
            &= \frac{d}{dt}\bigg|_{t=0}(\tilde \phi)^*\tilde\omega\\
        \end{align*}
        where $(\tilde \phi)^*$ is the flow induced by $\tilde X$. The last equality above follows
        from the fact that $i_*X=\tilde X$ and hence the inclusion of the flow of $X$ is the flow
        of $\tilde X$. To compute $L_X\omega$, then, it suffices to compute $L_{\tilde X}\tilde\omega$.
        We do this using the formula proved in Problem 2:
        \begin{align*}
            L_X\omega &= d(i_X\omega)+i_X(d\omega)\\
            &= d( (x^2+y^2+z^2)(ydx-xdy)) +i_X(3dx\wedge dy\wedge dz)\\
            &= -(x^2+y^2)dx\wedge dy+xzdx\wedge dy+yzdx\wedge dz-2z^2dx\wedge dy,\\
            L_Y\omega &= d(i_Y\omega)+i_Y(d\omega)\\
            &= d(x^2dz+y^2dz-yzdy-xzdx)+i_Y(3dx\wedge dy\wedge dz)\\
            &= 3xdx\wedge dz+3ydy\wedge dz-3ydy\wedge dz-3xdx\wedge dz\\
            &=0.
        \end{align*}

        Similarly, we can compute
        \begin{align*}
            L_X\eta &= d(i_X\eta)+i_X(d\eta)\\
            &= 0 + i_X(2dx\wedge dy)\\
            &= -2zx dy + 2zy dx,\\
            L_Y\eta &= d(i_Y\eta)+i_Y(d\eta)\\
            &= d(x^2+y^2)+i_Y(2dx\wedge dy)\\
            &= 2xdx+2ydy-2ydy-2xdx\\
            &= 0.
        \end{align*}

        Finally, we compute the Lie bracket (again using the logic as above),
        \begin{align*}
            L_XY &= [X,Y]\\
            &=-xz\frac{\partial}{\partial y}+zy\frac{\partial}{\partial x}-\left( zy\frac{\partial}{\partial x}-xz\frac{\partial}{\partial y}-2xy\frac{\partial}{\partial z}+2xy\frac{\partial}{\partial z} \right)\\
            &=0.
        \end{align*}
    \item We compute
        \begin{align*}
            d\eta &= di^*\tilde\eta = i^*(d\tilde\eta)\\
            &=i^*(2dx\wedge dy).
        \end{align*}
        On the other hand,
        \begin{align*}
            \lambda\omega &= \lambda i^*(\tilde\omega)\\
            &= \lambda\left(\frac{x^2}{z}dx\wedge dy + \frac{y^2}{z}dx\wedge dy+zdx\wedge dy\right)\\
            &= \frac{\lambda}{z} dx\wedge dy,
        \end{align*}
        where we have used the relation $xdx+ydy+zdz=0$. For $z\neq 0$, then, we find that $\lambda(x,y,z)=2z$.
        When $z=0$, we obtain the relation $xdx+ydy=0$, i.e. that $dx$ and $dy$ are proportional.
        Thus $d\eta=0$ for $z=0$ and hence $\lambda(x,y,z)=2z$ everywhere.
    \item Note that 
        \begin{align*}
            i_X\omega(Y) &= \omega(X,Y)=\tilde\omega(i_*X,i_*Y)=\tilde\omega(\tilde X,\tilde Y)\\
            &=i_{\tilde X}\tilde\omega(\tilde Y)
        \end{align*}
        and so we compute
        \begin{align*}
            i_X\omega\wedge i_Y\omega &= \left( (x^2+y^2+z^2)(ydx-xdy) \right)\wedge\left( (x^2+y^2)dz-yzdy-xzdx \right)\\
            &=-(x^2+y^2+z^2)(x^2+y^2)\tilde\omega
        \end{align*}
        and hence we find that $\phi(x,y,z)=-(x^2+y^2)$.
    \item Suppose there exists an $\bar\omega\in\Omega^2(P_2(\R))$ such that $\omega=\pi^*\bar\omega$.
        By definition, for $p\in S^2$,
        \[\omega_p(X,Y)=\bar\omega_{\pi(p)}(\pi_*X,\pi_*Y).\]
        Since $\pi$ is a local diffeomorphism, any nonzero tangent vector $X\in T_pS^2$ is sent to a nonzero
        tangent vector $\pi_*X\in T_{\pi(p)}P_2(\R)$.
        Now since $\omega$ vanishes nowhere on $S^2$, we find that at any $p\in S^2$, there exist $X,Y\in T_pS^2$
        such that $\bar\omega(\pi_*X,\pi_*Y)\neq0$. Hence, by the surjectivity of $\pi$, we find that $\bar\omega$
        is a nonvanishing two-form on $P_2(\R)$, which contradicts nonorientability.
\end{enumerate}



\end{document}
