\documentclass{../mathnotes}

\usepackage{tikz-cd}
\usepackage{todonotes}

\title{Introduction to Algebraic Topology Midterm II}
\author{Nilay Kumar}
\date{Last updated: \today}


\begin{document}

\maketitle

\begin{prop}
    Problem 1
\end{prop}
\begin{proof}
    Let $G=\langle a,b\mid aba^{-1}b\rangle$ be the fundamental group of the Klein bottle
    as in Figure 1.
    \begin{enumerate}[(a)]
        \item Consider the subgroup $H$ generated by $a^3$ and $b^2$. We wish to find a covering
            space $p:\tilde K\to K$ such that $\tilde K$ is associated to $H$ by the Galois correspondence.
            Consider the Klein bottle $\tilde K$ depicted in Figure 2, which is a sixfold cover of $K$.
            If the fundamental group of $\tilde K$ is generated by $x$ and $y$, with the relation
            $xyx^{-1}y$, we find that under $p$ (defined in the obvious way along the dashed lines)
            $p_*(x)=a^3$ and $p_*(y)=b^2$, and hence $p_*(\pi_1(\tilde K,\tilde x))=H$ as desired
            (by uniqueness via the Galois correspondence).
        \item Next, consider the three-fold cover $p_1:\tilde K_1\to K$ lying between $\tilde K$ and $K$
            given by the diagram in Figure 3. Note that topologically $K_1$ is again a Klein bottle.
            Using the same logic as above, if we define $p_1$ following the dashed lines as indicated in the
            figure, we find that if $x,y$ are the generators of $\pi_1(\tilde K_1,\tilde x_1)$ with relation
            $xyx^{-1}y$, then $(p_1)_*(x)=a^3$ and $(p_1)_*(y)=b$. Hence $(p_1)_*(\pi_1(\tilde K_1,\tilde x_1))$
            is the subgroup of $\pi_1(K,x)$ generated by $a^3,b$.
        \item Similarly, consider the two-fold cover $p_2:\tilde K_2\to K$ lying between $\tilde K$ and $K$
            given by the diagram in Figure 4. This space is topologically again a Klein bottle. Defining
            $p_2$ accordingly, we find that if $x,y$ are the generators of $\pi_1(\tilde K_2,\tilde x_2)$
            with relation $xyx^{-1}y$ then $(p_2)_*(x)=a$ and $(p_2)_*(y)=b^2$. Hence $(p_2)_*(\pi_1(\tilde K_2,\tilde x_2))$
            is the subgroup of $\pi_1(K,x)$ generated by $a,b^2$.
        \item Recall the algebraic fact that any subgroup of index two is normal; this immediately
            shows that $q_1$ and $p_2$ are normal covering spaces, as both these covering maps induce
            index two subgroups in the fundamental groups of their base spaces. It remains to check $p,p_1,$ and $q_2$.
            Note that checking normality of a covering reduces to checking normality of the associated
            image subgroup under the covering space map, which in turn can be reduced to checking that conjugation
            of the generators (by the homomorphism property of conjugation) returns an element of the subgroup.
            Hence for $p_1$ we must check that conjugating $a^3$ and $b$ by $a$ and $b$ yield elements in the subgroup
            generated by $a^3$ and $b$; this is easy via the relation $aba^{-1}b$: $ba^3b^{-1}$ is already
            in the subgroup while $aba^{-1}=b^{-1}$ which is also contained in the subgroup.
            Similarly, $q_2$ is shown to be a normal covering by noting that $ab^2a^{-1}=b^{-2}$ and
            $b^2a^3b^{-2}$ are both contained in the appropriate subgroup. Finally, $p$ is a normal covering,
            as $ab^2a^{-1}=b^{-2}$ and
            \begin{align*}
                ba^3b^{-1}&=b(b^{-1}a^{3}b^{-1})b^{-1}\\
                &=a^3b^{-2}
            \end{align*}
            are in the appropriate subgroup, where I have used the identity $a^3ba^{-3}b=1$, verified as follows.
            \begin{align*}
                a^3b&=a^2b^{-1}a,\\
                a^{-3}b&=ba^{-1}b^2a^{-1}b^2a^{-1}b^2,\\
                a^3ba^{-3}b&=a^2b^{-1}aba^{-1}b^2a^{-1}b^2a^{-1}b^2\\
                &=ab^2a^{-1}b^2\\
                &=b^{-2}b^2=1.
            \end{align*}
            Of course, this is much more easily seen topologically, by the fact that the change of basepoint homomorphism
            on lifts takes loops to loops.

            Thus we find that every covering in this diagram is normal, which makes computing
            deck groups relatively easy. Recall that in the case of a normal covering, the deck group is
            isomorphic to the quotient of the fundamental group of the base space by the subgroup $H$ associated
            to the covering. First note $G(q_1)=\Z_2$ as we are forcing the generator $b\in\pi_1(\tilde K_1,\tilde x_1)$
            to square to zero (this is also clear by eyeballing the picture -- one can swap the upper and lower halves of
            the six-fold covering). Similarly, $G(q_2)=\Z_3$ as we are forcing the generator $a\in\pi_1(\tilde K_2,\tilde x_2)$
            to cube to zero (again, topologically this is simply permuting the thirds of the six-fold covering).
            Next, $G(p_1)=\Z_3$ as we are forcing the generator $a\in\pi_1(K,x)$ to cube to zero and $G(p_2)=\Z_2$
            as we are forcing the generator $b\in\pi_1(K,x)$ to square to zero. Finally, we find that $G(p)=\Z_2\times \Z_3=\Z_6$
            as we are forcing $a\in\pi_1(K,x)$ to cube to zero and $b\in\pi_1(K,x)$ to square to zero.
    \end{enumerate}
\end{proof}

\begin{prop}
    Problem 2
\end{prop}
\begin{proof}\hfill
\begin{enumerate}
    \item See Figure 5 for a drawing of the simplicial structure chosen for $X=S^2\vee S^1\vee S^1$.
        Recall from Hatcher that the simplicial homology for the torus $T^2=S^1\times S^1$ is as follows:
        $H_0^\Delta(T^2)=\Z, H_1^\Delta(T^2)=\Z^2, H_2^\Delta(T^2)=\Z$ and zero for higher groups.
        Associated to $X$ we obtain the chain complex 
        \begin{equation*}
            \begin{tikzcd}
                0\ar{r} & \Z^2\ar{r}{\partial_2} & \Z^5\ar{r}{\partial_1} & \Z^3\ar{r}{\partial_0} & 0
            \end{tikzcd}
        \end{equation*}
        with $ \partial_0x=\partial_0y=\partial_0z=0,\partial_1a=y-x,\partial_1b=z-y, \partial_1c=z-x,$ and $\partial_2U=\partial_2L=a+b-c.$
        Of course, $H^\Delta_0(X)=\Z$ by path-connectedness,
        \[H_1^\Delta(X)=\frac{\ker \partial_1}{\text{im }\partial_2}=\frac{\langle a+b-c,d,e\rangle}{\langle a+b-c\rangle}=\Z^2,\]
        and $H^\Delta_2(X)=\langle U-L\rangle=\Z$. Hence the homology groups of the torus and $X$ are isomorphic.
    \item The universal cover of the torus is of course $\R^2$, while the universal cover $\mathcal{U}X$ of $X$ is sketched in Figure 6
        (this was previously done on a homework). By contractibility, $H^\Delta_n(\R^2)=0$ for all $n$. On the
        other hand, since in $\mathcal{U}X$ we have spheres touching the universal cover of $S^1\vee S^1$
        at every intersection point, $H^\Delta_2(\mathcal{U}X)=\ker\partial_2$ which is clearly non-zero, since the difference $U-L$
        of the two 2-simplices comprising a sphere is send by $\partial_2$ to zero (as shown in the previous part). Hence the homologies
        of $\R^2$ and $\mathcal{U}X$ are not isomorphic.
\end{enumerate}
\end{proof}
\end{document}
