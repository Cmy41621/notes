\documentclass{../mathnotes}

\usepackage{tikz-cd}
\usepackage{amsmath}
\usepackage{todonotes}


\title{Introduction to Algebraic Topology: Class Notes}
\author{Nilay Kumar}
\date{Last updated: \today}


\begin{document}

\maketitle

%\setcounter{section}{-1}

\section*{Class 1}
\begin{rem}
    The goal of algebraic topology is to develop algebraic tools to study topological spaces. To this end, all maps are assumed
    to be continuous unless otherwise stated.
\end{rem}

\subsection{Homotopy}

\begin{defn}
    A \textbf{deformation retract} of a space $X$ onto a subspace $A$ is a family of maps
    $f_t:X\to X$ for $t\in I=[0,1]$ such that
    \begin{enumerate}
        \item $f_0=\id_X$
        \item $f_1(X)=A$
        \item $f_t|_A=\id_A$ for all $t$
        \item $f_t$ is continuous as a map $X\times I\to X$ sending $(x,t)\mapsto f_t(x)$.
    \end{enumerate}

\end{defn}

\begin{exmp}
    Consider the annulus $X=\left\{ x\in\C \mid 1/2\leq|x|\leq 3/2 \right\}$ with the subspace
    $A=\left\{ x\in\C \mid |x|=1 \right\}$. We can construct a deformation retract of $X$ onto $A$ as
    \[f_t(x)=\frac{x}{1-t+t|x|}.\]
\end{exmp}

\begin{defn}
    Given $f:X\to Y$, we define the \textbf{mapping cylinder} $M_f$ to be the quotient space
    \[(X\times I)\sqcup Y/\sim\]
    where $(x,1)\sim f(x)$. Note that $M_f$ deformation retracts to $Y$ by sliding each point $(x,t)$ along $\left\{ x \right\}\times I\subset M_f$
    to $\left\{ x \right\}\times \left\{ 1 \right\}=f(x)\in Y$.
\end{defn}

\begin{exmp}
    If $f:X\to Y$ is an inclusion, say of the circle into the plane, then the mapping circle is
    simply a cylinder attached to the plane.
\end{exmp}

\begin{defn}
    A \textbf{homotopy} is a family of maps $f_t:X\to Y$ for $t\in I$ such that $F:X\times I\to Y$ given by $F(x,y)=f_t(x)$ is continuous.
    We say that two maps $f_0$ and $f_1$ from $X\to Y$ are \textbf{homotopic} if there exists a homotopy $f_t$ connecting them. We write
    this as $f_0\simeq f_1$.
\end{defn}

\begin{defn}
    A \textbf{retraction} of $X$ onto $A$ is a map $r:X\to X$ such that
    \begin{enumerate}
        \item $r(X)=A$
        \item $r|_A=\id_A$
    \end{enumerate}
    In this case, $A$ is called a \textbf{retract} of $X$. Note that $r^2=r$, as $r$ is the identity on its image.
    \todo{does this characterize $r$?}
\end{defn}

\begin{exmp}
    A deformation retract is a homotopy between the identity map to a retraction.
\end{exmp}

\begin{exmp}
    Let $X$ be any space with $x_0\in X$. Consider the map $f:X\to X$ given by $x\mapsto x_0$. This is clearly a retract.
\end{exmp}

One might ask when there exists a deformation retract $X\to \left\{ x_0 \right\}$. For example, given an annulus and some $x_0$
inside, does there exist a deformation retract? The answer is in fact no, and later we develop some tools to show this. Similarly
for the disjoint union of two disks, with $x_0$ contained in one of them.

\begin{defn}
    Let $A\subset X$. A homotopy $f_t:X\to Y$ such that $f_t|_A$ is independent of $t$. Then we say that $f_t$ is a \textbf{homotopy relative to $A$}.
\end{defn}

\begin{exmp}
    For example, a deformation retract of $X$ onto $A$ is a homotopy rel $A$.
\end{exmp}
\begin{exmp}
    Consider the unit disk $X=\left\{ z\in\C\mid |z|\leq 1 \right\}$ with $f_t(z)=ze^{2\pi it}$. In this case, $f_t$ is
    a homotopy rel $O$. Note that this is a homotopy from the identity to itself.
\end{exmp}

\begin{rem}
    Let $f_t:X\to X$ be a deformation retract of $X$ onto $A$. Let $r=f_1$ be the resulting retract and let $i:A\to X$ be
    the inclusion. Then $r\circ i=\id_A$ and $i\circ r\simeq \id_X$ (via the homotopy $f_t$). In this sense $r$ and $i$
    are inverses up to homotopy.
\end{rem}

\begin{defn}
    A map $f:X\to Y$ is a \textbf{homotopy equivalence} if there exists a $g:Y\to X$ such that $f\circ g\simeq\id_Y$ and $g\circ f\simeq\id_X$.
    Then we say that $X$ and $Y$ are \textbf{homotopy equivalent} and that they have the same \textbf{homotopy type} written $X\simeq Y$.
\end{defn}

\begin{defn}
    A space that is homotopy equivalent to a point is called \textbf{contractible}. This is equivalent to asking that the identity
    map be homotopic to a constant map, i.e. \textbf{nullhomotopic}.
\end{defn}

\begin{exmp}
    The spaces $\R^n$ and $D^n$ are contractible, while $S^n$ is not.
\end{exmp}

\begin{rem}
    It will be useful to think of the sphere as $S^n=D^n/\partial D^n$.
\end{rem}

\subsection{Operations on spaces}

We are of course already familiar with products and quotients.

\begin{defn}
    Given a space $X$, we define the \textbf{cone of $X$} $CX$ to be $X\times I/X\times\left\{ 0 \right\}$.
    The prototypical example is given by $S^1\times I/S^1\times \left\{ 0 \right\}$.
\end{defn}

\begin{defn}
    Given a space $X$, we define the \textbf{suspension of $X$} $SX$ to be $X\times \left\{ 0 \right\}$ collapsed to a point
    and $X\times \left\{ 1 \right\}$ collapsed to another point. Note that suspension is a functor, and we can suspend maps as
    well.
\end{defn}

\section*{Class 2}

\begin{defn}
    We define the \textbf{wedge sum} of two topological spaces as follows. Pick $x_0\in X,y_0\in Y$, and define
    \[X\vee Y=X\sqcup Y/\sim\]
    where $x_0\sim y_0$.
\end{defn}

\begin{defn}
    We define the \textbf{join} of two spaces to be
    \[X*Y=X\times Y\times I/\sim\]
    where $(x,y_1,0)\sim (x,y_2,0)$ and $(x_1,y,1)\sim (x_2,y,1)$. In other words, at $t=0$,
    the product collapses to $X$ while at $t=1$, it collapses to $Y$. As an example, consider $X=Y=I$.
    In this case, the product is a cube and the join becomes a tetrahedron. Note that one can also
    consider the join $X*Y$ as formal convex linear combinations of points of $X$ and $Y$.
\end{defn}

\begin{defn}
    We define the \textbf{smash product} of two spaces with $x_0\in X,y_0\in Y$ as
    \[X\wedge Y=X\times Y/X\vee Y,\]
    where the wedge sum is taken at $x_0,y_0$ and we are consider $X\vee Y=X\times \left\{ y_0 \right\}\cup \left\{ x_0 \right\}\times Y$
    as sitting inside $X\times Y$. One can it is true that $S^n\wedge S^m=S^{n+m}$.
\end{defn}

\subsection{CW-complexes}

The motivation of CW-complexes is to build higher dimensional surfaces from polygons. The analogy is to
the torus, which can be represented by a square in the plane with opposite edges oriented. This, in fact,
can be extended to higher-genus surfaces. Note that the interior of such a polygon is a 2-cell, i.e. open 2-disk.
The edges, on the other hand, are 1-cells, i.e. open 1-disks/intervals. Finally, we also have 0-cells as points.
To build a torus, then, we take a 0-cell and add two 1-cells as the characteristic circles of the torus, and
we attach a 2-cell to the 1-cells in the appropriate manner.

\begin{defn}
    We define a \textbf{cell complex} (or \textbf{CW-complex}), built inductively as follows:
    \begin{enumerate}
        \item Start with $X^0$, as discrete set of points;
        \item Form the \textbf{$n$-skeleton} $X^{n}$ from $X^{n-1}$ by attaching $n$-cells $e_\alpha^n$
            via maps $\phi:S^{n-1}=\partial D^n\to X^{n-1}$. Hence $X^n=X^{n-1}\sqcup D^n_\alpha/x\sim \phi_\alpha(x)$,
            where $x\in\partial D^n$. Hence, setwise $X^n=X^{n-1}\sqcup e^n_\alpha$ where each $e^n_\alpha$ is an open $n$-disk.
        \item Stop at some finite $n$ to obtain a \textbf{finitely-generated cell complex} $X=X^n$ or continue indefinitely to
            obtain a cell complex $X=\cap_{n=0}^\infty X^n$. In the infinite case, we consider the \textbf{weak topology}, where
            $A\subset X$ is open if and only if $A\cap X^n$ is open for all $n$.
    \end{enumerate}
\end{defn}

\begin{exmp}
    Let us consider some examples of cell complexes. The most obvious such example is a graph, which is a 1-dimensional
    complex, with vertices as 0-cells and edges as 1-cells. We can construct a sphere $S^1$ by simply taking a 0-cell $x_0$ and
    attaching a 1-cell (via the constant attaching map sending boundaries to $x_0$). The same holds for $S^n$, where we take an
    $n$-cell and attach its boundary to a 0-cell (the attaching map again constant).

    Next, we can construct the torus via a 0-cell $e^0$, 2 1-cells $e^1_1$ and $e^1_2$, and a 2-cell $e^2$.
    The attaching maps for the 1-cells are obvious. The boundary $S^1$ of $e^2$ is attached by going first positively
    along $e^1_1$, then positively along $e^1_2$, then negatively along $e^1_1$, and finally, negatively along $e^1_2$.

    As a final example, recall the definition of the real projective space $\R P^n$. We can treat it as a quotient of
    $D^n$ by identifying antipodal points. But this is simply $D^n$ with antipodal boundary points identified. But the boundary
    is simply $S^{n-1}$, and hence identifying its boundary's antipodal points yields $\R P^{n-1}$. In this sense, we can consider
    $\R P^n$ as $\R P^{n-1}\sqcup D^n/x\sim\phi(x)$ where $\phi:S^{n-1}\to\R P^{n-1}$ is the quotient map. Hence we can start with
    a 0-cell and then attach a 1-cell by the antipodal quotient map to get $\R P^1$. More generally, we can write $\R P^n$
    as the complex consisting of a 0-cell, 1-cell, 2-cell, \ldots, $n$-cell, where the attaching maps are all antipodal quotient maps.
\end{exmp}

\begin{defn}
    Given a CW-complex $X$, each cell $e_\alpha^n$ has a \textbf{characteristic map} $\Phi_\alpha:D^n\to X$ which 
    \begin{enumerate}
        \item extends the attaching map $\phi_\alpha:S^{n-1}\to X^{n-1}$;
        \item is a homeomorphism from the interior of $D^n_\alpha$ to the cell $e^n_\alpha$.
    \end{enumerate}
    Note that $\Phi_\alpha$ is the composition
    $D^n_\alpha\hookrightarrow X^{n-1}\coprod_\alpha D^n_\alpha\to X^n\hookrightarrow X.$
\end{defn}

\begin{exmp}
    For the sphere, the characteristic map is $\Phi:D^n\to S^n$, which collapses $\partial D^n$ to a point.
    For $\R P^n$, on the other hand, the characteristic map is $\Phi:D^i\hookrightarrow \R P^i\subset\R P^n$ which
    identifies antipodal points on $\partial D^i$.
\end{exmp}

\begin{defn}
    A \textbf{subcomplex} of a CW-complex $X$ is a closed subspace $A\subset X$ that is a union of cells of $X$.
    We call $(X,A)$ a \textbf{CW pair}.
\end{defn}

\begin{rem}
    Note that $A$ is itself a CW-complex.
\end{rem}

\begin{exmp}
    Consider one of the 1-cells in the cell decomposition for the torus. This 1-cell, together with the 0-cell of the torus, yields
    a subcomplex. As another example, note that $\R P^k\subset\R P^n$ for $k\leq n$ is a subcomplex.

    As a non-example, note that $S^1$ is not a subcomplex of $S^2$ with the usual cell structure, though it is a subspace. However,
    one can choose a different cell structure for $S^2$ in which we create the equator from two 0-cells and two 1-cells and two 2-cells
    (the northern and southern hemispheres), and in this structure, $S^1$ is indeed a subcomplex of $S^2$.
More generally, one can obtain a CW-complex for $S^n$ by attaching two $n$-cells to $S^{n-1}$, though this becomes difficult to visualize.
\end{exmp}

\begin{defn}
    If $X,Y$ are CW complexes, their \textbf{product} $X\times Y$ has the structure of a CW complex with cells $e^m_\alpha\times e^n_\beta$
    where $e^m_\alpha$ ranges over cells in $X$ and $e^n_\beta$ ranges over cells in $Y$.
\end{defn}

\begin{exmp}
    The cell decomposition of $S^1\times S^1$ will be $\left\{ e^0,e^1 \right\}\times \left\{ e^0,e^1 \right\}=\left\{ e^0,e_1^1,e_2^1,e^2 \right\}$,
    which is the usual decomposition we have for the torus.
\end{exmp}

\begin{defn}
    If $(X,A)$ is a CW pair, then the \textbf{quotient} $X/A$ inherits a natural CW structure which is the usual cells structure away from $A$
    but has $e^0$ replacing $A$.
\end{defn}

\begin{exmp}
    If we take a 0-cell and a 1-cell of the torus as a subcomplex and quotient by it, we obtain a 2-sphere, as we are left with
    $\{e^0,e^2\}$.
\end{exmp}

\subsection{Homotopy equivalence}

Let us now present two criteria for determining homotopy equivalence.

\begin{lem}
    If $(X,A)$ is a CW pair consisting of a CW complex $X$ and a contractible subspace $A$, then the quotient map $X\to X/A$ is a homotopy
    equivalence.
\end{lem}

\begin{exmp}
    For example, one might use this to determine homotopy equivalence for graphs. Indeed, if an edge has distinct endpoints, one can simply collapses the edge
    to a point. After repeatedly applying the above, all edge are loops, and each component of $X$ is either a single vertex or a wedge of circles $\vee_m S^1$.
    Hence the question becomes: how can we prove that $\vee_m S^1\simeq\vee_n S^1$ if and only if $m=n$? We will soon develop the tools needed to answer this question.
\end{exmp}

\begin{exmp}
    Consider the space $X$ which is a union of a torus with a disk embedded inside it. Then $X/A$ is homotopy equivalent to $X$, and we get a croissant shape.
\end{exmp}

\begin{lem}
    Given spaces $X_0$ and $X_1$, we can attach $X_0$ to $X_1$ by identifying points in a subspace $A\subset X_1$ with points in $X_0$.
    That is, given a map from $A\to X_0$, we form a quotient space $X_0\sqcup_f X_1\equiv X_0\coprod X_1/a\sim f(a)$ for all $a\in A$.
    If $(X_1, A)$ is a CW pair and the two attaching maps $f,g:A\to X_0$ are homotopic then $X_0\sqcup_f X_1\simeq X_0\sqcup_g X_1$.
\end{lem}

\begin{exmp}
    Recall the mapping cylinder of two spaces $X$ and $Y$ with $f:X\to Y$. We consider $X_1=X\times I, A=X\times \{1\},X_0=Y$.

    Another example is given by $X_0=X_1=S^1\times D^2$ and $A=\partial X_1$. Taking $f:\partial X_1\to \partial X_0\subset X_0$ by
    $(x,y)\mapsto (y,x)$. Then $X_0\sqcup_f X_1=S^3$.

    Consider the 2-sphere with a subcomplex $A$ be an arc from the north to the south pole. Take this arc and send its points to $S^1$
    in a way that wraps around all of $S^1$. The attachment via this map yields another croissant. Of course, this map is homotopic to
    a constant map (since $A$ is contractible). Attaching via this constant map yields the wedge $S^2\vee S^1$. This shows that the croissant
    is homotopy equivalent to $S^2\vee S^1$.
\end{exmp}

\subsection{The fundamental group}

We will be associating topological spaces to algebraic structures (such as groups or rings) and
continuous maps get associated to group homomorphisms. In the case of the fundamental group, for example
we will obtain a covariant functor. The fundamental group will be useful as something that can tell various
spaces apart.

\section*{Class 3}

\begin{defn}
    A \textbf{path} in a space $X$ is a continuous map $f:I\to X$.
\end{defn}

\begin{defn}
    A \textbf{homotopy of paths} in $X$ is a family $f_t:I\to X$, $t\in I$ such that
    \begin{enumerate}
        \item the endpoints $f_t(0)=X_0$ and $f_t(1)=x_1$ are independent of $t$;
        \item the associated map $F:I\times I\to X$ is continuous.
    \end{enumerate}
    If these conditions are satisfied, then $f_0$ and $f_1$ are \textbf{homotopic}, denoted $f_0\simeq f_1$.
\end{defn}

\begin{exmp}
    Let $f_0,f_1$ be two paths in $\R^n$ with the same endpoints $x_0$ and $x_1$. Then $f_0\simeq f_1$
    via the linear homotopy $f_t(s)=(1-t)f_0(s)+tf_1(s)$. Here we have used only the convexity of $\R^n$;
    more generally, then, this is true in any convex subset of $\R^n$.
\end{exmp}

\begin{prop}
    The relation of homotopy on paths with fixed endpoints in any space is an equivalence relation.
    We will denote the homotopy class of a path $f$ by $[f]$.
\end{prop}
\begin{proof}
    Reflexivity is clear via the constant homotopy, and symmetric by taking $f_{1-t}$ in place of $f_t$.
    Finally, for transitivity, suppose that $f_0\simeq f_1$ via $f_t$ and $f_1=g_0$ and $g_0\simeq g_1$
    via $g_t$. Then $f_0\simeq g_1$ by the homotopy that traverses $f_{2t}$ and then $g_{2t-1}$. Note that
    this is well-defined at $t=1/2$ and that the associated map $H(s,t)$ is continuous because a function
    on the union of two closed sets is continuous if the restriction to each of the closed
    sets is continuous.
\end{proof}

\begin{defn}
    Let $f$ and $g$ be paths such that $f(1)=g(0)$. The \textbf{composition} or \textbf{product path}
    $f\diamond g$ is $f(2s)$ for $0\leq s\leq 1/2$ and $g(2s-1)$ for $1/2\leq s\leq 1$.
\end{defn}

\begin{rem}
    Note that the product operation on paths respects homotopy classes. In particular, if $f_0\simeq f_1$ and
    $g_0\simeq g_1$, via $f_t$ and $g_t$ respectively, and $f_0(1)=g_0(0)$, then $f_t\diamond g_t$ provides
    a homotopy between $f_0\diamond g_0$ and $f_1\diamond g_1$.
\end{rem}

\begin{defn}
    A \textbf{reparametrization} of a path $f$ is a composition $f\circ \phi$ where $\phi:I\times I$ is
    a continuous map such that $\phi(0)=0$ and $\phi(1)=1$.
\end{defn}

\begin{rem}
    Note that $f\circ\phi\simeq f$ via $f\circ \phi_t$ where $\phi_t(s)=(1-t)\phi(s)+ts$.
\end{rem}

\begin{defn}
    A \textbf{loop} is a path $f:I\to X$ with $f(0)=f(1)=x_0$. The point $x_0$ is the \textbf{basepoint}.
\end{defn}

\begin{defn}
    The set of all homotopy classes $[f]$ of loops $f:I\to X$ (or equivalently $f:S^1\to X$) at the basepoint
    $x_0$ is $\pi_1(X,x_0)$, the \textbf{fundamental group of $X$ at basepoint $x_0$}.
\end{defn}

\begin{prop}
    $\pi_1(X,x_0)$ is a group with respect to the product $[f][g]=[f\diamond g]$.
\end{prop}
\begin{proof}
    Since we are restricted to loops with a fixed basepoint $x_0$, we can always concatenate them.
    The product $[f][g]=[f\diamond g]$ is well-defined since we showed that the product respects
    homotopy classes. Let us now check the group axioms. The product of two loops at $x_0$ is
    again a loop at $x_0$, and hence the set is closed under the operation. Let $c$ be the
    constant path at $x_0$, that is $c(s)=x_0$ for all $s\in I$. Let $\phi$ be the reparametrization
    that traverses $f$ in the first half, and stays at the endpoint $f(1)$ for the second half.
    Then $f\circ\phi\simeq f\diamond c$ so $f\diamond c\simeq f$. Similarly, if $\phi$ stays at $f(0)$
    for the first half and then traverse $f$ in the second half, we find that $f\circ\phi\simeq c\diamond f$
    and hence $c\diamond f\simeq f$. Hence $c$ is the identity of $\pi_1(X,x_0)$.

    Next let us show that the group operation is associative. Given loops $f,g,h$, we want to show that
    $f\diamond(g\diamond h)\simeq (f\diamond g)\diamond h$. But note that
    $f\diamond(g\diamond h)\simeq (f\diamond g)\diamond h\circ\phi$
    where $\phi$ sends $1/2$ to $1/4$ and $3/4$ to $1/2$ and is linear everywhere else, and hence
    associativity holds.

    Next let us determine inverses. Given a path $f$, define the inverse path to be $\bar f(s)=f(1-s)$.
    Let $f$ be a loop.  First note that $f\diamond\bar f\simeq c$ via $h_t=f_t\diamond g_t$ where
    $f_t$ is $f$ on $s\in[0,1-t]$ and $f(1-t)$ on $s\in[1-t,1]$ and $g_t=\bar f_{t}$. This gives us
    the desired homotopy between $f\diamond\bar f$ to $c\diamond\bar c=c$. Similarly for the other direction.
    Hence $\bar f$ is an inverse for $f$. This concludes the proof.
\end{proof}

\begin{exmp}
    Suppose $X\subset\R^n$, $X$ convex with $x_0\in X$. Then $\pi_1(X,x_0)=0$ since any two loops
    $f$ and $g$ are homotopic via the linear homotopy $h_t(s)=(1-t)f(s)+tg(s)$.
\end{exmp}

It is natural to ask how $\pi_1(X,x_0)$ depends on the basepoint $x_0$; of course, $\pi_1(X,x_0)$ only
detects the path-component of $X$ that contains $x_0$. Hence let $x_1$ lie in the same path component of
$X$ as $x_0$ and let $h:I\to X$ be a path from $x_0$ to $x_1$ with inverse $\bar h(s)=h(1-s)$. Then if $f$
is a loop at $x_1$, then $h\diamond f\diamond \bar h$ is a loop at $x_0$.

\begin{prop}
    The map $\beta_h:\pi_1(X,x_1)\to\pi_1(X,x_0)$ given by $[f]\mapsto[h\diamond f\diamond \bar h]$
    is an isomorphism.
\end{prop}
\begin{proof}
    It $f_t$ is a homotopy of a loops based at $x_1$, then $h\diamond f_t\diamond\bar h$ is a homotopy
    of loops at $x_0$, so the map is well-defined. Next, note that $\beta_h$ is indeed a homomorphism,
    as $\beta_h[f\diamond g]=[h\diamond f\diamond \bar h\diamond h\diamond g\diamond \bar h]=\beta_h[f]\beta_h[g]$.
    It now suffices to find an inverse to $\beta_h$. Let us try $\beta_{\bar h}$. Note that
    $\beta_h\beta_{\bar h}[f]=\beta_h[\bar h\diamond f\diamond h]=[h\diamond\bar h\diamond f\diamond h\diamond\bar h]=[f]$.
    Similarly to show that $\beta_{\bar h}$ is a right-inverse. This shows that $\beta_h$ is an isomorphism.
\end{proof}

\section*{Class 4}

\begin{thm}
    The map $\Phi:\Z\to\pi_1(S^1)$ given by $n\mapsto [\omega_n]$ is an isomorphism, where
    $\omega_n(s)=\left( \cos(2\pi ns),\sin(2\pi ns) \right)$.
\end{thm}
\begin{proof}
    See Hatcher p.29-31.
\end{proof}

\section*{Class 5}

\begin{thm}[Brouwer's fixed point theorem]
     Every continuous map $h:D^2\to D^2$ has a fixed point, i.e. there exists an $x\in D^2$ such taht 
     $h(x)=x$.
\end{thm}
\begin{proof}
    Suppose not, i.e. that $h(x)\neq x$ for all $x\in D^2$. Then, define a map $r:D^2\to S^1$ where
    $r(x)$ is the point where the ray from $h(x)$ to $x$ hits $S^1=\partial D^2$. Notice that $r(x)$
    is continuous, and that if $x\in S^1$ then $r(x)=x$. In particular, $r$ is a retraction of $D^2$
    onto $S^1$. Let $f_0$ be any loop in $S^1$ based at $x_0$. In $D^2$, $f_0\simeq c$, where $c$ is
    the constant loop at $x_0$, via the linear homotopy $f_t$, for example. Since $r|_{S^1}=\id|_{S^1}$,
    $r\circ f_t$ is a homotopy in $S^1$ from $r\circ f_0=f_0$ to $c$. This is a contradiction, as
    $\pi_1(S^1)\neq 0$.
\end{proof}

\begin{thm}[Borsuk-Ulam]
    Let $f:S^2\to\R^2$ be a continuous map. Then there exists a pair of antipodal points -- say $x$
    and $-x$ in $S^2$ -- such that $f(x)=f(-x)$.
\end{thm}
\begin{proof}
    For sake of contradiction, assume that no such $x$ exists. Then define $g:S^2\to S^1$ by
    \[g(x)=\frac{f(x)-f(-x)}{|f(x)-f(-x)|}.\]
    This is well-defined, as the denominator is never zero and furthermore, $g(-x)=-g(x)$.
    Now define $\eta:I\to S^2\subset\R^3$ given by $s\mapsto(\cos 2\pi s,\sin 2\pi s,0)$, whose
    image sits on the equator, and let $h:I\to S^1$ be $g\circ\eta$, which yields a loop in $S^1$.
    Since $g(-x)=-g(x)$, we can write that $h(s+1/2)=-h(s)$ for all $s\in [0,1/2]$. If we lift the loop
    $h$ to a path $\tilde h:I\to \R$, we can write $\tilde h(s+1/2)=\tilde h(s)+q/2$ for some
    odd $q\in \Z$. \textit{A priori}, $q$ might depend (continuously) on $s$, but integrality
    forces it to be constant. Then we find that
    \[\tilde h(1)=\tilde h(1/2)+q/2,\tilde h(1/2)=\tilde h(0)+q/2,\]
    which implies that $\tilde h(1)=\tilde h(0)+q$. Hence $h$ is $q$ times a generator for $\pi_1(S^1)$.
    As $q$ is odd, $h$ cannot be nullhomotopic. However, $h=g\circ\eta:I\to S^2\to S^1$ and $\eta$
    is nullhomotopic in $S^2$ via, say $f_t$,then $g\circ\eta$ is nullhomotopic via $g\circ f_t$,
    which is a contradiction.
\end{proof}

\begin{prop}
    Suppose $X$ and $Y$ are path-connected. Then $\pi_1(X\times Y)\cong \pi_1(X)\times\pi_1(Y)$.
\end{prop}
\begin{proof}
    Recall that if $X\times Y$ is endowed with the product topology, then $f:Z\to X\times Y$
    is continuous if and only if $g:Z\to X$ and $h:Z\to Y$ are continuous, where $f(z)=(g(z),h(z))$.
    Hence a loop in $X\times Y$ based at $(x_0,y_0)$ is equivalent to a pair of loops (under
    the two projections) based at $x_0$ and $y_0$ respectively. Denote these loops by
    $g:I\to X$ and $h:I\to X$. Furthermore, a homotopy $f_t:I\to X\times Y$ is equivalent
    to a pair of homotopies $g_t,h_t$. Hence there is a bijection between $\pi_1(X\times Y,(x_0,y_0))$
    and $\pi_1(X,x_0)\times\pi_1(Y,y_0)$. This is easily checked to be a homomorphism.
\end{proof}

\subsection*{Induced homomorphisms}

Suppose $\phi:(X,x_0)\to (Y,y_0)$, i.e. $\phi:X\to Y$ with $\phi(x_0)=y_0$. Then $\phi$ induces
a homomorphism $\phi_*:\pi_1(X,x_0)\to\pi_1(Y,y_0)$, given by $[f]\mapsto [\phi\circ f]$. Note
that $\phi$ is indeed well-defined since a homotopy $f_t$ of loops in $X$ at $x_0$ gets sent
by $\phi$ to a homotopy of loops in $Y$ at $y_0$. Hence
\[\phi_*[f_0]=[\phi\circ f_0]=[\phi\circ f_1]=\phi_*[f_1].\]
Note also that it is a homomorphism, as $\phi(f\diamond g)=(\phi\circ f)\diamond(\phi\circ g)$.
Two important properties of these induced homomorphisms is that
\[ (\phi\circ\psi)_*=\phi_*\circ\psi_*,\]
i.e. it is covariant, and that
\[(\id_{(X,x_0)})_*=\id_{\pi_1(X,x_0)}.\]
These two properties show that $\pi_1$ is a covariant functor from pointed topological space to groups.
One simple consequence of this is that if $\psi$ is the inverse of $\phi$ then $\psi_*$ is the inverse
of $\phi_*$, since $\phi_*\circ\psi_*=(\phi\circ\psi)_*=\id_*=\id$ (and similarly for the other order
of composition).

\begin{prop}
    If $X$ retracts onto a subspace $A$, then the homomorphism $\iota_*:\pi_1(A,x_0)\to\pi_1(X,x_0)$
    induced by the inclusion $\iota:A\to X$ is injective.
\end{prop}
\begin{proof}
    If $v:X\to A$ is a retraction than $r\circ i=\id_A$, so $r_*\circ \iota_*=\id$, so $\iota_*$
    must be injective.
\end{proof}

\begin{prop}
    If $\phi:X\to Y$ is a homotopy equivalence, then the induced homomorphism $\phi_*:\pi_1(X,x_0)\to\pi_1(Y,y_0)$
    is an isomorphism for all $x_0\in X$.
\end{prop}

\begin{lem}
    If $\phi_t:X\to Y$ is a homotopy and $h$ is a path $\phi_t(x_0)$ formed by images of the basepoint
    $x_0\in X$ then the following diagram commutes.
    \begin{equation*}
        \begin{tikzcd}
            \pi_1(X,x_0)\arrow{rr}{(\phi_1)_*}\arrow{rd}{(\phi_0)_*}&&\pi_1(Y,\phi_1(x_0))\arrow{ld}{\beta_h}\\
            &\pi_1(Y,\phi_0(x_0))&
        \end{tikzcd}
    \end{equation*}
\end{lem}
\begin{proof}
    Let $h_t(s)=h(ts)$. This is the restriction of $h$ to $[0,t]$, reparametrized so the domain is still
    $[0,1]$. If $f$ is a loop in $X$ at $x_0$ then $h_t\diamond\phi_t\circ f\diamond \bar h_t$ is a
    homotopy of loops at $\phi_0(x_0)$ ($h$ is a path from $\phi_0(x_0)$ to $\phi_1(x_0)$). Restricting
    to $t=0$ and $t=1$ we have $(\phi_0)_*([f])=\beta_h\left( (\phi_1)_*[f] \right))$, homotopic.
\end{proof}

\begin{proof}[Proof of proposition]
    Let $\psi:Y\to X$ be the homotopy inverse of $\phi$, i.e. $\phi\circ\psi\simeq\id_Y$ and $\psi\circ\phi\simeq\id_X$.
    Consider the sequence of maps
    \begin{equation*}
        \begin{tikzcd}
            \pi_1(X,x_0)\arrow{r}{\phi_*}&\pi_1(Y,\phi(x_0))\arrow{r}{\psi_*}&\pi_1(X,\psi\circ\phi(x_0))\arrow{r}&\cdots
        \end{tikzcd}
    \end{equation*}
    Note that $\psi_*\circ\phi_*$ is an isomorphism since $\psi\circ\phi\simeq\id_X$ so $\psi_*\circ\phi_*=\beta_h$
    for some $h$, by the lemma above. Thus $\phi_*$ is injective; similarly $\phi_*\circ\psi_*$ is an isomorphism
    for $\phi_*$ is surjective.
\end{proof}

\section*{Class 6}

Followed Hatcher's proof of the Seifert-van Kampen theorem word-for-word.
\begin{rem}
    We will have our first midterm on Feb. 24.
\end{rem}


\section*{Covering spaces}

\begin{defn}
    A \textbf{covering space} of a space $X$ is a space $\tilde X$ together with
    a map $p:\tilde X\to X$ satisfying the following condition: there exists an open
    cover $\left\{ U_\alpha \right\}$ of $X$ such that for each $\alpha$, $p^{-1}(U_\alpha)$
    is a disjoint union of open sets in $\tilde X$, each of which is mapped by $p$
    homeomorphically onto $U_\alpha$.
\end{defn}

\begin{rem}
    Note that we do not require $p^{-1}(U_\alpha)$ to be nonempty, so $p$ need not
    be surjective.
\end{rem}

The distinctive feature of covering spaces is their behavior with respect to lifting of maps.
Recall that a \textbf{lift} of a map $f:Y\to X$ is a map $\tilde f:Y\to\tilde X$ such that $p\tilde f=f$.

Let us cite some useful lemmas.

\begin{lem}[Homotopy lifting property]
    Given a covering space $p:\tilde X\to X$, a homotopy $f_t:Y\to X$, and a
    map $\tilde f_0:Y\to\tilde X$ lifting $f_0$, then there exists a unique homotopy
    $\tilde f_t:Y\to\tilde X$ of $\tilde f_0$ that lifts $f_t$.
\end{lem}
\begin{proof}
    We omit this proof, as the version proved for $\pi_1(S^1)=0$ holds for covering spaces in general.
\end{proof}


\end{document}
