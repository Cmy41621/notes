\documentclass{../../mathnotes}

\usepackage{tikz-cd}
\usepackage{todonotes}

\title{Introduction to Algebraic Topology PSET 8}
\author{Nilay Kumar}
\date{Last updated: \today}


\begin{document}

\maketitle

\begin{prop}
    Hatcher exercise 2.1.11 
\end{prop}
\begin{proof}
    Let $\iota:A\to X$ be the inclusion of $A$ into $X$, and $r:X\to X$ be the retract of $X$ onto $A$.
    The composition $r\circ \iota:A\to A$ yields the identity $\id_A:A\to A$. The induced maps on the homology are
    $(r\circ\iota)_*=r_*\circ\iota_*=\id:H_n(A)\to H_n(A)$. This map is of course injective, which implies that
    $\iota_*:H_n(A)\to H_n(X)$ must be injective as well.
\end{proof}

\begin{prop}
    Hatcher exercise 2.1.12
\end{prop}
\begin{proof}
    Let us show that the relation of chain homotopy between chain maps is an equivalence relation.
    Consider $f_\#,g_\#,h_\#:C_n(A)\to C_{n+1}(B)$. The relation is clearly reflexive, as $f_\#\sim f_\#$
    by the zero morphism $0:C_n(A)\to C_{n+1}(B)$. Symmetry holds as follows: if $f_\#\sim g_\#$ via a
    chain homotopy $h$, then $g_\#\sim f_\#$ via the chain homotopy $-h$, because then 
    \begin{align*}
        f_\#-g_\#&=\partial h+h\partial\\
        g_\#-f_\#&=-(\partial h+h\partial)\\
        &=\partial (-h)+(-h)\partial.
    \end{align*}
    Finally, the relation is transitive, because given $f_\#\sim g_\#$ via $H_1$ and $g_\#\sim h_\#$
    via $H_2$, we can add the two commutation relations to obtain that
    \begin{align*}
        f_\#-h_\#&=\partial H_1+H_1\partial+\partial H_2+H_2\partial\\
        &=\partial(H_1+H_2)+(H_1+H_2)\partial,
    \end{align*}
    as desired.
\end{proof}

\begin{prop}
    Hatcher exercise 2.1.14
\end{prop}
\begin{proof}
    Consider the sequence
    \begin{equation*}
        \begin{tikzcd}
            0\ar{r} & \Z_4 \ar{r}{\phi} & \Z_8\oplus\Z_2\ar{r}{\psi} & \Z_4\ar{r} & 0
        \end{tikzcd}
    \end{equation*}
    with $\phi$ taking the generator of $\Z_4$ to $(2,1)\in\Z_8\oplus\Z_2$, which is clearly a well-defined
    injective map. If we now quotient $\Z_8\oplus\Z_2$ by $H=\text{im }\phi$, we obtain four cosets: $(0,0)H, (0,1)H, (1,0)H,$
    and $(1,1)H$, which is clearly isomorphic to $\Z_4$ (with $(0,1)H$ as the generator). This yields
    a short exact sequence.

    Consider now more generally the sequence of groups
    \begin{equation*}
        \begin{tikzcd}
            0\ar{r} & \Z_{p^m}\ar{r}{\phi} & A\ar{r}{\psi} & \Z_{p^n}\ar{r} & 0
        \end{tikzcd}
    \end{equation*}

    Note first that $A = \Z_{p^{m+n-k}} \oplus \Z_{p^{k}}$, for $k \leq \min(m,n)$, fits into the sequence. 
    Indeed, we let $\phi(1)=(p^{n-k},1)$, which is injective because the image along the first factor is injective.
    We now need to choose $\psi$ such that $\ker\psi=(jp^{n-k},j)$ for $j\in\Z_{p^m}$. We claim that choosing
    $\psi(1,0)=1$ fixes the map because $\psi(i,j)=i\psi(1,0)+j\psi(0,1)$ but since $\psi(jp^{n-k},p)=0$,
    we find that $\psi(0,1)=-p^{n-k}$. Hence, with this choice of $\psi(1,0)$, we find that
    $\psi(i,j)=i-p^{n-k}j.$ It is easy to see that $\ker\psi=\text{im }\phi$. The image of $\psi$
    is cyclic, generated by $\psi(1,0)$ (as $\psi(i,j)=(i-jp^{n-k})\psi(1,0)$), and has order $p^{m+n}/p^m$, 
    and hence isomorphic to $\Z_{p^n}$.
    Note, however, that it is unclear that these are the \textit{only} groups that fit into this sequence
    (though it might be possible to invoke the fundamental theorem of finitely generated abelian groups).

    Finally, consider the sequence
    \begin{equation*}
        \begin{tikzcd}
            0\ar{r} & \Z\ar{r}{\phi} & A\ar{r}{\psi} & \Z_{n}\ar{r} & 0
        \end{tikzcd}
    \end{equation*}
    We claim that $A=\Z\oplus\Z_d$ fits into this sequence for $d|n$. Indeed, we define $\phi(1)=(1,n/d)$
    and then $\psi(0,1)=1$, which forces $\psi(i,j)=(j-in/d)\psi(0,1)=j-in/d$. The sequence is clearly exact,
    and $\text{im }\psi$ is cyclic. It suffices to compute the order of $\text{im }\psi$. This is done by
    counting the number of lattice points of $\Z^2$ contained in the parallelogram spanned by $(0,d)$ and $(1,n/d)$,
    which is simply $d\cdot n/d=n$. Hence we obtain $\Z_n$, as desired.
    Again, it is not clear that these are the \textit{only} groups that fit into this sequence.
\end{proof}

\begin{prop}
    Hatcher exercise 2.1.15
\end{prop}
\begin{proof}
    Consider the exact sequence
    \begin{equation*}
        \begin{tikzcd}
            A\ar{r}{\alpha}&B\ar{r}{\beta}&C\ar{r}{\gamma}&D\ar{r}{\delta}&E.
        \end{tikzcd}
    \end{equation*}

    Exactness at $B$ requires $\ker\beta=\text{im }\alpha$, and hence $\alpha$ is surjective if and only
    if $\ker\beta=B$. Exactness at $D$ requires $\ker\delta=\text{im }\gamma$, and hence $\delta$ is injective
    if and only if $\text{im }\gamma=0$.
    Hence if (and only if) $\alpha$ is surjective and $\delta$ is injective then $\gamma=0$ and $\beta=0$ and the exactness at $C$
    (requiring that $\ker\gamma=\text{im }\beta$) forces $C=0$.

    Hence for a good pair $(X,A)$, we find that $H_n(X,A)=0$ if and only if the inclusion $A\to X$ induces
    isomorphisms on all homology groups, as the long exact sequence of theorem 2.13 splits into sequences
    \begin{equation*}
        \begin{tikzcd}
            0\ar{r}&\tilde H_n(A)\ar{r}{\iota_*}&\tilde H_n(X)\ar{r}&0
        \end{tikzcd}
    \end{equation*}
    for all $n$.
\end{proof}

\begin{prop}
    Let $A$ and $B$ be chain complexes. A chain map $f:A\to B$ is a \textnormal{chain homotopy equivalence}
    if there exists a chain map $g:B\to A$ such that $f\circ g\sim\id_B$ and $g\circ f\sim\id_A$ in the
    sense of chain homotopies.
    \begin{enumerate}[(a)]
        \item Prove that if $f:A\to B$ is a chain homotopy equivalence, then $f$ induces an isomorphism on homology.
        \item Give an example of chain complexes $A$ and $B$ with isomorphic homology but no chain homotopy equivalence
            between them. (Hint: let $A$ be $\Z$ in two consecutive gradings and zero everywhere else.)
    \end{enumerate}
\end{prop}
\begin{proof}\hfill
    \begin{enumerate}[(a)]
        \item Recall that chain-homotopic maps induce the same homorphism on homology. Hence $(f\circ g)_*=f_*\circ g_*=(\id_B)_*=\id_{H_n(B)}$
            and $(g\circ f)_*=g_*\circ f_*=(\id_A)_*=\id_{H_n(A)}$. As $\id_{H_n(B)}$ is injective, $f_*:H_n(A)\to H_n(B)$ must be as well, and 
            as $\id_{H_n(A)}$ is surjective, $f_*$ must be as well. Hence $f_*$ is an isomorphism.
        \item Consider the map of chain complexes $f:A_\bullet\to B_\bullet$ given by the first two rows of
            \begin{equation*}
                \begin{tikzcd}
                    \cdots\ar{r} & \Z_2\ar{r}\ar{d} & \Z_4\ar{r}\ar{d} & \Z_2\ar{r}\ar{d} & 0\ar{r} & \cdots\\
                    \cdots\ar{r}&0\ar{r}\ar{d}&\Z\ar{r}\ar{d}&\Z\ar{r}\ar{d}&0\ar{r}&\cdots\\
                    \cdots\ar{r} & \Z_2\ar{r} & \Z_4\ar{r} & \Z_2\ar{r} & 0\ar{r} & \cdots
                \end{tikzcd}
            \end{equation*}
            where each square clearly commutes. The homology groups of the two sequences are $H_\bullet(A)=0$ and $H_\bullet(B)=0$.
            However, there does not exist a chain homotopy equivalence between $A_\bullet$ and $B_\bullet$, as we now show.
            If there did exist one, there would exist a chain map $g:B_\bullet\to A_\bullet$ (between the second two rows)
            such that the appropriate compositions
            of $f$ and $g$ would be chain homotopic to $\id_B$ and $\id_A$ via some chain homotopy $h$. Of course, the only possible
            compositions are zero, and hence 
            \begin{equation*}
                \begin{tikzcd}
                    {} & \Z_4\ar{r}{\pi}\ar[swap]{dl}{h_1}\ar{d}{0} & \Z_2\ar{dl}{h_2}\\
                    \Z_2\ar{r}{\iota} & \Z_4 &
                \end{tikzcd}
            \end{equation*}
            there must exist $h_1,h_2$ such that $\id_{\Z_4}=\iota\circ h_1+h_2\circ\pi$, but $h_1$ can only be the zero map
            or the quotient map and $h_2$ can only be the zero map or the inclusion map. It is easy to see that none of these
            combinations recover $\id_{\Z_4}$.
    \end{enumerate}
\end{proof}


\end{document}
