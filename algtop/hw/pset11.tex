\documentclass{../../mathnotes}

\usepackage{tikz-cd}
\usepackage{todonotes}

\title{Introduction to Algebraic Topology PSET 11}
\author{Nilay Kumar}
\date{Last updated: \today}


\begin{document}

\maketitle

\begin{prop}
    Hatcher exercise 2.2.43
\end{prop}
\begin{proof} \hfill
    \begin{enumerate}[(a)]
        \item Given a chain complex of free abelian groups $C_n$, we wish to show that it
            splits as a direct sum of subcomplexes $0\to L_{n+1}\to K_n\to 0$ with at most two
            nonzero terms. First note that for any $n$ the first isomorphism theorem yields the
            exact sequence
            \begin{equation*}
                \begin{tikzcd}
                    0\ar{r} & \ker \partial_n\ar{r} & C_n\ar{r} & \text{im }\partial_n\ar{r} & 0.
                \end{tikzcd}
            \end{equation*}
            This exact sequence splits, as we can find a section $s$ of $\partial_n:C_n\to\text{im }\partial_n$.
            Indeed, choose a basis $\left\{ e_\alpha \right\}$ for $\text{im }\partial_n$ (as a subgroup of a free abelian group it must
            be free abelian as well), and define $s:\text{im }\partial_n\to C_n$ by taking $s(e_\alpha)$ to be
            any element $c_\alpha\in C_n$ such that $\partial_nc_\alpha=e_\alpha$. Hence we can write
            $C_n\cong \ker \partial_n\oplus \text{im }\partial_n$, which now allows us to decompose the chain
            complex $C_n$. Indeed, we take $K_n$ above to be $\ker \partial_n$ and $L_{n+1}$ to be $\text{im }\partial_{n+1}$:
            \begin{equation*}
                \begin{tikzcd}[row sep=tiny]
                    \;&\cdots\ar{r} & C_2\ar{r} & C_1\ar{r} & C_0\ar{r} & 0\\
                    \;&\;& 0\ar{r} & \text{im }\partial_1\ar{r} & \ker \partial_0\ar{r} & 0\\
                    \;& 0\ar{r} & \text{im }\partial_2\ar{r} & \ker\partial_1\ar{r} & 0\\
                    0\ar{r} & \text{im }\partial_3\ar{r} & \ker\partial_2\ar{r} & 0\\
                    \; & \; & \vdots
                \end{tikzcd}
            \end{equation*}
        \item Now suppose that the groups $C_n$ are finitely generated. Then the map $L_{n+1}\to K_n$ is
            a linear transformation of finite-dimensional vector space. Note now that $L_{n+1}\subset K_n$
            as $C_n$ is a chain complex and $\partial^2=0$ requires that $\text{im }\partial_{n+1}\subset\ker\partial_n$.
            Now, without loss of generality, we can write $\text{im }\partial_{n}\cong\Z^j$ and $\ker\partial_n\cong \Z^k$
            for $j\leq k$. By an appropriate change of basis, we can write the map $\partial_n$ as taking
            each basis vector in $\Z^j$ to a multiple of a basis vector in $\Z^k$. This yields a further splitting
            of the complex into summands of the form $0\to\Z\to 0$ and $0\to\Z\to\Z\to 0$, where the first type
            generate the subspace of $\Z^k$ not in the image of the map $L_{n+1}\to K_n$ and the latter type
            generate the image.
        \item Let $X$ be a CW complex with finitely many cells in each dimension. The chain complex $C_n$ 
            obtained by cellular homology satisfies the conditions necessary for the results of part $(b)$ to hold.
            From this chain complex we can compute the homology groups $H_n(X)$. If we now consider instead
            the chain complex $C_n$ with coefficients in $G$ (again obtained by cellular homology)\ldots
            I'm not sure how to relate the $\Z$ case to the $G$ case.
    \end{enumerate}
\end{proof}

\begin{prop}
    Hatcher exercise 2.3.1
\end{prop}
\begin{proof}
    Let $T_n(X,A)$ denote the torsion subgroup of $H_n(X,A;\Z)$.
    Consider the functors $(X,A)\mapsto T_n(X,A)$, with the obvious induced homomorphisms $T_n(X,A)\to T_n(Y,B)$
    and boundary maps $T_n(X,A)\to T_{n-1}(A)$. 
    Consider the CW pair $(\RP^2,A)$, where $A$ is the union of the zero- and one-cells. It is easy to see that
    $\tilde H_2(\RP^2)=0,\tilde H_1(\RP^2)=\Z_2,\tilde H_0(\RP^2)=0$, $\tilde H_2(A)=0,\tilde H_1(A)=\Z,\tilde H_0(A)=0$,
    and since $X/A=S^2$, we have that $\tilde H_2(X/A)=0,\tilde H_1(X/A)=\Z, \tilde H_0(X/A)=0$.
    Working now instead with the torsion functor, we obtain via the second axiom the sequence $0\to \Z_2\to 0$, which is
    not exact, which is a contradiction. If, instead, we were to work with the mod-torsion functor
    $MT_n(X,A)=H_n(X,A;\Z)/T_n(X,A)$, we would find the sequence $0\to\Z\to0\to\Z\to 0$, which is again not exact.
    
\end{proof}

\begin{prop}
    Hatcher exercise 2.3.4
\end{prop}
\begin{proof}
    We proceed by induction on the number of wedge summands. Consider first the CW pair $(X_1\vee X_2, X_1)$, i.e. $n=2$.
    We obtain a long exact sequence
    \begin{equation*}
        \begin{tikzcd}
            \cdots\ar{r}{\partial} & \tilde h_n(X_1)\ar{r}{i_*} & \tilde h_n(X)\ar{r}{q_*} & \tilde h_n(X_2)\ar{r}{\partial} & \tilde h_{n-1}(X_1)\ar{r}{i_*} & \cdots.
        \end{tikzcd}
    \end{equation*}
    From the retraction $q:X_1\vee X_2\to X_1$ we find that $i_*$ in the diagram above is injective, which yields
    a number of short exact sequences
    \begin{equation*}
        \begin{tikzcd}
            0\ar{r}{\partial} & \tilde h_n(X_1)\ar{r}{i_*} & \tilde h_n(X)\ar{r}{q_*} & \tilde h_n(X_2)\ar{r}{\partial} & 0,
        \end{tikzcd}
    \end{equation*}
    which, due to the existence of the retraction splits as $\tilde h_n(X)\cong \tilde h_n(X_1)\oplus \tilde h_n(X_2)$.
    
    Now suppose we have that $\tilde h_n(\vee_i^n X_i)\cong \oplus_i^n \tilde h_n(X_i)$. Then consider the CW pair
    $(\vee_i^n X_i\vee X_{n+1}, X_{n+1})$. By exactly the same reasoning as above (using the retraction from the wedge
    sum to $X_{n+1}$) we find that $\tilde h_n(\vee_i^n X_i\vee X_{n+1})\cong \oplus_i^n\tilde h_n(X_i)\oplus \tilde h_n(X_{n+1})$,
    which concludes the proof.
\end{proof}

\begin{prop}
    Hatcher exercise 3.1.6
\end{prop}
\begin{proof}\hfill
    \begin{enumerate}[(a)]
        \item Consider the usual simplicial structure for the torus $T^2$ (Hatcher p. 102) consisting
            of 1 0-simplex $v$, 3 1-simplices $a,b,c$, and 2 2-simplices $U,L$. Recall that
            we obtain the chain complex
            \begin{equation*}
                \begin{tikzcd}
                    0\ar{r} & \Z^2\ar{r}{a+b-c} & \Z^3\ar{r}{0} & \Z\ar{r} & 0
                \end{tikzcd}
            \end{equation*}
            with homology $H_0(T^2)=\Z,H_1(T^2)=\Z^2,H_2(T^2)=\Z$. The corresponding cochain complex is
            \begin{equation*}
                \begin{tikzcd}
                    0 & \Z^2\ar[swap]{l}{\delta_2} & \Z^3\ar[swap]{l}{\delta_1} & \Z\ar[swap]{l}{\delta_0} & 0\ar{l}
                \end{tikzcd}
            \end{equation*}
            where $\delta_0=\partial_1^*=0$ and $\delta_1=\partial_2^*$. Clearly $H^0(T^2)=\Z$. Next, note that
            $H^1(T^2)=\ker\delta_1/\text{im }\delta_0=\ker\delta_1=\ker\partial_2^*$. We can write a basis for $C_1^*$
            as $\{\alpha,\beta,\gamma\}$ dual to $a,b,c$ respectively. For any $\phi\in C_1^*$, we find that
            $\delta_1\phi=\phi\circ\partial_2$, but the image of $\partial_2$ is $a+b-c$, and hence
            $\ker\delta_1=\langle \alpha-\gamma,\beta-\gamma \rangle$ and $H^1(T^2)=\Z^2$. Finally,
            $H^2(T^2)=\Z^2/\text{im }\delta_1=\Z$ by rank-nullity.

            Now consider the case of $\Z_2$ coefficients. We obtain the chain complex
            \begin{equation*}
                \begin{tikzcd}
                    0\ar{r} & \Z_2^2\ar{r} & \Z_2^3\ar{r} & \Z_2\ar{r} & 0
                \end{tikzcd}
            \end{equation*}
            where $\partial_1=0$ and $\partial_2=a+b+c$. The homology is given $H_0(T^2;\Z_2)=\Z$,
            $H_1(T^2;\Z_2)=\Z_2^3/\Z_2=\Z_2^2$, $H_2(T^2;\Z_2)=\Z_2$. The associated cochain complex
            is given
            \begin{equation*}
                \begin{tikzcd}
                    0 & \Z_2^2\ar[swap]{l}{\delta_2} & \Z_2^3\ar[swap]{l}{\delta_1} & \Z_2\ar[swap]{l}{\delta_0} & 0\ar{l}
                \end{tikzcd}
            \end{equation*}
            where $\delta_0=0$. Clearly $H^0(T^2;\Z_2)=\Z_2$. Next, $H^1(T^2;\Z_2)=\ker\delta_1=\ker\partial_2^*$.
            As above, we can write a basis for $C_1^*$ to be $\{\alpha,\beta,\gamma\}$ dual to $a,b,c$ respectively.
            Since the image of $\partial_2$ is $a+b+c$, we find that $H^1(T^2;\Z_2)=\langle \alpha+\beta, \beta+\gamma\rangle=\Z_2^2$.
            Finally, $H^2(T^2;\Z_2)=\Z_2^2/\text{im }\delta_2=\Z_2$ by rank-nullity.
        \item Consider the usual simplicial structure for $\RP^2$ (Hatcher p. 102) consisting
            of 2 0-simplices $v,w$, 3 1-simplices $a,b,c$, and 2 2-simplices $U,L$. We obtain the chain
            complex
            \begin{equation*}
                \begin{tikzcd}
                    0\ar{r} & \Z^2\ar{r} & \Z^3\ar{r} & \Z^2\ar{r} & 0
                \end{tikzcd}
            \end{equation*}
            where $\partial_1a=\partial_1b=w-v, \partial_1c=0$, $\partial_2U=-a+b+c,\partial_2L=a-b+c$.
            The homology is given $H_0(\RP^2)=\Z, H_1(\RP^2)=\Z_2, H_2(\RP^2)=0$. The associated
            cochain complex is
            \begin{equation*}
                \begin{tikzcd}
                    0 & \Z^2\ar[swap]{l}{\delta_2} & \Z^3\ar[swap]{l}{\delta_1} & \Z^2\ar[swap]{l}{\delta_0} & 0\ar{l}.
                \end{tikzcd}
            \end{equation*}
            By definition, $H^0(\RP^2)=\ker\delta_0=\ker\partial_1^*$. Note that $\partial_1^*\nu$ is a morphism that maps
            $a\mapsto -1,b\mapsto -1,c\mapsto 0$ and $\partial_1^*\omega$ is a morphism that maps $a\mapsto1,b\mapsto1,c\mapsto0$
            and hence $H^0(\RP^2)=\langle \nu+\omega\rangle=\Z$. Next, $H^1(\RP^2)=\ker\delta_1/\text{im }\delta_0=\ker\delta_1/\Z$.
            Note that $\partial_2^*\alpha$ is a morphism that maps $U\mapsto -1,L\mapsto 1$, $\partial_2^*\beta$ is a morphism
            that maps $U\mapsto 1,L\mapsto -1$, and $\partial_2^*\gamma$ is a morphism that maps $U\mapsto 1,L\mapsto 1$. Thus
            $H^1(\RP^2)=\Z/\langle\alpha+\beta\rangle=0$. Finally, $H^2(\RP^2)=\langle\mu,\lambda\rangle/\text{im }\delta_1$.
            By above, we find that $H^2(\RP^2)=\langle\mu,\lambda\rangle/\langle\mu-\lambda,\mu+\lambda\rangle=
            \langle \mu-\lambda,\lambda\rangle/\langle\mu-\lambda,\mu+\lambda\rangle=\Z_2$.

            Now consider the case of $\Z_2$ coefficients. We obtain the chain complex
            \begin{equation*}
                \begin{tikzcd}
                    0\ar{r} & \Z_2^2\ar{r} & \Z_2^3\ar{r} & \Z_2^2\ar{r} & 0
                \end{tikzcd}
            \end{equation*}
            where $\partial_1a=\partial_1b=v+w,\partial_1c=0$ and $\partial_2U=\partial_2L=a+b+c$. The homology is easily
            found $H_0(\RP^2;\Z_2)=H_1(\RP^2;\Z_2)=H_2(\RP^2;\Z_2)=\Z_2$. The associated cochain complex is
            \begin{equation*}
                \begin{tikzcd}
                    0 & \Z_2^2\ar[swap]{l}{\delta_2} & \Z_2^3\ar[swap]{l}{\delta_1} & \Z_2^2\ar[swap]{l}{\delta_0} & 0\ar{l}.
                \end{tikzcd}
            \end{equation*}
            By definition, $H^0(\RP^2;\Z_2)=\ker\partial_1^*$. The morphisms $\partial_1^*\nu=\partial_1^*\omega$ map
            $a\mapsto 1,b\mapsto 1,c\mapsto 0$ and thus $H^0(\RP^2;\Z_2)=\langle\nu+\omega\rangle=\Z_2$. Next,
            $H^1(\RP^2;\Z_2)=\ker\delta_1/\text{im }\delta_0=\ker\delta_1/\Z_2$. Since $\delta_1\alpha=\delta_1\beta=\delta_1\gamma$
            take $U\mapsto 1,L\mapsto 1$, $\ker\delta_1=\langle\alpha+\beta,\beta+\gamma\rangle=\Z_2^2$, and hence
            $H^1(\RP^2;\Z_2)=\Z_2$. Finally, $H^2(\RP^2;\Z_2)=\ker\delta_2/\text{im }\delta_1=\Z_2^2/\Z_2=\Z_2$.
        \item Consider the usual simplicial structure for the Klein bottle $K$ (Hatcher p.~ 102) consisting
            of 1 0-simplex, 3 1-simplices, and 2 2-simplices. We obtain the chain complex
            \begin{equation*}
                \begin{tikzcd}
                    0\ar{r} & \Z^2\ar{r} & \Z^3\ar{r} & \Z\ar{r} & 0
                \end{tikzcd}
            \end{equation*}
            where $\partial_1=0,\partial_2U=a+b-c,\partial_2L=-a+b+c$. Clearly $H^0(K)=\Z$. Next, $H^1(K)=\ker\delta_1$ and since
            $\delta_1\alpha$ maps $U\mapsto 1,L\mapsto -1$, $\delta_1\beta$ maps $U\mapsto 1,L\mapsto 1$, and $\delta_1\gamma$ maps
            $\mapsto -1,L\mapsto 1$, we find that $H^1(K)=\langle \alpha+\gamma\rangle=\Z$. Finally, $H^2(K)=\ker\delta_2/\text{im }\delta_1=\Z^2/\Z^2=0$.

            Now consider the case of $\Z_2$ coefficients. We obtain the chain complex
            \begin{equation*}
                \begin{tikzcd}
                    0\ar{r} & \Z_2^2\ar{r} & \Z_2^3\ar{r} & \Z\ar{r} & 0.
                \end{tikzcd}
            \end{equation*}
            The associated cochain complex is given
            \begin{equation*}
                \begin{tikzcd}
                    0 & \Z_2^2\ar[swap]{l}{\delta_2} & \Z_2^3\ar[swap]{l}{\delta_1} & \Z_2\ar[swap]{l}{\delta_0} & 0\ar{l}.
                \end{tikzcd}
            \end{equation*}
            Again, $H^0(K;\Z_2)=\Z_2$, but now $H^1(K;\Z_2)=\Z_2^2$ as $\ker\delta_1=\langle\alpha+\beta,\beta+\gamma\rangle$.
            Finally, $H^2(K;\Z_2)=\Z_2$.
    \end{enumerate}
\end{proof}

\begin{prop}
    Hatcher exercise 3.1.8
\end{prop}
\begin{proof} \hfill
    \begin{enumerate}[(a)]
        \item Consider the pair $(X,A)=(D^n,S^{n-1})$, so $X/A=S^n$. The long exact sequence of cohomology
            groups for the pair $(X,A)$ has every third term $\tilde H^i(D^n)$ zero since $D^n$ is contractible
            and the chain and cochain groups are zero. Hence we find that $\tilde H^{i-1}(S^{n-1})\cong \tilde H^i(S^n)$.
            The result follows by induction as usual.
            Similarly, using the Mayer-Vietoris sequence for reduced cohomology with $X=S^n$ and $A,B$ the
            northern and southern hemispheres, we obtain $\tilde H^i(S^{n-1};G)\cong \tilde H^{i+1}(S^n;G)$,
            as the split term goes to zero by contractibility.
        \item Let $A$ be a closed subspace of $X$ that is a deformation retract of some neighborhood $V$.
            We obtain a commutative diagram
            \begin{equation*}
                \begin{tikzcd}
                    H^n(X,A;G) & H^n(X,V;G)\ar{l}\ar{r} & H^n(X-A,V-A;G)\\
                    H^n(X/A,A/A;G)\ar{u}{q^*} & H^n(X/A,V/A;G)\ar{l}\ar{u}{q^*}\ar{r} & H_n(X/A-A/A, V/A-A/A;G)\ar{u}{q^*}
                \end{tikzcd}
            \end{equation*}
            Note that in the long exact sequence of the triple $(X,V,A)$, the groups $H^n(V,A;G)$ are zero for all $n$
            because a deformation retraction of $V$ onto $A$ gives a homotopy equivalence of pairs $(V,A)\cong (A,A)$
            and $H^n(A,A;G)=0$. Hence the upper left map is an isomorphism. The deformation retraction of $V$ onto $A$
            induces a deformation retraction of $V/A$ onto $A/A$, and hence the same argument shows that the lower left
            map is an isomorphism. The remainder of the horizontal maps are isomorphisms via excision for cohomology.
            Finally, the right-hand vertical map $q^*$ is an isomorphism since $q$ restricts to a homeomorphism on the
            complement of $A$. It follows by the commutativity of the diagram, now, that $H^n(X,A;G)\cong H^n(X/A,A/A;G)\cong\tilde H^n(X/A;G)$.
        \item Consider the short exact sequence
            \begin{equation*}
                \begin{tikzcd}
                    0 & H^n(A;G)\ar{l} & H^n(X;G)\ar[swap]{l}{i^*} & H^n(X,A;G)\ar{l} & 0\ar{l}.
                \end{tikzcd}
            \end{equation*}
            Suppose $A$ is a retract of $X$, i.e. we have a map $r:X\to A$ such that $r\circ i=\id_A$,
            and hence $(r\circ i)^*=i^*\circ r^*=\id_{H^n(A;G)}$. The splitting lemma now implies
            that $H^n(X;G)\cong H^n(A;G)\oplus H^n(X,A;G)$, as desired.
    \end{enumerate}


\end{proof}


\end{document}
