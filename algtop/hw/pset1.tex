\documentclass{../../mathnotes}

\usepackage{tikz-cd}
\usepackage{todonotes}

\title{Introduction to Algebraic Topology PSET 1}
\author{Nilay Kumar}
\date{Last updated: \today}


\begin{document}

\maketitle

\begin{prop}
    If a space $X$ is contractible then $X$ is path-connected.
\end{prop}
\begin{proof}
    Take any $x,y\in X$. It suffices to show that there exists a continuous path $\gamma:I\to X$ such that
    $\gamma(0)=x$ and $\gamma(1)=y$. Since $X$ is contractible, the identity map $\id_X$ is homotopic to a
    constant map. In other words, there exists a point $z\in X$ and a homotopy $f_t:X\times I\to X$ such that
    $f_0=\id_X$ and $f_1(p)=z$ for all $p\in X$. Now consider the path $f_t(x):I\to X$. Clearly $f_0(x)=x$
    and $f_1(x)=z$. Similarly, for $f_t(y):I\to X$, we see that $f_0(y)=y$ and $f_1(y)=z$. Both of these paths
    $f_t(x),f_t(y)$ are continuous as they are simply restrictions of the full homotopy $f_t:X\times I\to X$.
    But given continuous paths from $x$ to $z$ and $y$ to $z$, we can concatenate to obtain a path from
    $x$ to $y$. This proves path-connectedness.
\end{proof}

\begin{prop}
    Let $X,Y,Z$ be topological spaces. Consider the maps
    \begin{equation*}
        \begin{tikzcd}
            X\arrow[bend left]{r}{f_0}\arrow[bend right]{r}{f_1}&Y\arrow[bend left]{r}{g_0}
            \arrow[bend right]{r}{g_1}&Z
        \end{tikzcd}
    \end{equation*}
    and suppose that $f_0\simeq f_1$ and $g_0\simeq g_1$. Then $g_0\circ f_0:X\to Z$ and $g_1\circ f_1:X\to Z$
    are homotopic.
\end{prop}
\begin{proof}
    There must exist a homotopy $F:X\times I\to Y$ such that $F_0=f_0$ and $F_1=f_1$, as well as a homotopy
    $G:Y\times I\to Z$ such that $G_0=g_0$ and $G_1=g_1$. Now consider the composition $H:X\times I\to Z$
    defined by $(x,t)\mapsto G_t(F_t(x))$. The composition $H$ is continuous by virtue of the continuity
    of the homotopies $F$ and $G$.
    Moreover, $H$ satisfies $H_0=g_0\circ f_0$ and $H_1=g_1\circ f_1$, and hence $H$ is a homotopy from
    $g_0\circ f_0$ to $g_1\circ f_1$.
\end{proof}

\begin{prop}
    Construct an explicit deformation retraction of $\R^n-\left\{ 0 \right\}$ onto $S^{n-1}$.
\end{prop}
\begin{proof}
    Consider the function $f:\left(\R^n-\left\{ 0 \right\}\right)\times I\to \R^n-\left\{ 0 \right\}$ given by
    \[f_t(x)=x-t\left(x - \frac{x}{|x|}\right).\]
    It is clear that $f$ is continuous. Moreover, at $t=0$, we have $f_0(x)=x$ and hence $f_0=\id_{\R^n-\left\{ 0 \right\}}$.
    At $t=1$, on the other hand, we see that $f_1(x)=x/|x|$, which is on $S^{n-1}$ and hence $f_1(\R^n-\left\{ 0 \right\})=S^{n-1}$
    ($f_1$ obviously surjects onto $S^{n-1}$). Finally, note that $f_t|_{S^{n-1}}=\id_{S^{n-1}}$ for all $t$ because
    $f_t(x)=x-tx+tx=x$ if $|x|=1$. This shows that $f$ is indeed the desired deformation retract.
\end{proof}

\begin{prop}\hspace{1mm}
    \begin{enumerate}[(a)]
        \item The composition of homotopy equivalences $X\to Y$ and $Y\to Z$ is a homotopy equivalence
            $X\to Z$. Furthermore, homotopy equivalence is an equivalence relation.
        \item The relation of homotopy among maps $X\to Y$ is an equivalence relation.
        \item A map homotopic to a homotopy equivalence is a homotopy equivalence.
    \end{enumerate}
\end{prop}
\begin{proof}\hspace{1mm}
    \begin{enumerate}[(a)]
        \item Let $X\simeq Y$ via $f:X\to Y$ and $g:Y\to X$ and $Y\simeq Z$ via $f':Y\to Z$ and $g':Z\to Y$.
            In other words, $g\circ f\simeq\id_X,f\circ g\simeq\id_Y$ and $g'\circ f'\simeq\id_Y,f'\circ g'\simeq\id_Z$.
            Note that
            \begin{align*}
                f'\circ f\circ g\circ g'&\simeq f'\circ \id_Y\circ g'\\
                &\simeq f'\circ g'\\
                &\simeq \id_Z
            \end{align*}
            and
            \begin{align*}
                g\circ g'\circ f'\circ f&\simeq g\circ\id_Y\circ f\\
                &\simeq g\circ f\\
                &\simeq \id_X.
            \end{align*}
            But this implies that $X\simeq Z$ via $f'\circ f:X\to Z$ and $g\circ g':Z\to X$, as desired. This shows that
            homotopy equivalence is a transitive relation. In fact, homotopy equivalence is an equivalence relation because
            $X\simeq X$ by the identity map, and if $X\simeq Y$ via $f,g$, then $Y\simeq X$ simply by switching $f,g$.
        \item Note first that any map $\phi:X\to Y$ is homotopic to itself by the constant homotopy $f:X\times I\to Y$ given by
            $(x,t)\mapsto\phi(x)$. Furthermore, if $f_t$ is a homotopy from $f_0$ to $f_1$, we can define a homotopy from $f_1$
            to $f_0$ by simply taking $f_{1-t}$. Finally, let $f_t$ be a homotopy from $f_0$ to $f_1$ and $g_t$ be a homotopy
            from $g_0$ to $g_1$, where $f_1=g_0$. We can define a homotopy $h:X\times I\to Y$ from $f_0$ to $g_1$ by traversing
            the homotopy $f_{2t}$ followed by $g_{2t}$. Hence the relation of homotopy among maps $X\to Y$ is an
            equivalence relation.
        \item Let $X,Y$ be topological spaces, with $X\simeq Y$ via $h:X\to Y$ and $h':Y\to X$. Let $f_0:X\to Y$ be a map
            homotopic to $h$, i.e. there exists a homotopy $f_t:X\times Y$ between $f_0$ and $f_1=h$. Then
            $f_0\circ h'\simeq h\circ h'\simeq \id_Y$ and $h'\circ f_0\simeq h'\circ h\simeq\id_X$ and hence $f_0$ provides
            a homotopy equivalence $X\simeq Y$.
    \end{enumerate}
\end{proof}

\begin{prop}
    A retract of a contractible space is contractible.
\end{prop}
\begin{proof}
    Let $X$ be a topological space and $r:X\to X$ be a retraction of $X$ onto a subspace $A$.
    It suffices to show that $A$ is contractible, i.e. that the identity map $\id_A$ is homotopic to a
    constant map (on $A$). Using the contractibility of $X$, we find a homotopy $f:X\times I\to X$ such
    that $f_0=\id_X$ and $f_1(p)=z$ for some $z\in X$. Consider the composition $r\circ f:X\times I\to X$
    restricted to $A$.
    It is clear that $r\circ f_0|_A=\id_A$ and that $r\circ f_1|_A=r(z)$ is a constant map on $A$.
    As the composition (and restriction) $r\circ f|_A$ is continuous, this gives us a homotopy between
    $\id_A$ and a constant map on $A$, proving that $A$ is contractible.
\end{proof}


\end{document}
