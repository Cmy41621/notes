\documentclass{../../mathnotes}

\usepackage{tikz-cd}
\usepackage{todonotes}

\title{Introduction to Algebraic Topology PSET 2}
\author{Nilay Kumar}
\date{Last updated: \today}


\begin{document}

\maketitle

\begin{prop}
    Let $X$ be a contractible space. Then the suspension $SX$ is contractible as well.
\end{prop}
\begin{proof}
    Note that it suffices to show that a product $X\times Y$ of two contractible spaces is contractible,
    as the suspension (in this case) is simply a quotient of a product by contractible subspaces.
    By contractibility of $X$ we have a homotopy $f_t:X\times I\to X$ from $\id_X$ to $c_X=x_0$.
    Similarly for $Y$ we have a homotopy $g_t:Y\times I\to Y$ from $\id_Y$ to $c_Y=y_0$.
    Consider the map $h_t:X\times Y\times I\to X\times Y$ given by $(x,y,t)\mapsto (f_t(x),g_t(y))$.
    Clearly $h_t$ is continuous, and at $t=0$ $h_0(x,y)=(f_0(x),g_0(y))=(x,y)$, while at $t=1$
    $h_1(x,y)=(f_1(x),g_1(y))=(x_0,y_0)$. Hence $h_t$ is a homotopy between $\id_{X\times Y}$
    and $c_{X\times Y}=(x_0,y_0)$. The statement of the theorem holds for the product $X\times I$.
\end{proof}

\begin{prop}
    The punctured $\RP^n$ is homotopy equivalent to $\RP^{n-1}$.
\end{prop}
\begin{proof}
    Recall that we can view $\RP^n$ as the quotient $D^n/\partial D^n$ where the antipodal points
    of $\partial D^n$ are identified. Removing the center of $D^n$, we find that the punctured
    $\RP^n$, which we denote by $X$, is expressed $X=D^n-\{0\}/\partial D^n$. It is
    now clear that $X$ is homotopy equivalent to $\partial D^n$ via the expression from class,
    \[f_t(x)=\frac{x}{1-t+t|x|},\]
    as it is a deformation retract.  But the boundary $\partial D^n$ that is left is
    precisely $\RP^{n-1}$ (recall the cell decomposition of the real projective space). Hence
    $X\simeq \RP^{n-1}$.
\end{proof}

\begin{prop}
    Consider the space obtained from $S^2$ by attaching $n$ 2-cells along any collection
    of $n$ circles in $S^2$. This space is homotopy equivalent to the wedge sum of $n+1$
    2-spheres.
\end{prop}
\begin{proof}
    Recall that any circle $S^1$ on $S^2$ is contractible to a point (simply pull it along while
    making the radius smaller). Hence attaching a single
    2-cell along a circle in $S^2$ and quotienting said circle to a point yields a space
    (by the theorem from class) homotopy equivalent to the wedge sum of 2 2-spheres. Iterating this,
    we obtain the desired result (note that we could further draw arcs between the various points
    of attachment and quotient by those arcs and obtain homotopy equivalent spaces where the 2-spheres
    are wedged all at the same point).
\end{proof}

\begin{prop}
    The subspace $X\subset\R^3$ formed by a Klein bottle intersecting itself in a circle
    is homotopy equivalent to $S^1\vee S^1\vee S^2$.
\end{prop}
\begin{proof}
    Consider the circle at which the Klein bottle intersects itself (see diagrams on next page). This circle is of course
    contractible to a point, and hence, quotienting, we find a space homotopy equivalent to the bottle
    that now intersects itself in a pinched point. Note that we can push in the outer portion of the handle
    into the bottle to make it more ``spherical'' and attach a line from the north pole of the sphere to
    the point of self-intersection. Now, drawing an arc from the north pole to the point of self-intersection
    and quotienting yields a space homotopy equivalent to the bottle; note that we have found one of the circles.
    The next circle comes from closing up the funnel into the sphere (from the bottom) into a circle lying
    inside the sphere. Hence we find that this subspace of the Klein bottle is homotopy equivalent to $S^1\vee S^1\vee S^2$.
\end{proof}

\begin{prop}
    Consider a CW complex $X$ that can be written as two contractible subcomplexes whose intersection
    is also contractible. Then $X$ is contractible.
\end{prop}
\begin{proof}
    Write $X=A_1\cup A_2$ as per the proposition. As $A_1\cup A_2$ is contractible, $X/(A_1\cap A_2)$
    is homotopic to $X$. Of course, crushing $A_1\cap A_2$ to a point means that we are left with a union
    space which is the wedge sum $X/(A_1\cap A_2)=A_1/(A_1\cap A_2)\vee A_2/(A_1\cap A_2)$. 
    Each of these wedge summands is contractible (as $A_1\cap A_2$ and $A_1,A_2$ are contractible).
    Hence, quotienting by either of them yields a contractible space, and thus $X$ is contractible.
\end{proof}

\end{document}
