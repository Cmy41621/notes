\documentclass{../../mathnotes}

\usepackage{tikz-cd}
\usepackage{todonotes}

\title{Lie Groups PSET 5}
\author{Nilay Kumar}
\date{Last updated: \today}


\begin{document}

\maketitle

\subsection*{Problem 1}

Let $\fr g=\fr{sl}_n\C$. Consider the Killing form $K(x,y)=\tr(\ad x\ad y)$ for $x,y\in\fr{sl}_n\C$.
Note that the adjoint representation $V$ is irreducible, as $\fr{sl}_n\C$ is simple, and hence contains no ideals.
But then, by a previous exercise (Kirillov 4.5) we find that the space of $\fr{sl}_n\C$-invariant bilinear forms on $V$
is either zero or one-dimensional. On the other hand, if we consider the defining representation of $\fr{sl}_n\C$,
we note that the bilinear form given by $\tr(xy)$ is also an invariant bilinear form (see Kirillov 5.47). But by
the earlier remark we may write
\begin{equation}
    K(x,y)=\tr(\ad x\ad y)=c\tr(xy)
    \label{eq:forms}
\end{equation}
for some constant $c\in\C$ (note that the case $c=0$ corresponds to the zero-dimensional case). It now remains
to compute $c$. To do so, let us compute it in the case of the basis element $x=y=h_1$ with
\begin{equation*}
    h_1=\begin{pmatrix}
        1&&\\&-1&\\&&\ddots
    \end{pmatrix}.
\end{equation*}
where the ellipsis denotes zeros along the rest of the diagonal.
The right-hand side of Eq. \ref{eq:forms} is easy to compute:
\begin{align*}
    c\tr(h_1^2)=c\tr
    \begin{pmatrix}
        1&&\\&1&\\&&\ddots
    \end{pmatrix}
    =2c.
\end{align*}
The left-hand side requires us to compute the matrix representation of the adjoint action of $h_1$, which is not trivial,
as it depends on $n$ and requires some matrix multiplication\footnote{This method is computational, but I couldn't find a better approach.}.
Let us write out the basis for $\fr{sl}_n\C$ for the cases $n=2,3,4$ rather schematically as:
\begin{align*}
    n=2: &
    \begin{pmatrix}
        h_1&^{\color{red}{2}}x_1\\y_1&-h_1
    \end{pmatrix}\\
    n=3: &
    \begin{pmatrix}
        h_1&^{\color{red}{2}}x_1&^{\color{red}{1}}x_3\\y_1&h_2-h_1&^{\color{red}{-1}}x_2\\y_3&y_2&-h_2
    \end{pmatrix}\\
    n=4: &
    \begin{pmatrix}
        h_1&^{\color{red}{2}}x_1&^{\color{red}{1}}x_4&^{\color{red}{1}}x_6\\y_1&h_2-h_1&^{\color{red}{-1}}x_2&^{\color{red}{-1}}x_5\\y_4&y_2&h_3-h_2&^{\color{red}{0}}x_3\\y_6&y_5&y_3&-h_3
    \end{pmatrix}
\end{align*}
The number in red above $x_1$ represents the result of the computation $[h_1,x_1]$, etc. Summing up the squares of these numbers (including the
symmetric counterparts for the $y_i$'s up to sign before squaring) clearly yields $\tr(\ad h_1\ad h_1)$. Just from the way matrix multiplication against
$h_1$ works out, it is clear that the computed roots (brackets) of $x_i$ will always be 1 in the first row (except for the special case $x_1$),
-1 in the second row, and 0 everywhere below that. Simple counting gives us the formula
\[\tr(\ad h_1\ad h_1)=2\left( 2^2+(n-2)+(-1)^2(n-2) \right)=4n.\]
This proves that the constant $c$ above has the value $2n$, and thus, in general, the Killing form on $\fr{sl}_n\C$ can be expressed as 
\[K(x,y)=\tr(\ad x\ad y)=2n\tr(xy).\]

\subsection*{Problem 2}

Let $\fr g\subset \fr{gl}_n\C$ be the subspace consisting of block-triangular matrices:
\[\fr g=\left\{ \begin{pmatrix}A&B\\&D\end{pmatrix}\right\}\]
where $A$ is a $k\times k$ matrix, $B$ is a $k\times(n-k)$ matrix, and $D$ is a $(n-k)\times(n-k)$ matrix.
It easy to see that $\fr g$ in fact forms a subalgebra (a parabolic subalgebra):
\begin{align*}
    \left[
    \begin{pmatrix}
        A&B\\&D
    \end{pmatrix},
    \begin{pmatrix}
        E&F\\&H
    \end{pmatrix}
    \right]&=
    \begin{pmatrix}
        AE&AF+BH\\&DH
    \end{pmatrix}
    -
    \begin{pmatrix}
        EA&EB+FD\\&HD
    \end{pmatrix}\\
    &=
    \begin{pmatrix}
        [A,E]&AF+BH-EB-FD\\&[D,H]
    \end{pmatrix}.
\end{align*}
The radical $\fr{rad}(\fr g)$ clearly contains everything of the form $\fr h=\begin{pmatrix}\lambda I&B\\0&\mu I \end{pmatrix}$, as the derived
series $D^i\fr h$ eventually becomes zero (evident because $[\fr h,\fr h]$ is nilpotent, i.e. strictly upper triangular from the formula above).
The converse is a little bit more computational - we wish to find the maximal solvable ideal of $\fr g$, i.e. an ideal whose
elements vanish after successive commutators. But looking at the above formula for the commutator of two elements of the parabolic subalgebra,
we see that the matrices on the diagonal must be in the radical of $\fr{gl}_n\C$. But this is simply the center of $\fr{gl}_n\C$, by reductivity
and hence the diagonal matrices must be scalar matrices. In the interests of finding the maximal such ideal we note that the upper-right matrix
can be left free, as taking commutators will yield a nilpotent regardless (and hence give us solvability).

Another way of seeing this is to note that since this ideal is clearly solvable, Kirillov 5.40, if we can show that $\fr g$ quotiented by
this ideal yields a semisimple Lie alegbra. But the quotient is simply the set of block diagonal matrices $A,D$ contained in the parabolic subalgebra
above where the matrices $A,D$ are in $\fr{gl}_n\C/\lambda I=\fr{sl}_n\C$. Hence we see that the quotient is $\fr{sl}_n\C\oplus\fr{sl}_n\C$, which
is clearly semisimple. Hence that ideal is in fact the radical $\fr{rad}(\fr g)$.

\subsection*{Problem 3}

Let us show that the bilinear form $\tr(xy)$ on the symplectic Lie algebra $\fr{sp}_n\C$ is non-degenerate. Suppose
there exists an $x\in\fr{sp}_n\C$ such that
\[\tr(xy)=\sum_{i,j}x_{ij}y_{ji}=0\]
for all $y\in\fr{sp}_n\C$. It suffices to show that $x$ is zero. To do this, we will take $y$ to be the conjugate transpose $x^\dagger$.
However, we must first show that $x^\dagger\in\fr{sp}_n\C$. Recall that for an element $a\in\fr{sp}_n\C$ must satisfy
\[a+J^{-1}a^\intercal J=0\]
where $J$ is $\begin{pmatrix}0&I_n\\-I_n&0\end{pmatrix}$, which satisfies $J^{-1}=J^\intercal=-J$. Simply taking the conjugate transpose of this equation gives us
\[a^\dagger+J^{-1}\bar a J=0,\]
as desired. We now see that
\[\tr(xx^\dagger)=\sum_{i,j}x_{ij}\bar x_{ij}=\sum_{i,j}|x_{ij}|^2,\]
but this is a purely positive quantity and hence cannot be zero (as required above) unless $x=0$. This proves that $\tr(xy)$ is nondegenerate
on $\fr{sp}_n\C$.

\subsection*{Problem 4}

Let $\fr g$ be a real Lie algebra with a positive definite Killing form. The ad-invariance of the Killing form implies that the adjoint action
acts as antisymmetric matrices, i.e. $\fr g\subset \fr{so}(\fr g)$. But this in fact forces the Killing form to be negative-definite:
\begin{align*}
    K(\ad X\ad X)=\sum_{i,j} (\ad X)_{ij}(\ad X)_{ji}=-\sum_{i,j}(\ad X)_{ij}^2<0,
\end{align*}
and hence $\fr g=0$.

\subsection*{Problem 5}

Let $\fr g=\fr{so}_3\R$ and $K(x,y)=\tr(\ad x\ad y)$ the Killing form on $\fr g$. For the purposes of determining the Casimir operator on $\fr g$ let
us compute the dual basis to the usual basis $J_x,J_y,J_z$. First note that the commutation relations
\begin{align*}
    [J_x,J_y]&=J_z\\
    [J_y,J_z]&=J_x\\
    [J_z,J_x]&=J_y
\end{align*}
yield the adjoint representations:
\begin{align*}
    \ad J_x&=\begin{pmatrix}0&0&0\\0&0&-1\\0&1&0\end{pmatrix}\\
    \ad J_y&=\begin{pmatrix}0&0&1\\0&0&0\\-1&0&0\end{pmatrix}\\
    \ad J_z&=\begin{pmatrix}0&-1&0\\1&0&0\\0&0&0\end{pmatrix}.
\end{align*}
Using this, we find that
\begin{align*}
    \tr(\ad J_x\ad J_x)&=\tr\begin{pmatrix}0&0&0\\0&-1&0\\0&0&-1\end{pmatrix}=-2\\
    \tr(\ad J_y\ad J_y)&=\tr\begin{pmatrix}-1&0&0\\0&0&0\\0&0&-1\end{pmatrix}=-2\\
    \tr(\ad J_z\ad J_z)&=\tr\begin{pmatrix}-1&0&0\\0&-1&0\\0&0&0\end{pmatrix}=-2,
\end{align*}
while all the other inner products are zero, as is easily computed. Rescaling by a factor to obtain the dual basis, we find the Casimir operator to be
\[C_K=\frac{1}{2}\left(J_x^2+J_y^2+J_z^2\right)\in U\fr{g}.\]


\end{document}
