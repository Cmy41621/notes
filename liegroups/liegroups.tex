\documentclass{../mathnotes}

\usepackage{tikz-cd}
\usepackage{amsmath}


\title{Lie Groups and Representations: Notes}
\author{Nilay Kumar}
\date{Last updated: \today}


\begin{document}

\maketitle

%\setcounter{section}{-1}

\begin{thm}[Fundamental theorem of Lie theory]
    The category of connected, simply-connected, Lie groups is equivalent to the category of Lie algebras.
\end{thm}
\begin{proof}
    Omitted.
\end{proof}

One can consider as an example the case of $SO(3)$ and $SU(2)$, which have identical Lie algebras. Another example is
$\fr{sl}(2,\R)$ ($2\times 2$ matrices with trace zero), which corresponds to $\widetilde{SL(2,\R)}$, the universal cover of $SL(2,\R)$.
Indeed, to understand connected Lie groups, it suffices to understand Lie algebras as well as their theory of coverings.
One direction of this theorem is easy to see, namely sending groups to algebras, but the difficulty arises in the other direction:
how does one lift a Lie algebra homomorphism to a Lie group homomorphism? One (not completely obvious) way to do this is to use the
Baker-Campbell-Hausdorff formula, which allows us to locally recover the group law from the Lie algebra.

Indeed, the Baker-Campbell-Hausdorff formula can be used to define the Lie bracket operation. Another way of doing this is of course to
treat the Lie algebra elements as left-invariant vector fields, and to then use the usual Lie bracket of vector fields.
Of course, for matrix groups we can obviously define the bracket as simply the commutator $[X,Y]=XY-YX$ (we can think of this more
abstractly as the derivative of the adjoint representation of the group).

\begin{defn}
    A \textbf{representation $(\pi, V)$ of a group} $G$ on a vector space $V$ is a homomorphism $\pi: G\to GL(V)$.
\end{defn}

\begin{rem}
    For us, $G$ will be a Lie group, and $V$ will be a finite-dimensional vector space over a field, usually $\C$.
\end{rem}

\begin{defn}
    A \textbf{representation $(\phi, V)$ of a Lie algebra} $\fr g$ on a vector space $V$ is a Lie algebra homomorphism $\phi:\fr g\to\fr{gl}(V)=\End V$.    
\end{defn}

\begin{rem}
    Given a representation $(\pi, V)$ of $G$, we obtain a representation of $\fr g$ simply by taking the derivative $(\pi_*,V)$,
    with $\pi_*(X)=d/dt\pi(e^{tX})|_{t=0}$. However, we cannot necessarily go the other way: given a Lie algebra representation, we cannot
    always lift to a Lie group representation. If a representation of $\fr g$ is $\pi_*$ for some $\pi$ a representation of $G$, then we call
    $\pi$ \textbf{integrable}.
\end{rem}

Note that for any Lie group $G$, we can construct an action of $G$ on itself. This is the adjoint action given by $g\mapsto c(g)\in\Diff(G)$ where
$c(g)h=ghg^{-1}$. Note however that $\Diff(G)$ is nonlinear and hence we obtain a nonlinear representation. Instead we can consider
$g\mapsto (c(g))_*: TG\to TG$, which restricts at the identity to a map $\fr g\to\fr g$. Indeed, one can check that we get a representation this way.

\begin{defn}
    The \textbf{Adjoint representation of the group $G$} is $(\Ad,\fr g)$ where $\Ad(g)=(c(g))_*(e)$.
\end{defn}

We can go further and differentiate this representation to obtain a Lie algebra representation.

\begin{defn}
    The \textbf{adjoint representation} of a Lie algebra $\fr g=\Lie G$ is $(\ad, \fr g)$ where $\ad(X)=\Ad'(e^{tX})=\frac{d}{dt}\bigg|_{t=0}\Ad(e^{tX})$.
\end{defn}

\end{document}
