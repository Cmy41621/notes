\documentclass{../mathnotes}

\usepackage{tikz-cd}
\usepackage{amsmath}
\usepackage{todonotes}


\title{Lie Groups and Representations: Notes}
\author{Nilay Kumar}
\date{Last updated: \today}


\begin{document}

\maketitle

%\setcounter{section}{-1}

\begin{thm}[Fundamental theorem of Lie theory]
    The category of connected, simply-connected, Lie groups is equivalent to the category of Lie algebras.
\end{thm}
\begin{proof}
    Omitted.
\end{proof}

One can consider as an example the case of $SO(3)$ and $SU(2)$, which have identical Lie algebras. Another example is
$\fr{sl}(2,\R)$ ($2\times 2$ matrices with trace zero), which corresponds to $\widetilde{SL(2,\R)}$, the universal cover of $SL(2,\R)$.
Indeed, to understand connected Lie groups, it suffices to understand Lie algebras as well as their theory of coverings.
One direction of this theorem is easy to see, namely sending groups to algebras, but the difficulty arises in the other direction:
how does one lift a Lie algebra homomorphism to a Lie group homomorphism? One (not completely obvious) way to do this is to use the
Baker-Campbell-Hausdorff formula, which allows us to locally recover the group law from the Lie algebra.

Indeed, the Baker-Campbell-Hausdorff formula can be used to define the Lie bracket operation. Another way of doing this is of course to
treat the Lie algebra elements as left-invariant vector fields, and to then use the usual Lie bracket of vector fields.
Of course, for matrix groups we can obviously define the bracket as simply the commutator $[X,Y]=XY-YX$ (we can think of this more
abstractly as the derivative of the adjoint representation of the group).

\begin{defn}
    A \textbf{representation $(\pi, V)$ of a group} $G$ on a vector space $V$ is a homomorphism $\pi: G\to GL(V)$.
\end{defn}

\begin{rem}
    For us, $G$ will be a Lie group, and $V$ will be a finite-dimensional vector space over a field, usually $\C$.
\end{rem}

\begin{defn}
    A \textbf{representation $(\phi, V)$ of a Lie algebra} $\fr g$ on a vector space $V$ is a Lie algebra homomorphism $\phi:\fr g\to\fr{gl}(V)=\End V$.    
\end{defn}

\begin{rem}
    Given a representation $(\pi, V)$ of $G$, we obtain a representation of $\fr g$ simply by taking the derivative $(\pi_*,V)$,
    with $\pi_*(X)=d/dt\pi(e^{tX})|_{t=0}$. However, we cannot necessarily go the other way: given a Lie algebra representation, we cannot
    always lift to a Lie group representation. If a representation of $\fr g$ is $\pi_*$ for some $\pi$ a representation of $G$, then we call
    $\pi$ \textbf{integrable}.
\end{rem}

Note that for any Lie group $G$, we can construct an action of $G$ on itself. This is the adjoint action given by $g\mapsto c(g)\in\Diff(G)$ where
$c(g)h=ghg^{-1}$. Note however that $\Diff(G)$ is nonlinear and hence we obtain a nonlinear representation. Instead we can consider
$g\mapsto (c(g))_*: TG\to TG$, which restricts at the identity to a map $\fr g\to\fr g$. Indeed, one can check that we get a representation this way.

\begin{defn}
    The \textbf{Adjoint representation of the group $G$} is $(\Ad,\fr g)$ where $\Ad(g)=(c(g))_*(e)$.
\end{defn}

We can go further and differentiate this representation to obtain a Lie algebra representation.

\begin{defn}
    The \textbf{adjoint representation} of a Lie algebra $\fr g=\Lie G$ is $(\ad, \fr g)$ where $\ad(X)=\Ad'(e^{tX})=\frac{d}{dt}\bigg|_{t=0}\Ad(e^{tX})$.
\end{defn}

\textbf{September 25, 2013}

For matrix groups, it's clear that the Adjoint representation is just conjugation (via matrix multiplication), i.e. $Ad(g)M=gMg^{-1}$.
For such we can use a trivialization and treat Lie algebra elements as matrices, i.e. given the left-invariant vector field $X_M=(g,gM)$,
$M$ is the associated element of the Lie algebra.
Then the adjoint representation for matrix Lie algebras is given by the commutator $ad(X)M=[X,M]$. One can check this by computing
\[ad(X)(Y)=\frac{d}{dt}\bigg|_{t=0}\left( e^{tX}Ye^{-tX} \right)=XY-YX\]

In general, the Adjoint and adjoint representations are interesting in that they are a distinguished feature of a given Lie group/algebra.
This will be a key fact used when trying to classify algebras.

\begin{exmp}
    Consider $G=SO(3)$ whose Lie algebra is $\Lie SO(3)=\R^3$, isomorphic to the space of anti-symmetric $3\times 3$ real matrices.
    Choosing the usual basis $L_i$ for $\fr{so}_3$, where we have a 1 and -1 in each basis element, we see that we obtain, in some sense,
    the cross product on $\R^3$ via the bracket operation.

    We can do something similar for $G=SU(2)$, and we find (as we know from the homework) that $\fr{su}_2=\fr{so}_3$. Of course, this is a
    rather special case; in higher dimensions there are no such isomorphism in general.
\end{exmp}

So far we have considered Lie algebras that are real vector spaces. It turns out, however, that representations are simplest for
complex vector spaces (as diagonalizability is easiest done). Hence we wish to study the ``complexification'' of $\fr g$.

\begin{defn}
    The \textbf{complexification of $\fr g$} is the complex vector space $\fr g_\C=\fr g\otimes_\R \C=\fr g\oplus i\fr g$.
    The Lie bracket of $\fr g$ extends straightforwardly to a Lie bracket of $\fr g_\C$.
\end{defn}

\begin{rem}
    One can think of $\fr g_\C$ as a real Lie algebra with $\dim\fr g_\C=2\dim\fr g$. In general, when we write something like $\fr g=\Lie G$,
    we will mostly treat $G$ as a real manifold and assume that $\fr g$ is real.
\end{rem}

\begin{defn}
    The Lie algebras $\fr g_1,\fr g_2$ are \textbf{real forms} of a complex Lie algebra $\fr g$ if $\fr g_1\otimes_\R\C=\fr g_2\otimes_\R\C=\fr g$.
\end{defn}

\begin{exmp}
    Take $\fr g_1=\fr{sl}_2\R$, i.e. $2\times 2$ traceless matrices. This is isomorphic to $\R^3$ as a vector space, but is not
    isomorphic to $\fr{su}_2=\fr{so}_3$. If we complexify this Lie algebra, we obtain $\fr g_1\otimes_\R\C=\fr{sl}_2\C$, which is the space
    of $2\times 2$ traceless complex matrices, isomorphic to $\C^3\cong\R^6$. So $\fr g_1$ is a real form of $\fr{sl}_2\C$; note that $SL(2,\R)$ is a 
    non-compact group.

    Consider $\fr g_2=\fr{su}_2$, which is the space of anti-hermitian $2\times 2$ traceless matrices. If we take $\fr g_2\oplus i\fr g_2$, we 
    get both anti-hermitian and hermitian traceless matrices, i.e. all complex traceless matrices. Indeed, this is just $\fr{sl}_2\C$ and hence
    $\fr g_2$ is a real form of $SL(2,\C)$. This is not isomorphic to $\fr g_1$! Since $SU(2)$ is compact, $\fr g_2$ is the ``compact real form.''
\end{exmp}

\begin{exmp}
    Note that $\fr{so}_3\otimes_\R\C\cong\fr{sl}_2\C$. Indeed, we can define complex orthogonal groups, such as $SO(3,\C)$a, that are the transformations
    of $\C^3$ that preserve $\langle\vec z,\vec w\rangle=z_iw^i$ (this is definitely not positive-definite!). Special to the dimension 3 is the fact that
    $SO(3,\C)\cong SL(2,\C)$ (up to double cover issues). In general, we can define $SO(n,\C)$ that give rise to complex Lie algebras that each have
    several real forms, including $\fr{so}(p,q)$.
\end{exmp}

\section{Representation Theory}

We have already seen the general definition of a representation of a group $G$ and a Lie algebra $\fr g$.
The fundamental theorem of Lie theory tells us that finite-dimensional representations of $\fr g$ life to finite-dimensional
representations of $G$.

\begin{exmp}
    In this class we will almost always be discussing representations over $\C$, i.e. $V=\C$.
    \begin{enumerate}[(i)]
        \item We can always consider the trivial representation $(\pi,\C)$ such that $\pi(g)=1$;
        \item For matrix groups $G$, we can construct the defining representation where $\pi$ is simply the inclusion map $G\subset GL(n,\C)$;
        \item The Adjoint and adjoint representations $(\text{Ad},\fr g)$ and $(\text{ad},\fr g)$;
        \item If $G$ acts on $M$, we can define $(\pi,\Hom(M;\C))$ such that $(\pi(g)f)(m)=f(g^{-1}m)$. Typically
            this yields an infinite-dimensional space;
        \item $SU(2)$ acts on $\C^2$ (by the defining representation) so we can consider $\C[z_1,z_2]$: define $(\pi(g)f)(z_1,z_2)=f(g^{-1}\cdot(z_1,z_2)$.
            Note that this action preserves homogeneous polynomials of degree $n$ and hence, by restriction, we have a representation of $SU(2)$ of dimension $n+1$;
        \item The group comes equipped with left and right actions on itself which yield the left and right regular representations: for the left case, we have $(\pi_L,\Hom(G;\C))$ such that
            $(\pi_L(g)f)(g)=f(g^{-1}g_0)$. For a Lie group, this will be infinite-dimensional;
    \end{enumerate}
\end{exmp}

\begin{defn}
    Given two representations $V_1,V_2$ of $G$, we define $\Hom_G(V_1,V_2)$ to be the vector space of all \textbf{equivariant} or \textbf{intertwining} maps from $V_1$ to $V_2$.
    In other words, these are maps $F$ such that the following diagram commutes:
    \begin{equation*}
        \begin{tikzcd}
            V_1\arrow[swap]{d}{\pi_1(g)}\arrow{r}{F} & V_2\arrow{d}{\pi_2(g)}\\ 
            V_1\arrow{r}{F}&V_2
        \end{tikzcd}
    \end{equation*}
    Another way to think about these maps is to note that $\Hom_G(V_1,V_2)=\Hom_\C(V_1,V_2)^G$, i.e. complex vector space morphisms invariant under the action of $G$.
    \todo{is this obvious}
\end{defn}

\begin{rem}
    Representations of a group $G$ form a category \textbf{Rep}(G) with representations (the vector spaces) as the objects and equivariant maps as morphisms.
\end{rem}

\begin{defn}
    Given two representations $(\pi_1,V_1),(\pi_2,V_2)$ of $G$, we can define a new, \textbf{direct sum representation}, denoted by $(\pi_3,V_1\oplus V_2)$ in the obvious way.
    The same holds for Lie algebras.
\end{defn}

Hence it's fairly easy to construct larger representations; what is substantially more difficult is to \emph{break down} representations in some non-trivial way.

\begin{defn}
    A \textbf{subrepresentation} $(\pi_W,W)$ of $(\pi,V)$ is a representation on $W\leqslant V$ such that $W$ is invariant under the action of $G$, i.e.
    if $w\in W$ then $\pi(g)w\in W$ for all $g\in G$. Similarly for Lie algebras.
\end{defn}

\begin{defn}
    A representation is \textbf{irreducible} if it has no non-trivial (not zero or the full space) subrepresentations,
    and \textbf{reducible} otherwise. A representation is called \textbf{completely reducible} if it is can be written as a direct sum
    of irreducible subrepresentations, i.e.  $(\pi, V)=\bigoplus_I(\pi_i,V_i)$ where $(\pi_i,V_i)$ are irreducible.
\end{defn}

In general, then, we have two distinct problems - the first is computing irreducible representations, and the second is decomposing
representations into their irreducible subrepresentations.
Note that there is a slight subtlety in that we can have representations that are reducible
but not completely reducible, as the following example depicts.

\begin{exmp}
    Consider the upper triangular matrices $B\leqslant GL(2,\R)$ acting on $\R^2$ (in the defining representation). It is a simple computation to check that the $x$-axis,
    $V_1=\R$, is invariant under this action (it is dilated). We would now like to say that $V=\R^2=V_1\oplus V_2$, for some $V_2$ complementary to $V_1$ - 
    but no such invariant subspace exists!

    As another example, consider $G=\R$. We can obtain representations $\pi(t)=e^{tA}$ for $A\in M_{n\times n}$. This representation
    is completely reducible only if $A$ is diagonalizable.

    Such pathologies do not occur, for example, when the action preserves an inner product. Indeed, under the action of an orthogonal group, given any invariant subspace
    we can simply find the orthogonal complement as an invariant subspace.
\end{exmp}

What is $\Hom_G(V,V)$? First for $V$ irreducible we claim that $\Hom_G(V,V)=\C$. To prove this we need to show that if $F$ is a linear map $V\to V$,
commuting with $\pi(g)$ for all $g$ then $V$ irreducible implies $F=\lambda I$. Let $\lambda$ be an eigenvalue of $F$. Consider $\lambda$-eigenspace $V_\lambda\subset V$
$v:Fv=\lambda v$. Claim: $V_\lambda$ is a subrepresentation (because $F$ commutes with all $\pi$): $F(\pi(g)v)=\pi(g)Fv=\lambda \pi(g)v$.


\end{document}
