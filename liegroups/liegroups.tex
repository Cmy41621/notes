\documentclass{../mathnotes}

\usepackage{tikz-cd}
\usepackage{amsmath}
\usepackage{todonotes}


\title{Lie Groups and Representations: Notes}
\author{Nilay Kumar}
\date{Last updated: \today}


\begin{document}

\maketitle

%\setcounter{section}{-1}

\begin{thm}[Fundamental theorem of Lie theory]
    The category of connected, simply-connected, Lie groups is equivalent to the category of Lie algebras.
\end{thm}
\begin{proof}
    Omitted.
\end{proof}

One can consider as an example the case of $SO(3)$ and $SU(2)$, which have identical Lie algebras. Another example is
$\fr{sl}(2,\R)$ ($2\times 2$ matrices with trace zero), which corresponds to $\widetilde{SL(2,\R)}$, the universal cover of $SL(2,\R)$.
Indeed, to understand connected Lie groups, it suffices to understand Lie algebras as well as their theory of coverings.
One direction of this theorem is easy to see, namely sending groups to algebras, but the difficulty arises in the other direction:
how does one lift a Lie algebra homomorphism to a Lie group homomorphism? One (not completely obvious) way to do this is to use the
Baker-Campbell-Hausdorff formula, which allows us to locally recover the group law from the Lie algebra.

Indeed, the Baker-Campbell-Hausdorff formula can be used to define the Lie bracket operation. Another way of doing this is of course to
treat the Lie algebra elements as left-invariant vector fields, and to then use the usual Lie bracket of vector fields.
Of course, for matrix groups we can obviously define the bracket as simply the commutator $[X,Y]=XY-YX$ (we can think of this more
abstractly as the derivative of the adjoint representation of the group).

\begin{defn}
    A \textbf{representation $(\pi, V)$ of a group} $G$ on a vector space $V$ is a homomorphism $\pi: G\to GL(V)$.
\end{defn}

\begin{rem}
    For us, $G$ will be a Lie group, and $V$ will be a finite-dimensional vector space over a field, usually $\C$.
\end{rem}

\begin{defn}
    A \textbf{representation $(\phi, V)$ of a Lie algebra} $\fr g$ on a vector space $V$ is a Lie algebra homomorphism $\phi:\fr g\to\fr{gl}(V)=\End V$.    
\end{defn}

\begin{rem}
    Given a representation $(\pi, V)$ of $G$, we obtain a representation of $\fr g$ simply by taking the derivative $(\pi_*,V)$,
    with $\pi_*(X)=d/dt\pi(e^{tX})|_{t=0}$. However, we cannot necessarily go the other way: given a Lie algebra representation, we cannot
    always lift to a Lie group representation. If a representation of $\fr g$ is $\pi_*$ for some $\pi$ a representation of $G$, then we call
    $\pi$ \textbf{integrable}.
\end{rem}

Note that for any Lie group $G$, we can construct an action of $G$ on itself. This is the adjoint action given by $g\mapsto c(g)\in\Diff(G)$ where
$c(g)h=ghg^{-1}$. Note however that $\Diff(G)$ is nonlinear and hence we obtain a nonlinear representation. Instead we can consider
$g\mapsto (c(g))_*: TG\to TG$, which restricts at the identity to a map $\fr g\to\fr g$. Indeed, one can check that we get a representation this way.

\begin{defn}
    The \textbf{Adjoint representation of the group $G$} is $(\Ad,\fr g)$ where $\Ad(g)=(c(g))_*(e)$.
\end{defn}

We can go further and differentiate this representation to obtain a Lie algebra representation.

\begin{defn}
    The \textbf{adjoint representation} of a Lie algebra $\fr g=\Lie G$ is $(\ad, \fr g)$ where $\ad(X)=\Ad'(e^{tX})=\frac{d}{dt}\bigg|_{t=0}\Ad(e^{tX})$.
\end{defn}

\textbf{September 25, 2013}

For matrix groups, it's clear that the Adjoint representation is just conjugation (via matrix multiplication), i.e. $Ad(g)M=gMg^{-1}$.
For such we can use a trivialization and treat Lie algebra elements as matrices, i.e. given the left-invariant vector field $X_M=(g,gM)$,
$M$ is the associated element of the Lie algebra.
Then the adjoint representation for matrix Lie algebras is given by the commutator $ad(X)M=[X,M]$. One can check this by computing
\[ad(X)(Y)=\frac{d}{dt}\bigg|_{t=0}\left( e^{tX}Ye^{-tX} \right)=XY-YX\]

In general, the Adjoint and adjoint representations are interesting in that they are a distinguished feature of a given Lie group/algebra.
This will be a key fact used when trying to classify algebras.

\begin{exmp}
    Consider $G=SO(3)$ whose Lie algebra is $\Lie SO(3)=\R^3$, isomorphic to the space of anti-symmetric $3\times 3$ real matrices.
    Choosing the usual basis $L_i$ for $\fr{so}_3$, where we have a 1 and -1 in each basis element, we see that we obtain, in some sense,
    the cross product on $\R^3$ via the bracket operation.

    We can do something similar for $G=SU(2)$, and we find (as we know from the homework) that $\fr{su}_2=\fr{so}_3$. Of course, this is a
    rather special case; in higher dimensions there are no such isomorphism in general.
\end{exmp}

So far we have considered Lie algebras that are real vector spaces. It turns out, however, that representations are simplest for
complex vector spaces (as diagonalizability is easiest done). Hence we wish to study the ``complexification'' of $\fr g$.

\begin{defn}
    The \textbf{complexification of $\fr g$} is the complex vector space $\fr g_\C=\fr g\otimes_\R \C=\fr g\oplus i\fr g$.
    The Lie bracket of $\fr g$ extends straightforwardly to a Lie bracket of $\fr g_\C$.
\end{defn}

\begin{rem}
    One can think of $\fr g_\C$ as a real Lie algebra with $\dim\fr g_\C=2\dim\fr g$. In general, when we write something like $\fr g=\Lie G$,
    we will mostly treat $G$ as a real manifold and assume that $\fr g$ is real.
\end{rem}

\begin{defn}
    The Lie algebras $\fr g_1,\fr g_2$ are \textbf{real forms} of a complex Lie algebra $\fr g$ if $\fr g_1\otimes_\R\C=\fr g_2\otimes_\R\C=\fr g$.
\end{defn}

\begin{exmp}
    Take $\fr g_1=\fr{sl}_2\R$, i.e. $2\times 2$ traceless matrices. This is isomorphic to $\R^3$ as a vector space, but is not
    isomorphic to $\fr{su}_2=\fr{so}_3$. If we complexify this Lie algebra, we obtain $\fr g_1\otimes_\R\C=\fr{sl}_2\C$, which is the space
    of $2\times 2$ traceless complex matrices, isomorphic to $\C^3\cong\R^6$. So $\fr g_1$ is a real form of $\fr{sl}_2\C$; note that $SL(2,\R)$ is a 
    non-compact group.

    Consider $\fr g_2=\fr{su}_2$, which is the space of anti-hermitian $2\times 2$ traceless matrices. If we take $\fr g_2\oplus i\fr g_2$, we 
    get both anti-hermitian and hermitian traceless matrices, i.e. all complex traceless matrices. Indeed, this is just $\fr{sl}_2\C$ and hence
    $\fr g_2$ is a real form of $SL(2,\C)$. This is not isomorphic to $\fr g_1$! Since $SU(2)$ is compact, $\fr g_2$ is the ``compact real form.''
\end{exmp}

\begin{exmp}
    Note that $\fr{so}_3\otimes_\R\C\cong\fr{sl}_2\C$. Indeed, we can define complex orthogonal groups, such as $SO(3,\C)$a, that are the transformations
    of $\C^3$ that preserve $\langle\vec z,\vec w\rangle=z_iw^i$ (this is definitely not positive-definite!). Special to the dimension 3 is the fact that
    $SO(3,\C)\cong SL(2,\C)$ (up to double cover issues). In general, we can define $SO(n,\C)$ that give rise to complex Lie algebras that each have
    several real forms, including $\fr{so}(p,q)$.
\end{exmp}

\section{Representation Theory}

We have already seen the general definition of a representation of a group $G$ and a Lie algebra $\fr g$.
The fundamental theorem of Lie theory tells us that finite-dimensional representations of $\fr g$ life to finite-dimensional
representations of $G$.

\begin{exmp}
    In this class we will almost always be discussing representations over $\C$, i.e. $V=\C$.
    \begin{enumerate}[(i)]
        \item We can always consider the trivial representation $(\pi,\C)$ such that $\pi(g)=1$;
        \item For matrix groups $G$, we can construct the defining representation where $\pi$ is simply the inclusion map $G\subset GL(n,\C)$;
        \item The Adjoint and adjoint representations $(\text{Ad},\fr g)$ and $(\text{ad},\fr g)$;
        \item If $G$ acts on $M$, we can define $(\pi,\Hom(M;\C))$ such that $(\pi(g)f)(m)=f(g^{-1}m)$. Typically
            this yields an infinite-dimensional space;
        \item $SU(2)$ acts on $\C^2$ (by the defining representation) so we can consider $\C[z_1,z_2]$: define $(\pi(g)f)(z_1,z_2)=f(g^{-1}\cdot(z_1,z_2)$.
            Note that this action preserves homogeneous polynomials of degree $n$ and hence, by restriction, we have a representation of $SU(2)$ of dimension $n+1$;
        \item The group comes equipped with left and right actions on itself which yield the left and right regular representations: for the left case, we have $(\pi_L,\Hom(G;\C))$ such that
            $(\pi_L(g)f)(g)=f(g^{-1}g_0)$. For a Lie group, this will be infinite-dimensional;
    \end{enumerate}
\end{exmp}

\begin{defn}
    Given two representations $V_1,V_2$ of $G$, we define $\Hom_G(V_1,V_2)$ to be the vector space of all \textbf{equivariant} or \textbf{intertwining} maps from $V_1$ to $V_2$.
    In other words, these are maps $F$ such that the following diagram commutes:
    \begin{equation*}
        \begin{tikzcd}
            V_1\arrow[swap]{d}{\pi_1(g)}\arrow{r}{F} & V_2\arrow{d}{\pi_2(g)}\\ 
            V_1\arrow{r}{F}&V_2
        \end{tikzcd}
    \end{equation*}
    Another way to think about these maps is to note that $\Hom_G(V_1,V_2)=\Hom_\C(V_1,V_2)^G$, i.e. complex vector space morphisms invariant under the action of $G$.
    \todo{is this obvious}
\end{defn}

\begin{rem}
    Representations of a group $G$ form a category \textbf{Rep}(G) with representations (the vector spaces) as the objects and equivariant maps as morphisms.
\end{rem}

\begin{defn}
    Given two representations $(\pi_1,V_1),(\pi_2,V_2)$ of $G$, we can define a new, \textbf{direct sum representation}, denoted by $(\pi_3,V_1\oplus V_2)$ in the obvious way.
    The same holds for Lie algebras.
\end{defn}

Hence it's fairly easy to construct larger representations; what is substantially more difficult is to \emph{break down} representations in some non-trivial way.

\begin{defn}
    A \textbf{subrepresentation} $(\pi_W,W)$ of $(\pi,V)$ is a representation on $W\leqslant V$ such that $W$ is invariant under the action of $G$, i.e.
    if $w\in W$ then $\pi(g)w\in W$ for all $g\in G$. Similarly for Lie algebras.
\end{defn}

\begin{defn}
    A representation is \textbf{irreducible} if it has no non-trivial (not zero or the full space) subrepresentations,
    and \textbf{reducible} otherwise. A representation is called \textbf{completely reducible} if it is can be written as a direct sum
    of irreducible subrepresentations, i.e.  $(\pi, V)=\bigoplus_I(\pi_i,V_i)$ where $(\pi_i,V_i)$ are irreducible.
\end{defn}

In general, then, we have two distinct problems - the first is computing irreducible representations, and the second is decomposing
representations into their irreducible subrepresentations.
Note that there is a slight subtlety in that we can have representations that are reducible
but not completely reducible, as the following example depicts.

\begin{exmp}
    Consider the upper triangular matrices $B\leqslant GL(2,\R)$ acting on $\R^2$ (in the defining representation). It is a simple computation to check that the $x$-axis,
    $V_1=\R$, is invariant under this action (it is dilated). We would now like to say that $V=\R^2=V_1\oplus V_2$, for some $V_2$ complementary to $V_1$ - 
    but no such invariant subspace exists!

    As another example, consider $G=\R$. We can obtain representations $\pi(t)=e^{tA}$ for $A\in M_{n\times n}$. This representation
    is completely reducible only if $A$ is diagonalizable.

    Such pathologies do not occur, for example, when the action preserves an inner product. Indeed, under the action of an orthogonal group, given any invariant subspace
    we can simply find the orthogonal complement as an invariant subspace.
\end{exmp}

What is $\Hom_G(V,V)$? First for $V$ irreducible we claim that $\Hom_G(V,V)=\C$. To prove this we need to show that if $F$ is a linear map $V\to V$,
commuting with $\pi(g)$ for all $g$ then $V$ irreducible implies $F=\lambda I$. Let $\lambda$ be an eigenvalue of $F$. Consider $\lambda$-eigenspace $V_\lambda\subset V$
$v:Fv=\lambda v$. Claim: $V_\lambda$ is a subrepresentation (because $F$ commutes with all $\pi$): $F(\pi(g)v)=\pi(g)Fv=\lambda \pi(g)v$.
\todo{what happened here?}

\begin{lem}[Schur's lemma]
    If $V$ is an irreducible representation of $G$, then $\Hom_G(V,V)=\C$. In words, ever linear map $F:V\to V$ commuting
    with all $\pi(g)$ is $\lambda I$ for $\lambda\in\C$.
\end{lem}

\begin{defn}
    We say that two representations are \textbf{equivalent}, $V_1\approx V_2$, if there exists an morphism $M:V_2\to V_1$ such that $\pi_1(g)=M\pi_2(g)M^{-1}$
    for all $g\in G$.
\end{defn}

\begin{lem}
    If $V_1,V_2$ are both irreducible and not equivalent, then $\Hom_G(V_1,V_2)=0$.
\end{lem}
\begin{proof}
    Consider such an $F:V_1\to V_2$. Since $F$ commutes with the group action, $\ker F\subset V_1$ is a subrepresentation
    and $\text{Im } F\subset V_2$ is a subrepresentation. By irreducibility, we see that $\ker F$ and $\text{Im} F$ are
    either $0$ and $V_1$ or $0$ and $V_2$ respectively. Hence the only possibility for $F$ is 0.
\end{proof}

More generally, suppose that $V=\bigoplus_{i\in\text{Irr}}n_i\cdot V_i$, i.e. we can completely split into irreducibles.
Then $n_i$ is called the \textbf{multiplicity} of $V_i$.  We can now write
\begin{align*}
    \Hom_G(V,V)&=\bigoplus_{i\in\text{Irr}}\Hom_G(n_i\cdot V_i,n_i\cdot V_i)\\
    &=\bigoplus_{i\in\text{Irr}}\End\C^{n_i}.
\end{align*}

\begin{cor}
    All irreducible representations of $G$ are one-dimensional if $G$ is commutative.
\end{cor}
\begin{proof}
    If $V$ is irreducible then all $\pi(g)$ commute with the action of $G$, i.e. for all $\pi(g_0)$ (for $g_0\in G$),
    we have that $\pi(g)\pi(g_0)=\pi(g_0)\pi(g)$. Schur's lemma now implies that $\pi(g)=\lambda_g$ for some $\lambda_g\in\C^\times$.
\end{proof}

\begin{exmp}
    Consider $G=U(1)$: irreducible representations are $\pi_n(e^{i\theta})=e^{in\theta}$ for $n\in\Z$.
    For $G=\R$, on the other hand, irreducible representations are $\pi_k(x)=e^{ikx}$ for $k\in\R$.
\end{exmp}

\begin{defn}
    Given $V_1,V_2$ representations of $G$, we can construct a \textbf{tensor product representation} on $V_1\otimes V_2$ by defining in the obvious way
    \[\pi(g)(v_1\otimes v_2)=\pi_{1}(g)v_1\otimes \pi_2(g)(v_2),\]
    for $v_1\in V_1$ and $v_2\in V_2$.
\end{defn}

This raises a big question: how does $V_1\otimes V_2$ decompose into irreducibles? For $V_1,V_2$ irreducible, $V_1\otimes V_2=\bigoplus_i n_i\cdot V_i$; what are the $n_i$?

\begin{defn}
    For a Lie algebra $\fr g$, given two representations $V_1,V_2$, we can construct a \textbf{tensor product representation} $V_1\otimes V_2$
    \begin{align*}
        \phi(X)=\phi_1\otimes 1 + 1\otimes \phi_2(X),
    \end{align*}
    which follows from the derivative of the tensor product representation at the group level.
\end{defn}

\begin{defn}
    Given $V_1,V_2$ representations of $G$, we can construct a representation on $\Hom_\C(V_1,V_2)$ given by
    \begin{align*}
        \pi(g)(L)=\pi_2(g)L\pi_1(g^{-1}),
    \end{align*}
    for some $L\in\Hom_\C(V_1,V_2)$. In the special case that $V_2$ is the trivial representation $V_2=\C$, then we have
    a representation on $V^*=\Hom(V_1,\C)$ given by the inverse transpose
    \begin{align*}
        \pi(g)(L)=L\pi_1(g)^{-1}.
    \end{align*}
    This is the dual representation.
\end{defn}

\begin{defn}
    We say that a representation of $G$, $(\pi, V)$, is \textbf{unitary} if $\pi(g)\in U(n)$ for all $g\in G$, i.e. all $\pi(g)$ preserve the standard Hermitian form.
    At the Lie algebra level, we say that a representation $(\phi,V)$ is \textbf{unitary} if $\phi(X)=-\phi(X)^\dagger$.
\end{defn}

From now on, let us focus on finite-dimensional representations.

\begin{thm}
    Unitary representations are completely reducible.
\end{thm}
\begin{proof}
    If $V_1\subset V$ is a subrepresentation, then we claim that $V_1^\perp$ is also a subrepresentation. If $V_2$ is a subrepresentation
    then $V=V_1\oplus V_2$, which is a sum of subrepresentations, and we can continue this process until we are left with irreducibles.
    Now let us prove the claim. Note that $V_1^\perp$ is preserved by $\pi(g)$ because
    \begin{align*}
        0=\langle\pi(g)v,v_1\rangle=\langle(\pi(g)^{-1}\pi(g)v,\pi(g)^{-1}v_1\rangle=\langle v,v_1'\rangle
    \end{align*}
    for some $v_1'\in V_1$. This implies that $v\in V_1^\perp$ implies that $\pi(g)v\in V_1^\perp$, as desired.
\end{proof}

\begin{cor}
    All representations of a finite group $G$ are completely reducible.
\end{cor}
\begin{proof}
    It suffices to show that there exists a $G$-invariant Hermitian inner product on $V$. Start with any Hermitian inner product $B(\cdot,\cdot)$ on $V$. 
    Then we define
    \[\langle\cdot,\cdot\rangle=\frac{1}{|G|}\sum_{g\in G}B(\pi(g)\cdot,\pi(g)\cdot).\]
    Then it's clear that 
    \[\langle\pi(u)v,\pi(u)w\rangle=\frac{1}{|G|}\sum_{g\in G}B(\pi(g)\pi(u)v,\pi(g)\pi(u)w)=\frac{1}{|G|}\sum_{g\in G}B(\pi(g)v,\pi(g)w)\]
    by the homomorphism property and relabeling, and we are done.
\end{proof}

This ``averaging trick'' is quite powerful, and can in fact be extended to compact Lie groups, as we shall see shortly.

\begin{defn}
    If $(\pi,V)$ is a representation of $G$, it's \textbf{character} is a function on $G$ given by
    \[\chi_V(g)=\tr \pi(g).\]
    Note that $\chi_V$ is in fact a function on the conjugacy classes of $G$. Additionally, it is easy to see that
    \begin{align*}
        \chi_{V_1\oplus V_2}&=\chi_{V_1}+\chi_{V_2}\\
        \chi_{V_1\otimes V_2}&=\chi_{V_1}\cdot\chi_{V_2}\\
        \chi_{(V_1,\C)}&=1
    \end{align*}
\end{defn}

\begin{defn}
    The \textbf{representation ring} $R(G)$ of a group $G$ is the ring with elements equivalence classes of representations $[V]$
    with $[V_1]+[V_2]=[V_1\oplus V_2]$ and $[V_1][V_2]=[V_1\otimes V_2]$. The inverses for the additive operation are given by ``formal
    differences'' with the equivalence relation $([V_1],[V_2])\sim([V_1]+[V],[V_2]+[V])$.
\end{defn}

\begin{lem}
    The complexified representation ring is precisely the functions on $G$ conjugation-invariant on $G$: $R(G)\otimes_\Z\C=\text{Fun}(G)^G$.
\end{lem}

Let us now consider how to construct irreducible representations of finite; we will start by using functions on the group, i.e. the regular representation.
Later we will generalize this compact groups.

There exists a left and right invariant Hermitian inner product on $Fun(G)$:
\begin{align*}
    \langle f_1,f_2\rangle =\frac{1}{|G|}\sum_{g\in G}\bar f_1(g)f_2(g).
\end{align*}
This implies that the left and right actions on $Fun(G)$ are unitary. We will prove that the $\chi_{V_i}$, for $V_i$ irreducible, yield
an orthonormal basis of the conjugation invariant functions $Fun(G)^G$. Indeed, the following theorem holds:
    
\begin{thm}
$\langle \chi_{V_1},\chi_{V_j}\rangle=\delta_{ij}$.
\end{thm}


As an aside, note that for $V$ an irreducible representation of $G$ a vector space over any field $k$, then $\End_G(V,V)$ is a division
algebra over $k$. This can be shown as follows. First note that $\End_G(v,V)$ are an algebra over $k$, where the product is composition.
If $F\in\End_G(V,V)$ then irreducibility of $V$ implies that $\ker F,\text{Im }F$ are $G$-invariant subspaces. If $\ker F=0$, we see that $F$
is an isomorphism and is thus invertible. Otherwise, if $\ker F=V$, then $F=0$. Of course, we can no longer use the proof of Schur's lemma
using eigenvalues, as our field $k$ is not necessarily closed; one way to proceed is to note that there is only one division algebra over $\C$ for
the case $k=\C$. If $k=\R$, there exist three division algebras over $\R$, i.e. $\R,\C,\HH$. This implies that $V$ irreducible yields three
possibilities for $\End_G(V,V)$. For this reason, we will stick to dealing with $\C$, as it is generally simpler.

\begin{cor}
    Given the decomposition $V=\oplus_i n_iV_i$, where $n_i\in\N$ is the multiplicity of $V_i$, then $\langle\chi_{V},\chi_{V_i}\rangle=n_i=\frac{1}{|G|}\sum_{g\in G}\bar\chi_V(g)\chi_{V_i}(g)$.
\end{cor}

Let us now try to prove the theorem above. The following theorem will be useful.

\begin{thm}
    If $(\pi^V,v),(\pi^W,W)$ are two irreducible representations of $G$ then $\langle \pi^V_{ab},\pi^W_{cd}\rangle$ is $\delta_{ac}\delta_{bd}/\dim V$ if $V$ is
    equivalent to $W$ and 0 otherwise, i.e. the matrix elements of irreducible representations are orthogonal.
\end{thm}
\begin{proof}
    Given a linear map $F:V\to W$, consider $\tilde F=\frac{1}{|G|}\sum_{g\in G}\pi^W(g)\pi^V(g^{-1})$. One can check that $\pi^W\circ \tilde F=\tilde F\circ \pi^V$,
    i.e. $\tilde F$ is an intertwining operator. Schur's lemma implies that $\tilde F=0$ if $V$ and $W$ are not equivalent but $\tilde F=\lambda I$ if they are.
    Then $\tr\tilde F=\lambda\dim V=\tr F$. This implies that $\tilde F=\tr F/\dim V \cdot I$.
\end{proof}

\begin{proof}
    We first show the orthogonality of matrix elements. Given a basis of $V$, $\pi_V(g)$ is a $\dim V\times \dim V$ matrix that
    gives $(\dim V)^2$ functions on $G$. Pick $F=F_{ac}$ which is 1 at $a,c$ and 0 otherwise. Then
    \begin{align*}
        \tilde F_{ac}=\frac{1}{|G|}\sum_{g\in G}\pi^W(g)F_{ac}\pi^V(g)=0\text{ if }V\nsim W.
    \end{align*}
    We can write this as a matrix formula for $(F_{ac})_{ij}(\tilde F_{ac})_{bd}=\frac{1}{|G|}\sum_{g\in G}\pi^W_{ba}(g)\bar{\pi}^V_{dc}(g)=0$,
    which is precisely the orthogonality relation for the matrix elements if $V,W$ are not equivalent. On the other hand, if $V,W$ are equivalent,
    Schur's lemma via the above theorem implies that $(\tilde F_{ac})_{bd}=\delta_{bd}\tr F_{ac}/\dim V=\delta_{bd}\delta_{ac}/\dim V$, which yields 
    the orthogonality.

    Now let us prove the theorem. $\langle\chi_V,\chi_W\rangle=\frac{1}{|G|}\sum_{g\in G}\sum_a\bar\pi_{aa}^V\sum_c\pi^W_{cc}$ is zero if $V\nsim W$
    and is $\sum_{a,c}\delta_{ac}/\dim V$ if $V\sim W$, and we are done.
\end{proof}

This leaves a few questions. First of all, what is the set $\hat G$ of inequivalent irreducible representations? Second, what are the functions $\chi_{V_i}$?
Moreover, what are the actual matrices $\pi_{V_i}(g)$ (upto conjugation)?


One might ask how to construct representations - i.e. how to produce linear spaces on which a group acts. One obvious thing to do is to consider the space
of functions on $G$, $Fun(G)$. We wish to find such irreducible representations. Note that $Fun(G)$ carries a representation of $G\times G$, the \textbf{regular
representation} $\pi_r$, given by $(\pi_r(g_l,g_r)f)(g)=f(g_l^{-1}gg_r).$ However, for any representation $\pi_V$ of $G$, we also have a representation of $G\times G$
on $V^*\otimes V$ by $\Phi(v)(g_l,g_r)(v^*\otimes v)=\pi_{V^*}(g_l)v^*\otimes \pi_V(g_r)v$ where $v^*\in V^*,v\in V$.
One thing to notice is that $V^*\otimes V=\End V$.

We claim that there exists an isomorphism of these two $G\times G$ representations given by $\oplus_{i\in\hat G}V_i^*\otimes V_i\leftrightarrow Fun(G)$.
Mapping from left to right (call it $m$), take $v^*\in V^*,v\in V$, let $m(v^*\otimes v)=v^*(\pi_V(g)v)\in Fun(G)$, which is simply a matrix element.
One can show that this $G\times G$ map is injective and surjective and that the orthogonality of matrix elements implies that this is a unitary map.

Another, more algebraic point of view, of the same fact is to consider $\C(G)$ as the algebra of complex-valued functions on $G$ with convolutions
as the product:
\[f_1\cdot f_2(g)=\frac{1}{|G|}\sum_{g\in G}f_1(gh^{-1})f_2(h).\]
This is an associative algebra of finite dimension over $\C$. Wedderburn's theorem tells us that $\C(G)=\oplus_i M(d_i,\C)$, i.e. this breaks up
into matrix algebras. One then finds that the dimensions of both side must be the same, i.e. $\dim\C(G)=|G|=\sum_id_i^2$. PIcking the right action
we get a representation $\pi$ on $Fun(G)$ (not irreducible) $\pi(g)f(h)=f(gh)$ with $Fun(G)=\oplus_{i\in\hat G}n_iV_i$ where $n_i=\dim V_i=\oplus_{i\in\hat G}V_i\otimes\Hom_G(V_i,Fun(G))$.



\textbf{a few classes later}

\section{Representations of $\fr{sl}_2\C$}

We wish to find the irreducible representations of $SU(2)$ by purely algebraic methods, as we found last class a series of representations, but it was not clear
that these were all of them. Note that for finite-dimensional representations of $SU(2)$ we have a one-to-one correspondence with representations of $\fr{su}_2$
(note carefully that this is not true for infinite-dimensional representations). Furthermore, we will in fact study the complexified representation: those of $\fr{sl}_2\C$,
i.e. all traceless $2\times 2$ complex matrices. In fact there is an equivalence of categories between representations of $\fr{su}_2\C$ and $\fr{sl}_2\C$: one direction
is given by restriction and the other is given by complexification.

\begin{thm}
    The finite-dimensional representations of $\fr{sl}_2\C$ are completely reducible.
\end{thm}
\begin{proof}
    One can use the equivalence of categories above along with the fact that representations of compact groups is completely reducible (via Haar measure).
    This proof is often called Weyl's unitary trick. There is a more algebraic proof, but we will consider this later.
\end{proof}

Let us pick now pick the standard basis of $\fr{sl}_2\C$ to be $e,f,h$ where $[e,f]=h,[h,e]=2e,$ and $[h,f]=-2f$. The plan of attack will be to first concentrate
on the action of $h$ on the representation ($h$ exponentiates to an abelian $\C^\times$).
\begin{defn}
    We say that $v\in V$ has \textbf{weight} $\lambda$ if $hv=\lambda v$. In other words, $\lambda$ is an eigenvalue of the $h$ action. Then set
    $V(\lambda)\subset V$ of vectors of weight $\lambda$ to be the $\lambda$-weight space of $V$.
\end{defn}

We claim that $eV(\lambda)=V(\lambda+2)$ and that $fV(\lambda)\subset V(\lambda-2)$. This is a simple computation.

\begin{defn}
    We say that $\lambda$ is a \textbf{highest weight} if $\text{Re}\lambda'\leq\text{Re}\lambda$ for all weights $\lambda'$.
\end{defn}

\subsection{Universal enveloping algebra}

Note that the Lie algebra is not an associative algebra, i.e. we have no multiplcation in $\fr g$. For any representation, on the other hand,
$\rho:\fr g\to\End V$ we can multiply $\rho(X_1)\rho(X_2)$. The resulting algebra is on that will satsify the homomorphism property
\[\rho(X_1)\rho(X_2)-\rho(X_2)\rho(X_1)=\rho([X_1,X_2]).\]
The idea here is to define an algebra with generators $X\in\fr g$, with some relations, linearity, and the Lie algebra homomorphism proprty.

\begin{defn}
    For any $k$-vector space $V$, the \textbf{tensor algebra} is the associative algebra
    \[T(V)=\bigoplus_{n\geq 0}V^{\otimes n}=k\oplus V\oplus (V\otimes V)\oplus\cdots.\]
    This is the associative algebra generated by elements $i(X)$, $x\in V$, subject to linearity relations.
\end{defn}
For this notation to make sense, we need to check that $i$ is injective, but we will not worry about this at the moment.

\begin{defn}
    The \textbf{universal enveloping algebra} is defined to be
    \[U(\fr g)=T(\fr g)/\left( XY-YX-[X,Y] \right).\]
    Note that if $\fr g$ is commutative, we are simply taking $U(\fr g)=T(\fr g)/(XY-YX)=S(\fr g)$, which is simply the symmetric algebra
    (this is essentially the same thing as polynomials on $\fr g^*$).
\end{defn}

We claim that any representation $\rho:\fr g\to\End V$ factors through $U(\fr g)$. Indeed, in this sense we see that for every Lie algebra representation
we have a $U(\fr g)$-module. This is, in fact, an equivalence of categories. One can think, more explicitly, of $U(\fr g)$ as the convolution algebra of
distributions on $G$ supported at $e$. 

Consider now $U(\fr{sl}_2\C)$. It has a center, unlike $\fr{sl}_2\C$. Take, for example, $C=ef+fh+h^2/2$ - one can check that this is contained in the center.
Schur's lemma implies that $C$ acts as a scalar on irreducible representations of $\fr g$. Indeed, one can characterize irreducible representations by this $C$,
which is occasionally called the ``infinitesimal character'' (as it is purely Lie algebraic in nature). Furthermore, one use these to prove complete reducibility
for semisimple Lie algebras. It turns out that we can compute this scalar by letting $C$ act on heighest weight vectors $v_\lambda$:
\[Cv_\lambda=(ef+fe+h^2/2)v_\lambda=(ef+\lambda^2/2)v_\lambda=(fe+h+\lambda^2/2)v_\lambda=\lambda\left( \lambda+1/2 \right)v_\lambda.\]

Ineed, if we have a finite-dimensional representation of $\fr{sl}_2\C$ then there exists a highest weight $\lambda$, with a $v_\lambda$.
Define $v^k=f^kv_\lambda/k!$. One can check that $hv^k=(\lambda-2k)v_\lambda$, and so all $v^k$ have different eigenvalues and thus are
linearly independent. Unfortunately, we do not know if any of these $v^k$ is zero, because if so, the higher terms will be zero as well.
It is easy to see that $fv^k=(k+1)v^{k+1}$. Finally, we claim that $ev^k=\left( \lambda-k+1 \right)v^{k-1}$ (this can be done via induction).

\end{document}
