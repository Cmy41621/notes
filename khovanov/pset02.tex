\documentclass{../mathnotes}

\usepackage{tikz-cd}
\usepackage{todonotes}

\newgeometry{margin=1.75in}

\title{Algebraic Topology I: PSET 2}
\author{Nilay Kumar\footnote{Collaborated with Matei Ionita.}}
\date{Last updated: \today}


\begin{document}

\maketitle

\subsection*{Hatcher 0.24}
Let $X$ and $Y$ be CW complexes with 0-cells $x_0$ and $y_0$ repsectively. Consider the quotient
$X*Y/(X*\{y_0\}\cup Y*\{x_0\})$. This space can be represented as a product $X\times Y\times I$
with the usual relations
\begin{align*}
    (x_1,y,0)&\sim (x_2,y,0)\\
    (x,y_1,1)&\sim (x,y_2,1)
\end{align*}
as well as those induced by the quotient
\begin{align*}
    (x_0,y,t)&\sim(x_0,y_0,0)\\
    (x,y_0,t)&\sim(x_0,y_0,0),
\end{align*}
which in turn induce the relations (from the ones above)
\begin{align*}
    (x,y,0)\sim(x_0,y,0)\sim(x_0,y_0,0)\\
    (x,y,1)\sim(x,y_0,1)\sim(x_0,y_0,0)
\end{align*}

On the other hand, the space $S(X\wedge Y)$ can be represented by a product $X\times Y\times I$
with the relations
\begin{align*}
    (x,y_0,t) &\sim(x_0,y_0,t)\\
    (x_0,y,t) &\sim(x_0,y_0,t)\\
    (x,y,1) &\sim(x_0,y_0,1)\\
    (x,y,0) &\sim (x_0,y_0,0)
\end{align*}
while quotienting by $S(x_0\wedge y_0)$ adds the relation
\begin{align*}
    (x_0,y_0,t)\sim (x_0,y_0,0).
\end{align*}
Inspecting the above relations, it becomes clear that the relations are identical, and
hence the spaces are homeomorphic, being the same quotients of $X\times Y\times I$.

Since $S(x_0\wedge y_0)$ is contractible in $S(X\wedge Y)$ and $X*y_0$ and $Y*x_0$ are both contractible
in $X*Y$, we find that $S(X\wedge Y)\simeq S(X\wedge Y)/S(x_0\wedge y_0)\cong X*Y/(X*y_0\cup Y*x_0)\simeq X*Y$,
as desired.


\subsection*{Hatcher 0.25}
Let $X$ be a CW complex with connected components $X_\alpha$. Consider the wedge sum of suspensions
$\bigvee_\alpha X_\alpha$. Denoting the number of connected components by $n$, attach to this wedge sum
a further wedge of $n-1$ copies of $S^1$. We wish to show that $\bigvee^{n-1} S^1\bigvee_\alpha SX_\alpha$
is homotopy equivalent to $SX$. This is done as follows: consider the wedge sum $S^1\vee SX_\alpha\vee SX_{\alpha+1}$.
We can homotope the loop down the suspension of a zero cell in each suspension until the loop connects the
unconnected vertices of the two suspensions; quotienting by this contractible loop yields a homotopy equivalent
shape which is precisely $S(X_\alpha\cup X_{\alpha+1})$. Repeating this for each of the $n-1$ copies of $S^1$
shows that the space is homotopy equivalent to $SX$.

Suppose, in particular, that $X$ is a finite graph. In this case, the suspension of each connected component
$X_\alpha$ is homotopy equivalent to a wedge sum of spheres (suspensions of loops in the graph) as unlooped
edges yield contractible sheets. Hence by the result above, we find that $SX$ is homotopy equivalent to a wedge
sum of circles and spheres.


\subsection*{Hatcher 1.1.16}
In this problem we invoke Hatcher's Proposition 1.17, which states that a retraction of $X$ to $A$
induces, at the level of fundamental groups, an injection from $\pi_1(A)\to\pi_1(X)$.
\begin{enumerate}[(a)]
    \item There is no injection from $\Z=\pi_1(S^1)$ to the trivial group.
    \item Clearly $\pi_1(S^1\times S^1)=\Z^2$, and $\pi_1(S^1\times D^2)=\pi(S^1)=\Z$. However,
        there is no injection from $\Z\times\Z$ because if $(1,0)\mapsto a$ and $(0,1)\mapsto b$,
        then $(b,0)$ and $(0,a)$ both map to $ab$.
    \stepcounter{enumi}
    \item The wedge sum of two disks, $D^2\vee D^2$, is clearly contractible, as one can explicity
        construct a nullhomotopy of any loop (treating the spaces as two separate convex regions),
        or alternatively using van Kampen's theorem. On the other hand, the fundamental group of $S^1\vee S^1$
        is nontrivial (the free product $\Z*\Z$).
    \stepcounter{enumi}
    \item The M\"obius band deformation retracts onto its core circle, and hence has fundamental group $\Z$,
        as does the boundary circle. It is clear, however, that twice the homotopy class of the core circle is
        sent to that of the boundary circle. This would imply the existence of a map, namely the morphism on
        fundamental groups induced by the retraction, $\Z\to\Z$, taking twice the generator to the generator
        which is impossible.
\end{enumerate}

\subsection*{Hatcher 1.1.17}
We wish to construct infinitely many nonhomotopic retractions $S^1\vee S^1\to S^1$.
There is an obvious one taking all points in one of the circles to the mirrored points in the others,
and keeping the points of the other circle fixed. The induced morphism $r_*$ on fundamental groups
has image the generator of the fundamental group of $S^1$. However, we can also take
force the points of the first circle to loop around the second circle twice, which yields
at the level of fundamental groups twice the generator of the fundamental group of $S^1$.
Clearly we can extend this to $\Z$ (the negatives are obtained by reversing the direction of the
circle. These maps are clearly nonhomotopic, as the maps they induce on the fundamental groups are
not the same.

\subsection*{Hatcher 1.2.4}
Let $X\subset\R^3$ be the union of $n$ lines through the origin and consider the space $Y=\R^3-X$.
$Y$ clearly deformation retracts onto the sphere punctured at $2n$ points. By a simple stereographic
projection based at one of these punctures, then, we find that $Y$ is homeomorphic to $\R^2$ punctured
at $2n-1$. This, in turn, deformation retracts to the $2n-1$ wedge product of circles, and hence $Y$
has fundamental group $\Z^{*(2n-1)}$.

\subsection*{Hatcher 1.2.8}
Consider two tori $S^1\times S^1$ where the first circle of one is identified with the second circle of
the other. If we apply the van Kampen theorem to open neighborhoods of the tori, we find that the fundamental
group of the identification is given by the free product of two copies of $\Z\times\Z$ quotiented by
the relation that one of the generators in the first is identified with one in the second. In other
words, we can write $\pi_1(X)=\langle a,b,c,d\mid aba^{-1}b^{-1}, cdc^{-1}d^{-1}, bc^{-1}\rangle$. In other words,
$b=c$ which commutes with both $a$ and $d$, and we find that $\pi(X)=\Z*\Z\times\Z$.

\subsection*{Hatcher 1.2.11}
Let $X=S^1\vee S^1$ and $f:X\to X$ be a basepoint-preserving map. The mapping torus $T_f$ can be viewed
at the result of attaching two cells to the wedge sum $X\vee S^1$; as a result, by Hatcher's Proposition
1.26, we find that $\pi_1(T_f,x_0)=\Z^{*3}/\langle acf_*(a)^{-1}c^{-1}, bcf_*(b)^{-1}c^{-1}\rangle$, where
$a,b$ are the generators of the fundamental group of $X$ and $c$ is the generator of the additional $S^1$.

Next let $X=S^1\times S^1$. Again, we can view the mapping torus $T_f$ as the result of attaching cells to
the sum $X\vee S^1$. This time, however, as the torus is a two-dimensional cell complex composed of two
one-cells and one two-cell (not counting the basepoint), we must attach two two-cells and one three-cell.
The three-cell is irrelevant for the purposes of computing $\pi_1$ (c.f. Hatcher Problem 1.2.6) so it suffices
to examine how the two-cells are attached. Note, however, that what we obtain is essentially the same as
for the case of $S^1\vee S^1$ (due to the cell structure of $S^1\times S^1$) except that there is an extra
2-cell making $a$ and $b$ commute, where $a$ and $b$ are the generators of the fundamental group of the torus.
Hence we find that $\pi_1(T_f,x_0)=\Z^{*3}/\langle [a,b],acf_*(a)^{-1}c^{-1}, bcf_*(b)^{-1}c^{-1}\rangle$

\subsection*{Hatcher 1.3.4}
See diagrams below.

\subsection*{Hatcher 1.3.9}
Let $X$ have finite fundamental group and $f:X\to S^1$ be a continuous map. Then $f_*=0$, as $\pi_1(S^1)=\Z$,
which has no finite subgroups. In view of the universal cover $p:\R\to S^1$, then, we find that the map $f$
lifts to a map $\tilde f:X\to\R$. Clearly $\tilde f$ is nullhomotopic and so projecting down by $p$ we find
that $f$ is nullhomotopic.

\subsection*{Hatcher 1.3.10}
See diagrams below.

\end{document}
