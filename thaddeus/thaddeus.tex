\documentclass{article}
\usepackage[utf8]{inputenc}

%\usepackage[margin=1in]{geometry}
\usepackage{amsmath}
\usepackage{amssymb}
\usepackage{amsthm}
\usepackage{enumerate}
\usepackage{tikz-cd}

\newcommand{\nm}[1]{\;\textnormal{#1}\;}
\newcommand{\ra}[0]{\rightarrow}
\newcommand{\fa}[0]{\;\forall}
\newcommand{\R}{\mathbb{R}}
\newcommand{\Q}{\mathbb{Q}}
\newcommand{\Z}{\mathbb{Z}}
\newcommand{\F}{\mathbb{F}}
\newcommand{\C}{\mathbb{C}}
\newcommand{\CP}{\mathbb{C}\mathbb{P}}
\newcommand{\RP}{\mathbb{R}\mathbb{P}}
\newcommand{\Proj}{\mathbb{P}}
\newcommand{\N}{\mathbb{N}}
\newcommand{\p}{\partial}
\newcommand{\fr}{\mathfrak}
\newcommand{\OO}{\mathcal{O}}


\DeclareMathOperator{\Ker}{Ker}
\DeclareMathOperator{\Tr}{Tr}
\DeclareMathOperator{\Res}{Res}
\DeclareMathOperator{\ord}{ord}
\DeclareMathOperator{\Hom}{Hom}
\DeclareMathOperator{\length}{length}
\DeclareMathOperator{\res}{Res}
\DeclareMathOperator{\Int}{Int}
\DeclareMathOperator{\Ext}{Ext}
\DeclareMathOperator{\Aut}{Aut}
\DeclareMathOperator{\Gal}{Gal}
\DeclareMathOperator{\Sym}{Sym}
\DeclareMathOperator{\Lie}{Lie}
\DeclareMathOperator{\Pro}{Proj}
\DeclareMathOperator{\id}{Id}
\DeclareMathOperator{\tr}{tr}
\DeclareMathOperator{\irr}{irr}
\DeclareMathOperator{\supp}{supp}
\DeclareMathOperator{\trdeg}{trdeg}
\DeclareMathOperator{\Spec}{Spec}
\DeclareMathOperator{\Nm}{Nm}
\DeclareMathOperator{\hgt}{ht}
\theoremstyle{plain}
\newtheorem{thm}{Theorem}
\newtheorem*{thm*}{Theorem}
\newtheorem{lem}[thm]{Lemma}
\newtheorem*{lem*}{Lemma}
\newtheorem{cor}[thm]{Corollary}
\newtheorem*{cor*}{Corollary}
\newtheorem{prop}[thm]{Proposition}
\newtheorem*{prop*}{Proposition}
\newtheorem{exc}{Exercise}

\theoremstyle{definition}
\newtheorem{defn}{Definition}
\newtheorem{exmp}{Example}

\theoremstyle{remark}
\newtheorem*{rem}{Remark}

\title{Commutative Algebra \`a la M.~Thaddeus}
\author{Notes by Nilay Kumar}
\date{}

\begin{document}

\maketitle

The following result will be quoted extensively.

\begin{lem}[Nakayama, Azumaya, Krull]
     Let $A$ be a ring, $M$ a finitely generated $A$-module, and $I$ an ideal of $A$.
     Suppose that $IM=M$. Then there exists an element $a\in A$ of the form $a=1+x$,
     $x\in I$ such that $aM=0$. If, moreover, if $I$ is contained in the Jacobson radical
     $\mathcal{J}(A),$ then $M=0$.
\end{lem}
\begin{proof}
    We follow Matsumura.
    Let $M=Aw_1+\cdots+Aw_s$. We induct on $s$. Let $M'=M/Aw_s$. By the induction
    hypothesis there exist $x\in I$ such that $(1+x)M'=0$, i.e. $(1+x)M\subset Aw_s$.
    Note that for the base case $s=1$, the module is trivial, so we can take $x=0$.
    Since $M=IM$, we have that
    $(1+x)M=I(1+x)M\subset I(Aw_s)=Iw_s$, and hence we can write $(1+x)w_s=yw_s$ for some
    $y\in I$. Then $(1+x-y)(1+x)M=0$, and $(1+x-y)(1+x)\equiv 1\mod I$ proving the
    first assertion. If in particular $I\subset\mathcal{J}(A)$ then $1+x$ is not contained
    in any maximal, and thus a unit. This implies that $M=0$.
\end{proof}

The following corollary is often useful.
\begin{cor}
    Let $A$ be a ring, $M$ an $A$-module, $N$ and $N'$ submodules of $M$, and $I$ an
    ideal of $A$. Suppose that $M=N+IN'$, and that either (a) $I$ is nilpotent or (b)
    $I\subset\mathcal{J}(A)$ and $N'$ is finite. Then $M=N$.
\end{cor}
\begin{proof}
    In case (a) we have $M/N=I(M/N)=I^2(M/N)=\cdots=0.$ In case (b), apply Nakayama's
    Lemma to $M/N$.
\end{proof}

\subsection*{November 11, 2014}

Recall the following theorem from last time.
\begin{thm}
    A ring $R$ is Artinian if and only if $R$ is Noetherian of Krull dimension 0.
\end{thm}

\begin{exmp}
    An algebra $R$ over $k$ a field is Artinian if and only if $R$ is a finite-dimensional
    vector space, e.g.
    \[R=k[x,y]/(x^m,y^n,f(x,y),g(x,y)).\]
    Note, incidentally, if we have $f,g$ monomials (i.e.~a monomial ideal), we can draw a lattice to represent
    this ring, and we obtain Young diagrams.
\end{exmp}

\begin{exc}
    A ring $R$ is Artinian if and only if $\Spec R$ is finite and discrete.
\end{exc}

\begin{exc}
    $\Spec R=\Spec R_1\sqcup \Spec R_2$ as a ringed space if and only if $R\cong R_1\times R_2$.
\end{exc}

\begin{thm}[Structure theorem for Artinian rings]
    Any Artinian ring $R$ is a product of finitely many Artinian local rings.
\end{thm}
\begin{proof}
    C.f. Atiyah-MacDonald. However, there may be a proof through thinking geometrically
    via the above exercises.
\end{proof}

\begin{cor}
    If $(R,\fr m)$ is Artinian local then $\fr m$ is nilpotent.    
\end{cor}
\begin{proof}
    Apply Nakayama's lemma to the chain of $R$-modules
    \[\fr m\supset \fr m^2\supset \fr m^3\supset \cdots\]
    which stabilizes.
\end{proof}

\begin{cor}
    For $R$ Noetherian, $\fr p$ is a prime minimal among those containing some ideal $\mathcal{I}\leq R$,
    if and only if $R_{\fr p}/\mathcal{I}_{\fr p}$ is Artinian.
\end{cor}
\begin{proof}
    The ring $R_{\fr p}/\mathcal{I}_{\fr p}$ is Noetherian, and the primes in this ring correspond
    to the primes $\mathcal{I}\subset\fr q\subset\fr p$. So $\fr p$ is the only one if and only if $R_{\fr p}/\mathcal{I}_{\fr p}$
    is Artinian.
\end{proof}

\begin{prop}
    Suppose $R$ is Noetherian and $\fr p$ is a prime containing a nonzerodivisor $x$. Then $\hgt\fr p\geq 1$.
\end{prop}
\begin{proof}
    The nilradical $\sqrt{0}\leq R$ is, in general, the intersection of all primes in $R$. Consider the
    chain
    \[\sqrt{0}=\bigcap_{\fr q\in R}\fr q\subset \bigcap_{\substack{\fr q\in R\\\fr q\neq\fr q_1}}\fr q
        \subset \bigcap_{\substack{\fr q\in R\\\fr q\neq\fr q_1,\fr q_2}}\fr q\subset\cdots\]
    for fixed primes $\fr q_i$. Since $R$ is Noetherian, this chain must stabilize (in this case, to some prime), and hence $\sqrt{0}$
    can be written as the intersection of finitely many primes in $R$.
    Thus
    \[\fr p\supset \sqrt{0}=\fr p_1\cap\cdots\cap\fr p_n\]
    for some primes $\fr p_i$. Now, $\fr p$ must contain some
    $\fr p_i$. Suppose it didn't: taking $x_i\in \fr p_i\setminus \fr p$, we find that $x_1\cdots x_n\in\sqrt{0}\setminus\fr p$,
    a contradiction. To establish that the height is one, now, it suffices to show that each of these minimal primes
    consist entirely of zerodivisors, so as not to be equal to $\fr p_i$. So let $\fr q$ be any minimal prime.
    Suppose $\fr q$ contains a nonzerodivisor $x$. Since $R_{\fr q}$ is Artinian, for $x=x/1\in R_{\fr q}$,
    the chain
    \[(x)\supset (x)^2\supset (x)^3\supset\cdots\]
    stabilises, i.e. $(x)^{n+1}=(x)^n$ and $x^n=yx^{n+1}$ for $y\in R_{\fr q}$.
    Then $x^n(1-xy)=0$ which implies that $1-xy=0$ and hence $x$ is a unit, i.e. $x\notin \fr q_{\fr q}$. This
    shows that $x\notin\fr q$.\footnote{We have used the following easy fact: if $x$ is a
        nonzerodivisor in $R$ then $x$ is a nonzerodivisor in $R_{\fr q}$. Show this!}
\end{proof}

Now we prove an estimate in the opposite direction. But first we introduce the following notion.

\begin{defn}
    The $n^\text{th}$ \textbf{symbolic power} of $\fr p\subset R$ is defined to be
    \[\fr p^{(n)}=\{x\in R\mid sx\in\fr p^n\text{ for some }s\in R\setminus\fr p\},\]
    i.e. $\fr p^{(n)}$ is the inverse image of $\fr p_{\fr p}^n\leq R_{\fr p}$.
    In particular $(\fr p^{(n)})_{\fr p}=(\fr p_{\fr p})^{n}$.
\end{defn}

\begin{rem}
    The geometric inuition behind this definition is unclear, though the algebraic value lies in
    the following results.
\end{rem}

\begin{lem}
    If $xy\in \fr p^{(n)}$ and $x\notin\fr p$ then $y\in\fr p^{(n)}$.
\end{lem}
\begin{proof}
    If $xy\in\fr p^{(n)}$ then $(xy)/1\in(\fr p_{\fr p})^n$, which implies that $y/1\in(\fr p_{\fr p})^n$
    since $x/1$ is a unit. This shows that $y\in\fr p^{(n)}$.
\end{proof}

\begin{thm}[Krull's Hauptidealsatz]
    If $R$ is Noetherian and $\fr p$ is minimal among primes containing an element $x\in R$ then
    $\hgt\fr p\geq 1$.
\end{thm}

\begin{rem}
    Consider the example of a variety. If we consider the subvariety cut out
    by an irreducible element $f$ in the coordinate ring, the ideal $(f)$ is obviously minimal
    containing $f$. This, together with the two estimates above, force the codimension of this
    subvariety to be exactly 1.
\end{rem}

\begin{proof}
    Fix a prime $\fr q\subsetneq\fr p$. By construction, $x\notin\fr q$. It is sufficient to
    show, now, that $\hgt\fr q=0$, which is the same as showing that $\dim R_{\fr q}=0$.
    Replacing $R$ with $R_{\fr p}$, we assume without loss of generality that $(R,\fr p)$ is local.\footnote{In
    other words, a counterexample remains a counterexample after localization.}
    Since $\fr p$ is minimal among primes containing $(x)$, we find, by
    the above Corollary, that $R/(x)$ is Artinian. We have a descending chain $\fr q^{(n)}$ in $R$
    whose image in $R/(x)$ is a descending chain of ideals in $R/(x)$.
    This chain stabilizes, and hence the corresponding chain $(x)+\fr q^{(n)}$ in $R$ must stabilize:
    \[(x)+\fr q^{(n+1)}=(x)+\fr q^{(n)}\supset\fr q^{(n)}.\]
    Thus, for all $f\in\fr q^{(n)}$, \[f=rx+g,\] for $r\in R,g\in\fr q^{(n+1)}$.
    In particular, $rx\in\fr q^{(n)}$, $x\notin\fr q$, so by the above Lemma, $r\in\fr q^{(n)}$.
    Now $\fr q^{(n)}=(x)\fr q^{(n)}+\fr q^{(n+1)}$, and $\fr q^{(n)}/\fr q^{(n+1)}=(x)\fr q^{(n)}/\fr q^{(n+1)}$,
    so by Nakayama's lemma (everything in sight is finitely generated) we find that $\fr q^{(n)}=\fr q^{(n+1)}$
    since $(x)$ is in the Jacobson radical of $R$. Hence
    \[(\fr q_{\fr q})^n=(\fr q^{(n)})_{\fr q}=(\fr q^{(n+1)})_{\fr q}=(\fr q_{\fr q})^{n+1}\]
    and applying Nakayama's lemma again, we find that $(\fr q_{\fr q})^n=0$.
    So if $0\subset\mathcal{I}\subset\fr q$, for $\mathcal{I}$ prime, then 
    \[\fr q_{\fr q}\subset \sqrt{0_{\fr q}}\subset\sqrt{\mathcal{I}_{\fr q}}\subset \sqrt{\fr q_{\fr q}}.\]
    It follows that $\mathcal{I}_{\fr q}=\fr q_{\fr q}$, and hence $\mathcal{I}=\fr q$ and $\dim R_{\fr q}=0$.
\end{proof}

\begin{cor}
    If $R$ is Noetherian and $\fr p$ is a minimal prime over a nonzerodivisor $x$ then $\hgt\fr p=1$.
\end{cor}

\subsection*{November 12, 2014}



\end{document}

