\documentclass{mathnotes}

\title{Notes on Topological and Differentiable Manifolds}
\author{Nilay Kumar}
\date {}

\begin{document}

\maketitle

\section{Elementary Topology}

Let us begin with the definition of a topology:
\begin{defn}
    A \textbf{topology} on a set $X$ is a collection $\mathcal{T}$ of subsets of $X$, called \textbf{open sets},
    satisfying the following properties:
    \begin{enumerate}
        \item $X$ and $\varnothing$ are elements of $\mathcal{T}$.
        \item $\mathcal{T}$ is closed under finite intersections: If $U_1\dots U_n\in\mathcal{T}$, then their intersection
            $U_1\cap\dots\cap U_n$ is in $\mathcal{T}$.
        \item $\mathcal{T}$ is closed under arbitrary unions: If $U_1\dots U_n\dots$ is any (finite or infinite) collection
            of elements of $\mathcal{T}$, then their union $\cup_\alpha U_\alpha$ is in $\mathcal{T}$.
    \end{enumerate}
    A pair $(X,\mathcal{T}$) consisting of a set $X$ and a topology $\mathcal{T}$ on $X$ is called a \textbf{topological space}.
    The elements of a topological space are usually called its \textbf{points}.
\end{defn}

\begin{defn}
    If $X$ is a topological space and $q\in X$, a \textbf{neighborhood} of $q$ is just an open set containing $q$. More generally, a neigborhood
    of a subset $K\subset X$ is an open set containing $K$.
\end{defn}

\begin{defn}
    If $X$ is a topological space and $\left\{ q_i \right\}$ is any sequence of points in $X$, we say that the sequence \textbf{converges} to $q\in X$, and
    $q$ is the \textbf{limit} of the sequence, if for every neighborhood $U$ of $q$ there exists $N$ such that $q_i\in U$ for all $i\geq N$. We denote this as
    $q_i\ra q$ or $\lim_{i\ra\infty}q_i=q$.
\end{defn}

\begin{exmp}
    Let $Y$ be a trivial topological space (i.e. the only open sets are $X$ and $\varnothing$). Each point has only 1
    neighborhood: $X$ itself. Thus, any sequence can be entirely contained in the neighborhood $X$, and consequently, any sequence
    converges to any point in $X$.
\end{exmp}

\begin{exmp}
    Let $X$ be a discrete topological space (i.e. all every subset of $X$ is open). Take any sequence of points $\left\{ q_i \right\}$.
    If the sequence converges to $q$, every open set containing $q$ must contain all but a finite elements of the sequence. By virtue of
    the discrete topology, there exists an open set that contains only $q$. Obviously, then, there must exist an $N$ such that
    $q_i=q$ for all $i\geq N$. Consequently, the only convergent sequences in $X$ are the ones that are ``eventually constant.''
\end{exmp}

\begin{defn}
    If $X$ and $Y$ are topological spaces, a map $f:X\ra Y$ is said to be \textbf{continuous} if for every open set
    $U\subset Y$, $f^{-1}(U)$ is open in $X$.
\end{defn}

\begin{lem}
    Let $X, Y, Z$ be topological spaces.
    \begin{enumerate}
        \item Any constant map $f:X\ra Y$ is continuous.
        \item The identity map $\id: X\ra X$ is continuous.
        \item If $f: X\ra Y$ is continuous, so is the restriction of $f$ to any open subset of $X$.
        \item If $f: X\ra Y$ and $g:Y \ra Z$ are continuous, so is their composition $g \circ f: X\ra Z$.
    \end{enumerate}
\end{lem}
\begin{proof}
    Let us begin with the constant map. Suppose $f$ maps $X$ to the constant $\lambda\in Y$. We wish to show
    that the preimage of $f$ of every open set $U$ in $Y$ is open. There are two cases: $U$ either does or does not contain
    $\lambda$. If it does, $f^{-1}(U)=X$; otherwise, $f^{-1}(U)=\varnothing.$ As both $X$ and $\varnothing$ are open sets,
    $f$ is continuous.

    The continuity of the identity map follows trivially from the fact that $\id$ maps any open set back to the same open set.

    To prove the third statement, take any open set $U$ in $Y$. $U$ can be written as a union of points in and outside
    $f(V)\subset Y$: $U=U_i\cup U_o$. We want to show that $g^{-1}(U)$ is open in $V$. Since $g^{-1}(U_o)=\varnothing$, which is open,
    and $g^{-1}(U_i)\subset V$ and is open in $X$ by the continuity of $f$, $g^{-1}(U_o\cup U_i)=g^{-1}(U_o)\cup g^{-1}(U_i)$ is open in $V$.

    To prove the fourth statement, it suffices to show that $(g\circ f)^{-1}(U)$, with $U\subset Z$ open, is open in $X$.
    First note that $(g\circ f)^{-1}(U)=f^{-1}(g^{-1}(U))$. Since $g$ is continuous, $g^{-1}(U)$ is an open set in $Y$.
    Similarly, $f^{-1}$ of an open set in $Y$ is open in $X$ as $f$ is continuous, and we are done.
\end{proof}

\begin{lem}[Local Criterion for Continuity]
    A map $f:X\ra Y$ between topological spaces is continuous if and only if each point of $X$ has a neighborhood on which
    (the restriction of) $f$ is continuous.
\end{lem}
\begin{proof}
    If $f$ is continuous, each point of $X$ a neighborhood on which $f$ is continuous; namely, $X$ itself.
    
    To prove the converse, suppose that each point of $X$ has a neighborhood on which $f$ is continuous - we wish to show that for any
    open set $U\subset Y,$ $f^{-1}(U)$ is open in $X$. By continuity at each point, we know that any point $x\in f^{-1}(U)$ has a neighborhood
    $V_x$ on which $f$ is continuous. In other words, $(f|_{V_x})^{-1}(U)=f^{-1}(U)\cap V_x$ is open in $X$ and is contained in $f^{-1}(U)$.
    As $f^{-1}(U)$ is the union of such sets for all $V_x$, and these sets are open, it follows that $f^{-1}(U)$ is open, and we are done.
\end{proof}

\begin{defn}
    If $X$ and $Y$ are topological spaces, a \textbf{homeomorphism} from $X$ to $Y$ is defined to be a continuous bijective map
    $\phi:X\ra Y$ with continuous inverse. If there exists a homeomorphism betwee $X$ and $Y$, we say that $X$ and $Y$ are \textbf{homeomorphic}
    or \textbf{topologically equivalent}. Sometimes this is abbreviated $X\approx Y$.
\end{defn}

\begin{exc}
    Show that homeomorphisms are an equivalence relation.
\end{exc}
\begin{proof}
    To show that homeomorphisms are an equivalence relation, we show
    \begin{itemize}
        \item $X\approx X$: The identity map $\id$ is a homeomorphism from $X$ to $X$.
        \item $X\approx Y \implies Y\approx X$: There exists a homeomorphism from $X$ to $Y$. Its inverse is clearly
            a homeomorphism from $Y$ to $X$.
        \item $X\approx Y \nm{and} Y\approx Z \implies X\approx Z$: As the composition of two homeomorphisms is also a homeomorphism
            (from elementary set theory), the homeomorphism from $X$ to $Z$ is simply the composition of the homeomorphisms
            from $Y$ to $Z$ and from $X$ to $Y$, respectively.
    \end{itemize}
\end{proof}

\begin{exmp}
    Any open ball in $\mathbb{R}^n$ is homeomorphic to any other open ball. The homeomorphism can be constructed simply by
    composition translations $x\mapsto x+x_0$ and dilations $x\mapsto cx$. This shows that size is not a topological property.
\end{exmp}

\begin{exmp}
    If $\mathbb{B}^n$ is the open unit ball, we can define $F:\mathbb{B}^n\ra \mathbb{R}^n$ by
    \[y=F(x)=\frac{x}{1-|x|^2}.\]
    The inverse is given by
    \[x=F^{-1}(y)=\frac{2y}{1+\sqrt{1+4|y|^2}}.\]
    As both are continuous and bijective, $F$ is a homeomorphism, so $\mathbb{R}^n$ is homeomorphic to $\mathbb{B}^n$. This
    shows that boundedness is not a topological property.
\end{exmp}

\begin{exmp}
    Take the surface of the unit sphere in $\mathbb{R}^3$ and the surface of the cube of side 2, centered at the origin. There exists
    a homeomorphism between these two surfaces,
    \[\phi(x,y,z)=\frac{(x,y,z)}{\sqrt{x^2+y^2+z^2}}\]
    whose inverse is given by
    \[\phi^{-1}(x,y,z)=\frac{(x,y,z)}{\max(|x|,|y|,|z|)}.\]
    Thus, corners are not a topological property either.
\end{exmp}

\begin{exmp}
    Now for an example of a continuous bijection that is not a homeomorphism by failure of its inverse to be continuous.
    Let $X$ be the interval $[0,1)\subset \mathbb{R}$, and let $\mathbb{S}^1$ denote the unit circle in $\mathbb{R}^2$
    (both with the Euclidean metric topologies). Define a map $a:X\ra \mathbb{S}^1$ by $a(t)=(\cos 2\pi t, \sin 2\pi t)$.
    It is clear that the map is continuous and bijective. The inverse, however, is not continuous. To see why, take any
    neighborhood of the point $(1,0)$ - the inverse of $a$ on this neighborhood will inevitably contain the point $0\in[0,1).$
    Thus the preimage of open sets is not necessarily open, and thus $a^{-1}$ is not continuous.
\end{exmp}

\begin{defn}
    A map $f:X\ra Y$ is said to be an \textbf{open map} if for any open set $U\subset X$, the image set $f(U)$ is open in $Y$.
    A map can be open but not continuous, continuous but not open, both, or neither.
\end{defn}

\begin{defn}
    We say that a continuous map $f:X\ra Y$ between topological spaces is a \textbf{local homeomorphism} if every point
    $x\in X$ has a neighborhood $U\subset X$ such that $f(U)$ is an open subset of $Y$ and $f|_U:U\ra f(U)$ is a homeomorphism.
\end{defn}

\begin{exc}
    Show that:
    \begin{enumerate}
        \item every local homeomorphism is an open map.
        \item every homeomorphism is a local homeomorphism.
        \item a bijective continuous open map is a homeomorphism.
        \item a bijective local homeomorphism is a homeomorphism.
    \end{enumerate}
\end{exc}
\begin{proof}
    \hspace{1mm} 
    \begin{enumerate}
        \item Take any open set $U\in X$. For every $x\in U$, there exists some neighborhood $V=U_x\cap U$
            of $x$ for which the local homeomorphism $f(V)$ is open. Note that $U$ is the union of all such $V$ (for various $x$)
            and thus $f(U)$ is the union of all $f(V)$, and as the union of open sets must be an open set, $f(U)$ must be open.
            Consequently, $f$ is an open map.
        \item Take $f$ to be a homeomorphism from $X$ to $Y$. $f$ is trivially a local homeomorphism; the neighborhood needed by
            the definition is simply $X$ itself.
        \item Let $f$ be a bijective, continuous, open map. To show that $f$ is a homeomorphism, we must show that $f^{-1}$
            is a continuous map. In other words, we must show that the preimage of $f$ on open sets of $Y$ are open
            sets in $X$. This is true as $f$ is open and bijective: open sets in $X$ are taken to open sets in $Y$ (open), and
            all open sets in $Y$ are images of open sets in $X$ (bijective).
        \item Let $f$ be a bijective local homeomorphism. 
    \end{enumerate}
\end{proof}















\end{document}
