\documentclass{../mathnotes}

\usepackage{tikz-cd}
\usepackage{todonotes}

\title{Hartshorne Solutions}
\author{Matei Ionita and Nilay Kumar}
\date{Last updated: \today}


\begin{document}

\maketitle

\subsection*{Problem I.1.1}
\begin{enumerate}[(a)]
    \item Let $Y$ be the plane curve $y-x^2=0$. The coordinate ring $A(Y)=k[x,y]/(y-x^2)$
        is isomorphic to $k[x]$, as any power of $y$ is simply replaced by $x^2$.
    \item Let $Z$ be the plane curve $xy-1=0$. The corrdinate ring $A(Z)=k[x,y]/(xy-1)$
        is clearly not isomorphic to a polynomial ring of one variable, as there are
        non-constant elements of $A(Z)$ (namely, powers of $y$) that are invertible.
    \item We assume that the $\characteristic k\neq 2$.
        Let $f$ be an irreducible quadratic polynomial in $k[x,y]$, i.e.
        \[f(x,y)=x^2+axy+by^2+cx+dy+e,\]
        and let $W$ be the conic defined by $f$ (we can choose $f$ to be monic without
        loss of generality).
        
        Suppose the degree two terms are a perfect square.
        Then we may change variables to obtain $z^2+bz+dy+e$ (with new coefficients).
        Completing the square (here we use $\characteristic k\neq 2$) in $z$ and changing
        variables again, we obtain $w^2-y=0$, which yields the curve $Y$ from part
        (a) above.

        If the degree two terms do not factor perfectly, \todo{finish}
\end{enumerate}

\subsection*{Problem I.1.2}
Let $Y\subset \A^3$ be the set $Y=\left\{ (t,t^2,t^3)\mid t\in k\right\}$.
Consider $\fr a=(y-x^2,z-x^3)\subset k[x,y,z]$. It is clear that $Y=I(\fr a)$\todo{why? division}.
The coordinate ring is thus given $A(Y)=k[x,y,z]/(y-x^2,z-x^3)$. Note that
$\dim A(Y)=1$ by dimension theory (as $z-x^3$ is not a zero divisor in $k[x,y,z]$
and $y-x^2$ is not a zero divisor in $k[x,y,z]/(z-x^3)$), and hence $Y$
has dimension one. Indeed, it is easy to see that $A(Y)\cong k[x]$ as $y$
is replaced by $x^2$ and $z$ is replaced by $x^3$.

\subsection*{Problem I.3.1}
\begin{enumerate}[(a)]
    \item By the results of problem 1.1, we know that any conic in $\A^2$ can be written as either
        a variety $Y$ defined by $y-x^2=0$ or a variety $Z$ defined by $xy-1=0$. We know that
        $A(Y)=k[x]$ and $A(Z)=k[x,x^{-1}]$. Note that $A(Y)\cong A(\A^1)$, and hence by Corollary
        3.7, $Y\cong\A^1$ as affine varieties. It remains to show that $Z$ is isomorphic to
        $\A^1-\{0\}$. Note first that $xy-1=0$ can be parametrized as $(t,t^{-1})$, which
        suggests the map $\phi:Z\to\A^1-\{0\}$ given by $\phi(t,t^{-1})=t$ as well as the reverse
        $\psi:\A^1-\{0\}\to Z$ given by $\psi(x)=(x,x^{-1})$. It is easy to check $\phi$ and $\psi$
        are morphisms with $\psi\circ\phi=\id_Z$ and $\phi\circ\psi=\id_{\A^1-\{0\}}$.
    \item Let $B$ be a proper open subset of $\A^1$. By definition of the Zariski topology, we can
        write $B=\A^1\setminus\{p_1,\ldots, p_n\}$ where $p_i$ are a finite set of points in $\A^1$.
        The ring of regular functions of $\A^1$ is $\mathcal{O}(\A^1)=k[x]$. In $B$, however,
        polynomials that vanish only at any of the $p_i$ are globally invertible, and hence
        $\mathcal{O}(B)=k[x,(x-p_1)^{-1},\ldots,(x-p_n)^{-1}]$. These two rings are clearly
        not isomorphic, which completes the proof.
    \item In the projective plane, we can write a conic as $F(x,y,z)=ax^2+2bxy+2cxz+dy^2+2eyz+fz^2$,
        which can be rewritten under an appropriate change of variables as $x^2+y^2+z^2$\todo{do this} (assuming
        $\characteristic k\neq 2$). Hence every
        conic in the projective plane is isomorphic, and it will suffice to show that there exists
        a conic that is isomorphic to $\Proj^1$.
        This follows From the result of exercise I.3.4: the $2$-uple embedding 
        $\rho_2:\Proj^1\to\Proj^2$ is an isomorphism onto its image
        \[\rho_2(a_0,a_1)=(a_0^2,a_0a_1,a_1^2),\]
        which clearly traces out a conic $xz-y^2$.
    \item This is hard!! \todo{finish}
    \item If an affine variety $X$ is isomorphic to a projective variety $Y$, then we must have
        that $\mathcal{O}(X)=\mathcal{O}(Y)=k$. But for $k[x_1,\ldots,x_n]/I(X)=k$, $I(X)$
        must be maximal. Hence $I(X)=(x_1-a_1,\ldots,x_n-a_n)$, i.e. $X$ is just a point.
\end{enumerate}

\subsection*{Problem I.3.14}
\begin{enumerate}[(a)]
    \item Note first that $\phi$ is continuous, as the preimage of any closed subset $V\subset \Proj^n$
        is the projective cone $\overline{C(V)}$, which is closed in $\Proj^{n+1}$. 
        Furthermore, the point at which the line connecting any $Q$ and $P$ to the hypersurface
        (choose $x_0=0$ without loss of generality) is given by 
        \[\phi(Q)=[Q_1-\frac{Q_0P_1}{P_0}:\cdots:Q_{n+1}-\frac{Q_0P_{n+1}}{P_0}],\]
        where $P_i$ and $Q_i$, are the $i$th components of $P$ and $Q$, respectively (the coordinates
        are written as for a point in $\Proj^n$).
        It is easy to see that $\phi$
        pulls back regular functions to regular functions: given $g/h:\Proj^n\to k$, $g(\phi(Q))/h(\phi(Q))$
        is regular as well, since inserting $\phi(Q)$ (as above) will retain homogeneity as well as
        keep the denominator non-zero (as $h$ has no zeroes).
    \item The twisted cubic is given parametrically by $[x:y:z:w]=[t^3:t^2u:tu^2:u^3]$. We wish
        to project from $P=[0:0:1:0]$ onto the hyperplane $z=0$. This yields the points
        $[t^3:t^2u:u^3]\in\Proj^2$. Note that these points satisfy the equation $x_0^2x_2-x_1^3=0$.
        But this is precisely the projective closure of the cuspidal cubic $y^3=x^2$.
\end{enumerate}


\subsection*{Problem I.3.15}
\begin{enumerate}[(a)]
    \item Let $X\subset\A^n$ and $Y\subset\A^m$ be affine varieties. Consider the product
        $X\times Y\subset\A^{n+m}$ with the induced Zariski topology. Suppose that $X\times Y$
        is a union of two closed subsets $Z_1\cup Z_2$. Let $X_i=\left\{ x\in X\mid x\times Y\subset Z_i \right\}$
        for $i=1,2$. The irreducibility of $Y$ guarantees that $X_1\cup X_2=X$: if there were
        an $x$ for which $x\times Y$ were not contained in a $Z_i$, this would yield a
        covering of $Y$ by closed sets $Z_1\cap Y$ and $Z_2\cap Y$. Now consider the inclusion
        map $\iota:X\to X\times Y$; if $\iota$ is continuous then $X_i$ must be closed, as
        $\iota^{-1}(Z_i)=X_i$. But the inclusion is obviously continuous, as any closed set
        in $X\times Y$ is defined by the vanishing of polynomials $f_\alpha(x_1,\ldots,x_n,y_1,\ldots, y_m)$,
        whose pullback to $X$ is $f_\alpha(x_1,\ldots,x_n,0,\ldots,0)$, which is by definition
        a closed set of $X$. But if $X_i$ are closed and cover $X$, either $X_1=X$ or $X_2=X$
        and thus $Z_1=X\times Y$ or $Z_2=X\times Y$, i.e. $X\times Y$ is irreducible.
    \item Consider the map $\psi:A(X)\otimes_kA(Y)\to A(X\times Y)$ given by taking $(f\otimes g)(x,y)$
        to $f(x)g(y)$. This map is clearly onto, as it produces the coordinate functions
        $x_1,\ldots,x_n,y_1,\cdots,y_m$.\footnote{One might worry that generators may be missing from
        $A(X)$ or $A(Y)$ and hence that $\psi$ may not produce all the generators of $A(X\times Y)$.
        This is actually not a problem: if $x_i\in I(X)$ then $x_i\in I(X\times Y)$ as well.}
        Injectivity is less obvious.
    \item
    \item It suffices to show that $\dim A(X)\otimes_k A(Y)=\dim A(X)+\dim A(Y)$. By Noether
        normalization, $A(X)$ is module-finite over the polynomial ring $k[t_1,\ldots,t_{d_1}]$
        and $A(Y)$ is module-finite over the polynomial ring $k[s_1,\ldots,s_{d_2}]$ with
        $d_1=\dim A(X)$ and $d_2=\dim A(Y)$. In other words, every element of $A(X)$ or $A(Y)$
        is the solution to some polynomial over the above polynomial rings, respectively.
        Next note that $R=k[t_1,\ldots,t_{d_1}, s_1,\ldots,s_{d_2}]$ must inject into
        $A(X)\otimes_kA(Y)$ via a map $\phi$. 
        Recall that every element in the tensor product can be written as a sum of elementary tensors
        $x\otimes y$ with $x\in A(X),y\in A(Y)$. Hence every element in the tensor product must
        also solve some polynomial over the ring $R$, i.e. $A(X)\otimes_kA(Y)$ is module-finite
        over $R$ and $\dim A(X)\otimes_kA(Y)=d_1+d_2$, as desired.
\end{enumerate}

\subsection*{Problem I.3.17}
\begin{enumerate}[(a)]
    \item Any conic in $\Proj^2$ is isomorphic to $\Proj^1$ by problem I.3.1, and hence it
        suffices to show that $\Proj^1\subset\Proj^2$ is normal. If we parametrize the
        projective line by $x=0$, its coordinate ring becomes the graded ring $k[y,z]$.
        Then $A(Y)$ is a UFD, and it is straightforward to check that as is the degree-zero
        subring of any localization of $A(Y)$. Hence, by theorem I.3.4b, any conic
        in $\Proj^1$ is normal.
    \item 
    \item Consider the cuspidal cubic $C$ in $\A^2$ given by $y^2-x^3=0$. The ring of regular
        functions on $C$ is isomorphic to the coordinate ring $k[x,y]/(y^2-x^3)$. The local
        ring at $(0,0)$ is given by the localization
        \[\mathcal{O}_{(0,0),C}=\left( k[x,y]/(y^2-x^3) \right)_{(x,y)}=k[x,y]_{(x,y)}/(y^2-x^3).\]
        Now notice that the quotient $y/x$ sitting in the fraction field of $\mathcal{O}_{(0,0),C}$
        solves the monic polynomial $t^2-x\in\mathcal{O}_{(0,0),C}[t]$, but $y/x\notin\mathcal{O}_{(0,0),C}$.
        Hence $C$ is not normal.
    \item Let $Y$ be an affine variety with coordinate ring $A(Y)\cong\mathcal{O}(Y)$. Suppose
        first that $\mathcal{O}(Y)$ is normal (integrally closed in its field of fractions).
        Let us show that $Y$ is normal. It suffices to show
        that, more generally, the localization $S^{-1}A$ of a normal domain $A$ is normal.
        If $x$ is an element of the fraction field of $A$ integral over $S^{-1}A$, it must solve
        a polynomial
        \[x^n+\frac{a_{n+1}}{s_{n-1}}x^{n-1}+\cdots+\frac{a_0}{s_0}\]
        for $a_i\in A,s_0\in S$. If we denote by $r=s_0\cdots s_{n-1}$ the product of the $s_i$,
        and multiply by $r^n$, we obtain
        \[(rx)^n+\frac{a_{n-1}r}{s_{n-1}}(r^{n-1}x^{n-1})+\cdots+\frac{a_0r^n}{s_0}.\]
        This implies that $rx$ is integral over $A$, and so $rx\in A$ and $x\in S^{-1}A$.
        Thus, as $\mathcal{O}_{P,Y}$ is a localization of $\mathcal{O}(Y)$,
        which is normal, we find that $Y$ is normal.

        Now suppose instead that $Y$ is normal (i.e. $\mathcal{O}_{P,Y}$ is normal for every
        $P\in Y$); we wish to show that $\mathcal{O}(Y)$ is normal.\todo{finish}
    \item Let $Y$ be an affine variety with $A(Y)\cong\mathcal{O}(Y)$ its coordinate ring.
        The integral closure $N$ of $\mathcal{O}(Y)$ in its field of fractions
        is of course a normal domain. By theorem I.3.9A, $N$ is a finitely generated
        $k$-algebra, and hence by the equivalence of categories of corollary I.3.8,
        we obtain a a variety $\tilde Y$ such that $\mathcal{O}(\tilde Y)=N$. Consider
        now a normal variety $Z$ with a dominant morphism $\phi:Z\to Y$ 
\end{enumerate}

\subsection*{Problem I.3.21}
\begin{enumerate}[(a)]
    \item It suffices to show that the addition and inversion maps are morphisms of varieties.
        But this follows from Lemma 3.6, as $\mu(a,b)=a+b$ and $\iota(a)=-a$ clearly define
        regular functions.
    \item Note that $\G_m$ is, as a variety, simply $\A^1-\{0\}$, which in turn is isomorphic to 
        an affine variety (c.f. problem I.3.1). Hence $\G_m$ is an affine variety, and the
        multiplication and inversion maps are morphisms again by Lemma 3.6.
    \item We define the group operation $\cdot$ on $\Hom(X,G)$ as
        \[(f\cdot g)(x)=\mu(f(x),g(x)),\]
        where $f,g\in\Hom(X,G)$ and $\mu$ is the operation on $G$ and inversion as
        \[f^{-1}(x)=\iota(f(x)),\]
        where $\iota$ is the inversion on $G$. Thus defined, $\Hom(X,G)$ becomes a group
        by virtue of the group structure on $G$.
    \item By part (c), $\Hom(X,\G_a)$ inherits a group structure from $\G_a$, while the group
        structure on $\mathcal{O}(X)$ is the usual one. Any $f\in\Hom(X,\G_a)$ defines a regular
        function on $X$, and hence $f\in\mathcal{O}(X)$. Conversely, any regular function
        $\tilde f\in\mathcal{O}(X)$ is a morphism from $X$ to $\G_a=\A^1$ (by Lemma 3.1) and
        hence contained in $\Hom(X,\G_a)$. The set equality $\Hom(X,\G_a)=\mathcal{O}(X)$ clearly
        extends to a group isomorphism, as the additive structure is clearly preserved.
    \item By part (c), $\Hom(X,\G_m)$ inherits a group structure from $\G_m$, while the group
        of units $H$ in $\mathcal{O}(X)$ is the group of invertible, globally regular functions
        on $X$. Just as in part (d), we have the setwise equality $\Hom(X,\G_m)=H$, which extends
        to a group isomorphism, as the multiplicative structure is preserved.
\end{enumerate}

\end{document}
