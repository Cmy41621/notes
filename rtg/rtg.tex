\documentclass{../mathnotes}

\title{RTG Notes}
\author{Nilay Kumar}
\date{Last updated: \today}


\begin{document}

\maketitle

\setcounter{section}{-1}

\section{Introduction}

\subsection{Adam}

What is algebraic geometry? It is the study of solutions to systems of polynomial equations.

\begin{defn}
    An \textbf{algebraic variety} $V(f_1,\ldots,f_s$ is the set of points $p$ such that $f_1(p)=f_2(p)=\cdots=0$
\end{defn}

One nice thing is that we can work over different fields and get different results. In fact, we can often get
hard number theoretic problems. Take for example $V(x^n+y^n-z^n)$ over $\Q$ -- analyzing this necessitates Fermat's
last theorem!

\begin{exmp}
    Let us, for example, find rational solutions to $x^2+y^2=z^2$. We can rewrite this as $X^2+Y^2$ by a
    substitution. It's clear that we have a circle which goes through $(0,1)$. If we assume there exists a
    rational point $(p,q)$ also on the circle, it's clear that the line going from $(0,1)$ and $(p,q)$ has rational
    slope. Conversely, any line with rational slope will also hit some rational point on the circle. Thus we can
    parametrize the solutions to our problem by rational slopes $m$.
\end{exmp}

In light of this example, how do we work backwards from parametrizations (toric varieties) to varieties?

\begin{exmp}
    Let $C=\left\{ (t,t^2,t^3)\in\R^3|t\in\R \right\}$. $C$ is, in fact, an algebraic variety:
    \[C=V(y-x^2,z-x^3).\]
    Are there other choices? Well we could try to start with the parametrizations and try to eliminate the parameter.
    Consider the homomorphism $\phi:\R[x,y,z]\to\R[t]$ given by $\phi(x)=t,\phi(y)=t^2,\phi(z)=t^3$. Clearly the set
    we want is $\ker\phi$, i.e. the set of polynomials in $x,y,z$ that vanish when $x=t,y=t^2,z=t^3$. We've reduced 
    the problem to determining the kernel of some ring homomorphism.
\end{exmp}

The general question is, given a homomorphism $\phi:\R[x_1,\ldots,x_s]\to\R[r,s,t]$, how do we describe the kernel?
Let's look at another example.

\begin{exmp}
    Consider the parametrizations
    \[\phi(x_1)=s^4,\phi(x_2)=s^3t,\phi(x_3)=st^3,\phi(x_4)=t^4.\]
    One equation is $x_1x_4-x_3x_2$. However, it's clear that these problems can quickly get rather difficult.
\end{exmp}

These are the types of questions we will generally be talking about.

\subsection{Pablo}

We will be focusing on representation theory and geometry. What is representation theory? It is the study of
homomorphisms from a group $G$ into $GL(V)=GL_n$ where $V$ is some vector space of dimension $n$.

As a non-trivial example, consider the group $\left\{ (x,y)\in\C^2|y^2=x^3+x \right\}$ unioned with some point.
Why is this a group? It's certainly non-obvious, and this kinda relates back to Adam's talk about solution sets.
Indeed, in certain fields, such as the complex numbers, the set of points may look like something nice -- take, for
example, the torus. The torus is the direct product of two $U(1)$ groups!

One example of how representation theory can be useful is in the concept of braid groups, where we talk about
braidings pieces of string. It turns out that operations on these braids yield groups, which can be analyzed via
representation theory. As another example, one might consider the group $\Z$ and a representation
\[n\mapsto
    \left(
    \begin{array}[]{cc}
        1 & n\\
          & 1
    \end{array}
    \right).
    \]
Another example  is the matrix Lie group $SL_2$ of invertible matrices with determinant 1, which has the inclusion
map as a representation. Note that representations need not be injective or surjective. For example, we could have a representation
$U(1)\times U(1)\to GL_1(\C)$ that takes $(z,w)\mapsto z$ or $(z,w)\mapsto zw$, etc. with $(a,b)\in\Z^2$.

Pablo recently learned the following thing that he finds rather cool. Take some arbitrary $n\times n$ matrix. Computing determinants
is not the most fun thing in the world. Let us take $n=4$ for example. We can write, it turns out:
\[\det(\lambda I-A)=\lambda^4-\tr(A)\lambda^3+a\lambda^2-b\lambda+\det(A).\]
Is there a nice way of figuring out what $a,b$ are?

Well consider a map $A:\C^4\to\C^4$. One can consider $\Lambda^4 A:\Lambda^4\C^4\to\Lambda^4\C$. But these new vector spaces
are one-dimensional, and the map is simply the determinant map of $A$. Thus $\det(A)=\tr(\Lambda^4A)$. It turns out, then that
the coefficients are of the form $\tr(\Lambda^iA)$! In some sense, this is all due to how alternating tensors work.

But where does the geometry come in? We will get geometry from algebra! For example, we can have groups that act on vector spaces and
thus groups acting on direct sums of vector spaces. One can consider the very big vector space $\oplus_{n\geq 0}\Sym^n(V)$. This, in fact,
forms a ring and hence we have a group acting on polynomials! Let's look at as simple case.
Let $G=\Z/2\Z$ and the vector space $V=\C^2$. One can imagine that the non-identity element in the group acts as $-I$. Consider
what's fixed in the ring: $\C[x,y]^{\Z/2\Z}=\C[x^2,xy,y^2]\cong \C[a,b,c]$. We can consider the solutions to $ac-b^2=0$. We can draw a picture -- this is geometry!

\subsection{Dan}

We're going to talk about symplectic geometry. Let's talk about some preliminaries. We're typically going to be working with what are called \textbf{symplectic
manifolds} $(M^{2n},\omega)$. This means that $M^{2n}$ is a $2n$-dimensional smooth manifold (with or without boundary, compact or non-compact).
One can consider spaces such as $\R^{2n},S^{2n},B^{2n}$ etc. (where the last one is a ball). The $\omega$ is a closed differential two-form ($d\omega=0$) that
is nondegenerate, i.e. $\omega\wedge\cdots\wedge\omega$ ($n$ times) is nowhere zero. What the two-form does is carry geometric information -- it can tell us,
for example, what the angle between two vectors in the tangent space of our manifold at some point. Note that the non-degenerateness will give us a
volume/orientation form. In addition, there are interesting subtleties that stem from the cohomology class of $\omega$. Note that the symplectic conditions
are rather strong -- not every manifold can be given a symplectic geometry.

Why do we care? What's the motivation for symplectic geometry? These ideas stem from the ideas of parameter spaces. We can think of our symplectic manifold
as a shape whose points correspond to configurations of a physical system. For example, take the earth-sun-moon system in Newtonian mechanics.
What's the configuration space for this system? One needs the position of each body, which gives us $\R^9$, but we also need the momentum of each body, which is
another $\R^9$. In total, then, our parameter space is $\R^{18}$. It turns out to be productive to think of this as the cotangent bundle $T^*\R^9$.
It turns out that this parameter space has a natural symplectic form: $\omega_{std}=dx_1\wedge dy_1+\cdots+dx_n\wedge dy_n$, where $(x,y)\in\R^9\times\R^9$.
We leave it as an exercise to show that this form is, in fact, symplectic. More importantly, the form $\omega_{std}$ is an irreplaceable part of the physics
of the parameter space $\R^{18}=T^*\R^9$.

This motivates some of the fundamental questions of symplectic geometry. For example, when are two symplectic manifolds equivalent? This will motivate
the definition of symplectic equivalence. What is the geometry of symplectic manifold ``like''?

People have been thinking about symplectic geometry since the 1960s. However, the basic geometry of symplectic manifolds was not well understood until
the mid-eighties. One of the important questions was: how different is symplectic geometry from volume-preserving geometry? For example, a symplectic form
induces a volume form, and so one might ask how much of the symplectic geometry is captured in this volume form? Let's look at a precise question.
Define two symplectic manifolds $(M_1,\omega_1),(M_2,\omega_2)$ to be \textbf{symplectomorphic} if there exists a $\phi:M_1\to M_2$ that is a
diffeomorphism with $\phi^*(\omega_2)=\omega_1$. We leave it as an exercise to show that a symplectomorphism preserves the volume form, i.e. the induced volume
form on $M_2$ pulls back to the induced volume form on $M_1$. Hence, a symplectomorphism is volume-preserving. But if we have a volume-preserving map,
is it symplectic? The answer is no, in general, and this problem was easy. Much more interesting was the question: given a volume-preserving map,
can we approximate it (arbitrarily closely) by symplectic maps? If the answer were yes, since volume-preserving geometry is relatively easy, we'd see that
symplectic geometry is really quite trivial. People worried about this for a while, until in 1985 when Gromov (and Eliashberg) proved what is now
known as Gromov non-squeezing. Gromov thought about $B^{2n}(R)=\left\{ x^2+y_1^2+\cdots+x_n^2+y_n^2\leq R \right\}\subset(\R^{2n},\omega_{std})$
and $Z(R)=\left\{ x_1^2+y_1^2\leq R \right\}\subset (\R^{2n},\omega_{std}$), i.e. a ball and an infinite cylinder. 

\begin{thm}[Gromov non-squeezing]
    Symplectically, $B^{2n}(r)$ embeds in $Z(R)$ if and only if $r\leq R$.
\end{thm}

The significance of this theorem is that $B(1.01)$ easily admits a volume-preserving embedding into $Z(1)$ (easy fact of volume-preserving geometry) since the only
invariant in the volume-preserving world is volume. This was the first example of what's called \textbf{symplectic rigidty}.
We're going to be thinking about examples of some ``flexibility'' -- we want to understand just how rigid symplectic geometry really is.

\section{WWHS}

Here's a rough syllabus of things we will be learning about (in no particular order):
\begin{itemize}
    \item $SL_2(\C)$ and representation theory
    \item Lie algebras
    \item Root systems
    \item Toric varieties
    \item Flag varieties
    \item Weyl character formula
    \item Highest weight theory
    \item Littlewood-Richardson Rule
    \item Borel-Weil-Bott
\end{itemize}

\subsection{Assignment 1}

\begin{enumerate}

    \item Take $x\in\fr g$ and $g\in G$. Now consider
        \begin{align*}
            \exp\left(gXg^{-1}t\right)&=1+gXg^{-1}t+(gXg^{-1})^2t^2/2+\cdots\\
            &=g\left(1+Xt+X^2t^2/2+\cdots\right)g^{-1}\\
            &=g\exp(Xt)g^{-1}.
        \end{align*}
        Hence, $gXg^{-1}\in\fr g$.

    \item Given a representation $r:G\to GL_n(\C)$, we can consider the differential at the identity,
        $dr:T_eG=\fr g\to T_eGL_n(\C)=M_{n\times n}(\C)$. We wish to show that $dr$ is a Lie algebra homomorphism.

    \item Define $\pi:\fr g\to\fr{gl}(V\otimes W)$ by
        \[\pi(X)(v\otimes w)=\pi_1(X)v\otimes w+v\otimes \pi_2(X)w.\]
        We wish to show that $\pi$ is a Lie algebra homomorphism, i.e. that $\pi([X,Y])=[\pi(X),\pi(Y)]$.
        The left-hand side yields
        \begin{align*}
            \pi([X,Y])(v\otimes w)&=\pi_1([X,Y])v\otimes w+v\otimes \pi_2([X,Y])\\
            &=[\pi_1(X),\pi_1(Y)]v\otimes w+v\otimes[\pi_2(X),\pi_2(Y)]w
        \end{align*}
        \textbf{FINISH****}

        Now note that given any $X\in\fr g$, we can recover the group element $e^{tX}\in G$. This element is sent by the Lie
        group homomorphism $\rho:G\to GL(V\otimes W)$ to an action
        \[e^{tX}(v\otimes w)=e^{tX}v\otimes e^{tX}w\]
        by definition of the representation of tensor products. Then, to descend back down to $\fr{gl}(V\otimes W)$, we simply take a derivative:
        \begin{align*}
            \frac{d}{dt}\left(e^{tX}v\otimes e^{tX}w\right)\bigg|_{t=0}&=\left(Xe^{tX}v\otimes e^{tX}w+e^{tX}v\otimes Xe^{tX}w\right)\bigg|_{t=0}\\
            &=Xv\otimes w+v\otimes Xw.
        \end{align*}
        But this action is precisely that carried out by $\pi$, and thus the given diagram commutes.

    \item Elementary computation for vector fields from differential geometry. 

\end{enumerate}


\end{document}
