\documentclass{../mathnotes}

\usepackage{enumerate}
\usepackage{todonotes}
\usepackage{tikz-cd}

\title{Commutative Algebra}
\author{Matei Ionita and Nilay Kumar}
\date{September 2013}

\begin{document}

\maketitle

\section*{Class 1}

\begin{defn}
Given a ring $R$, a \textbf{finite-type} $R$-algebra is any $R$-algebra $A$ which can be generated \textit{as an $R$-algebra} by finitely many elements over $R$. Equivalently, $A\cong R[x_1,\ldots, x_n]/I$.
\end{defn}

\begin{defn}
A ring map $\phi:A\to B$ is \textbf{finite} (or $B$ is finite over $A$) if there exist finitely many elements of $B$ that generate $B$ \textit{as an $A$-module}. Equivalently, there exists a surjective map $A^{\oplus n}\to B$ as $A$-modules.
\end{defn}

\begin{exmp}
Consider $A=k[x_1,x_2]/(x_1x_2-1)\cong k[t,t^{-1}]\subset k(t)$. The map $k[x_1,x_2]\twoheadrightarrow A$ is finite but not injective. On the other hand, $k[x_1]\to A$ is injective but not finite. The map $k[y]\overset{\phi}{\to}A$ given by $y\mapsto x_1+x_2$ works, as one can show.
\end{exmp}


\begin{thm*}[Noether Normalization]
Let $k$ be a field, and $A$ be a finite-type $k$-algebra. Then there exists an $r\geq0$ and a finite injective $k$-algebra map $k[y_1,\ldots,y_r]\to A$.
\end{thm*}

Before we prove the theorem let us state some useful lemmas.

\begin{lem}
\label{L1}
If $A\to B$ is a ring map such that $B$ is generated as an $A$-algebra by $x_1,\ldots,x_n\in B$ and each $x_i$ satisfies a monic equation
\[x_n^{d_n}+\phi(a_{n-1})x_{n-1}^{d_{n-1}}+\ldots+\phi(a_1)=0\]
over $A$, then $\phi$ is finite.
\end{lem}
\begin{proof}
The map $A^{\oplus d_1\ldots d_n}\to B$ given by
\[(a_{i_1},\ldots,a_{i_n})\mapsto\sum\phi(a_{i_1},\ldots,a_{i_n})x_1^{i_1}\cdots x_n^{i_n}\]
is surjective.
\end{proof}

\begin{defn}
Given a ring map $A\to B$ we say that an element $b\in B$ is \textbf{integral} over $A$ if there exists a monic $P(T)\in A[T]$ such that $P(b)=0$ in $B$.
\end{defn}

\begin{lem}[Horrible lemma]
\label{L2}
Suppose $f\in k[x_1,\ldots, x_n]$ is non-zero. Pick natural numbers $e_1\gg e_2\gg\ldots\gg e_{n-1}$. Then $f(x_1+x_n^{e_1},\ldots, x_{n-1}+x_n^{e_{n-1}}, x_n)$ is of the form $a x_n^N+\text{lower order terms}$, where $a\in k^\times$.
\end{lem}
\begin{proof}
Write $f=\sum_{I\in k}a_I x^I$ with $a_I\neq 0$ for all $I\in k$, where $k$ is a finite set of multi-indices. Substituting, we get something of the form
\[(x_1+x_n^{e_1})^{i_i}\cdots (x_{n-1}+x_n^{e_{n-1}})^{i_{n-1}}x_n^{i_n}=x_n^{i_1e_1+\ldots+i_{n-1}e_{n-1}+i_n}.\]
It suffices to show that if $I,I'\in k$, for distinct $I,I'$ we have that
\[i_1e_1+\ldots+i_{n-1}e_{n-1}+i_n\neq i'_1e_1+\ldots+ i'_{n-1}e_{n-1}+i'_n.\]
If $I$ is lexicographically larger than $I'$ then the left hand side is greater than the right hand side.
\end{proof}

\begin{lem}
\label{L3}
Suppose we have $A\to B\to C$ ring maps. If $A\to C$ is finite, then $B\to C$ is finite.
\end{lem}
\begin{proof}
Trivial.
\end{proof}

\begin{lem}
\label{L4}
Suppose we have $A\to B\to C$ ring maps. If $A\to B$ and $B\to C$ are finite, then $A\to C$ is finite as well.
\end{lem}
\begin{proof}
Trivial.
\end{proof}

We now have enough machinery to prove Noether normalization.

\begin{proof}
Let $A$ be as in the theorem. We write $A=k[x_1,\ldots, x_n]/I$. We proceed by induction on $n$. For $n=0$, we simply have $A=k$, and we can take the identity map $k\to A$, which is clearly finite and injective. Now suppose the statement holds for $n-1$, i.e. for algebras generated by $n-1$ or fewer elements. If the generators $x_1,\ldots,x_n$ are algebraically independent over $k$ (i.e. $I=0$), we are done and we may take $r=n$ and $y_i=x_i$. If not, pick a non-zero $f\in I$. For $e_1\gg e_2\gg\ldots\gg e_{n-1}\gg 1$, set
\[y_1=x_1-x_n^{e_1},\ldots y_{n-1}=x_{n-1}-x_n^{e_{n-1}},x_n=x_n,\]
and consider $f(x_1,\ldots, x_n)=f(y_1+x_n^{e_1},\ldots,y_{n-1}+x_n^{e_{n-1}},y_n)$. By  Lemma \ref{L2}, we see that this polynomial is monic in $x_n$ and hence, since $x_i$ are integral over $A$, we conclude (by Lemma \ref{L1}) that $A=k[x_1,\ldots,x_n]/I$ is finite over $B=k[y_1,\ldots, y_{n-1}]$. To show that $B\to A$ is injective, let $J =$ Ker$(B \to A)$ and replace $B$ by $B/J$. Now $B/J \to A$ is injective, and by Lemma \ref{L3} it is finite. But since $B/J$ is finite over $k[y_1,\ldots,y_r]$ by the induction hypothesis, $A$ must be as well (see Lemma \ref{L4}), and we are done.
\end{proof}


\section*{Class 2}

    Let $A$ be a ring. Then we define the \textbf{spectrum of} $A$, $\Spec A$, to be the set of prime ideals of $A$.
    Note that $\Spec(-)$ is a contravariant functor in the sense that if $\phi:A\to B$ is a ring map we get a map $\Spec(\phi):\Spec(B)\to\Spec(A)$ given by $q\mapsto \phi^{-1}(\fr q)$. In order for this to work we want $\fr q$ prime in $B \Leftrightarrow \phi^{-1}(\fr q)$ prime in $A$. Notice that $\fr q$ prime implies that $B/\fr q$ is a domain. But $\phi$ induces a homomorphism $A/\phi^{-1}(\fr q) \to B/\fr q$ and this homomorphism must preserve multiplication. In particular, if the product of two elements in $A/\phi^{-1}(\fr q)$ is 0, then so is the product of their images in $B/\fr q$. So $B/\fr q$ domain implies that $A/\phi^{-1}(\fr q)$ is a domain. Then $\phi^{-1}(\fr q)$ is prime. The converse can be proved in the same way.
    

\begin{rem}
    Abuse of notation: often we write $A\cap \fr q$ for $\phi^{-1}(\fr q)$ even if $\phi$ is not injective. Note also that $\Spec(-)$ is in fact a functor from $\mathbf{Ring}$ to $\mathbf{Top}$, though we will postpone discussion about topology until later.
\end{rem}

\begin{exmp}
    Consider $\Spec(\C[x])$. Since $\C[x]$ is a PID (and thus a UFD), the primes are principal ideals generated by irreducibles, i.e.
    linear terms. Hence $\Spec(\C[x])=\left\{ (0),(x-\lambda) | \lambda\in C \right\}$. Consider $\phi:\C[x]\to\C[y]$, given by $x\mapsto y^2$.
    Set $\fr q_\lambda=(y-\lambda)$ and $\fr p_\lambda=(x-\lambda)$. Then $\Spec(\phi)(\fr q_\lambda)=\fr p_{\lambda^2}$. Why is this? 
    First note that $\phi(\fr p_{\lambda^2})=(x^2-\lambda^2)=(x-\lambda)(x+\lambda)\subset\fr p_\lambda$, which gives us an inclusion.
    %We have that $x-\lambda^2\mapsto y^2-\lambda^2=(y+\lambda)(y-\lambda)\in q_\lambda$.
    Additionally, we have that $\Spec(\phi)\left( (0) \right)=(0)$. Since this is everything in $\Spec\C[y]$, we have equality.  Note that the fibres of $\Spec(\phi)$ are finite!
\end{exmp}

Indeed, the goal of the next couple lectures will be to show that the fibres of maps on spectra of a finite ring map are finite.

Let us start by considering the following setup. Let $\phi: A\to B$ be a ring map and $\fr p\subset A$ a prime ideal. What is the fibre of $\Spec(\phi)$ over $\fr p$? First of all, note that if $\phi^{-1}(\fr q)=\fr q\cap A=\fr p$, then $\fr p B=\phi(\phi^{-1}(\fr q))B\subset\fr q$.

\begin{lem}
\label{L5}
    If $I\subset A$ is an ideal in a ring $A$ then the ring map $A\to A/I$ induces via $\Spec(-)$ a bijection $\Spec(A/I)\leftrightarrow V(I)=\left\{ \fr p\in\Spec(A) | I\subset \fr p \right\}$.
\end{lem}
\begin{proof}
    We use the fact that the ideals of $A/I$ are in 1-to-1 correspondence with ideals of $A$ containing $I$. We wish to extend this to prime ideals. By the third isomorphism theorem, given $J\subset I\subset A$, we have that $A/I\cong(A/J)/(I/J)$. We see that $A/I$ is a domain iff $I/J$ is prime in $A/J$ iff $I$ is prime in $A$; this gives us the 1-to-1 correspondence.
\end{proof}

%\begin{rem}
    %The \textbf{Zariski topology} has as closed subsets the sets $V(I)$.
%\end{rem}
\begin{rem}
Consider next the following two diagrams. 
\[
\begin{tikzcd}
B\arrow{r} & B/\fr pB\\
A\arrow{u}{\phi}\arrow{r} & A/\fr p\arrow[swap]{u}{\bar\phi}
\end{tikzcd}
\longleftrightarrow
\begin{tikzcd}
\Spec B\arrow[swap]{d}{\Spec(\phi)} & \arrow{l}\arrow{d}{\Spec(\bar\phi)}\Spec(B/\fr pB)\\
\Spec A & \arrow{l}\Spec(A/\fr p)
\end{tikzcd}
\]
Clearly the point $\fr p \in$ Spec $A$ corresponds to $(0) \in$ Spec$(A/\fr p)$. Thus, by Lemma \ref{L5}, points in the fibre of $\Spec(\phi)$ over $\fr p$ are in 1-1 correspondence with points in the fibre of $\Spec(\bar \phi)$ over $(0)\in\Spec(A/\fr p)$. This fact will be very important for our proofs later on.
\end{rem}

\begin{lem}
\label{L6}
    If $k$ is a field, then $\Spec(k)$ has exactly one point. If $k$ is the fraction field of a domain $A$, then $\Spec(k)\to\Spec(A)$
    maps the unique point to $(0)\in\Spec(A)$.
\end{lem}
\begin{proof}
    The only ideals of a field $k$ are $(0)$ and $k$ itself. The sole prime ideals is thus $(0)$ and hence $\Spec(k)$ has only one point.
    If $k$ is the fraction field of the domain $A$ then we have an injective map $A\to k$ which clearly pulls $(0)\subset k$ back to $(0)\subset A$. 
\end{proof}

Next we wish to invert some elements in $B/\fr p B$. More specifically, since we are interested in the ideals of $B/\fr p B$ that are mapped to $(0)$ by Spec$(\bar \phi)$, we would like to 'throw out' the other ones. We do this by creating inverses for elements of $A/\fr p - \{0\}$, such that none of them will be primes anymore. (See example \ref{E3} below for how this works.) This leads to a very general notion of localization, which we discuss in detail for the rest of the lecture.

\begin{defn}
    Let $A$ be a ring. A \textbf{multiplicative subset} of $A$ is a subset $S\subset A$ such that $1\in S$ and if $a,b\in S$,
    then $ab\in S$.
\end{defn}

\begin{defn}
    Given a multiplicative subset $S$, we can define the \textbf{localization of $A$ with respect to $S$}, $S^{-1}A$, as the set of pairs $(a,s)$ with $a\in A,s\in S$ modulo the equivalence relation
    $(a,s)\sim(a',s')\iff \exists s''\in S$ such that $s''(as'-a's)=0$ in $A$. Elements of $S^{-1}A$ are denoted $\frac{a}{s}$. Addition proceeds
    as usual. One checks that this is indeed a ring.
\end{defn}

\begin{lem}
\label{L7}
    The ring map $A\to S^{-1}A$ given by $a\mapsto \frac{a}{1}$ induces a bijection $\Spec(S^{-1}A)\leftrightarrow\left\{ \fr p\subset A|S\cap \fr p=\varnothing \right\}$.
\end{lem}
\begin{proof}
It's easy to show that $\Spec(\phi) (S^{-1}(A)) \subset \{ \fr p\subset A|S\cap \fr p=\varnothing \}$. Let $\fr q$ be a prime in $S^{-1}A$; if $\phi^{-1}$ contains some $s\in S$, then $\phi(s) \in \fr q$. But $\phi(s)$ is a unit, so $\fr q = S^{-1}A$. For the converse, \todo{how did this work again?}
\end{proof}
Note that any element of $S$ becomes invertible in $S^{-1}A$ so it is not in any prime ideal of $S^{-1}A$. 

\begin{exmp}
\label{E3}
    Suppose $A=\C[x]\to B=\C[y]$ with $x\mapsto 5y^2+3y+2$. Then $\Spec(\phi)^{-1}\left( (x) \right)=\Spec\left( (A/\fr p-\left\{ 0 \right\})^{-1} B/\fr p B \right)=\Spec\left( (\C^\times)^{-1}\C[y]/(5y^2+3y+2) \right)=\Spec\left(\C[y]/(5y^2+3y+2) \right)$. There are two points in this space, since this quadratic factors into two prime ideals containing the ideal generated by this quadratic (see Lemma \ref{L5}). More generally, one may refer to the following diagram, which will be very useful next lecture.
\end{exmp}

\[
\begin{tikzcd}
B\arrow{r}&B/\fr pB\arrow{r} & \bar\phi\left(A/\fr p-\{0\}\right)^{-1}B/\fr pB\\
A\arrow{r}\arrow{u} & A/\fr p\arrow{r}\arrow{u}&\text{Fr}(A/\fr p)=\left(A/\fr p-\{0\}\right)^{-1}A/\fr p\arrow{u}
\end{tikzcd}
\]

Given $S\subset A$ multiplicative, and an $A$-module $M$, we can form an $S^{-1}A$-module
\[S^{-1}M=\left\{ \frac{m}{s}|m\in M,s\in S \right\}/\sim\]
where the equivalence relation is the same as before. The construction $M\to S^{-1}M$ is a functor $\mathbf{Mod}_A\to\mathbf{Mod}_{S^{-1}A}$.

\begin{lem}
\label{L8}
    The localization functor $M\to S^{-1}M$ is exact.
\end{lem}
\begin{proof}
    Suppose the sequence $0\to M'\overset{\alpha}{\to} M\overset{\beta}{\to} M''\to 0$ is exact. We wish to show that the sequence $0\to S^{-1}M'\overset{S^{-1}\alpha}{\to} S^{-1}M\overset{S^{-1}\beta}{\to} S^{-1}M''\to 0$ is exact. Let us first show that this sequence is exact at $S^{-1}M$, i.e. that $\text{Im} (S^{-1}\alpha)=\ker(S^{-1}\beta)$. Pick $m'/s\in S^{-1}M'$. We take $S^{-1}\alpha(m'/s)=\alpha(m')/s$ and then compute $S^{-1}\beta(\alpha(m')/s)=\beta(\alpha(m'))/s=0$ by the given exactness. This shows the inclusion $\text{Im} (S^{-1}\alpha)\subset\ker(S^{-1}\beta)$. Next, choose an element $m/s\in \ker(S^{-1}\beta).$ Then $\beta(m)/s=0$ in $S^{-1}M''$, i.e. there exists a $t\in S$ such that $t\beta(m)=0$ in $M''$. Since $\beta$ is a $A$-module homomorphism, $t\beta(m)=\beta(tm)$ and so $tm\in\ker(\beta)=\text{Im}(\alpha)$. Therefore $tm=\alpha(m')$ for some $m'\in M'$. Hence we have $m/s=\alpha(m')/st=(S^{-1}\alpha)(m'/st)\in\text{Im}(S^{-1}\alpha)$, which demonstrates the reverse inclusion.
    
    The rest of the proof is left as a exercise.
\end{proof}

\begin{rem}
An exact functor is one that preserves quotients. What Lemma \ref{L8} says is that if $N\subset M$ then $S^{-1}M/S^{-1}N \cong S^{-1}(M/N)$. In particular, if $I\subset A$ is an ideal, then $S^{-1}(A/I) = S^{-1}A/S^{-1}I$.
\end{rem}

\begin{rem}
If $A \overset{\phi}{\to} B$, then $S^{-1}B$ is an $S^{-1}A$-algebra and $S^{-1}B \cong \big(\phi(S)\big)^{-1} B$.
\end{rem}

\begin{defn}
Let $A$ be a ring and $\fr p\subset A$ be a prime ideal, then $A_{\fr p} = (A-\fr p)^{-1}A$ is the \textbf{local ring of $A$ at $\fr p$} (or the localization of $A$ at $\fr p$). If $M$ is an $A$-module, then we set $M_{\fr p} = (A-\fr{p})^{-1}M$.
\end{defn}

\begin{defn}
A \textbf{local ring} is a ring with a unique maximal ideal.
\end{defn}

\begin{lem}
\label{L9}
$A_{\fr{p}}$ is a local ring.
\end{lem}
\begin{proof}
Consider the quotient $A_\fr p/\fr pA_\fr p$.  By the remark above, we 
can factor $A_{\fr{p}} / \fr{p}A_{\fr{p}} = (A - \fr{p})^{-1} (A/\fr{p})$. This is justified because $A_\fr p=(A-\fr p)^{-1}A$ by definition and because $\fr pA_\fr p=(A-\fr p)^{-1}\fr p$ for some reason. Next, by the remark directly above, if we let $\phi:A\to A/\fr p$ be the natural surjection, then $(A-\fr p)^{-1}(A/\fr p)=(\phi(A-\fr p))^{-1}(A/\fr p)=(A/\fr p-\{0\})^{-1}(A/\fr p)$. But this is just the fraction field of $A/\fr p$, i.e. $A_\fr p/\fr pA_\fr p$ is a field. Hence $\fr pA_\fr p$ is maximal.

This is the unique maximal ideal because by Lemma \ref{L7}, the primes of $A_{\fr p}$ are the primes $\fr{q}\subset A$ that do not intersect $A-\fr{p}$. This implies that $q\subset p$, and thus $\fr q$ cannot be maximal unless $\fr q = \fr p$.
\end{proof}




\section*{Class 3}

\begin{lem}
\label{L10}
\begin{enumerate}[(a)]
Let $A \overset{\phi}{\to} B$ be a finite ring map. Then
\item for $I \subset A$ ideal, the ring map $A/I \to B/IB$ is finite;
\item for $S\subset A$ multiplicative subset, $S^{-1}A \to S^{-1}B$ is finite;
\item for $A\to A'$ ring map, $A' \to B \otimes_A A'$ is finite.
\end{enumerate}
\end{lem}
\begin{proof}
\begin{enumerate}[(a)]
\item Consider the following diagram:
\[
\begin{tikzcd}
B\arrow{r} & B/IB\\
A\arrow{u}\arrow{r} & A/I\arrow{u}
\end{tikzcd}
\]
By Lemma $\ref{L4}$ we see that if the map $B\to B/IB$ is finite, then so is $A\to B/IB$, which would imply that (by Lemma $\ref{L3}$) $A/I\to B/IB$ is finite. But $B\to B/IB$ is obviously finite, as it is generated as a $B$-module by $\{1\}$.
\item Since $A\to B$ is finite, there exists a surjection $A^{\oplus n}\twoheadrightarrow B$. The statement that $S^{-1}A\to S^{-1}B$ is finite follows immediately from the fact that localization is exact and hence preserves surjectivity of $(S^{-1}A)^{\oplus n}\twoheadrightarrow S^{-1}B$.
\item We haven't yet discussed tensor products, so we will leave this for now.
\end{enumerate}
\end{proof}

\begin{lem}
\label{L11}
Suppose $k$ is a field, $A$ is a domain and $k \to A$ a finite ring map. Then $A$ is a field.
\end{lem}
\begin{proof}
Since $A$ is an algebra, multiplication by an element $a\in A$ defines a $k$-linear map $A\to A$. The map is also injective: Ker$(a) = \{ a' \in A | aa' = 0 \} = \{ 0 \}$, because $A$ has no zero divisors. But, since dim$_k (A)$ is finite, injectivity implies surjectivity. Then there exists $a''$ such that $aa'' = 1$, so $a$ is a unit.
\end{proof}

\begin{lem}
\label{L12}
Let $k$ be a field and $k\to A$ a finite ring map. Then:
\begin{enumerate}[(a)]
\item Spec$(A)$ is finite.
\item there are no inclusions among prime ideals of $A$.
\end{enumerate}
In other words, $\Spec(A)$ is a finite discrete topological space with respect to the Zariski topology.
\end{lem}
\begin{proof}
For some $\fr p$ prime in $A$, $A/\fr p$ is a domain and the natural map $k\to A/\fr p$ is finite since $k\to A$ and $A\to A/\fr p$ are both finite. 
By Lemma $\ref{L11}$ we see that $A/\fr p$ must be a field, and that $\fr p$ must be maximal. Hence all primes of $A$ are maximal. This shows $(b)$, as there can be no inclusions among maximal ideals. Moreover, by the Chinese remainder theorem (see Lemma $\ref{L13}$ below) the map $A \to A/\fr m_1 \times \ldots \times A/\fr m_n$ is surjective.
Since $A$ and its quotients are vector spaces, this translates into a statement about their dimension: $\dim_k A \geq \sum_i \dim_k A/\fr m_i \geq n$. Thus $n$ is finite, which shows $(a)$.
\end{proof}

\begin{lem} [Chinese remainder theorem]
\label{L13}
Let $A$ be a ring, and $I_1, ... , I_n$ ideals of $A$ such that $I_i + I_j = A , \forall i\neq j$. Then there exists a surjective ring map $A \twoheadrightarrow A/I_1 \times ... \times A/I_n$ with kernel $I_1 \cap ... \cap I_n = I_1 ... I_n$.
\end{lem}
\begin{proof}
Omitted.
\end{proof}

\begin{lem}
\label{L14}
Let $A \overset{\phi}{\to} B$ be a finite ring map. The fibres of Spec$(\phi)$ are finite.
\end{lem}
\begin{proof}
Consider the following diagram:
\[
\begin{tikzcd}
B\arrow{r}&B/\fr pB\arrow{r} & B_\fr p/\fr pB_\fr p=\bar\phi\left(A/\fr p-\{0\}\right)^{-1}B/\fr pB\\
A\arrow{r}\arrow{u}{\phi} & A/\fr p\arrow{r}\arrow{u}{\bar\phi}&\text{Fr}(A/\fr p)=\left(A/\fr p-\{0\}\right)^{-1}A/\fr p\arrow{u}
\end{tikzcd}
\]
By $(a)$ and $(b)$ of Lemma \ref{L10}, $\bar\phi$ and $\text{Fr}(A/\fr p)\to B_\fr p/\fr pB_\fr p$ are finite. Now recall that the points in the fibre of $\Spec(\phi)$ over $\fr p\in\Spec(A)$ correspond to points in the fibre of $\Spec(\bar \phi)$ over $(0)\in\Spec(A/\fr p)$. If we now look at the third column of the diagram, we see that since $\text{Fr}(A/\fr p)$ is a field, Lemma \ref{L12} implies that $\Spec(B_\fr p/\fr pB_\fr p)$ is finite. Hence there must be a finite number of points in $\Spec(B_\fr p/\fr pB_\fr p)$ that map to $(0)\in\Spec(\text{Fr}(A/\fr p))$, and thus (again arguing via correspondence), the points in the fibre of $\Spec(\phi)$ over $\fr p\in\Spec(A)$ must be finite.
\end{proof}

\begin{lem}
\label{L15}
Suppose that $A\subset B$ is a finite extension (i.e. there exists a finite injective map $A\to B$). Then Spec$(B) \to $ Spec$(A)$ is surjective.
\end{lem}
\begin{proof}
We want to reduce the problem to the case where $A$ is a local ring. For this, let $p\subset A$ be a prime. By part b of Lemma \ref{L10}, the map $A_p \to B_p$ is finite. By Lemma \ref{L8}, the same map is injective. Then we can replace $A$ and $B$ in the statement of the lemma by $A_p$ and $B_p$.

Now, assuming that $A$ is local, $p$ is the maximal ideal of $A$, and we denote it by $m$ in what follows. The following statements are equivalent:
\begin{align*}
 \exists q \subset B \text{ lying over } m &\Leftrightarrow \exists q\subset B \text{ such that } mB \subset q
 \\   & \Leftrightarrow B/mB \neq 0
\end{align*}
But the last statement is always true, since Nakayama's lemma (see below) says that $mB = B$ implies $B = 0$.
\end{proof}

\begin{lem} [Nakayama's lemma]
\label{L16}
Let $A$ be a local ring with maximal ideal $m$, and let $M$ be a finite $A$-module such that $M = mM$. Then $M=0$.
\end{lem}
\begin{proof}
Let $x_1, ... , x_r \in M$ be generators of $M$. Since $M = mM$ we can write $x_i = \sum_{j=1}^r a_{ij} x_j$, for some $a_{ij} \in m$. Then define the $r\times r$ matrix $B = 1_{r\times r} - (a_{ij})$. The above relation for the generators translates into:
\[     B \left(  \begin{array} {c} x_1 \\ \vdots \\ x_r  \end{array}  \right)  = 0   \]
Now consider $B^{\text{ad}}$, the matrix such that $B^{\text{ad}} B =$ det$(B) 1_{r\times r}$. Multiplying the above equation on the left by $B^{\text{ad}}$ we obtain:
\[    \text{det}(B) \left(  \begin{array} {c} x_1 \\ \vdots \\ x_r  \end{array}  \right)  = 0   \]
Thus det$(B) x_i = 0$ for all $i$. If we assume that the generators of $M$ are nonzero, the fact that det$(B)$ annihilates all generators implies that it is equal to 0. But, by expanding out the determinant of $B = 1_{r\times r} - (a_{ij})$, we see that it is of the form $1+a$ for some $a\in m$. Since $(A,m)$ is a local ring, this implies that det$(a)$ is a unit. A unit cannot be zero in $(A, m)$, so this is a contradiction. Thus all generators of $M$ are zero, and $M=0$.
\end{proof}

\begin{lem} [Going up for finite ring maps]
\label{L17}
Let $A \to B$ be a finite ring map, $p$ a prime ideal in $A$ and $q$ a prime ideal in $B$ which belongs to the fibre of $p$. If there exists a prime $p'$ such that $p \subset p' \subset A$, then there exists a prime $q'$ such that $q\subset q' \subset B$ and $q'$ belongs to the fibre of $p'$.
\[
\begin{tikzcd}
B & & q\arrow{d}\arrow[dashed, hook]{r} & ?\arrow[dashed]{d} \\
A\arrow{u} &   & p\arrow[hook]{r} & p'
\end{tikzcd}
\]
\end{lem}
\begin{proof}
Consider $A/p \to B/q$. This is injective since $p = A \cap q$ and finite by Lemma \ref{L3}. $p'/p$ is a prime ideal in $A/p$, and by Lemma \ref{L15} its preimage is nonempty. Thus there exists a prime $q'/q$ in $A/p$ which maps to $p'/p$, and this corresponds to a prime $q'$ in $B$ that contains $q$.
\end{proof}



\section*{Class 4}

\begin{lem}
\label{L18}
The following are equivalent for a ring $A$:
\begin{enumerate}[(1)]
\item $A$ is local;
\item Spec$(A)$ has a unique closed point;
\item $A$ has a maximal ideal $m$ such that every element of $A-m$ is invertible;
\item $A$ is not zero and $x\in A \Rightarrow x\in A^*$ or $1-x \in A^*$.
\end{enumerate}
\end{lem}

\begin{proof}
$(1) \Leftrightarrow (2)$ In the Zariski topology for Spec$(A)$, a closed set looks like $V(\fr p)$ for some prime $\fr p$. Therefore a closed point is a maximal ideal.
\\
\\
$(1) \Rightarrow (3)$ Let $m\subset A$ be the maximal ideal and take $x\not \in m$. Then $V(x) = \emptyset$, and, by Lemma \ref{L19}, $x$ is invertible.
\\
\\
$(3) \Rightarrow (4)$ If $x\not \in m$ then $x$ is invertible, so assume $x\in m$. But then $1-x \not \in m$, since this would imply $1\in m$. Therefore $1-x$ is invertible.
\\
\\
$(4) \Rightarrow (1)$ Let $m = A - A^*$. It's easy to show that $m$ is an ideal. Moreover, $m$ is maximal: assume $m\subset I$ and $m\neq I$, then $I$ must contain a unit, and so $I = A$. There can be no other maximal ideal, since all elements of $A-m$ are units.
\end{proof}


\begin{lem}
\label{L19}
For $x\in A$, $A$ local, $V(x) = \emptyset \Leftrightarrow x\in A^*$.
\end{lem}
\begin{proof}
The $\Leftarrow$ direction is trivial. For the converse, note that by Lemma \ref{L7}:
\[ V(x) = \emptyset \Rightarrow \Spec(A/xA) = \emptyset \Leftrightarrow A/xA = 0 \Leftrightarrow x \text{ unit} \]
\end{proof}

\begin{exmp}
\label{E4}
Examples of local rings:
\begin{enumerate} [(a)]
\item fields, the maximal ideal is $(0)$.
\item $\C[[z]]$, power series ring, the maximal ideal is $(z)$. Note that something of the form $z-\lambda$ is invertible by some power series, and thus cannot be maximal.
\item for $X$ topological space and $x\in X$, $O_{X,x}$, the ring of germs of continuous $\C$-valued functions at $x$. The maximal ideal is $m_x = \{(U,f)\in O_{X,x} | f(x) = 0 \}$. Note that, if $g\not \in m_x$, then $g\neq 0$ on a neighborhood of $x$, because of continuity. Therefore $g$ is invertible on this neighborhood. Then, by Lemma \ref{L18}, $m_x$ is maximal.
\item for $k$ a field, $k[x]/(x^n)$, the maximal ideal is $(x)/(x^n)$.
\end{enumerate}
\end{exmp}

For the rest of the lecture, we examine the closedness of maps on spectra.
\begin{defn}
Let $X$ be a topological space, $x,y\in X$. We say that \textbf{$x$ specializes to $y$} or \textbf{$y$ is a generalization of $x$} if $y\in \overline{\{x\}}$. We denote this as $x\leadsto y$.
\end{defn}

\begin{exmp}
\label{E5}
In Spec$\Z$ we have $(0) \leadsto (p)$ for all primes $p$, but not $(p) \leadsto (0)$ or $(p) \leadsto (q)$, unless $p = q$.
\end{exmp}

\begin{lem}
\label{L20}
The closure of $\fr p$ in Spec$(A)$ is $V(\fr p)$. In particular, $\fr p \leadsto \fr q$ iff $\fr p \subset \fr q$.
\end{lem}
\begin{proof}
\[       \overline{\{\fr p\}} = \bigcap_{I\subset \fr p} V(I) = V(\bigcap_{I \subset \fr p} I) = V(\fr p)      \]
\end{proof}

\begin{lem}
\label{L21}
The image of Spec$(A_{\fr p}) \to \Spec(A)$ is the set of all generators of $\fr p$.
\end{lem}
\begin{proof}
By Lemma \ref{L7}, there is a bijection between primes of $A_{\fr p}$ and primes of $A$ contained in $\fr p$. But the latter are all ideals generated by a subset of the generators of $\fr p$, and in particular the generators themselves.
\end{proof}

\begin{defn}
A subset $T$ of a topological space is \textbf{closed under specialization} if $x\in T$ and $x \leadsto y$ imply $y\in T$.
\end{defn}

Notation: for $f\in A$, let $D(f) = \Spec(A) - (f) = \{\fr q \in A | f\not \in \fr q\}$. Obviously $D(f)$ is open.

\begin{lem}
\label{L22}
Let $A \to B$ be a ring map. Set $T = \text{Im} \big(\Spec(B) \to \Spec(A) \big)$. If $T$ is closed under specialization then $T$ is closed.
\end{lem}
\begin{proof}
Suppose $\fr p\in\bar T$. Then every open neighborhood of $\fr p$ contains a point of $T$. Now pick $f\in A\setminus{\fr p}$. Then $D(f)\subset\Spec A$ is an open neighborhood of $\fr p$. Then there exists a $\fr q\subset B$ with $\Spec(\phi)(\fr q)\in D(f)$, which implies that ther exists a $\fr q\subset B$ such that $\phi(f)\neq\fr q$. Hence $B_f\neq 0$.

Thus we see $\phi(f)\cdot 1\neq 0$ for all $f\in A\setminus\fr p$. Hence $B_\fr p\neq 0$ ($1\neq 0$) and thus $\Spec(B_\fr p)\neq\varnothing$. We conclude (Lemma \ref{L21}) that there exists a $\fr q'\subset B$ such that $\fr p'=\phi^{-1}(\fr q')\in T$ is a generalization of $\fr p$, i.e. $\fr p$ is a specialization of a point of $T$, and we conclude that $\fr p\in T$.
\todo{understand this}
\end{proof}

\begin{lem}
\label{L23}
If going up holds from $A$ to $B$, then $\Spec(\phi)$ is closed as a map of topological spaces.
\end{lem}
\begin{proof}
Let $Z\subset \Spec(B)$ be a closed subset; we want to show that its image is closed. In the Zariski topology closed sets look like $V(J)$ for some prime $J$, and by Lemma \ref{L7} we have $Z = \text{Im} \big(\Spec(B/J) \to \Spec(B) \big)$. Then:
\[
\begin{tikzcd}
\; & \Spec(B/J)\arrow{dl}\arrow{dr}{\cong} & \\
\Spec(B)\arrow{d}[swap]{\Spec(\phi)} & & Z\arrow{d}{\Spec(\phi)}\arrow[hook]{ll} \\
\Spec(A) & & \Spec(\phi)(Z) \arrow[hook]{ll}
\end{tikzcd}
\]
Note that $\Spec(\phi)(Z) = \text{Im}\big(\Spec(B/J) \to \Spec(A)\big)$. By Lemma \ref{L22} it suffices to show that $\Spec(\phi)(Z)$ is closed under specialization. That is, if there exists some prime $\fr p' \subset A$ which specializes to another prime $\fr p$ and is the image of a prime $\fr q' \subset B$, then $\fr p$ is also the image of some prime $\fr q \subset B$. Suppose we have the solid part of the diagram; by going up we can find $\fr q$ fitting into the diagram below, therefore $p\in \Spec(\phi)(Z)$ as long as $p' \in \Spec(\phi)(Z)$.
\[
\begin{tikzcd}
B & & J\arrow[hook]{r} & q'\arrow{d}\arrow[dashed, hook]{r} & ?\arrow[dashed]{d} \\
A\arrow{u} & &   & p'\arrow[hook]{r} & p
\end{tikzcd}
\]
\todo{understand this}
\end{proof}


\section*{Class 5: Krull dimension}

\begin{defn}\hspace{1mm}
\begin{enumerate}[(a)]
\item A topological space $X$ is \textbf{reducible} if it can be written as the union $X=Z_1\cup Z_2$ of two closed, proper subsets $Z_i$ of $X$. A topological space is \textbf{irreducible} if it is not reducible.
\item A subset $T\subset X$ is called \textbf{irreducible} iff $T$ is irreducible as a topological space with the induced topology.
\item An \textbf{irreducible component} of $X$ is a maximal irreducible subset of $X$.
\end{enumerate}
\end{defn}

\begin{exmp}\hspace{1mm}
\begin{enumerate}[(a)]
\item In $\R^n$ with the usual topology, the only irreducible subsets are the singletons. This is true in general for any Hausdorff topological space.
\item $\Spec\Z$ is irreducible.
\item If $A$ is a domain, then $\Spec A$ is irreducible. This is because $(0)\in V(I)\iff I\subset(0)\iff I\subset(0)\iff I=(0)\implies V(I)=\Spec A$.
\item $\Spec(k[x,y]/(xy))$ is reducible because it is $V(x)\cup V(y)$: geometrically speaking, the coordinate axes
\end{enumerate}
\end{exmp}

\begin{lem}
\label{L24}
Let $X$ be a topological space.
\begin{enumerate}[(a)]
\item If $T\subset X$ is irreducible so is $\bar T\subset X$;
\item An irreducible component of $X$ is closed;
\item $X$ is the union of its irreducible components, i.e. $X=\cup_{i\in I}Z_i$ where $Z_i\subset X$ are closed and irreducible with no inclusions among them. 
\end{enumerate}
\end{lem}
\begin{proof}
Omitted.
\end{proof}

\begin{lem}
\label{L25}
Let $X=\Spec A$ where $A$ is a ring. Then,
\begin{enumerate}[(a)]
\item $V(I)$ is irreducible if and only if $\sqrt{I}$ is a prime;
\item Any closed irreducible subset of $X$ is of the form $V(\fr p)$, $\fr p$ a prime;
\item Irreducible components of $X$ are in one-to-one correspondence with the minimal primes of $A$
\end{enumerate}
\end{lem}

\begin{proof}\hspace{1mm}
\begin{enumerate}[(a)]
\item $V(I)=\{\fr p:I\subset\fr p\}=V(\sqrt{I})$ so we may replace $I$ by $\sqrt{I}$. For the backwards direction, let $I$ be a prime. Then $A/I$ is a domain, so $\Spec(A/I)=V(I)$ by a previous lemma (this is true both as sets and topologies), which is irreducible by the example $(c)$ above. Conversely, if $V(I)$ is irreducible and $ab\in I$, then
\[V(I)=V(I,a)\cup V(I,b).\]
By irreduciblity we have that $V(I)=V(I,a)$ or $V(I)=V(I,b)$. This implies that either $a\in I$ or $b\in I$ by Lemma \ref{L26} below.
\item Omitted.
\item Omitted.
\end{enumerate}
\end{proof}

\begin{lem}
\label{L26}
$\sqrt{I}=\cap_{I\subset\fr p} \fr p$
\end{lem}
\begin{proof}
That the left-hand side is included in the right-hand side is clear. Conversely, suppose $f$ is contained in the right-hand side. Then $\Spec((A/I)_f)=\varnothing$ and hence $(A/I)_f=0$ as a ring. This implies that $f^n\cdot 1=0$ in $A/I$, and hence that $f^n\in I$.
\end{proof}

\begin{defn}
Let $X$ be a topological space. We set
\[\dim X=\sup\left\{ n | \exists Z_0 \subsetneq Z_1\subsetneq \cdots\subsetneq Z_n\subset X \right\}\]
with $Z_i\subset X$ irreducible and closed. We call $\dim X$ the \textbf{Krull} or \textbf{combinatorial} dimension of $X$. Furthermore, for $x\in X$ and for $U\ni x$ open subsets of $X$, we set
\[\dim_x X=\min_U\dim U,\]
which is called the \textbf{dimension of $X$ at x}.
\end{defn}

\begin{lem}
\label{L27}
Let $A$ be a ring. The dimension of $\Spec A$ is
\[\dim\Spec A=\sup\{n | \exists \fr p_0\subsetneq\fr p_1\subsetneq\cdots\subsetneq p_n\subset A \},\]
(for $\fr p_i$ primes) and is called the dimension of $A$.
\end{lem}
\begin{proof}
Clear from Lemma \ref{L25}.
\end{proof}

\begin{lem}
\label{L28}
Let $A$ be a ring. Then
\[\dim A=\sup_{\fr p\subset A}\dim A_\fr p=\sup_{\fr m\subset A}\dim A_\fr m.\]
\end{lem}

\begin{defn}
If $\fr p\subset A$ is prime, then the \textbf{height} of $\fr p$ is
\[\text{ht}(\fr p)=\dim A_\fr p.\]
Informally, one might think of this as the ``codimension'' of $V(p)$ in $\Spec A$.
\end{defn}

\begin{exc}
If $\fr p\subset A$ is a prime, then $\fr p$ is a minimal prime if and only if $\text{ht}(\fr p)=0$.
\end{exc}

Let us now prove the lemma.
\begin{proof}
Any chain of primes in $A$ has a last one. If we consider
\[\fr p_0\subsetneq\cdots\subsetneq\fr p_n\]
we can localize to get the chain
\[\fr p_0A_{\fr p_n}\subsetneq\cdots\subsetneq\fr p_nA_{\fr p_n}\]
in $A_{\fr p_n}$.
\end{proof}


\begin{lem}
\label{L29}
Let $A\overset{\phi}{\to}B$ be a finite ring map such that $\Spec \phi$ is surjective. Then $\dim A=\dim B$.
\end{lem}
\begin{proof}
By our description of fibres of $\Spec(\phi)$ in the proofs of Lemma \ref{L12} and \ref{L14}, there are no strict inclusions among primes in a fibre. If we take the chain
\[\fr q_0\subsetneq\fr q_1\subsetneq\cdots\fr q_n\]
in $B$ then $A\cap\fr q_0\subsetneq\cdots\subsetneq A\cap\fr q_n$ is a chain in $A$. Hence $\dim B\leq\dim A$. On the other hand, let $\fr p_0\subsetneq\cdots\subsetneq\fr p_n$ be a chain of primes in $A$. Pick $\fr q_0$ lying over $\fr p_0$ in $B$ (since $\Spec(\phi)$ is surjective). We can now use going up to succesively pick $\fr q_0\subset\fr q_1\subset\cdots\subset\fr q_n$ lying over $\fr p_1\subset\cdots\subset\fr p_n$ (a previous lemma showed that going up holds for finite ring maps). We conclude that $\dim B\geq\dim A$.
\end{proof}

\begin{rem}
There are a few remarks to be made here:
\begin{enumerate}[(a)]
\item The proof shows that if $A\overset{\phi}{\to}B$ has going up and $\Spec(\phi)$ is surjective, then $\dim A=\dim B$. The same statement holds for going down in place of going up.
\item By Noether normalization together with Lemma \ref{L29}, we can conclude that the dimension of a finite-type algebra over a field $k$ is equal to the dimension of $k[t_1,\ldots, t_r]$ for some $r$.
\item It will turn out that $\dim k[t_1,\ldots, t_r]=r$. For now all we can say is that it is certainly greater than $r$ because we can construct the chain
\[(0)\subset(t_1)\subset(t_1,t_2)\subset\ldots\subset(t_1,\ldots,t_r).\]
\end{enumerate}
\end{rem}


Now we talk for a bit about dimension 0 rings.

\begin{defn}
An ideal $I\subset A$ is \textbf{nilpotent} if there exists $n\geq 1$ such that $I^n = 0$. It is \textbf{locally nilpotent} if $\forall x\in I, \exists n\geq 1$ such that $x^n=0$.
\end{defn}

\begin{lem}
\label{L30}
For $\fr p \subset A$ prime, the following are equivalent:
\begin{enumerate} [(a)]
\item $\fr p$ minimal
\item ht$(\fr p) = 0$
\item the maximal ideal $\fr pA_{\fr p}$ of $A_{\fr p}$ is locally nilpotent
\end{enumerate}
\end{lem}

\begin{proof}
(a)$\Leftrightarrow $(b) follows from the description of Spec$(A_{\fr p})$ in Lemma \ref{L21}. The rest follows from Lemma \ref{L31}, stated below.
\end{proof}

\begin{lem}
\label{L31}
If $(A, m)$ is local, the following are equivalent:
\begin{enumerate} [(a)]
\item dim$(A) = 0$
\item Spec$(A) = \{m\}$
\item $m$ is locally nilpotent
\end{enumerate}
\end{lem}

\begin{proof}
(b) $\Rightarrow$ (c) If $f\in m$ is not nilpotent, then $A_f \neq 0$, so Spec$(A_f) \neq 0$, so $\exists \fr p\subset A, f \not \in \fr p$, which is a contradiction; hence $\fr p = m$.
\end{proof}


\begin{defn}
A ring is \textbf{Noetherian} if every ideal is finitely generated.
\end{defn}

\begin{lem}
\label{L32}
Let $I\subset A$ be an ideal. If $I$ is locally nilpotent and finitely generated, then $I$ is nilpotent. In particular, if $A$ is Noetherian then all locally nilpotent ideals are nilpotent.
\end{lem}

\begin{proof}
If $I = (f_1, ... , f_n)$ and $f_i^{e_i} = 0$, then consider:
\[     (a_1 f_1 + ... + a_n f_n)^{(e_1 - 1) + ... + (e_n -1) + 1} = \sum (\text{binomial coefficient}) a_1^{i_1}...a_n^{i_n} f_1^{i_1}...f_n^{i_n} =0   \]
Since in each term at least one of the $i_j$ will be $\geq e_j$, which will make $f_j^{i_j} = 0$.
\end{proof}

\section*{Class 6}

\begin{defn}
Let $A$ be a ring and $M$ be an $A$-module. We say that $M$ is \textbf{Artinian} ring if it satisfies the descending chain condition on ideals. We say that $A$ is Artinian if $A$ is Artinian as an $A$-module.
\end{defn}

\begin{lem}
\label{L33}
Let
\[0\to M'\to M\to M''\to 0\]
be a short exact sequence of $A$-modules. If $M'$ and $M''$ are Artinian (of length $m,n$) then $M$ is as well (of length $\max(m,n)$).
\end{lem}
\begin{proof}
Suppose $M\subset M_1\subset\ldots$ are submodules of $M$. By assumption, there exists an $n$ such that $M_n\cap M'=M_{n+1}\cap M'=\cdots$ and there exists an $m$ such that $\pi(M_m)=\pi(M_{m+1})=\cdots$. Then $M_t=M_{t+1}=\cdots$ for $t=\max(m,n)$.
\end{proof}

\begin{lem}
\label{L34}
A Noetherian local ring of dimension 0 is Artinian.
\end{lem}
\begin{proof}
Using Lemmas \ref{L31} and \ref{L32} we get that $\fr m^n=0$ for some $n\geq 1$. So $0=\fr m^n\subset\fr m^{n-1}\subset\cdots\subset\fr m\subset A$. Then $\fr m^i/\fr m^{i+1}=(A/\fr m)^{\oplus r_i}$ is an $A/\fr m$-module generated by finitely many elements (since $A$ is Noetherian). So it is clear that $\fr m^i/\fr m^{i+1}$ is Artinian as an $A/\fr m$-module, hence over $A$. Apply Lemma \ref{L33} repeatedly.
\end{proof}

\begin{lem}
\label{L35}
If $A$ is Noetherian then so is
\begin{enumerate}[(a)]
\item $A/I$ for $I\subset A$ an ideal;
\item $S^{-1}A$ with $S\subset A$ multiplicative;
\item $A[x_1,\ldots,x_n]$;
\item any localization of a finite-type $A$-algebra.
\end{enumerate}
\end{lem}
\begin{proof}
Omitted.
\end{proof}

\begin{rem}
Any finite-type algebra over a field or over $\Z$ is Noetherian.
\end{rem}

\begin{thm}[Hauptidealsatz, v.1]
\label{L36}
Let $(A,\fr m)$ be a Noetherian local ring. If $\fr m=\sqrt{(x)}$ for some $x\in\fr m$ then $\dim A\leq 1$.
\end{thm}
\begin{proof}
Take $\fr p\subset A, \fr p\neq\fr m$. We will show $\text{ht}(p)=0$ and the theorem will follow. Observe that $x\notin\fr p$ because if it were, by primeness of $\fr p$, $\sqrt{(x)}$ would be contained in $\fr p$, which is a contradiction. Set for $n\geq 1$,
\[\fr p^{(n)}=\{a\in A | \frac{a}{1}\in\fr p^nA_\fr p\}.\]
We will use later that $\fr p^{(n)}A_\fr p=\fr p^nA_\fr p$ (proof omitted). The ring $B=A/(x)$ is local and Noetherian with nilpotent maximal ideal (since $\fr m=\sqrt{(x)}$). By Lemma \ref{L34} $B$ is Artinian. Hence
\[\frac{\fr p+(x)}{(x)}\supset\frac{\fr p^{(2)}+(x)}{(x)}\supset\frac{\fr p^{(3)}+(x)}{(x)}\supset\cdots\]
stabilizes and $\fr p^{(n)}+(x)=\fr p^{(n+1)}+(x)$ for some $n$. Then every $f\in\fr p^{(n)}$ is of the form $f=ax+b$ where $a\in A,b\in\fr p^{(n+1)}$. This implies that $\frac{a}{1}\cdot\frac{x}{1}=\frac{f-b}{1}\in\fr p^nA_\fr p$ and $\frac{x}{1}$ is a unit in $A_\fr p$. Thus $\frac{a}{1}\in\fr p^nA_\fr p$ and $a\in\fr p^{(n)}$. Hence $\fr p^{(n)}=x\fr p^{(n)}+\fr p^{(n+1)}$. Since $x\in\fr m$ and $\fr p^{(n)}$ and $\fr p^{(n+1)}$ are finite $A$-modules, Nakayama's lemma implies that $\fr p^{(n)}=\fr p^{(n+1)}$. Going back to $A_\fr p$, we get $\fr p^{(n)}A_\fr p=\fr p^{(n+1)}A_\fr p$, which implies that $\fr p^nA_\fr p=\fr p^{n+1}A_\fr p$. By Nakayama's lemma, $\fr p^nA_\fr p=0$. Finally, by Lemma \ref{L30} $\dim A_\fr p=0$, i.e. $\text{ht}(\fr p)=0$.
\end{proof}

\begin{lem}
\label{L37}
In the situation of the previous theorem, $\dim A=0$ if and only if $x$ is nilpotent and $\dim A=1$ if and only if $x$ is not nilpotent.
\end{lem}
\begin{proof}
By Lemma \ref{L31}, $\dim A=0$ if and only if $\fr m$ is locally nilpotent.
\end{proof}

\begin{lem}
\label{L38}
If $(A,\fr m)$ is a local Noetherian ring and $\dim A=1$ then there exists an $x\in M$ such that $\fr m=\sqrt{(x)}$.
\end{lem}
\begin{proof}
Since the dimension of $A$ is 1 there must exist primes other than $\fr m$, $\fr p_i$ which are all minimal. To finish the proof, we will use two facts: first, that a Noetherian ring has finitely many minimal ideals and secondly, that one can find $x\in\fr m$ with $x\notin\fr p_i$ for $i\in I$. We shall prove these lemmas below next. Assuming these facts, $V(x)=\{\fr m\}$, which implies that $\sqrt{(x)}=\fr m$.
\end{proof}

\begin{lem}[Prime avoidance]
\label{L39}
Let $A$ be a ring, $I\subset A$ an ideal, and $\fr p_1,\ldots,\fr p_n\subset A$ primes. If $I\not\subset\fr p_i$ for all $i$ then $I\not\subset\fr p_1\cup\cdots\cup\fr p_n$ (i.e. we can find a function vanishing on $I$ but not on $\fr p_i$, Urysohn's lemma).
\end{lem}
\begin{proof}
We proceed by induction on $n$. It's clearly true for $n=1$. We may assume that there are no inclusions among $\fr p_1,\cdots,\fr p_n$ (drop smaller ones). Pick $x\in I,x\notin\fr p_1\cup\cdots\cup\fr p_{n-1}$ (induction hypothesis). If $x\notin\fr p_n$, we are done; if $\fr p_1,\ldots,\fr p_{n-1}\subset\fr p_n$ then $\fr p_j\subset\fr p_n$ for some $j$ ($\fr p_n$ is prime). This contradicts previous mangling of the primes. So $\fr p_1,\ldots,\fr p_{n-1}\not\subset\fr p_n$ and $I\not\subset\fr p_n$ which implies (since $\fr p_n$ is prime) that $\fr p_1\cdots \fr p_{n-1}I\not\subset\fr p_n$. Pick $y\in\fr p_1\cdots\fr p_{n-1}I$ with $y\notin\fr p_n$. Then $x+y$ works. Indeed, $x+y\in I$, $x+y\notin\fr p_j$ for $j=1,\cdots,n-1$, and $x+y\notin\fr p_n$ ($x\in\fr p_n$ but not $y$).
\end{proof}

\begin{lem}
\label{L40}
Let $A$ be a Noetherian ring. Then
\begin{enumerate}[(a)]
\item For all ideals $I\subset A$, there exists a list of primes $\fr p_1,\ldots, \fr p_n$ such that $I\subset\fr p_i$ and $\fr p_1\cdots\fr p_n\subset I$;
\item The set of primes minimal over $I$ is a subset of this list;
\item $A$ has a finite number of minimal primes (i.e. the spectrum has a finite number of irreducible components)
\end{enumerate}
\end{lem}
\begin{proof}\hspace{1mm}
\begin{enumerate}[(a)]
\item Look at $\mathcal{I}=\{I\subset A | \text{(a) does not hold}\}$. If $\mathcal{I}\neq\varnothing$ there must exist an $I\in\mathcal{I}$ maximal with respect to inclusion (since $A$ is Noetherian). So if $ab\in I$ and $a\notin I,b\notin I$ then $\fr p_i\supset(I,a)$ and $(I,a)\supset\fr p_1\cdots\fr p_n$, and $\fr q_j\supset(I,b)$ and $(I,b)\supset\fr q_1\cdots\fr q_m$. This implies that $I\supset(I,a)(I,b)\supset\fr p_1\cdots\fr p_n\fr q_1\cdots\fr q_m$ and $I\subset\fr p_i,I\subset\fr q_j$. This can't happen because $I\in\mathcal{I}$ and hence we conclude that $I$ is a prime which is a contradiction.
\item If $I$ is minimal in $\fr p$ then $\fr p_1\cdots\fr p_n\subset\fr p$ and $\fr p_j\subset \fr p$ for some $j$, i.e. $\fr p_j=\fr p$ and $\fr p_{\text{min}}\supset I$.
\item Apply $(a)$ and $(b)$ to $I=(0)$.
\end{enumerate}
\end{proof}






\section*{Class 7}
For a local Noetherian ring $(A,m)$ set $\dim A = $ the Krull dimension of $A$, and $d(A) = \text{min } \{d| \exists x_1, \dots, x_d \in m \text{ such that } m = \sqrt{(x_1, \dots, x_d)}\}$. We've seen already that:
\[        \dim A = 0 \Leftrightarrow m \text{ nilpotent} \Leftrightarrow d(A) = 0   \;\;\; (\text{Lemma } \ref{L31} + \text{Lemma } \ref{L32})  \]
\[       \dim A = 1 \Leftrightarrow d(A) = 1 \;\;\; (\text{Lemma } \ref{L36} + \text{Lemma } \ref{L38})    \]

\begin{thm} [Krull Hauptidealsatz, v. 2]
\label{L41}
$\dim A = d(A)$.
\end{thm}
\begin{proof}
We first prove that $\dim(A) \leq d(A)$ by induction on $d$. Let $x_1, \dots,x_d \in m$ such that $m = \sqrt{(x_1, \dots, x_d)}$. Because $A$ is Noetherian, $\forall q\subsetneq m, q\neq m$ there exists a $q\subset p \subset m$ such that there exists no prime strictly between $p$ and $m$. Hence it suffices to show $\text{ht}(p)\leq d-1$ for such a $p$. We may assume $x_d \not \in p$ (by reordering). Then $m = \sqrt{(p, x_d)}$ because there exists no prime strictly between $p$ and $m$ (+ Lemma \ref{L26}). Hence:
\[      x_i^{n_i} = a_i x_d + z_i \;\;\;\; (*)      \]
For some $n_i\geq 1, z_i \in p, a_i \in A$. Then we have:

Hence $\sqrt{\overline{(x_d)}} = $ maximal ideal of $A(z_1, \dots, z_{d-1})$. By Theorem \ref{L36} $\dim A/(z_1, \dots , z_{d-1}) \leq 1$. Then $p$ is minimal over $(z_1, \dots, z_{d-1})$. By Lemma \ref{L42}, $pA_p$ is minimal over $(z_1, \dots, z_{d-1})A_p$. Finally, by the induction hypothesis $\dim(A_p) \leq d-1 \Rightarrow \text{ht}(p) \leq d-1$.
\\
\\
Now we prove that $d(A) \leq \dim(A)$. We may assume that $\dim(A) \geq 1$. Let $p_1, \dots, p_n$ be the finite number of minimal primes of $A$. (By Lemma \ref{L40}) Pick $y\in m, y \not \in p_i$ for $i=1, \dots, n$. (Such a $y$ exists by Lemma \ref{L39}.) Then:
\[         \dim \big(A/(y)\big) \leq \dim(A) -1        \]
Because all chains of primes in $A/(y)$ can be seen as a chain of primes in $A$ that can be extended by one of the $p_i$). Then by the induction hypothesis there exists $\bar x_1 , \dots , \bar x_{\dim(A)-}$ in $m/(y)$ such that $m/(y) = \sqrt{(\bar x_1, \dots, \bar x_{\dim(A)-1})}$. It follows that $m = \sqrt{(\bar x_1, \dots, \bar x_{\dim(A)-1}, y)}$.
\end{proof}


\begin{lem}
\label{L42}
Let $A$ be a ring, $I\subset A$ an ideal, $I\subset p$ prime, $S\subset A$ a multiplicative subset, $S\cap p = \emptyset$. Then $p$ minimal over $I \Leftrightarrow S^{-1}p$ is minimal over $S^{-1}I$ of $S^{-1}A$.
\end{lem}

\begin{proof}
See Lemma \ref{L7}.
\end{proof}




\begin{lem}
\label{L43}
Let $(A,\fr m)$ be a Noetherian local ring. Then the dimension of $A$ is less than or equal to the number of generators of $\fr m=\dim_{A/\fr m}\left(\fr m/\fr m^2\right)$. In particular, $\dim A<\infty$.
\end{lem}
\begin{proof}
The inequality is clear because if $\fr m=(x_1,\ldots,x_n)$ then $\fr m=\sqrt{(\fr m_1,\ldots,\fr m_n)}.$ Equality follows from one of Nakayama's many lemmas:
\begin{itemize}
\item if $M$ is finite and $\fr m M=M$, then $M=0$;
\item if $N\subset M$, $M=\fr mM+N$, everything finite, then $M=N$;
\item if $x_1,\ldots,x_t\in M$ which generate $M/\fr mM$, then $x_1,\ldots, x_n$ generate $M$.
\end{itemize}
\end{proof}

\begin{rem}
Note that there do indeed exist infinte-dimensional Noetherian rings. Constructing them is not particularly fun.
\end{rem}

\begin{lem}
\label{L44}
Let $A$ be a Noetherian ring. Let $I=(f_1,\ldots,f_c)$ be an ideal generated by $c$ elements ($c$ somehow stands for codimension). If $\fr p$ is a minimal prime over $I$, then $\text{ht}(p)\leq c$.
\end{lem}
\begin{proof}
Combine Theorem \ref{L41} and \ref{L42}.
\end{proof}

\begin{lem}
\label{L45}
Let $A$ be a Noetherian ring, $\fr p\subset A$ prime. If $\text{ht}(\fr p)=c$ then there exist $f_1,\ldots, f_c\in A$ such that $\fr p$ is minimal over $I=(f_1,\ldots,f_c)$.
\end{lem}
\begin{proof}
By Theorem $\ref{L41}$ there exists $x_1,\ldots, x_c\in\fr pA_\fr p$ such that $\fr pA_\fr p=\sqrt{(x_1,\ldots,x_c)}$. Write $x_i=f_i/g_i,f_i\in\fr p$ and $g_i\in A,g_i\notin\fr p$. Then $I=(f_1,\ldots, f_c)$ satisfies $IA_\fr p=(x_1,\ldots,x_c)A_\fr p$ with Lemma \ref{L42}.
\end{proof}

\begin{lem}
\label{L46}
Let $(A,\fr m)$ be a Noetherian local ring. Let $x\in\fr m$. Then $\dim(A/xA)\in\{\dim A,\dim A-1\}$. If $x$ is not contained in any minimal prime of $A$, e.g. if $x$ is a nonzerodivisor, then $\dim(A/xA)=\dim A-1$.
\end{lem}
\begin{proof}
If $x_1,\ldots,x_t$ map to $\bar x_1,\ldots,\bar x_t$ in $A/xA$ such that $\fr m_{A/xA}=\sqrt{(\bar x_1,\ldots, \bar x_t)}$. Then $\fr m_A=\sqrt{(x_1,\ldots,x_t,x)}$. Hence $\text{d}(A)\leq\text{d}(A/xA)+1$. Conversely, $\text{d}(A)\leq \text{d}(A/xA)$ is easy. Thus $\text{d}(A/xA)\in\{\text{d}(A),\text{d}(A)-1\}$ and hence the same for dimension by Theorem \ref{L41}.
\end{proof}

\begin{lem}
\label{L47}
A nonzerodivisor of any ring is not contained in a minimal prime.
\end{lem}
\begin{proof}
Let $x\in A$ be a nonzerodivisor. Then the map $A\overset{a}{\to}A$ is injective. By exactness of localization, $x/1$ is a nonzerodivisor in $A_\fr p$ for all minimal $\fr p$. Hence $x$ is not nilpotent in $A_\fr p$. Note also that $x/1\notin\fr p A_\fr p$ because $\fr pA_\fr p$ is locally nilpotent when $\fr p$ is minimal by Lemma \ref{L30}.
\end{proof}

\begin{exmp}\hspace{1mm}
\begin{itemize}

\item Consider $A=\left(k[x,y]/(xy)\right)_{(x,y)}$. It's clear from a previous homework exercise that $\dim A=1$ (the primes look like $(x),(y),$ and $(x,y)$). Note that if we consider $A/(x)$, which is now a domain as $(x)$ is prime in $A$, $(x,y)$ is now simply $(y)$, and the chain we are left with is $(0)\subset (y)$. Hence $\dim A/(x)=1$.
\item Consider $A=k[x,y,z]_{(x,y,z)}$. By one of the lemmas we have just proved above, since $\fr m=(x,y,z)$ has 3 generators, it's clear that $\dim A\leq 3$. However, it must be at least 3 due to the presence of the chain $(0)\subset (x)\subset (x,y)\subset (x,y,z)$. Hence $\dim A=3$.

\item $\dim \left(k[x,y,z]/(x^2 + y^2 + z^2) \right)_{(x,y,z)} = 2 = 3-1$, since $(x^2 + y^2 + z^2)$ is not a zerodivisor
\item $\dim \left(k[x,y,z]/(x^2 + y^2 + z^2, x^3 + y^3 + z^3) \right)_{(x,y,z)} = 1 = 3-2$. It suffices to check that $x^3 + y^3 + z^3$ is not $0$ in the domain $k[x,y,z]/(x^2 + y^2 + z^2)$.
\item $\dim \left( k[x,y,z]/(xy, yz, xz) \right)_{(x,y,z)} = 1$. This is because $\dim A/(x+y+z) = 0$, and we've seen in the problem sets that $(x+y+z)$ is not a minimal prime.
\end{itemize}
\end{exmp}


\section*{Class 8}

\begin{thm}[Hilbert Nullstellensatz]
\label{L48}
Let $k$ be a field. For any finite-type $k$-algebra $A$ we have:
\begin{enumerate}[(i)]
\item If $\fr m\subset A$ is a maximal ideal then $A/\fr m$ is a finite extension of $k$;
\item If $I\subset A$ is a radical ideal (i.e. $I=\sqrt{I}$) then $I=\cap_{I\subset\fr m}\fr m$.
\end{enumerate}
\end{thm}

\begin{rem}
Note that if $k=\bar k$ then this says that the residue fields at maximal ideals are equal to $k$. In particular, every maximal ideal of $k[x_1,\ldots,x_n]$ is of the form $(x_1-\lambda_1,\ldots,x_n-\lambda_n)$ for some $\lambda_i\in k$.

In every ring, if $I$ is radical then $I=\cap_{\fr p\supset I}\fr p$. Hence closed subsets of $\Spec A$ are in one-to-one correspondence with radical ideals. Part $(ii)$ of the theorem says that if $A$ is a finite-type $k$-algebra then closed points are dense in all closed subsets.
\end{rem}

\begin{proof}
Let us prove $(i)$ first. Note that $B=A/\fr m$ is a finite-type $k$-algebra which is a field. By Noether normalization there exists some $k[t_1,\ldots,t_r]\subset B$ for some $r\geq 0$. Now, by Lemma \ref{L15}, the map $\Spec B\to\Spec k[t_1,\ldots, t_r]$ is surjective. Since $\Spec B$ is simply a point, we can conclude that $r=0$. Hence $\dim_k B\leq\infty$.

The proof of $(ii)$ follows from $(i)$. We omit it.
\end{proof}

\begin{lem}
\label{L49}
Let $k$ be a field and $A \overset{\phi}{\to} B$ be a homomorphism of finite type $k$-algebras. Then $\Spec \phi$ maps closed points to closed points.
\end{lem}

\begin{proof}
We have to show that $m\in B$ maximal implies $\phi^{-1}(m)$ maximal. We look at $k \subset A/\phi^{-1}(m) \subset B/m$. Note that the latter is a finite field extension of $k$, by Theorem \ref{L48}. Then $\dim_k A/\phi^{-1}(m) < \infty$. Then by Lemma \ref{L11} $A/\phi^{-1}(m)$ is a field. 
\end{proof}

\begin{lem}
\label{L51}
For $k$ field and $A$ finite type $k$-algebra, $\dim(A) = 0 \Leftrightarrow \dim_k A < \infty$.
\end{lem}

\begin{proof}
By Noether normalization there exists a finite map $k[t_1, \dots, k_r] \hookrightarrow A$. Then by Lemma \ref{L29} $\dim(A) = \dim(k[t_1 \dots t_r]) \geq r$. Hence $(\Rightarrow)$ follows. For the converse use Lemma \ref{L12}.
\end{proof}
Our goal is now to construct a ``good'' dimension theory for finite type algebras over fields.

\begin{lem}\hspace{1mm}
\label{L51}
\begin{enumerate}[(a)]
\item For $X$ a topological space with irreducible components $Z_i$ then $\dim(X) = \sup \dim(Z_i)$;
\item For a ring $A$, $\dim(A) = \sup_{p\subset A \text{ minimal}} \dim(A/p)$.
\end{enumerate}
\end{lem}
\begin{proof}
Omitted.
\end{proof}

\begin{defn}
Let $k\subset K$ be a field extension. The \textbf{transcendence degree} $\trdeg_k K = \sup \{n|\; \exists x_1, \dots, x_n \text{ algebraically independent over } k\}$. This means that the map $k[t_1 \dots t_n] \to K$ that takes $t_i \to x_i$ is injective.
\end{defn}


\begin{lem}
\label{L52}
Let $k$ be a field, then every maximal ideal $m$ of the ring $k[x_1 \dots x_n]$ can be generated by $n$ numbers, and $\dim(k[x_1 \dots x_n])_{m} = n$.
\end{lem}

\begin{proof}
By Theorem \ref{L48}, the residue field $\kappa = k[x_1 \dots x_n]/m$ is finite over $k$. Let $\alpha_i \in \kappa$ be the image of $x_i$. We look at the chain:
\[        k = \kappa_0 \subset \kappa_1 = k(\alpha_1) \subset \dots \subset \kappa = k(\alpha_1, \dots , \alpha_n)        \]
We know from field theory that $x_i \in k[\alpha_1, \dots, \alpha_i]$. Choose $f_i \in k[x_1 \dots x_i]$ such that $f(\alpha_1, \dots, \alpha_{i-1}, x_i)$ is the minimal polynomial of $\alpha_i$ over $\kappa_{i-1}$. Then $f_i(\alpha_1, \dots, \alpha_i) = 0$, so $f_i \subset m$. Now we claim that $\kappa_i \cong k[x_1, \dots, x_i]/(f_1, \dots, f_i)$. We prove this by induction:
\[       k[x_1, \dots, x_i]/(f_1, \dots, f_i)   \cong k[x_1, \dots, x_{i-1}]/(f_1, \dots, f_{i-1}) [x_i] /(f_i)        \]
If we let $i=n$, this proves the first statement of the lemma. Finally, we have a chain of primes:
\[      (0) \subset (f_1) \subset \dots \subset (f_1 \dots f_n) = m        \]
because $ k[x_1, \dots, x_i]/(f_1, \dots, f_i) \cong \kappa_i [x_{i+1}, \dots, x_n]$. Therefore $\dim(k[x_1, \dots, x_{n}])_m \geq n$. But by Lemma \ref{L41} it is at most $n$, so this finishes the proof.
\end{proof}

\begin{lem}
\label{L53}
$\dim(k[x_1, \dots, x_n]) = n$.
\end{lem}
\begin{proof}
Omitted.
\end{proof}

\begin{rem}
For a Noetherian local ring $(A,\fr m)$ we have:
\[  \dim A \leq \text{minimum number of generators of } \fr m = \dim_{A/\fr m} \fr m/\fr m^2 \]
$(A,\fr m)$ is called \textbf{regular} if we have equality. The above shows that $k[x_1, \dots, x_n]_\fr m$ is regular for all maximal ideals $\fr m$.
\end{rem}

\begin{lem}
\label{L54}
Let $k$ be a field and $A$ be a finite type $k$-algebra. Then:
\begin{enumerate} [(a)]
\item the integer $r$ from Noether Normalization is equal to $\dim A$;
\item if $A$ is a domain, then $\dim A = \text{trdeg}_k (\text{f.f.} A)$.
\end{enumerate}
\end{lem}

\begin{proof}\hspace{1mm}
\begin{enumerate} [(a)]
\item follows from Lemma \ref{L29} and Lemma \ref{L53}
\item follows from $(a)$ and the fact:
\end{enumerate}
\[       k[t_1 \dots t_r] \subset A \text{ finite}  \overset{L \ref{L10}}{\Rightarrow}  k(t_1 \dots t_r) \subset S^{-1}A \text{ finite}   \overset{L \ref{L11}}{\Rightarrow} S^{-1}A \text{ is the f.f. of } A \]
Then:
\[     k(t_1 \dots t_r) \subset \text{f.f.}(A)  \Rightarrow \text{trdeg}_k \text{f.f.}(A) = \text{trdeg}_k k(t_1 \dots t_r) = r     \]
The last two equalities should be familiar from field theory.
\end{proof}

\begin{rem}
If $k\to A$ is a finite type domain then $\dim(A) = \dim(A_f) \forall f\in A, f\neq 0$. We may regard this as a very weak form of ``equidimensionality''.
\end{rem}
\begin{rem}
So far we missed proving an important result; we will do so later. We will want to show that for $A$ finite type domain over a field, $p \subset A$ prime, we have $\dim(A) = \dim(A/p) + \text{ht}(p)$. Intuitively it's clear why this should be so: take a chain in $A$ and some $p$ in this chain, then $\dim(A/p)$ counts elements containing $p$, and ht$(p)$ counts elements contained in $p$.
\end{rem}

\begin{defn}
Let $A\to B$ be a ring map. The \textbf{integral closure} of $A$ in $B$ is $B' = \{b\in B | \text{b is integral over} A\}$. We say that \textbf{$B$ is integral over $A$} iff $B'=B$.
\end{defn}

\begin{lem}
\label{L55}
If $A\to B$ finite, then $B$ is integral over $A$.
\end{lem}

\begin{proof}
Pick $b\in B$. Choose $b_1, \dots b_n, \in B$ such that $B = \sum A b_i$. 
Write, for $a_ij\in A$,
\[bb_i=\sum_{a_{ij}}b_j.\]
Let $M=(a_{ij})\in\text{Mat}(n\times n, A)$ and let $P(T)\in A[T]$ be the characteristic polynomial of $M$. By Cayley-Hamilton, $P(M)=0$, which implies that $P(b)=0$.
\end{proof}


\section*{Class 9}
\begin{lem}
\label{L56}
The integral closure of a ring $A$ is an $A$-algebra.
\end{lem}
\begin{proof}
Suppose $b,b' \in B'$, we want to show that $b+b', bb' \in B'$. Let $C$ be the $A$-algebra generated by $b,b'$. Then $C$ is finite over $A$ by Lemma \ref{L1}. Then by Lemma \ref{L55} $C$ is integral over $A$, so $C\subset B'$.
\end{proof}

\begin{lem}
\label{L57}
If $A\to B \to C$ are ring maps then:
\begin{enumerate}
\item $A\to B, B\to C$ integral $\Rightarrow A\to C$ integral;
\item $A\to C$ integral $\Rightarrow B\to C$ integral.
\end{enumerate}
\end{lem}
\begin{proof}
Omitted.
\end{proof}

\begin{defn}
A \textbf{normal domain} is a domain which is integrally closed in its field of fractions. (In other words, it is equal to its integral closure in its field of fractions.)
\end{defn}

\begin{lem}
\label{L58}
For a field $k$, $k[x_1, \dots, x_n]$ is a normal domain.
\end{lem}
\begin{proof}
Polynomial rings are UFDs, so this follows from Lemma \ref{L59}.
\end{proof}

\begin{lem}
\label{L59}
A UFD is a normal domain.
\end{lem}
\begin{proof}
Suppose that $a/b \in$ f.f.$(A)$ is in least terms (we can always reduce a fraction to least terms, due to unique factorization) and is integral over $A$. Thus there exist some $a_i \in A$ such that:
\[         \left( \frac{a}{b} \right)^n +  a_1 \left( \frac{a}{b} \right)^{n-1} + \dots + a_n = 0       \]
\[       a^n + a_1 a^{n-1} b + \dots + a_n b^n = 0    \]
Therefore $a^n \in (b)$, which, unless $b$ is a unit, contradicts the fact that $a, b$ are relatively prime. Therefore the only elements of the field of fractions that are integral over $A$ are those of $A$ itself.
\end{proof}

\begin{lem}
\label{L60}
Let $R$ be a domain with field of fractions $K$, and let $a_0, \dots, a_{n-1}$, $b_0, \dots, b_{m-1} \in R$. If $x^n + a_{n-1} x^{n-1} + \dots + a_0$ divides $x^m + b_{m-1} x^{m-1} + \dots + b_0$, then $a_i$ are integral over the $\Z$ subalgebra of $R$ generated by $\{b_j\}$.
\end{lem}
\begin{proof}
Choose some field extension $L$ of $K$ with $\beta_1, \beta_m \in L$ such that:
\[   x^m + b_{m-1}x^{m-1} + \dots + b_0 = \prod_{i=1}^m (x - \beta_i)  \]
Then by unique factorization in $L[x]$ we get:
\[     x^n + a_{n-1}x^{n-1} + \dots + a_0 = \prod_{j} (x - \beta_j)       \]
Where $j$ runs over a subset of $\{1, \dots, m\}$. But this means that:
\[     a_i \in \Z[b_0, \dots, b_{m-1} , \beta_1, \dots, \beta_m] \supset \Z[b_0, \dots, b_{m-1}]     \]
By Lemmas \ref{L1} and \ref{L55}, the inclusion is integral.
\end{proof}

\begin{lem}
\label{L61}
Let $R\subset A$ be a finite extension of domains, $R$ normal. For $a\in A$ we have:
\begin{enumerate} [(1)]
\item the coefficients of the minimal polynomial of $a$ over the field of fractions of $R$ are in $R$;
\item $\Nm (a) \in R$, where $\Nm$ denotes the norm.
\end{enumerate}
\end{lem}
\begin{proof}
Apply Lemma \ref{L60}. For example, $a$ must satisfy a monic polynomial with coefficients in $R$, and the minimal polynomial must divide that.
\end{proof}

\begin{lem}
\label{L62}
Suppose $R\subset A$ is a finite extension of domains, and $R$ is normal. Suppose also that $f\in A$, $\fr p \subset A$ prime with $V(f) \subset V(\fr p)$. Then setting $f_0 = \Nm_{\text{ff}(A)/\text{ff}R} (f)$ we have:
\begin{enumerate}
\item $f_0 \in R$;
\item $R \cap \fr p = \sqrt{(f_0)}$.
\end{enumerate} [(1)]
\end{lem}
\begin{proof}
$(1)$ follows by Lemma \ref{L61}. (See also the argument by Tate in Mumford's red book.) For part $(2)$, let
\[    x^d + r_1 x^{d-1} + \dots r_d    \]
be the minimal polynomial of $f$ over f.f.$(R)$, with $r_i \in R$. (This is possible by Lemma \ref{L61}.) Then $f_0 = r_d^e$ for some $e\geq 1$. So:
\[     f^d + r_1 f^{d-1} + \dots + r_d = 0   \Rightarrow r_d \in (f) \Rightarrow f_0 \in (f)   \]
We already know that $f \in \fr p$ by assumption $V(f) = V(\fr p)$, so we get $\sqrt{f_0} \subset R \cap \fr p$. Conversely, if $r \in R \cap \fr p$ we have $r^n\in (f)$ for some $n$, because $V(f) = V(\fr p)$. Say $r^n = af$, then:
\[         (r^n)^{[\text{f.f.}(A) : \text{f.f.}(R)]} = \Nm (r^n) = \Nm(a) \Nm(f) = \{\text{sth in }R\} f_0       \]
Then $r \in \sqrt{f_0}$.
\end{proof}
\begin{rem}
This lemma says that the image (under a finite map) of an irreducible hypersurface is an irreducible hypersurface.
\end{rem}
Now we use this lemma to prove the missing link from dimension theory.
\begin{thm}
\label{L63}
Given a finite type $k$-algebra $A$ which is a domain and a height $1$ prime $\fr p$, then:
\[       \dim A = \dim(A/\fr p) + 1   \]
\end{thm}
\begin{proof}
By Lemma \ref{L45}, $\fr p$ minimal over $(f)$ for some $f\in A$. Say $\fr p, \fr p_1, \dots, \fr p_n$ are all the distinct minimal primes over $(f)$. (Lemma \ref{L40} says they are finitely many.) Then $\fr p_1 \dots \fr p_n \not \subset \fr p$. (Prime avoidance.) Then we can pick $g\in p_1 \dots p_n$, $g\not \in \fr p$. After replacing $A$ by $A_g$ and $\fr p$ by $\fr p A_g$ and $f$ by $f/1$ we may assume $\fr p$ is the only prime minimal over $(f)$, i.e. $V(f) = V(\fr p)$. By an earlier remark, $\dim A = \dim A_g$, so the statement of the theorem doesn't change if we do the replacement. 
\\
\\
Now by Noether Normalization we choose a finite injective map $k[t_1, \dots, t_d] \hookrightarrow A$. Set $f_0 = \Nm(f) \in k[t_1, \dots, t_d]$, by Lemma \ref{L62} and $\fr p \subset k[t_1, \dots, t_d] = \sqrt{(f_0)}$, again by Lemma \ref{L62}. Since $k[t_1, \dots, t_d]$ is a UFD, we can write $f_0 = cf^e$, for some $c\in k^*, e\geq 1$. Then $\sqrt{(f_0)} = (f_0)$ and we see that $k[t_1, \dots, t_d]/(f_0) \hookrightarrow A/\fr p$ is a finite injective map. Thus:
\[       \text{trdeg}_k (\text{f.f.} (A/\fr p)) \subset \text{trdeg}_k (\text{f.f.} (k[t_1, \dots, t_d]/(f_0))) = d-1    \]
\end{proof}

\begin{exmp}
We compute the integral closure of $k[x,y]/(y^2 - x^3)$ in its field of fractions. (This is called ``normalization''.)
\[       k[x,y]/(y^2 - x^3) \subset \text{f.f.} (k[x,y]/(y^2 - x^3))    \]
To get some element in the integral closure which is not in the ring, we look at the equation:
\[       y^2 - x^3 = 0 \Rightarrow \frac{y}{x} = x^{1/2}    \]
We see that $t = y/x$ is both in the integral closure and in the field of fractions. We therefore add it to the ring and see what happens. We construct the map:
\begin{align*}      k[x,y]/(y^2 - x^3) &\to k[t] \\
x &\to t^2 \\
y &\to t^3
\end{align*}
We need to check that this induces an isomorphism of fraction fields, namely maps $y/x$ to $t$. We also need to check that the map is integral (it is, because $t$ is integral). We are now done, because $k[t]$ is a UFD and therefore a normal domain.
\end{exmp}

\section*{Class 10}

\todo{Matei fill in stuff here}

\begin{lem}
\label{L64}
\end{lem}

\begin{defn}
A \textbf{graded ring} is a ring $A$ together with a given direct sum decomposition $A=\oplus_{d\geq 0}A_d$ such that $A_dA_e\subset A_{d+e}$. A \textbf{graded module} $M$
over $A$ is an $A$-module $M$ equipped with a direct sum decomposition $M=\oplus_{d\in\Z}Md$ such that $A_dM_e\subset M_{d+e}$. We say that $B$ is a \textbf{graded $A$-algebra} if there is a direct sum decomposition as $R$-modules.
\end{defn}

\begin{exmp}
If we take $A=k[x_1,\ldots,x_d]$ and $A_d$ to be the homogeneous polynomials of degree $d$, we see that $A$ is a graded ring.
\end{exmp}

\begin{thm}
\label{L65}
Let $M$ be a finitely-generated, graded, $A$-module, where $A$ is a graded $k$-algebra which is generated (as a $k$-algebra) by a finite number of elements of degree \todo{$\trdeg$?} 1. Then the function $d\mapsto \dim_k(M_d)$ is a numerical polynomial. This function is known as the \textbf{Hilbert polynomial}.
\end{thm}

\begin{defn}
A function $f:\Z\to\Z$ is a \textbf{numerical polynomial} iff there exists an $r\geq 0,a_i\in \Z$ such that
\[f(d)=\sum_{i=0}^r a_i\binom{d}{i}\]
for all $d\gg 0$.
\end{defn}

\begin{lem}
\label{L66}
If $f:\Z\to\Z$ is a function and $d\mapsto f(d)-f(d-1)$ is a numerical polynomial, then so is $f$.
\end{lem}

Let us now prove the theorem.

\begin{proof}
We may assume that $A=k[x_1,\ldots, x_d]$, graded as in the above example. The proof proceeds by induction on $n$. Let us consider three distinct cases. In the first, we suppose that $x_n$ is a nonzerodivisor on $M$. Then we have a short exact sequence
\[0\to M\overset{x_n}{\to}M\to M/x_nM\to 0.\]
Note that the multiplication by $x_n$ shifts the grading by 1 and that
\[-\dim M_{d-1}+\dim M_d-\dim \left( M/x_nM\right)_d=0.\]
Now $M/x_nM$ is a finitely-generated graded module \todo{why? morphism shifts grading} as $k[x_1,\ldots,x_{d-1}]$ so we are done by induction and Lemma \ref{L66}.

Next consider the case where $x_n^eM=0$ for some $e\geq 0$. In this case we get a short exact sequence
\[0\to x_nM\to M\to M/x_nM\to 0.\]
Note that $x_n^{e-1}x_nM=0$. Hence we are done by induction on $e$ and $n$.

Finally, consider the general case. Let $N=\{m\in M | x_n^e m=0\text{ for some }e\}$. Then
we get an exact sequence
\[0\to N\to M\to M/N\to 0.\]
At $N$, this follows from the nilpotent cases, as $A$ is Noetherian and $M$ being finitely-generated implies that $N$ is. At $M/N$ this follows from the nonzerodivisor case.
\end{proof}

\begin{defn}
Let $(A,\fr m, k)$ be a Noetherian local ring. Set
\[\text{gr}_\fr m A=\bigoplus_{n\geq0}\fr m^n/\fr m^{n+1}.\]
This is a graded $k$-algebra generated by $\fr m/\fr m^2$ over $k$. Hence
$n\mapsto\dim_k(\fr m^n/\fr m^{n+1})$ is a numerical polynomial by Theorem \ref{L65}.
We denote by $d(A)$ the degree of this polynomial; if $\fr m^n=0$ for some $n$, we take -1.
\end{defn}

\begin{thm*}
For $(A,\fr m,k)$ a Noetherian local ring, $d(A)=\dim A-1$.
\end{thm*}
\begin{proof}
We will only sketch the proof of this theorem.
The result is clear for $\dim A=0$. Suppose $\dim A>0$. Suppose you can find an $x\in\fr m$ such that $x$ is a nonzerodivisor in $A$ and $\bar x\in\fr m/\fr m^2$ is a nonzerodivisor in $\text{gr}_\fr m A$. Then $\dim A/xA=\dim A-1$. The sequences
\[0\to\fr m^{n-1}/\fr m^n\overset{x}{\to}\fr m^n/\fr m^{n+1}\to \bar{\fr m}^n/\bar{\fr m}^{n+1}\to 0 \]
are exact, where $\bar{\fr m}\subset A/xA$ is the maximal ideal. Then $d(A/xA)=d(A)-1$, and we are done by induction on $\dim A$.

This argument will not work if such an $x$ does not exist. Roughly speaking, one finds an $x$ such that $\bar x$ is in the none of the minimal primes of $\text{gr}_\fr mA$ and one shows that $d(A/xA)$ does actually drop by 1.
\end{proof}

\begin{thm*}
If $B$ is a graded $k$-algebra generated by finitely many elements of degree 1, then $\dim B-1$ is the degree of the numerical polynomial $n\mapsto\dim_k B_n$.
\end{thm*}

\begin{cor*}
$\dim A=\dim\text{gr}_\fr m A.$
\end{cor*}

\end{document}




































