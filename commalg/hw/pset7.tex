\documentclass{../../mathnotes}

\usepackage{enumerate}
\usepackage{todonotes}
\usepackage{tikz-cd}

\title{Commutative Algebra: Problem Set 7}
\author{Nilay Kumar}
\date{Last updated: \today}


\begin{document}

\maketitle

\subsection*{Problem 2}

Let us find a basis for $L(D)$ when $D=2v_0+3v_1$. Recall the definition
\[L(D)=\left\{ f\in K^\times \mid (f)+D\geq 0 \right\}.\]
Since the principal divisor of $f$ is just defined as $(f)=\sum_v v(f) v$, we see that the condition on $f\in L(2v_0+3v_1)$ allows
poles of up to order 2 at 0 and poles of up to order 3 at 1. It is easy to see, then, that the following set of rational functions satisfy
the condition:
\[\left\{1,\frac{1}{x},\frac{1}{x^2},\frac{1}{x-1},\frac{1}{(x-1)^2},\frac{1}{(x-1)^3}\right\}.\]
Any product of these can be decomposed by partial fractions into these basis elements. It is clear that these are in fact a basis
for $L(D)$ as a $k$-vector space (the dimension is 6, which falls into the bound we proved in class as $\deg D=5$).

Next consider $D=2v_0+2v_\infty$. Here we are allowed poles at 0 of up to order 2 and poles at infinity up to order 2. The basis is then
\[\left\{ 1,x,x^2,\frac{1}{x},\frac{1}{x^2} \right\}.\]
For exactly the same reasons as above, we see that this gives us a basis for $L(D)$ as a $k$-vector space (again, the bound checks out).


\subsection*{Problem 3}

For simplicity, let us consider the case of $k=\C$. Then any squarefree cubic may be written as $f(t)=\prod_{i=1}^3(x-\lambda_i)$ for $\lambda_i$
distinct. We wish to classify the discrete valuations on $K=\text{Frac}\left(\C[x,y]/(y^2-\prod_{i=1}^3(x-\lambda_i))\right)$ over $\C(x)$. Note first
that we have a degree two extension and hence for every discrete valuation $v$ on $\C(x)$ we must have finitely many extensions $w_i$ on $K$ of $v$
such that $w_i=e_iv$. However, we also know that $\sum_i e_i=2$, and hence there are only two possibilities: given an valuation $v$ on $\C(x)$, there
is either one valuation on $K$ that restricts to $v$ with ramification 2 or there are two valuations on $K$ that restrict to $v$ each with ramification 1.
Let us proceed in cases.

Let $w$ be a valuation on $K$. Suppose $w(x-\lambda_j)>0$ for some $j$. Then, restricting to $\C(x)$, we see that $w|_{\C(x)}=\text{ord}_{t=\lambda_j}$.
But then we see that
\begin{align*}
    2w(y)&=w(y^2)=n\left(w(x-\lambda_1)+w(x-\lambda_2)+w(x-\lambda_3)\right)=n
\end{align*}
for some integral $n\leq 2$.
But this implies that $n=2$ (given the bound earlier). Hence $w$ is the only valuation lying over $\text{ord}_{t=\lambda_j}$. Indeed, we can
use this now to obtain exactly what $w$ does to all elements in $K$, i.e. $w(y)=1$, etc.

The other case is that $w(x-\lambda_j)<0$ (as $w(x-\lambda_j)\neq 0$). Restricting to $\C(x)$ we see that $w|_{\C(x)}=\text{ord}_{t=\infty}$.
But then we see that
\begin{align*}
    2w(y)=w(y^2)=m\left( w(x-\lambda_1)+w(x-\lambda_2)+w(x-\lambda_3) \right)=-3m
\end{align*}
for some integral $m\leq 2$. But this implies that $m=2$, as otherwise $w(y)$ will not be integral. Thus $w$ is the only valuation
lying over $\text{ord}_{t=\lambda_j}$. Indeed, $w(y)=-3$ and we can deduce the value of $w$ on all elements in $K$.

So far we have examined the valuations that lie over $\lambda_i$ and $\infty$. But what about valuations lying over other points in $\Proj^1$?
It turns out that there are two unique valuations lying over every other such point, i.e. these points in $K$ are not ramified. To see this, let
us appeal to some geometric reasoning. In terms of complex geometry, we see that $y^2=\prod_{i=1}^3(x-\lambda_i)$ is simply a Riemann surface
upon which $y$ is holomorphic. It is easy graphically to see that this surface is a torus (and can be more rigorously shown via Abel's map and
Jacobi inversion) and thus has one handle. But the number of ramified valuations in such a Riemann surface is simply four, as one can see in the
diagrams below - these are precisely the $\lambda_i$ and $\infty$. Hence the other points must each have two distinct valuations lying over them.



\end{document}
