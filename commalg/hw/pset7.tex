\documentclass{../../mathnotes}

\usepackage{enumerate}
\usepackage{todonotes}
\usepackage{tikz-cd}

\title{Commutative Algebra: Problem Set 7}
\author{Nilay Kumar}
\date{Last updated: \today}


\begin{document}

\maketitle

\subsection*{Problem 2}

Let us find a basis for $L(D)$ when $D=2v_0+3v_1$. Recall the definition
\[L(D)=\left\{ f\in K^\times \mid (f)+D\geq 0 \right\}.\]
Since the principal divisor of $f$ is just defined as $(f)=\sum_v v(f) v$, we see that the condition on $f\in L(2v_0+3v_1)$ allows
poles of up to order 2 at 0 and poles of up to order 3 at 1. It is easy to see, then, that the following set of rational functions satisfy
the condition:
\[\left\{1,\frac{1}{x},\frac{1}{x^2},\frac{1}{x-1},\frac{1}{(x-1)^2},\frac{1}{(x-1)^3}\right\}.\]
Any product of these can be decomposed by partial fractions into these basis elements. It is clear that these are in fact a basis
for $L(D)$ as a $k$-vector space (the dimension is 6, which falls into the bound we proved in class as $\deg D=5$).

Next consider $D=2v_0+2v_\infty$. Here we are allowed poles at 0 of up to order 2 and poles at infinity up to order 2. The basis is then
\[\left\{ 1,x,x^2,\frac{1}{x},\frac{1}{x^2} \right\}.\]
For exactly the same reasons as above, we see that this gives us a basis for $L(D)$ as a $k$-vector space (again, the bound checks out).


\subsection*{Problem 3}

\subsection*{Problem 4}



\end{document}
