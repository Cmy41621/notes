\documentclass{../../mathnotes}

\usepackage{enumerate}
\usepackage{todonotes}
\usepackage{tikz-cd}

\title{Commutative Algebra: Problem Set 8}
\author{Nilay Kumar}
\date{Last updated: \today}


\begin{document}

\maketitle

\subsection*{Problem 2}

Let $k$ be an algebraically closed field. Let $A=k[x,y]/(f)$ where $f$ is an irreducible polynomial. Let $K$ be
the fraction field of $A$. Let $C=\left\{ (s,t)\in k^2\mid f(s,t)=0 \right\}.$ Recall that the maximal ideals of
$A$ correspond one-to-one with points of the curve $C$. Now suppose that every point of $C$ is nonsingular. Fix any
point $p\in C$ (equivalently a maximal ideal $\fr m$). By a previous homework exercise, we see that the ring $A_\fr m$
is a regular local ring. Moreover, it is a domain, as $(f)$ is prime. Hence, since the dimension of $k[x,y]$ is 2 and $f$
is a nonzerodivisor, $A=k[x,y]/(f)$ has dimension one. By Tag 00PD of the Stacks Project we find that $A$ is in fact
a discrete valuation ring. The discrete valuation ring comes equipped with a valuation that is centered on $A$ (one can
see this by looking at the uniformizer); this valuation is, in fact, unique. One way to see this is to simply look at the definition
of a valuation as a projection from $K^*$ to $K^*/A^*$ (see Tag 00I8). This establishes the claim.

Let us consider the self-intersecting curve $C$ given by $y^2-x^3-x^2$ as an example of a singularity (at $x=y=0$) for which this theorem
does not hold. Note that we can consider the function field of $C$, $\text{Frac}(k[x,y]/(y^2-x^3-x^2))$, as a field extension of degree
three over $k(t)$ and $k(s)$ with the injective maps given by $t\mapsto x+y$, $s\mapsto x-y$. To see why this must be degree three,
simply note that we can rewrite the defining equation of the curve as $(x+y)(x+y)-2(x+y)-x^3$, and similarly for $x-y$, which shows that we
are adding the element $x$ via a degree three polynomial. Graphically we can now obviously see that there is more than one valuation
at $(0,0)$, as one can take the order of vanishing in $(x-y)$ and the order of vanishing in $(x+y)$, which is geometrically just
the order of vanishing after projecting the singularity onto the lines $y=\pm x$. We can also show this more formally, but this is rather tedious.
First note that we have an automorphism that takes $y\mapsto -y$ and hence we must have that $v_1(x-y)=v_2(x+y)$. Thus it suffices to show
that $v_1(x-y)\neq v_1(x+y)$.
Also note that since the singular point is clearly centered above $\text{ord}_{t=s=0}$, we have that $v_1(x-y)=v_2(x+y)=e\cdot 1$ for $1\leq e\leq 3$.
To proceed, we will make use of the following three use facts:
\begin{enumerate}[(i)]
    \item $v_1(x+y)+v_1(x-y)=3v_1(x)$;
    \item $2x(y+x)+x^3=(y+x)^2$;
    \item $v_1(x+y)\in\left\{ 1,2,3 \right\}$.
\end{enumerate}
Let us proceed by cases. If $v_1(x+y)=1$ then $(i)$ implies that $v_1(x-y)\neq 1$ by simple divisbility arguments, and we are done.
Next, if $v_1(x+y)=2$ then $(i)$ implies that $v_1(x-y)\neq 2)$ by similar arguments, and we are done. Finally, consider what happens
when $v_1(x+y)=3$. Then, if further $v_1(x)=1$, then $(ii)$ yields a contradiction $3=6$. If instead $v_1(x)\geq 2$, let $v_1(x)=a$.
Since $3+a<3a$, $(ii)$ now yields that $3+a=6\implies a=3$. But then $(i)$ implies that $v_1(x-y)=9-3=6\neq v_!(x+y)$, and we are done.



%In summary, if we denote the valuations for $x+y,x-y$ by $v_1,v_2$ respectively, we note by extension theory for valuations that
%the sum of ramification indices must be 3 due to the extension being degree three. We now wish to show that $v_1$ and $v_2$ are in
%fact distinct; we can do this with a bit of casework by checking that 


\subsection*{Problem 3}

Let $k=\C$ be the field of complex numbers. Let $f=1+x^n+y^n\in k[x,y]$ for some positive integer $n$. Let $K$ be the
fraction field of $A=k[x,y]/(f)$. It is clear that the algebraic curve defined by $f$ is nonsingular (we showed this in a previous
homework exercise), and hence the valuations of $K$ centered on $A$ are in one-to-one correspondence with points of the curve.
Hence the only missing valuation in this case should be the valuation at infinity, as it will evaluate orders of vanishing to be negative.
%The valuations of $K/k$ in general should be  

Given the discussion in the previous exercise, we would expect that every singularity yields extra valuations centered at $A$, thus ruining
the one-to-one correspondence. However, this guess would be incorrect, as the curve $C$ given by $y^2=x^3$ shows. This is the cuspidal cubic, which
at the cusp has tangent lines that limit to a horizontal line. Thus, geometrically speaking (or more formally using the arguments of the previous
exercise), it is clear that we are projecting only onto this one horizontal line to determine orders of vanishing, and hence, unlike the previous
example where we had two lines $y=\pm x$ and thus two valuations, in this case we will have only one.


\end{document}
