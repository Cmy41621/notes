\documentclass{article}
\usepackage[utf8]{inputenc}

%\usepackage[margin=1in]{geometry}
\usepackage{amsmath}
\usepackage{amssymb}
\usepackage{amsthm}
\usepackage{enumerate}
\usepackage{tikz-cd}

\newcommand{\nm}[1]{\;\textnormal{#1}\;}
\newcommand{\ra}[0]{\rightarrow}
\newcommand{\fa}[0]{\;\forall}
\newcommand{\R}{\mathbb{R}}
\newcommand{\Q}{\mathbb{Q}}
\newcommand{\Z}{\mathbb{Z}}
\newcommand{\F}{\mathbb{F}}
\newcommand{\C}{\mathbb{C}}
\newcommand{\CP}{\mathbb{C}\mathbb{P}}
\newcommand{\RP}{\mathbb{R}\mathbb{P}}
\newcommand{\Proj}{\mathbb{P}}
\newcommand{\N}{\mathbb{N}}
\newcommand{\p}{\partial}
\newcommand{\fr}{\mathfrak}
\newcommand{\OO}{\mathcal{O}}


\DeclareMathOperator{\Ker}{Ker}
\DeclareMathOperator{\Tr}{Tr}
\DeclareMathOperator{\Res}{Res}
\DeclareMathOperator{\ord}{ord}
\DeclareMathOperator{\Hom}{Hom}
\DeclareMathOperator{\length}{length}
\DeclareMathOperator{\res}{Res}
\DeclareMathOperator{\Int}{Int}
\DeclareMathOperator{\Ext}{Ext}
\DeclareMathOperator{\Aut}{Aut}
\DeclareMathOperator{\Gal}{Gal}
\DeclareMathOperator{\Sym}{Sym}
\DeclareMathOperator{\Lie}{Lie}
\DeclareMathOperator{\Pro}{Proj}
\DeclareMathOperator{\id}{Id}
\DeclareMathOperator{\tr}{tr}
\DeclareMathOperator{\irr}{irr}
\DeclareMathOperator{\supp}{supp}
\DeclareMathOperator{\trdeg}{trdeg}
\DeclareMathOperator{\Spec}{Spec}
\DeclareMathOperator{\Nm}{Nm}
\theoremstyle{plain}
\newtheorem{thm}{Theorem}
\newtheorem*{thm*}{Theorem}
\newtheorem{lem}[thm]{Lemma}
\newtheorem*{lem*}{Lemma}
\newtheorem{cor}[thm]{Corollary}
\newtheorem*{cor*}{Corollary}
\newtheorem{prop}[thm]{Proposition}
\newtheorem*{prop*}{Proposition}
\newtheorem{exc}{Exercise}

\theoremstyle{definition}
\newtheorem{defn}{Definition}
\newtheorem{exmp}{Example}

\theoremstyle{remark}
\newtheorem*{rem}{Remark}

\title{Commutative Algebra \`a la A.~J.~de Jong}
\author{Notes by Matei Ionita and Nilay Kumar}
\date{Fall 2013}

\begin{document}

\maketitle

\section*{Class 1}

\begin{defn}
Given a ring $R$, a \textbf{finite-type} $R$-algebra is any $R$-algebra $A$ which can be generated \textit{as an $R$-algebra} by finitely many elements over $R$. Equivalently, $A\cong R[x_1,\ldots, x_n]/I$.
\end{defn}

\begin{defn}
A ring map $\phi:A\to B$ is \textbf{finite} (or $B$ is finite over $A$) if there exist finitely many elements of $B$ that generate $B$ \textit{as an $A$-module}. Equivalently, there exists a surjective map $A^{\oplus n}\to B$ as $A$-modules.
\end{defn}

\begin{exmp}
Consider $A=k[x_1,x_2]/(x_1x_2-1)\cong k[t,t^{-1}]\subset k(t)$. The map $k[x_1,x_2]\twoheadrightarrow A$ is finite but not injective. On the other hand, $k[x_1]\to A$ is injective but not finite. The map $k[y]\overset{\phi}{\to}A$ given by $y\mapsto x_1+x_2$ works, as one can show.
\end{exmp}


\begin{thm*}[Noether Normalization]
Let $k$ be a field, and $A$ be a finite-type $k$-algebra. Then there exists an $r\geq0$ and a finite injective $k$-algebra map $k[y_1,\ldots,y_r]\to A$.
\end{thm*}

Before we prove the theorem let us state some useful lemmas.

\begin{lem}
\label{L1}
If $A\to B$ is a ring map such that $B$ is generated as an $A$-algebra by $x_1,\ldots,x_n\in B$ and each $x_i$ satisfies a monic equation
\[x_n^{d_n}+\phi(a_{n-1})x_{n-1}^{d_{n-1}}+\ldots+\phi(a_1)=0\]
over $A$, then $\phi$ is finite.
\end{lem}
\begin{proof}
The map $A^{\oplus d_1\ldots d_n}\to B$ given by
\[(a_{i_1},\ldots,a_{i_n})\mapsto\sum\phi(a_{i_1},\ldots,a_{i_n})x_1^{i_1}\cdots x_n^{i_n}\]
is surjective.
\end{proof}

\begin{defn}
Given a ring map $A\to B$ we say that an element $b\in B$ is \textbf{integral} over $A$ if there exists a monic $P(T)\in A[T]$ such that $P(b)=0$ in $B$.
\end{defn}

\begin{lem}[Horrible lemma]
\label{L2}
Suppose $f\in k[x_1,\ldots, x_n]$ is non-zero. Pick natural numbers $e_1\gg e_2\gg\ldots\gg e_{n-1}$. Then $f(x_1+x_n^{e_1},\ldots, x_{n-1}+x_n^{e_{n-1}}, x_n)$ is of the form $a x_n^N+\text{lower order terms}$, where $a\in k^\times$.
\end{lem}
\begin{proof}
Write $f=\sum_{I\in k}a_I x^I$ with $a_I\neq 0$ for all $I\in k$, where $k$ is a finite set of multi-indices. Substituting, we get something of the form
\[(x_1+x_n^{e_1})^{i_i}\cdots (x_{n-1}+x_n^{e_{n-1}})^{i_{n-1}}x_n^{i_n}=x_n^{i_1e_1+\ldots+i_{n-1}e_{n-1}+i_n}.\]
It suffices to show that if $I,I'\in k$, for distinct $I,I'$ we have that
\[i_1e_1+\ldots+i_{n-1}e_{n-1}+i_n\neq i'_1e_1+\ldots+ i'_{n-1}e_{n-1}+i'_n.\]
If $I$ is lexicographically larger than $I'$ then the left hand side is greater than the right hand side.
\end{proof}

\begin{lem}
\label{L3}
Suppose we have $A\to B\to C$ ring maps. If $A\to C$ is finite, then $B\to C$ is finite.
\end{lem}
\begin{proof}
Trivial.
\end{proof}

\begin{lem}
\label{L4}
Suppose we have $A\to B\to C$ ring maps. If $A\to B$ and $B\to C$ are finite, then $A\to C$ is finite as well.
\end{lem}
\begin{proof}
Trivial.
\end{proof}

We now have enough machinery to prove Noether normalization.

\begin{proof}
Let $A$ be as in the theorem. We write $A=k[x_1,\ldots, x_n]/I$. We proceed by induction on $n$. For $n=0$, we simply have $A=k$, and we can take the identity map $k\to A$, which is clearly finite and injective. Now suppose the statement holds for $n-1$, i.e. for algebras generated by $n-1$ or fewer elements. If the generators $x_1,\ldots,x_n$ are algebraically independent over $k$ (i.e. $I=0$), we are done and we may take $r=n$ and $y_i=x_i$. If not, pick a non-zero $f\in I$. For $e_1\gg e_2\gg\ldots\gg e_{n-1}\gg 1$, set
\[y_1=x_1-x_n^{e_1},\ldots y_{n-1}=x_{n-1}-x_n^{e_{n-1}},x_n=x_n,\]
and consider $f(x_1,\ldots, x_n)=f(y_1+x_n^{e_1},\ldots,y_{n-1}+x_n^{e_{n-1}},y_n)$. By  Lemma \ref{L2}, we see that this polynomial is monic in $x_n$ and hence, since $x_i$ are integral over $A$, we conclude (by Lemma \ref{L1}) that $A=k[x_1,\ldots,x_n]/I$ is finite over $B=k[y_1,\ldots, y_{n-1}]$. To show that $B\to A$ is injective, let $J =$ Ker$(B \to A)$ and replace $B$ by $B/J$. Now $B/J \to A$ is injective, and by Lemma \ref{L3} it is finite. But since $B/J$ is finite over $k[y_1,\ldots,y_r]$ by the induction hypothesis, $A$ must be as well (see Lemma \ref{L4}), and we are done.
\end{proof}


\section*{Class 2}

    Let $A$ be a ring. Then we define the \textbf{spectrum of} $A$, $\Spec A$, to be the set of prime ideals of $A$.
    Note that $\Spec(-)$ is a contravariant functor in the sense that if $\phi:A\to B$ is a ring map we get a map $\Spec(\phi):\Spec(B)\to\Spec(A)$ given by $q\mapsto \phi^{-1}(\fr q)$. In order for this to work we want $\fr q$ prime in $B \Leftrightarrow \phi^{-1}(\fr q)$ prime in $A$. Notice that $\fr q$ prime implies that $B/\fr q$ is a domain. But $\phi$ induces a homomorphism $A/\phi^{-1}(\fr q) \to B/\fr q$ and this homomorphism must preserve multiplication. In particular, if the product of two elements in $A/\phi^{-1}(\fr q)$ is 0, then so is the product of their images in $B/\fr q$. So $B/\fr q$ domain implies that $A/\phi^{-1}(\fr q)$ is a domain. Then $\phi^{-1}(\fr q)$ is prime. The converse can be proved in the same way.
    

\begin{rem}
    Abuse of notation: often we write $A\cap \fr q$ for $\phi^{-1}(\fr q)$ even if $\phi$ is not injective. Note also that $\Spec(-)$ is in fact a functor from $\mathbf{Ring}$ to $\mathbf{Top}$, though we will postpone discussion about topology until later.
\end{rem}

\begin{exmp}
    Consider $\Spec(\C[x])$. Since $\C[x]$ is a PID (and thus a UFD), the primes are principal ideals generated by irreducibles, i.e.
    linear terms. Hence $\Spec(\C[x])=\left\{ (0),(x-\lambda) | \lambda\in C \right\}$. Consider $\phi:\C[x]\to\C[y]$, given by $x\mapsto y^2$.
    Set $\fr q_\lambda=(y-\lambda)$ and $\fr p_\lambda=(x-\lambda)$. Then $\Spec(\phi)(\fr q_\lambda)=\fr p_{\lambda^2}$. Why is this? 
    First note that $\phi(\fr p_{\lambda^2})=(x^2-\lambda^2)=(x-\lambda)(x+\lambda)\subset\fr p_\lambda$, which gives us an inclusion.
    %We have that $x-\lambda^2\mapsto y^2-\lambda^2=(y+\lambda)(y-\lambda)\in q_\lambda$.
    Additionally, we have that $\Spec(\phi)\left( (0) \right)=(0)$. Since this is everything in $\Spec\C[y]$, we have equality.  Note that the fibres of $\Spec(\phi)$ are finite!
\end{exmp}

Indeed, the goal of the next couple lectures will be to show that the fibres of maps on spectra of a finite ring map are finite.

Let us start by considering the following setup. Let $\phi: A\to B$ be a ring map and $\fr p\subset A$ a prime ideal. What is the fibre of $\Spec(\phi)$ over $\fr p$? First of all, note that if $\phi^{-1}(\fr q)=\fr q\cap A=\fr p$, then $\fr p B=\phi(\phi^{-1}(\fr q))B\subset\fr q$.

\begin{lem}
\label{L5}
    If $I\subset A$ is an ideal in a ring $A$ then the ring map $A\to A/I$ induces via $\Spec(-)$ a bijection $\Spec(A/I)\leftrightarrow V(I)=\left\{ \fr p\in\Spec(A) | I\subset \fr p \right\}$.
\end{lem}
\begin{proof}
    We use the fact that the ideals of $A/I$ are in 1-to-1 correspondence with ideals of $A$ containing $I$. We wish to extend this to prime ideals. By the third isomorphism theorem, given $J\subset I\subset A$, we have that $A/I\cong(A/J)/(I/J)$. We see that $A/I$ is a domain iff $I/J$ is prime in $A/J$ iff $I$ is prime in $A$; this gives us the 1-to-1 correspondence.
\end{proof}

%\begin{rem}
    %The \textbf{Zariski topology} has as closed subsets the sets $V(I)$.
%\end{rem}
\begin{rem}
Consider next the following two diagrams. 
\[
\begin{tikzcd}
B\arrow{r} & B/\fr pB\\
A\arrow{u}{\phi}\arrow{r} & A/\fr p\arrow[swap]{u}{\bar\phi}
\end{tikzcd}
\longleftrightarrow
\begin{tikzcd}
\Spec B\arrow[swap]{d}{\Spec(\phi)} & \arrow{l}\arrow{d}{\Spec(\bar\phi)}\Spec(B/\fr pB)\\
\Spec A & \arrow{l}\Spec(A/\fr p)
\end{tikzcd}
\]
Clearly the point $\fr p \in$ Spec $A$ corresponds to $(0) \in$ Spec$(A/\fr p)$. Thus, by Lemma \ref{L5}, points in the fibre of $\Spec(\phi)$ over $\fr p$ are in 1-1 correspondence with points in the fibre of $\Spec(\bar \phi)$ over $(0)\in\Spec(A/\fr p)$. This fact will be very important for our proofs later on.
\end{rem}

\begin{lem}
\label{L6}
    If $k$ is a field, then $\Spec(k)$ has exactly one point. If $k$ is the fraction field of a domain $A$, then $\Spec(k)\to\Spec(A)$
    maps the unique point to $(0)\in\Spec(A)$.
\end{lem}
\begin{proof}
    The only ideals of a field $k$ are $(0)$ and $k$ itself. The sole prime ideals is thus $(0)$ and hence $\Spec(k)$ has only one point.
    If $k$ is the fraction field of the domain $A$ then we have an injective map $A\to k$ which clearly pulls $(0)\subset k$ back to $(0)\subset A$. 
\end{proof}

Next we wish to invert some elements in $B/\fr p B$. More specifically, since we are interested in the ideals of $B/\fr p B$ that are mapped to $(0)$ by Spec$(\bar \phi)$, we would like to 'throw out' the other ones. We do this by creating inverses for elements of $A/\fr p - \{0\}$, such that none of them will be primes anymore. (See example \ref{E3} below for how this works.) This leads to a very general notion of localization, which we discuss in detail for the rest of the lecture.

\begin{defn}
    Let $A$ be a ring. A \textbf{multiplicative subset} of $A$ is a subset $S\subset A$ such that $1\in S$ and if $a,b\in S$,
    then $ab\in S$.
\end{defn}

\begin{defn}
    Given a multiplicative subset $S$, we can define the \textbf{localization of $A$ with respect to $S$}, $S^{-1}A$, as the set of pairs $(a,s)$ with $a\in A,s\in S$ modulo the equivalence relation
    $(a,s)\sim(a',s')\iff \exists s''\in S$ such that $s''(as'-a's)=0$ in $A$. Elements of $S^{-1}A$ are denoted $\frac{a}{s}$. Addition proceeds
    as usual. One checks that this is indeed a ring.
\end{defn}

\begin{lem}
\label{L7}
    The ring map $A\to S^{-1}A$ given by $a\mapsto \frac{a}{1}$ induces a bijection $\Spec(S^{-1}A)\leftrightarrow\left\{ \fr p\subset A|S\cap \fr p=\varnothing \right\}$.
\end{lem}
\begin{proof}
It's easy to show that $\Spec(\phi) (S^{-1}(A)) \subset \{ \fr p\subset A|S\cap \fr p=\varnothing \}$. Let $\fr q$ be a prime in $S^{-1}A$; if $\phi^{-1}$ contains some $s\in S$, then $\phi(s) \in \fr q$. But $\phi(s)$ is a unit, so $\fr q = S^{-1}A$. We omit the proof of the converse.
\end{proof}
Note that any element of $S$ becomes invertible in $S^{-1}A$ so it is not in any prime ideal of $S^{-1}A$. 

\begin{exmp}
\label{E3}
    Suppose $A=\C[x]\to B=\C[y]$ with $x\mapsto 5y^2+3y+2$. Then $\Spec(\phi)^{-1}\left( (x) \right)=\Spec\left( (A/\fr p-\left\{ 0 \right\})^{-1} B/\fr p B \right)=\Spec\left( (\C^\times)^{-1}\C[y]/(5y^2+3y+2) \right)=\Spec\left(\C[y]/(5y^2+3y+2) \right)$. There are two points in this space, since this quadratic factors into two prime ideals containing the ideal generated by this quadratic (see Lemma \ref{L5}). More generally, one may refer to the following diagram, which will be very useful next lecture.
\end{exmp}

\[
\begin{tikzcd}
B\arrow{r}&B/\fr pB\arrow{r} & \bar\phi\left(A/\fr p-\{0\}\right)^{-1}B/\fr pB\\
A\arrow{r}\arrow{u} & A/\fr p\arrow{r}\arrow{u}&\text{Fr}(A/\fr p)=\left(A/\fr p-\{0\}\right)^{-1}A/\fr p\arrow{u}
\end{tikzcd}
\]

Given $S\subset A$ multiplicative, and an $A$-module $M$, we can form an $S^{-1}A$-module
\[S^{-1}M=\left\{ \frac{m}{s}|m\in M,s\in S \right\}/\sim\]
where the equivalence relation is the same as before. The construction $M\to S^{-1}M$ is a functor $\mathbf{Mod}_A\to\mathbf{Mod}_{S^{-1}A}$.

\begin{lem}
\label{L8}
    The localization functor $M\to S^{-1}M$ is exact.
\end{lem}
\begin{proof}
    Suppose the sequence $0\to M'\overset{\alpha}{\to} M\overset{\beta}{\to} M''\to 0$ is exact. We wish to show that the sequence $0\to S^{-1}M'\overset{S^{-1}\alpha}{\to} S^{-1}M\overset{S^{-1}\beta}{\to} S^{-1}M''\to 0$ is exact. Let us first show that this sequence is exact at $S^{-1}M$, i.e. that $\text{Im} (S^{-1}\alpha)=\ker(S^{-1}\beta)$. Pick $m'/s\in S^{-1}M'$. We take $S^{-1}\alpha(m'/s)=\alpha(m')/s$ and then compute $S^{-1}\beta(\alpha(m')/s)=\beta(\alpha(m'))/s=0$ by the given exactness. This shows the inclusion $\text{Im} (S^{-1}\alpha)\subset\ker(S^{-1}\beta)$. Next, choose an element $m/s\in \ker(S^{-1}\beta).$ Then $\beta(m)/s=0$ in $S^{-1}M''$, i.e. there exists a $t\in S$ such that $t\beta(m)=0$ in $M''$. Since $\beta$ is a $A$-module homomorphism, $t\beta(m)=\beta(tm)$ and so $tm\in\ker(\beta)=\text{Im}(\alpha)$. Therefore $tm=\alpha(m')$ for some $m'\in M'$. Hence we have $m/s=\alpha(m')/st=(S^{-1}\alpha)(m'/st)\in\text{Im}(S^{-1}\alpha)$, which demonstrates the reverse inclusion.
    
    The rest of the proof is left as a exercise.
\end{proof}

\begin{rem}
An exact functor is one that preserves quotients. What Lemma \ref{L8} says is that if $N\subset M$ then $S^{-1}M/S^{-1}N \cong S^{-1}(M/N)$. In particular, if $I\subset A$ is an ideal, then $S^{-1}(A/I) = S^{-1}A/S^{-1}I$.
\end{rem}

\begin{rem}
If $A \overset{\phi}{\to} B$, then $S^{-1}B$ is an $S^{-1}A$-algebra and $S^{-1}B \cong \big(\phi(S)\big)^{-1} B$.
\end{rem}

\begin{defn}
Let $A$ be a ring and $\fr p\subset A$ be a prime ideal, then $A_{\fr p} = (A-\fr p)^{-1}A$ is the \textbf{local ring of $A$ at $\fr p$} (or the localization of $A$ at $\fr p$). If $M$ is an $A$-module, then we set $M_{\fr p} = (A-\fr{p})^{-1}M$.
\end{defn}

\begin{defn}
A \textbf{local ring} is a ring with a unique maximal ideal.
\end{defn}

\begin{lem}
\label{L9}
$A_{\fr{p}}$ is a local ring.
\end{lem}
\begin{proof}
Consider the quotient $A_\fr p/\fr pA_\fr p$.  By the remark above, we 
can factor $A_{\fr{p}} / \fr{p}A_{\fr{p}} = (A - \fr{p})^{-1} (A/\fr{p})$. This is justified because $A_\fr p=(A-\fr p)^{-1}A$ by definition and because $\fr pA_\fr p=(A-\fr p)^{-1}\fr p$ for some reason. Next, by the remark directly above, if we let $\phi:A\to A/\fr p$ be the natural surjection, then $(A-\fr p)^{-1}(A/\fr p)=(\phi(A-\fr p))^{-1}(A/\fr p)=(A/\fr p-\{0\})^{-1}(A/\fr p)$. But this is just the fraction field of $A/\fr p$, i.e. $A_\fr p/\fr pA_\fr p$ is a field. Hence $\fr pA_\fr p$ is maximal.

This is the unique maximal ideal because by Lemma \ref{L7}, the primes of $A_{\fr p}$ are the primes $\fr{q}\subset A$ that do not intersect $A-\fr{p}$. This implies that $q\subset p$, and thus $\fr q$ cannot be maximal unless $\fr q = \fr p$.
\end{proof}




\section*{Class 3}

\begin{lem}
\label{L10}
\begin{enumerate}[(a)]
Let $A \overset{\phi}{\to} B$ be a finite ring map. Then
\item for $I \subset A$ ideal, the ring map $A/I \to B/IB$ is finite;
\item for $S\subset A$ multiplicative subset, $S^{-1}A \to S^{-1}B$ is finite;
\item for $A\to A'$ ring map, $A' \to B \otimes_A A'$ is finite.
\end{enumerate}
\end{lem}
\begin{proof}
\begin{enumerate}[(a)]
\item Consider the following diagram:
\[
\begin{tikzcd}
B\arrow{r} & B/IB\\
A\arrow{u}\arrow{r} & A/I\arrow{u}
\end{tikzcd}
\]
By Lemma $\ref{L4}$ we see that if the map $B\to B/IB$ is finite, then so is $A\to B/IB$, which would imply that (by Lemma $\ref{L3}$) $A/I\to B/IB$ is finite. But $B\to B/IB$ is obviously finite, as it is generated as a $B$-module by $\{1\}$.
\item Since $A\to B$ is finite, there exists a surjection $A^{\oplus n}\twoheadrightarrow B$. The statement that $S^{-1}A\to S^{-1}B$ is finite follows immediately from the fact that localization is exact and hence preserves surjectivity of $(S^{-1}A)^{\oplus n}\twoheadrightarrow S^{-1}B$.
\item We haven't yet discussed tensor products, so we will leave this for now.
\end{enumerate}
\end{proof}

\begin{lem}
\label{L11}
Suppose $k$ is a field, $A$ is a domain and $k \to A$ a finite ring map. Then $A$ is a field.
\end{lem}
\begin{proof}
Since $A$ is an algebra, multiplication by an element $a\in A$ defines a $k$-linear map $A\to A$. The map is also injective: Ker$(a) = \{ a' \in A | aa' = 0 \} = \{ 0 \}$, because $A$ has no zero divisors. But, since dim$_k (A)$ is finite, injectivity implies surjectivity. Then there exists $a''$ such that $aa'' = 1$, so $a$ is a unit.
\end{proof}

\begin{lem}
\label{L12}
Let $k$ be a field and $k\to A$ a finite ring map. Then:
\begin{enumerate}[(a)]
\item Spec$(A)$ is finite.
\item there are no inclusions among prime ideals of $A$.
\end{enumerate}
In other words, $\Spec(A)$ is a finite discrete topological space with respect to the Zariski topology.
\end{lem}
\begin{proof}
For some $\fr p$ prime in $A$, $A/\fr p$ is a domain and the natural map $k\to A/\fr p$ is finite since $k\to A$ and $A\to A/\fr p$ are both finite. 
By Lemma $\ref{L11}$ we see that $A/\fr p$ must be a field, and that $\fr p$ must be maximal. Hence all primes of $A$ are maximal. This shows $(b)$, as there can be no inclusions among maximal ideals. Moreover, by the Chinese remainder theorem (see Lemma $\ref{L13}$ below) the map $A \to A/\fr m_1 \times \ldots \times A/\fr m_n$ is surjective.
Since $A$ and its quotients are vector spaces, this translates into a statement about their dimension: $\dim_k A \geq \sum_i \dim_k A/\fr m_i \geq n$. Thus $n$ is finite, which shows $(a)$.
\end{proof}

\begin{lem} [Chinese remainder theorem]
\label{L13}
Let $A$ be a ring, and $I_1, ... , I_n$ ideals of $A$ such that $I_i + I_j = A , \forall i\neq j$. Then there exists a surjective ring map $A \twoheadrightarrow A/I_1 \times ... \times A/I_n$ with kernel $I_1 \cap ... \cap I_n = I_1 ... I_n$.
\end{lem}
\begin{proof}
Omitted.
\end{proof}

\begin{lem}
\label{L14}
Let $A \overset{\phi}{\to} B$ be a finite ring map. The fibres of Spec$(\phi)$ are finite.
\end{lem}
\begin{proof}
Consider the following diagram:
\[
\begin{tikzcd}
B\arrow{r}&B/\fr pB\arrow{r} & B_\fr p/\fr pB_\fr p=\bar\phi\left(A/\fr p-\{0\}\right)^{-1}B/\fr pB\\
A\arrow{r}\arrow{u}{\phi} & A/\fr p\arrow{r}\arrow{u}{\bar\phi}&\text{Fr}(A/\fr p)=\left(A/\fr p-\{0\}\right)^{-1}A/\fr p\arrow{u}
\end{tikzcd}
\]
By $(a)$ and $(b)$ of Lemma \ref{L10}, $\bar\phi$ and $\text{Fr}(A/\fr p)\to B_\fr p/\fr pB_\fr p$ are finite. Now recall that the points in the fibre of $\Spec(\phi)$ over $\fr p\in\Spec(A)$ correspond to points in the fibre of $\Spec(\bar \phi)$ over $(0)\in\Spec(A/\fr p)$. If we now look at the third column of the diagram, we see that since $\text{Fr}(A/\fr p)$ is a field, Lemma \ref{L12} implies that $\Spec(B_\fr p/\fr pB_\fr p)$ is finite. Hence there must be a finite number of points in $\Spec(B_\fr p/\fr pB_\fr p)$ that map to $(0)\in\Spec(\text{Fr}(A/\fr p))$, and thus (again arguing via correspondence), the points in the fibre of $\Spec(\phi)$ over $\fr p\in\Spec(A)$ must be finite.
\end{proof}

\begin{lem}
\label{L15}
Suppose that $A\subset B$ is a finite extension (i.e. there exists a finite injective map $A\to B$). Then Spec$(B) \to $ Spec$(A)$ is surjective.
\end{lem}
\begin{proof}
We want to reduce the problem to the case where $A$ is a local ring. For this, let $p\subset A$ be a prime. By part b of Lemma \ref{L10}, the map $A_p \to B_p$ is finite. By Lemma \ref{L8}, the same map is injective. Then we can replace $A$ and $B$ in the statement of the lemma by $A_p$ and $B_p$.

Now, assuming that $A$ is local, $p$ is the maximal ideal of $A$, and we denote it by $m$ in what follows. The following statements are equivalent:
\begin{align*}
 \exists q \subset B \text{ lying over } m &\Leftrightarrow \exists q\subset B \text{ such that } mB \subset q
 \\   & \Leftrightarrow B/mB \neq 0
\end{align*}
But the last statement is always true, since Nakayama's lemma (see below) says that $mB = B$ implies $B = 0$.
\end{proof}

\begin{lem} [Nakayama's lemma]
\label{L16}
Let $A$ be a local ring with maximal ideal $m$, and let $M$ be a finite $A$-module such that $M = mM$. Then $M=0$.
\end{lem}
\begin{proof}
Let $x_1, ... , x_r \in M$ be generators of $M$. Since $M = mM$ we can write $x_i = \sum_{j=1}^r a_{ij} x_j$, for some $a_{ij} \in m$. Then define the $r\times r$ matrix $B = 1_{r\times r} - (a_{ij})$. The above relation for the generators translates into:
\[     B \left(  \begin{array} {c} x_1 \\ \vdots \\ x_r  \end{array}  \right)  = 0   \]
Now consider $B^{\text{ad}}$, the matrix such that $B^{\text{ad}} B =$ det$(B) 1_{r\times r}$. Multiplying the above equation on the left by $B^{\text{ad}}$ we obtain:
\[    \text{det}(B) \left(  \begin{array} {c} x_1 \\ \vdots \\ x_r  \end{array}  \right)  = 0   \]
Thus det$(B) x_i = 0$ for all $i$. If we assume that the generators of $M$ are nonzero, the fact that det$(B)$ annihilates all generators implies that it is equal to 0. But, by expanding out the determinant of $B = 1_{r\times r} - (a_{ij})$, we see that it is of the form $1+a$ for some $a\in m$. Since $(A,m)$ is a local ring, this implies that det$(a)$ is a unit. A unit cannot be zero in $(A, m)$, so this is a contradiction. Thus all generators of $M$ are zero, and $M=0$.
\end{proof}

\begin{lem} [Going up for finite ring maps]
\label{L17}
Let $A \to B$ be a finite ring map, $p$ a prime ideal in $A$ and $q$ a prime ideal in $B$ which belongs to the fibre of $p$. If there exists a prime $p'$ such that $p \subset p' \subset A$, then there exists a prime $q'$ such that $q\subset q' \subset B$ and $q'$ belongs to the fibre of $p'$.
\[
\begin{tikzcd}
B & & q\arrow{d}\arrow[dashed, hook]{r} & ?\arrow[dashed]{d} \\
A\arrow{u} &   & p\arrow[hook]{r} & p'
\end{tikzcd}
\]
\end{lem}
\begin{proof}
Consider $A/p \to B/q$. This is injective since $p = A \cap q$ and finite by Lemma \ref{L3}. $p'/p$ is a prime ideal in $A/p$, and by Lemma \ref{L15} its preimage is nonempty. Thus there exists a prime $q'/q$ in $A/p$ which maps to $p'/p$, and this corresponds to a prime $q'$ in $B$ that contains $q$.
\end{proof}



\section*{Class 4}

\begin{lem}
\label{L18}
The following are equivalent for a ring $A$:
\begin{enumerate}[(1)]
\item $A$ is local;
\item Spec$(A)$ has a unique closed point;
\item $A$ has a maximal ideal $m$ such that every element of $A-m$ is invertible;
\item $A$ is not zero and $x\in A \Rightarrow x\in A^*$ or $1-x \in A^*$.
\end{enumerate}
\end{lem}

\begin{proof}
$(1) \Leftrightarrow (2)$ In the Zariski topology for Spec$(A)$, a closed set looks like $V(\fr p)$ for some prime $\fr p$. Therefore a closed point is a maximal ideal.
\\
\\
$(1) \Rightarrow (3)$ Let $m\subset A$ be the maximal ideal and take $x\not \in m$. Then $V(x) = \emptyset$, and, by Lemma \ref{L19}, $x$ is invertible.
\\
\\
$(3) \Rightarrow (4)$ If $x\not \in m$ then $x$ is invertible, so assume $x\in m$. But then $1-x \not \in m$, since this would imply $1\in m$. Therefore $1-x$ is invertible.
\\
\\
$(4) \Rightarrow (1)$ Let $m = A - A^*$. It's easy to show that $m$ is an ideal. Moreover, $m$ is maximal: assume $m\subset I$ and $m\neq I$, then $I$ must contain a unit, and so $I = A$. There can be no other maximal ideal, since all elements of $A-m$ are units.
\end{proof}


\begin{lem}
\label{L19}
For $x\in A$, $A$ local, $V(x) = \emptyset \Leftrightarrow x\in A^*$.
\end{lem}
\begin{proof}
The $\Leftarrow$ direction is trivial. For the converse, note that by Lemma \ref{L7}:
\[ V(x) = \emptyset \Rightarrow \Spec(A/xA) = \emptyset \Leftrightarrow A/xA = 0 \Leftrightarrow x \text{ unit} \]
\end{proof}

\begin{exmp}
\label{E4}
Examples of local rings:
\begin{enumerate} [(a)]
\item fields, the maximal ideal is $(0)$.
\item $\C[[z]]$, power series ring, the maximal ideal is $(z)$. Note that something of the form $z-\lambda$ is invertible by some power series, and thus cannot be maximal.
\item for $X$ topological space and $x\in X$, $O_{X,x}$, the ring of germs of continuous $\C$-valued functions at $x$. The maximal ideal is $m_x = \{(U,f)\in O_{X,x} | f(x) = 0 \}$. Note that, if $g\not \in m_x$, then $g\neq 0$ on a neighborhood of $x$, because of continuity. Therefore $g$ is invertible on this neighborhood. Then, by Lemma \ref{L18}, $m_x$ is maximal.
\item for $k$ a field, $k[x]/(x^n)$, the maximal ideal is $(x)/(x^n)$.
\end{enumerate}
\end{exmp}

For the rest of the lecture, we examine the closedness of maps on spectra.
\begin{defn}
Let $X$ be a topological space, $x,y\in X$. We say that \textbf{$x$ specializes to $y$} or \textbf{$y$ is a generalization of $x$} if $y\in \overline{\{x\}}$. We denote this as $x\leadsto y$.
\end{defn}

\begin{exmp}
\label{E5}
In Spec$\Z$ we have $(0) \leadsto (p)$ for all primes $p$, but not $(p) \leadsto (0)$ or $(p) \leadsto (q)$, unless $p = q$.
\end{exmp}

\begin{lem}
\label{L20}
The closure of $\fr p$ in Spec$(A)$ is $V(\fr p)$. In particular, $\fr p \leadsto \fr q$ iff $\fr p \subset \fr q$.
\end{lem}
\begin{proof}
\[       \overline{\{\fr p\}} = \bigcap_{I\subset \fr p} V(I) = V(\bigcap_{I \subset \fr p} I) = V(\fr p)      \]
\end{proof}

\begin{lem}
\label{L21}
The image of Spec$(A_{\fr p}) \to \Spec(A)$ is the set of all generators of $\fr p$.
\end{lem}
\begin{proof}
By Lemma \ref{L7}, there is a bijection between primes of $A_{\fr p}$ and primes of $A$ contained in $\fr p$. But the latter are all ideals generated by a subset of the generators of $\fr p$, and in particular the generators themselves.
\end{proof}

\begin{defn}
A subset $T$ of a topological space is \textbf{closed under specialization} if $x\in T$ and $x \leadsto y$ imply $y\in T$.
\end{defn}

Notation: for $f\in A$, let $D(f) = \Spec(A) - (f) = \{\fr q \in A | f\not \in \fr q\}$. Obviously $D(f)$ is open.

\begin{lem}
\label{L22}
Let $A \to B$ be a ring map. Set $T = \text{Im} \big(\Spec(B) \to \Spec(A) \big)$. If $T$ is closed under specialization then $T$ is closed.
\end{lem}
\begin{proof}
Suppose $\fr p\in\bar T$. Then every open neighborhood of $\fr p$ contains a point of $T$. Now pick $f\in A\setminus{\fr p}$. Then $D(f)\subset\Spec A$ is an open neighborhood of $\fr p$. Then there exists a $\fr q\subset B$ with $\Spec(\phi)(\fr q)\in D(f)$, which implies that ther exists a $\fr q\subset B$ such that $\phi(f)\neq\fr q$. Hence $B_f\neq 0$.

Thus we see $\phi(f)\cdot 1\neq 0$ for all $f\in A\setminus\fr p$. Hence $B_\fr p\neq 0$ ($1\neq 0$) and thus $\Spec(B_\fr p)\neq\varnothing$. We conclude (Lemma \ref{L21}) that there exists a $\fr q'\subset B$ such that $\fr p'=\phi^{-1}(\fr q')\in T$ is a generalization of $\fr p$, i.e. $\fr p$ is a specialization of a point of $T$, and we conclude that $\fr p\in T$.
\end{proof}

\begin{lem}
\label{L23}
If going up holds from $A$ to $B$, then $\Spec(\phi)$ is closed as a map of topological spaces.
\end{lem}
\begin{proof}
Let $Z\subset \Spec(B)$ be a closed subset; we want to show that its image is closed. In the Zariski topology closed sets look like $V(J)$ for some prime $J$, and by Lemma \ref{L7} we have $Z = \text{Im} \big(\Spec(B/J) \to \Spec(B) \big)$. Then:
\[
\begin{tikzcd}
    {}& \Spec(B/J)\arrow{dl}\arrow{dr}{\cong} & \\
\Spec(B)\arrow{d}[swap]{\Spec(\phi)} & & Z\arrow{d}{\Spec(\phi)}\arrow[hook]{ll} \\
\Spec(A) & & \Spec(\phi)(Z) \arrow[hook]{ll}
\end{tikzcd}
\]
Note that $\Spec(\phi)(Z) = \text{Im}\big(\Spec(B/J) \to \Spec(A)\big)$. By Lemma \ref{L22} it suffices to show that $\Spec(\phi)(Z)$ is closed under specialization. That is, if there exists some prime $\fr p' \subset A$ which specializes to another prime $\fr p$ and is the image of a prime $\fr q' \subset B$, then $\fr p$ is also the image of some prime $\fr q \subset B$. Suppose we have the solid part of the diagram; by going up we can find $\fr q$ fitting into the diagram below, therefore $p\in \Spec(\phi)(Z)$ as long as $p' \in \Spec(\phi)(Z)$.
\[
\begin{tikzcd}
B & & J\arrow[hook]{r} & q'\arrow{d}\arrow[dashed, hook]{r} & ?\arrow[dashed]{d} \\
A\arrow{u} & &   & p'\arrow[hook]{r} & p
\end{tikzcd}
\]
\end{proof}


\section*{Class 5: Krull dimension}

\begin{defn}\hspace{1mm}
\begin{enumerate}[(a)]
\item A topological space $X$ is \textbf{reducible} if it can be written as the union $X=Z_1\cup Z_2$ of two closed, proper subsets $Z_i$ of $X$. A topological space is \textbf{irreducible} if it is not reducible.
\item A subset $T\subset X$ is called \textbf{irreducible} iff $T$ is irreducible as a topological space with the induced topology.
\item An \textbf{irreducible component} of $X$ is a maximal irreducible subset of $X$.
\end{enumerate}
\end{defn}

\begin{exmp}\hspace{1mm}
\begin{enumerate}[(a)]
\item In $\R^n$ with the usual topology, the only irreducible subsets are the singletons. This is true in general for any Hausdorff topological space.
\item $\Spec\Z$ is irreducible.
\item If $A$ is a domain, then $\Spec A$ is irreducible. This is because $(0)\in V(I)\iff I\subset(0)\iff I\subset(0)\iff I=(0)\implies V(I)=\Spec A$.
\item $\Spec(k[x,y]/(xy))$ is reducible because it is $V(x)\cup V(y)$: geometrically speaking, the coordinate axes
\end{enumerate}
\end{exmp}

\begin{lem}
\label{L24}
Let $X$ be a topological space.
\begin{enumerate}[(a)]
\item If $T\subset X$ is irreducible so is $\bar T\subset X$;
\item An irreducible component of $X$ is closed;
\item $X$ is the union of its irreducible components, i.e. $X=\cup_{i\in I}Z_i$ where $Z_i\subset X$ are closed and irreducible with no inclusions among them. 
\end{enumerate}
\end{lem}
\begin{proof}
Omitted.
\end{proof}

\begin{lem}
\label{L25}
Let $X=\Spec A$ where $A$ is a ring. Then,
\begin{enumerate}[(a)]
\item $V(I)$ is irreducible if and only if $\sqrt{I}$ is a prime;
\item Any closed irreducible subset of $X$ is of the form $V(\fr p)$, $\fr p$ a prime;
\item Irreducible components of $X$ are in one-to-one correspondence with the minimal primes of $A$
\end{enumerate}
\end{lem}

\begin{proof}\hspace{1mm}
\begin{enumerate}[(a)]
\item $V(I)=\{\fr p:I\subset\fr p\}=V(\sqrt{I})$ so we may replace $I$ by $\sqrt{I}$. For the backwards direction, let $I$ be a prime. Then $A/I$ is a domain, so $\Spec(A/I)=V(I)$ by a previous lemma (this is true both as sets and topologies), which is irreducible by the example $(c)$ above. Conversely, if $V(I)$ is irreducible and $ab\in I$, then
\[V(I)=V(I,a)\cup V(I,b).\]
By irreduciblity we have that $V(I)=V(I,a)$ or $V(I)=V(I,b)$. This implies that either $a\in I$ or $b\in I$ by Lemma \ref{L26} below.
\item Omitted.
\item Omitted.
\end{enumerate}
\end{proof}

\begin{lem}
\label{L26}
$\sqrt{I}=\cap_{I\subset\fr p} \fr p$
\end{lem}
\begin{proof}
That the left-hand side is included in the right-hand side is clear. Conversely, suppose $f$ is contained in the right-hand side. Then $\Spec((A/I)_f)=\varnothing$ and hence $(A/I)_f=0$ as a ring. This implies that $f^n\cdot 1=0$ in $A/I$, and hence that $f^n\in I$.
\end{proof}

\begin{defn}
Let $X$ be a topological space. We set
\[\dim X=\sup\left\{ n | \exists Z_0 \subsetneq Z_1\subsetneq \cdots\subsetneq Z_n\subset X \right\}\]
with $Z_i\subset X$ irreducible and closed. We call $\dim X$ the \textbf{Krull} or \textbf{combinatorial} dimension of $X$. Furthermore, for $x\in X$ and for $U\ni x$ open subsets of $X$, we set
\[\dim_x X=\min_U\dim U,\]
which is called the \textbf{dimension of $X$ at x}.
\end{defn}

\begin{lem}
\label{L27}
Let $A$ be a ring. The dimension of $\Spec A$ is
\[\dim\Spec A=\sup\{n | \exists \fr p_0\subsetneq\fr p_1\subsetneq\cdots\subsetneq p_n\subset A \},\]
(for $\fr p_i$ primes) and is called the dimension of $A$.
\end{lem}
\begin{proof}
Clear from Lemma \ref{L25}.
\end{proof}

\begin{lem}
\label{L28}
Let $A$ be a ring. Then
\[\dim A=\sup_{\fr p\subset A}\dim A_\fr p=\sup_{\fr m\subset A}\dim A_\fr m.\]
\end{lem}

\begin{defn}
If $\fr p\subset A$ is prime, then the \textbf{height} of $\fr p$ is
\[\text{ht}(\fr p)=\dim A_\fr p.\]
Informally, one might think of this as the ``codimension'' of $V(p)$ in $\Spec A$.
\end{defn}

\begin{exc}
If $\fr p\subset A$ is a prime, then $\fr p$ is a minimal prime if and only if $\text{ht}(\fr p)=0$.
\end{exc}

Let us now prove the lemma.
\begin{proof}
Any chain of primes in $A$ has a last one. If we consider
\[\fr p_0\subsetneq\cdots\subsetneq\fr p_n\]
we can localize to get the chain
\[\fr p_0A_{\fr p_n}\subsetneq\cdots\subsetneq\fr p_nA_{\fr p_n}\]
in $A_{\fr p_n}$.
\end{proof}


\begin{lem}
\label{L29}
Let $A\overset{\phi}{\to}B$ be a finite ring map such that $\Spec \phi$ is surjective. Then $\dim A=\dim B$.
\end{lem}
\begin{proof}
By our description of fibres of $\Spec(\phi)$ in the proofs of Lemma \ref{L12} and \ref{L14}, there are no strict inclusions among primes in a fibre. If we take the chain
\[\fr q_0\subsetneq\fr q_1\subsetneq\cdots\fr q_n\]
in $B$ then $A\cap\fr q_0\subsetneq\cdots\subsetneq A\cap\fr q_n$ is a chain in $A$. Hence $\dim B\leq\dim A$. On the other hand, let $\fr p_0\subsetneq\cdots\subsetneq\fr p_n$ be a chain of primes in $A$. Pick $\fr q_0$ lying over $\fr p_0$ in $B$ (since $\Spec(\phi)$ is surjective). We can now use going up to succesively pick $\fr q_0\subset\fr q_1\subset\cdots\subset\fr q_n$ lying over $\fr p_1\subset\cdots\subset\fr p_n$ (a previous lemma showed that going up holds for finite ring maps). We conclude that $\dim B\geq\dim A$.
\end{proof}

\begin{rem}
There are a few remarks to be made here:
\begin{enumerate}[(a)]
\item The proof shows that if $A\overset{\phi}{\to}B$ has going up and $\Spec(\phi)$ is surjective, then $\dim A=\dim B$. The same statement holds for going down in place of going up.
\item By Noether normalization together with Lemma \ref{L29}, we can conclude that the dimension of a finite-type algebra over a field $k$ is equal to the dimension of $k[t_1,\ldots, t_r]$ for some $r$.
\item It will turn out that $\dim k[t_1,\ldots, t_r]=r$. For now all we can say is that it is certainly greater than $r$ because we can construct the chain
\[(0)\subset(t_1)\subset(t_1,t_2)\subset\ldots\subset(t_1,\ldots,t_r).\]
\end{enumerate}
\end{rem}


Now we talk for a bit about dimension 0 rings.

\begin{defn}
An ideal $I\subset A$ is \textbf{nilpotent} if there exists $n\geq 1$ such that $I^n = 0$. It is \textbf{locally nilpotent} if $\forall x\in I, \exists n\geq 1$ such that $x^n=0$.
\end{defn}

\begin{lem}
\label{L30}
For $\fr p \subset A$ prime, the following are equivalent:
\begin{enumerate} [(a)]
\item $\fr p$ minimal
\item ht$(\fr p) = 0$
\item the maximal ideal $\fr pA_{\fr p}$ of $A_{\fr p}$ is locally nilpotent
\end{enumerate}
\end{lem}

\begin{proof}
(a)$\Leftrightarrow $(b) follows from the description of Spec$(A_{\fr p})$ in Lemma \ref{L21}. The rest follows from Lemma \ref{L31}, stated below.
\end{proof}

\begin{lem}
\label{L31}
If $(A, m)$ is local, the following are equivalent:
\begin{enumerate} [(a)]
\item dim$(A) = 0$
\item Spec$(A) = \{m\}$
\item $m$ is locally nilpotent
\end{enumerate}
\end{lem}

\begin{proof}
(b) $\Rightarrow$ (c) If $f\in m$ is not nilpotent, then $A_f \neq 0$, so Spec$(A_f) \neq 0$, so $\exists \fr p\subset A, f \not \in \fr p$, which is a contradiction; hence $\fr p = m$.
\end{proof}


\begin{defn}
A ring is \textbf{Noetherian} if every ideal is finitely generated.
\end{defn}

\begin{lem}
\label{L32}
Let $I\subset A$ be an ideal. If $I$ is locally nilpotent and finitely generated, then $I$ is nilpotent. In particular, if $A$ is Noetherian then all locally nilpotent ideals are nilpotent.
\end{lem}

\begin{proof}
If $I = (f_1, ... , f_n)$ and $f_i^{e_i} = 0$, then consider:
\[     (a_1 f_1 + ... + a_n f_n)^{(e_1 - 1) + ... + (e_n -1) + 1} = \sum (\text{binomial coefficient}) a_1^{i_1}...a_n^{i_n} f_1^{i_1}...f_n^{i_n} =0   \]
Since in each term at least one of the $i_j$ will be $\geq e_j$, which will make $f_j^{i_j} = 0$.
\end{proof}

\section*{Class 6}

\begin{defn}
Let $A$ be a ring and $M$ be an $A$-module. We say that $M$ is \textbf{Artinian} ring if it satisfies the descending chain condition on ideals. We say that $A$ is Artinian if $A$ is Artinian as an $A$-module.
\end{defn}

\begin{lem}
\label{L33}
Let
\[0\to M'\to M\to M''\to 0\]
be a short exact sequence of $A$-modules. If $M'$ and $M''$ are Artinian (of length $m,n$) then $M$ is as well (of length $\max(m,n)$).
\end{lem}
\begin{proof}
Suppose $M\subset M_1\subset\ldots$ are submodules of $M$. By assumption, there exists an $n$ such that $M_n\cap M'=M_{n+1}\cap M'=\cdots$ and there exists an $m$ such that $\pi(M_m)=\pi(M_{m+1})=\cdots$. Then $M_t=M_{t+1}=\cdots$ for $t=\max(m,n)$.
\end{proof}

\begin{lem}
\label{L34}
A Noetherian local ring of dimension 0 is Artinian.
\end{lem}
\begin{proof}
Using Lemmas \ref{L31} and \ref{L32} we get that $\fr m^n=0$ for some $n\geq 1$. So $0=\fr m^n\subset\fr m^{n-1}\subset\cdots\subset\fr m\subset A$. Then $\fr m^i/\fr m^{i+1}=(A/\fr m)^{\oplus r_i}$ is an $A/\fr m$-module generated by finitely many elements (since $A$ is Noetherian). So it is clear that $\fr m^i/\fr m^{i+1}$ is Artinian as an $A/\fr m$-module, hence over $A$. Apply Lemma \ref{L33} repeatedly.
\end{proof}

\begin{lem}
\label{L35}
If $A$ is Noetherian then so is
\begin{enumerate}[(a)]
\item $A/I$ for $I\subset A$ an ideal;
\item $S^{-1}A$ with $S\subset A$ multiplicative;
\item $A[x_1,\ldots,x_n]$;
\item any localization of a finite-type $A$-algebra.
\end{enumerate}
\end{lem}
\begin{proof}
Omitted.
\end{proof}

\begin{rem}
Any finite-type algebra over a field or over $\Z$ is Noetherian.
\end{rem}

\begin{thm}[Hauptidealsatz, v.1]
\label{L36}
Let $(A,\fr m)$ be a Noetherian local ring. If $\fr m=\sqrt{(x)}$ for some $x\in\fr m$ then $\dim A\leq 1$.
\end{thm}
\begin{proof}
Take $\fr p\subset A, \fr p\neq\fr m$. We will show $\text{ht}(p)=0$ and the theorem will follow. Observe that $x\notin\fr p$ because if it were, by primeness of $\fr p$, $\sqrt{(x)}$ would be contained in $\fr p$, which is a contradiction. Set for $n\geq 1$,
\[\fr p^{(n)}=\{a\in A | \frac{a}{1}\in\fr p^nA_\fr p\}.\]
We will use later that $\fr p^{(n)}A_\fr p=\fr p^nA_\fr p$ (proof omitted). The ring $B=A/(x)$ is local and Noetherian with nilpotent maximal ideal (since $\fr m=\sqrt{(x)}$). By Lemma \ref{L34} $B$ is Artinian. Hence
\[\frac{\fr p+(x)}{(x)}\supset\frac{\fr p^{(2)}+(x)}{(x)}\supset\frac{\fr p^{(3)}+(x)}{(x)}\supset\cdots\]
stabilizes and $\fr p^{(n)}+(x)=\fr p^{(n+1)}+(x)$ for some $n$. Then every $f\in\fr p^{(n)}$ is of the form $f=ax+b$ where $a\in A,b\in\fr p^{(n+1)}$. This implies that $\frac{a}{1}\cdot\frac{x}{1}=\frac{f-b}{1}\in\fr p^nA_\fr p$ and $\frac{x}{1}$ is a unit in $A_\fr p$. Thus $\frac{a}{1}\in\fr p^nA_\fr p$ and $a\in\fr p^{(n)}$. Hence $\fr p^{(n)}=x\fr p^{(n)}+\fr p^{(n+1)}$. Since $x\in\fr m$ and $\fr p^{(n)}$ and $\fr p^{(n+1)}$ are finite $A$-modules, Nakayama's lemma implies that $\fr p^{(n)}=\fr p^{(n+1)}$. Going back to $A_\fr p$, we get $\fr p^{(n)}A_\fr p=\fr p^{(n+1)}A_\fr p$, which implies that $\fr p^nA_\fr p=\fr p^{n+1}A_\fr p$. By Nakayama's lemma, $\fr p^nA_\fr p=0$. Finally, by Lemma \ref{L30} $\dim A_\fr p=0$, i.e. $\text{ht}(\fr p)=0$.
\end{proof}

\begin{lem}
\label{L37}
In the situation of the previous theorem, $\dim A=0$ if and only if $x$ is nilpotent and $\dim A=1$ if and only if $x$ is not nilpotent.
\end{lem}
\begin{proof}
By Lemma \ref{L31}, $\dim A=0$ if and only if $\fr m$ is locally nilpotent.
\end{proof}

\begin{lem}
\label{L38}
If $(A,\fr m)$ is a local Noetherian ring and $\dim A=1$ then there exists an $x\in M$ such that $\fr m=\sqrt{(x)}$.
\end{lem}
\begin{proof}
Since the dimension of $A$ is 1 there must exist primes other than $\fr m$, $\fr p_i$ which are all minimal. To finish the proof, we will use two facts: first, that a Noetherian ring has finitely many minimal ideals and secondly, that one can find $x\in\fr m$ with $x\notin\fr p_i$ for $i\in I$. We shall prove these lemmas below next. Assuming these facts, $V(x)=\{\fr m\}$, which implies that $\sqrt{(x)}=\fr m$.
\end{proof}

\begin{lem}[Prime avoidance]
\label{L39}
Let $A$ be a ring, $I\subset A$ an ideal, and $\fr p_1,\ldots,\fr p_n\subset A$ primes. If $I\not\subset\fr p_i$ for all $i$ then $I\not\subset\fr p_1\cup\cdots\cup\fr p_n$ (i.e. we can find a function vanishing on $I$ but not on $\fr p_i$, Urysohn's lemma).
\end{lem}
\begin{proof}
We proceed by induction on $n$. It's clearly true for $n=1$. We may assume that there are no inclusions among $\fr p_1,\cdots,\fr p_n$ (drop smaller ones). Pick $x\in I,x\notin\fr p_1\cup\cdots\cup\fr p_{n-1}$ (induction hypothesis). If $x\notin\fr p_n$, we are done; if $\fr p_1,\ldots,\fr p_{n-1}\subset\fr p_n$ then $\fr p_j\subset\fr p_n$ for some $j$ ($\fr p_n$ is prime). This contradicts previous mangling of the primes. So $\fr p_1,\ldots,\fr p_{n-1}\not\subset\fr p_n$ and $I\not\subset\fr p_n$ which implies (since $\fr p_n$ is prime) that $\fr p_1\cdots \fr p_{n-1}I\not\subset\fr p_n$. Pick $y\in\fr p_1\cdots\fr p_{n-1}I$ with $y\notin\fr p_n$. Then $x+y$ works. Indeed, $x+y\in I$, $x+y\notin\fr p_j$ for $j=1,\cdots,n-1$, and $x+y\notin\fr p_n$ ($x\in\fr p_n$ but not $y$).
\end{proof}

\begin{lem}
\label{L40}
Let $A$ be a Noetherian ring. Then
\begin{enumerate}[(a)]
\item For all ideals $I\subset A$, there exists a list of primes $\fr p_1,\ldots, \fr p_n$ such that $I\subset\fr p_i$ and $\fr p_1\cdots\fr p_n\subset I$;
\item The set of primes minimal over $I$ is a subset of this list;
\item $A$ has a finite number of minimal primes (i.e. the spectrum has a finite number of irreducible components)
\end{enumerate}
\end{lem}
\begin{proof}\hspace{1mm}
\begin{enumerate}[(a)]
\item Look at $\mathcal{I}=\{I\subset A | \text{(a) does not hold}\}$. If $\mathcal{I}\neq\varnothing$ there must exist an $I\in\mathcal{I}$ maximal with respect to inclusion (since $A$ is Noetherian). So if $ab\in I$ and $a\notin I,b\notin I$ then $\fr p_i\supset(I,a)$ and $(I,a)\supset\fr p_1\cdots\fr p_n$, and $\fr q_j\supset(I,b)$ and $(I,b)\supset\fr q_1\cdots\fr q_m$. This implies that $I\supset(I,a)(I,b)\supset\fr p_1\cdots\fr p_n\fr q_1\cdots\fr q_m$ and $I\subset\fr p_i,I\subset\fr q_j$. This can't happen because $I\in\mathcal{I}$ and hence we conclude that $I$ is a prime which is a contradiction.
\item If $I$ is minimal in $\fr p$ then $\fr p_1\cdots\fr p_n\subset\fr p$ and $\fr p_j\subset \fr p$ for some $j$, i.e. $\fr p_j=\fr p$ and $\fr p_{\text{min}}\supset I$.
\item Apply $(a)$ and $(b)$ to $I=(0)$.
\end{enumerate}
\end{proof}






\section*{Class 7}
For a local Noetherian ring $(A,m)$ set $\dim A = $ the Krull dimension of $A$, and $d(A) = \text{min } \{d| \exists x_1, \dots, x_d \in m \text{ such that } m = \sqrt{(x_1, \dots, x_d)}\}$. We've seen already that:
\[        \dim A = 0 \Leftrightarrow m \text{ nilpotent} \Leftrightarrow d(A) = 0   \;\;\; (\text{Lemma } \ref{L31} + \text{Lemma } \ref{L32})  \]
\[       \dim A = 1 \Leftrightarrow d(A) = 1 \;\;\; (\text{Lemma } \ref{L36} + \text{Lemma } \ref{L38})    \]

\begin{thm} [Krull Hauptidealsatz, v. 2]
\label{L41}
$\dim A = d(A)$.
\end{thm}
\begin{proof}
We first prove that $\dim(A) \leq d(A)$ by induction on $d$. Let $x_1, \dots,x_d \in m$ such that $m = \sqrt{(x_1, \dots, x_d)}$. Because $A$ is Noetherian, $\forall q\subsetneq m, q\neq m$ there exists a $q\subset p \subset m$ such that there exists no prime strictly between $p$ and $m$. Hence it suffices to show $\text{ht}(p)\leq d-1$ for such a $p$. We may assume $x_d \not \in p$ (by reordering). Then $m = \sqrt{(p, x_d)}$ because there exists no prime strictly between $p$ and $m$ (+ Lemma \ref{L26}). Hence:
\[      x_i^{n_i} = a_i x_d + z_i \;\;\;\; (*)      \]
For some $n_i\geq 1, z_i \in p, a_i \in A$. Then we have:

Hence $\sqrt{\overline{(x_d)}} = $ maximal ideal of $A(z_1, \dots, z_{d-1})$. By Theorem \ref{L36} $\dim A/(z_1, \dots , z_{d-1}) \leq 1$. Then $p$ is minimal over $(z_1, \dots, z_{d-1})$. By Lemma \ref{L42}, $pA_p$ is minimal over $(z_1, \dots, z_{d-1})A_p$. Finally, by the induction hypothesis $\dim(A_p) \leq d-1 \Rightarrow \text{ht}(p) \leq d-1$.
\\
\\
Now we prove that $d(A) \leq \dim(A)$. We may assume that $\dim(A) \geq 1$. Let $p_1, \dots, p_n$ be the finite number of minimal primes of $A$. (By Lemma \ref{L40}) Pick $y\in m, y \not \in p_i$ for $i=1, \dots, n$. (Such a $y$ exists by Lemma \ref{L39}.) Then:
\[         \dim \big(A/(y)\big) \leq \dim(A) -1        \]
Because all chains of primes in $A/(y)$ can be seen as a chain of primes in $A$ that can be extended by one of the $p_i$). Then by the induction hypothesis there exists $\bar x_1 , \dots , \bar x_{\dim(A)-}$ in $m/(y)$ such that $m/(y) = \sqrt{(\bar x_1, \dots, \bar x_{\dim(A)-1})}$. It follows that $m = \sqrt{(\bar x_1, \dots, \bar x_{\dim(A)-1}, y)}$.
\end{proof}


\begin{lem}
\label{L42}
Let $A$ be a ring, $I\subset A$ an ideal, $I\subset p$ prime, $S\subset A$ a multiplicative subset, $S\cap p = \emptyset$. Then $p$ minimal over $I \Leftrightarrow S^{-1}p$ is minimal over $S^{-1}I$ of $S^{-1}A$.
\end{lem}

\begin{proof}
See Lemma \ref{L7}.
\end{proof}




\begin{lem}
\label{L43}
Let $(A,\fr m)$ be a Noetherian local ring. Then the dimension of $A$ is less than or equal to the number of generators of $\fr m=\dim_{A/\fr m}\left(\fr m/\fr m^2\right)$. In particular, $\dim A<\infty$.
\end{lem}
\begin{proof}
The inequality is clear because if $\fr m=(x_1,\ldots,x_n)$ then $\fr m=\sqrt{(\fr m_1,\ldots,\fr m_n)}.$ Equality follows from one of Nakayama's many lemmas:
\begin{itemize}
\item if $M$ is finite and $\fr m M=M$, then $M=0$;
\item if $N\subset M$, $M=\fr mM+N$, everything finite, then $M=N$;
\item if $x_1,\ldots,x_t\in M$ which generate $M/\fr mM$, then $x_1,\ldots, x_n$ generate $M$.
\end{itemize}
\end{proof}

\begin{rem}
Note that there do indeed exist infinte-dimensional Noetherian rings. Constructing them is not particularly fun.
\end{rem}

\begin{lem}
\label{L44}
Let $A$ be a Noetherian ring. Let $I=(f_1,\ldots,f_c)$ be an ideal generated by $c$ elements ($c$ somehow stands for codimension). If $\fr p$ is a minimal prime over $I$, then $\text{ht}(p)\leq c$.
\end{lem}
\begin{proof}
Combine Theorem \ref{L41} and \ref{L42}.
\end{proof}

\begin{lem}
\label{L45}
Let $A$ be a Noetherian ring, $\fr p\subset A$ prime. If $\text{ht}(\fr p)=c$ then there exist $f_1,\ldots, f_c\in A$ such that $\fr p$ is minimal over $I=(f_1,\ldots,f_c)$.
\end{lem}
\begin{proof}
By Theorem $\ref{L41}$ there exists $x_1,\ldots, x_c\in\fr pA_\fr p$ such that $\fr pA_\fr p=\sqrt{(x_1,\ldots,x_c)}$. Write $x_i=f_i/g_i,f_i\in\fr p$ and $g_i\in A,g_i\notin\fr p$. Then $I=(f_1,\ldots, f_c)$ satisfies $IA_\fr p=(x_1,\ldots,x_c)A_\fr p$ with Lemma \ref{L42}.
\end{proof}

\begin{lem}
\label{L46}
Let $(A,\fr m)$ be a Noetherian local ring. Let $x\in\fr m$. Then $\dim(A/xA)\in\{\dim A,\dim A-1\}$. If $x$ is not contained in any minimal prime of $A$, e.g. if $x$ is a nonzerodivisor, then $\dim(A/xA)=\dim A-1$.
\end{lem}
\begin{proof}
If $x_1,\ldots,x_t$ map to $\bar x_1,\ldots,\bar x_t$ in $A/xA$ such that $\fr m_{A/xA}=\sqrt{(\bar x_1,\ldots, \bar x_t)}$. Then $\fr m_A=\sqrt{(x_1,\ldots,x_t,x)}$. Hence $\text{d}(A)\leq\text{d}(A/xA)+1$. Conversely, $\text{d}(A)\leq \text{d}(A/xA)$ is easy. Thus $\text{d}(A/xA)\in\{\text{d}(A),\text{d}(A)-1\}$ and hence the same for dimension by Theorem \ref{L41}.
\end{proof}

\begin{lem}
\label{L47}
A nonzerodivisor of any ring is not contained in a minimal prime.
\end{lem}
\begin{proof}
Let $x\in A$ be a nonzerodivisor. Then the map $A\overset{a}{\to}A$ is injective. By exactness of localization, $x/1$ is a nonzerodivisor in $A_\fr p$ for all minimal $\fr p$. Hence $x$ is not nilpotent in $A_\fr p$. Note also that $x/1\notin\fr p A_\fr p$ because $\fr pA_\fr p$ is locally nilpotent when $\fr p$ is minimal by Lemma \ref{L30}.
\end{proof}

\begin{exmp}\hspace{1mm}
\begin{itemize}

\item Consider $A=\left(k[x,y]/(xy)\right)_{(x,y)}$. It's clear from a previous homework exercise that $\dim A=1$ (the primes look like $(x),(y),$ and $(x,y)$). Note that if we consider $A/(x)$, which is now a domain as $(x)$ is prime in $A$, $(x,y)$ is now simply $(y)$, and the chain we are left with is $(0)\subset (y)$. Hence $\dim A/(x)=1$.
\item Consider $A=k[x,y,z]_{(x,y,z)}$. By one of the lemmas we have just proved above, since $\fr m=(x,y,z)$ has 3 generators, it's clear that $\dim A\leq 3$. However, it must be at least 3 due to the presence of the chain $(0)\subset (x)\subset (x,y)\subset (x,y,z)$. Hence $\dim A=3$.

\item $\dim \left(k[x,y,z]/(x^2 + y^2 + z^2) \right)_{(x,y,z)} = 2 = 3-1$, since $(x^2 + y^2 + z^2)$ is not a zerodivisor
\item $\dim \left(k[x,y,z]/(x^2 + y^2 + z^2, x^3 + y^3 + z^3) \right)_{(x,y,z)} = 1 = 3-2$. It suffices to check that $x^3 + y^3 + z^3$ is not $0$ in the domain $k[x,y,z]/(x^2 + y^2 + z^2)$.
\item $\dim \left( k[x,y,z]/(xy, yz, xz) \right)_{(x,y,z)} = 1$. This is because $\dim A/(x+y+z) = 0$, and we've seen in the problem sets that $(x+y+z)$ is not a minimal prime.
\end{itemize}
\end{exmp}


\section*{Class 8}

\begin{thm}[Hilbert Nullstellensatz]
\label{L48}
Let $k$ be a field. For any finite-type $k$-algebra $A$ we have:
\begin{enumerate}[(i)]
\item If $\fr m\subset A$ is a maximal ideal then $A/\fr m$ is a finite extension of $k$;
\item If $I\subset A$ is a radical ideal (i.e. $I=\sqrt{I}$) then $I=\cap_{I\subset\fr m}\fr m$.
\end{enumerate}
\end{thm}

\begin{rem}
Note that if $k=\bar k$ then this says that the residue fields at maximal ideals are equal to $k$. In particular, every maximal ideal of $k[x_1,\ldots,x_n]$ is of the form $(x_1-\lambda_1,\ldots,x_n-\lambda_n)$ for some $\lambda_i\in k$.

In every ring, if $I$ is radical then $I=\cap_{\fr p\supset I}\fr p$. Hence closed subsets of $\Spec A$ are in one-to-one correspondence with radical ideals. Part $(ii)$ of the theorem says that if $A$ is a finite-type $k$-algebra then closed points are dense in all closed subsets.
\end{rem}

\begin{proof}
Let us prove $(i)$ first. Note that $B=A/\fr m$ is a finite-type $k$-algebra which is a field. By Noether normalization there exists some $k[t_1,\ldots,t_r]\subset B$ for some $r\geq 0$. Now, by Lemma \ref{L15}, the map $\Spec B\to\Spec k[t_1,\ldots, t_r]$ is surjective. Since $\Spec B$ is simply a point, we can conclude that $r=0$. Hence $\dim_k B\leq\infty$.

The proof of $(ii)$ follows from $(i)$. We omit it.
\end{proof}

\begin{lem}
\label{L49}
Let $k$ be a field and $A \overset{\phi}{\to} B$ be a homomorphism of finite type $k$-algebras. Then $\Spec \phi$ maps closed points to closed points.
\end{lem}

\begin{proof}
We have to show that $m\in B$ maximal implies $\phi^{-1}(m)$ maximal. We look at $k \subset A/\phi^{-1}(m) \subset B/m$. Note that the latter is a finite field extension of $k$, by Theorem \ref{L48}. Then $\dim_k A/\phi^{-1}(m) < \infty$. Then by Lemma \ref{L11} $A/\phi^{-1}(m)$ is a field. 
\end{proof}

\begin{lem}
\label{L51}
For $k$ field and $A$ finite type $k$-algebra, $\dim(A) = 0 \Leftrightarrow \dim_k A < \infty$.
\end{lem}

\begin{proof}
By Noether normalization there exists a finite map $k[t_1, \dots, k_r] \hookrightarrow A$. Then by Lemma \ref{L29} $\dim(A) = \dim(k[t_1 \dots t_r]) \geq r$. Hence $(\Rightarrow)$ follows. For the converse use Lemma \ref{L12}.
\end{proof}
Our goal is now to construct a ``good'' dimension theory for finite type algebras over fields.

\begin{lem}\hspace{1mm}
\label{L51}
\begin{enumerate}[(a)]
\item For $X$ a topological space with irreducible components $Z_i$ then $\dim(X) = \sup \dim(Z_i)$;
\item For a ring $A$, $\dim(A) = \sup_{p\subset A \text{ minimal}} \dim(A/p)$.
\end{enumerate}
\end{lem}
\begin{proof}
Omitted.
\end{proof}

\begin{defn}
Let $k\subset K$ be a field extension. The \textbf{transcendence degree} $\trdeg_k K = \sup \{n|\; \exists x_1, \dots, x_n \text{ algebraically independent over } k\}$. This means that the map $k[t_1 \dots t_n] \to K$ that takes $t_i \to x_i$ is injective.
\end{defn}


\begin{lem}
\label{L52}
Let $k$ be a field, then every maximal ideal $m$ of the ring $k[x_1 \dots x_n]$ can be generated by $n$ numbers, and $\dim(k[x_1 \dots x_n])_{m} = n$.
\end{lem}

\begin{proof}
By Theorem \ref{L48}, the residue field $\kappa = k[x_1 \dots x_n]/m$ is finite over $k$. Let $\alpha_i \in \kappa$ be the image of $x_i$. We look at the chain:
\[        k = \kappa_0 \subset \kappa_1 = k(\alpha_1) \subset \dots \subset \kappa = k(\alpha_1, \dots , \alpha_n)        \]
We know from field theory that $x_i \in k[\alpha_1, \dots, \alpha_i]$. Choose $f_i \in k[x_1 \dots x_i]$ such that $f(\alpha_1, \dots, \alpha_{i-1}, x_i)$ is the minimal polynomial of $\alpha_i$ over $\kappa_{i-1}$. Then $f_i(\alpha_1, \dots, \alpha_i) = 0$, so $f_i \subset m$. Now we claim that $\kappa_i \cong k[x_1, \dots, x_i]/(f_1, \dots, f_i)$. We prove this by induction:
\[       k[x_1, \dots, x_i]/(f_1, \dots, f_i)   \cong k[x_1, \dots, x_{i-1}]/(f_1, \dots, f_{i-1}) [x_i] /(f_i)        \]
If we let $i=n$, this proves the first statement of the lemma. Finally, we have a chain of primes:
\[      (0) \subset (f_1) \subset \dots \subset (f_1 \dots f_n) = m        \]
because $ k[x_1, \dots, x_i]/(f_1, \dots, f_i) \cong \kappa_i [x_{i+1}, \dots, x_n]$. Therefore $\dim(k[x_1, \dots, x_{n}])_m \geq n$. But by Lemma \ref{L41} it is at most $n$, so this finishes the proof.
\end{proof}

\begin{lem}
\label{L53}
$\dim(k[x_1, \dots, x_n]) = n$.
\end{lem}
\begin{proof}
Omitted.
\end{proof}

\begin{rem}
For a Noetherian local ring $(A,\fr m)$ we have:
\[  \dim A \leq \text{minimum number of generators of } \fr m = \dim_{A/\fr m} \fr m/\fr m^2 \]
$(A,\fr m)$ is called \textbf{regular} if we have equality. The above shows that $k[x_1, \dots, x_n]_\fr m$ is regular for all maximal ideals $\fr m$.
\end{rem}

\begin{lem}
\label{L54}
Let $k$ be a field and $A$ be a finite type $k$-algebra. Then:
\begin{enumerate} [(a)]
\item the integer $r$ from Noether Normalization is equal to $\dim A$;
\item if $A$ is a domain, then $\dim A = \text{trdeg}_k (\text{f.f.} A)$.
\end{enumerate}
\end{lem}

\begin{proof}\hspace{1mm}
\begin{enumerate} [(a)]
\item follows from Lemma \ref{L29} and Lemma \ref{L53}
\item follows from $(a)$ and the fact:
\end{enumerate}
\[       k[t_1 \dots t_r] \subset A \text{ finite}  \overset{L \ref{L10}}{\Rightarrow}  k(t_1 \dots t_r) \subset S^{-1}A \text{ finite}   \overset{L \ref{L11}}{\Rightarrow} S^{-1}A \text{ is the f.f. of } A \]
Then:
\[     k(t_1 \dots t_r) \subset \text{f.f.}(A)  \Rightarrow \text{trdeg}_k \text{f.f.}(A) = \text{trdeg}_k k(t_1 \dots t_r) = r     \]
The last two equalities should be familiar from field theory.
\end{proof}

\begin{rem}
If $k\to A$ is a finite type domain then $\dim(A) = \dim(A_f) \forall f\in A, f\neq 0$. We may regard this as a very weak form of ``equidimensionality''.
\end{rem}
\begin{rem}
So far we missed proving an important result; we will do so later. We will want to show that for $A$ finite type domain over a field, $p \subset A$ prime, we have $\dim(A) = \dim(A/p) + \text{ht}(p)$. Intuitively it's clear why this should be so: take a chain in $A$ and some $p$ in this chain, then $\dim(A/p)$ counts elements containing $p$, and ht$(p)$ counts elements contained in $p$.
\end{rem}

\begin{defn}
Let $A\to B$ be a ring map. The \textbf{integral closure} of $A$ in $B$ is $B' = \{b\in B | \text{b is integral over} A\}$. We say that \textbf{$B$ is integral over $A$} iff $B'=B$.
\end{defn}

\begin{lem}
\label{L55}
If $A\to B$ finite, then $B$ is integral over $A$.
\end{lem}

\begin{proof}
Pick $b\in B$. Choose $b_1, \dots b_n, \in B$ such that $B = \sum A b_i$. 
Write, for $a_ij\in A$,
\[bb_i=\sum_{a_{ij}}b_j.\]
Let $M=(a_{ij})\in\text{Mat}(n\times n, A)$ and let $P(T)\in A[T]$ be the characteristic polynomial of $M$. By Cayley-Hamilton, $P(M)=0$, which implies that $P(b)=0$.
\end{proof}


\section*{Class 9}
\begin{lem}
\label{L56}
The integral closure of a ring $A$ is an $A$-algebra.
\end{lem}
\begin{proof}
Suppose $b,b' \in B'$, we want to show that $b+b', bb' \in B'$. Let $C$ be the $A$-algebra generated by $b,b'$. Then $C$ is finite over $A$ by Lemma \ref{L1}. Then by Lemma \ref{L55} $C$ is integral over $A$, so $C\subset B'$.
\end{proof}

\begin{lem}
\label{L57}
If $A\to B \to C$ are ring maps then:
\begin{enumerate}
\item $A\to B, B\to C$ integral $\Rightarrow A\to C$ integral;
\item $A\to C$ integral $\Rightarrow B\to C$ integral.
\end{enumerate}
\end{lem}
\begin{proof}
Omitted.
\end{proof}

\begin{defn}
A \textbf{normal domain} is a domain which is integrally closed in its field of fractions. (In other words, it is equal to its integral closure in its field of fractions.)
\end{defn}

\begin{lem}
\label{L58}
For a field $k$, $k[x_1, \dots, x_n]$ is a normal domain.
\end{lem}
\begin{proof}
Polynomial rings are UFDs, so this follows from Lemma \ref{L59}.
\end{proof}

\begin{lem}
\label{L59}
A UFD is a normal domain.
\end{lem}
\begin{proof}
Suppose that $a/b \in$ f.f.$(A)$ is in least terms (we can always reduce a fraction to least terms, due to unique factorization) and is integral over $A$. Thus there exist some $a_i \in A$ such that:
\[         \left( \frac{a}{b} \right)^n +  a_1 \left( \frac{a}{b} \right)^{n-1} + \dots + a_n = 0       \]
\[       a^n + a_1 a^{n-1} b + \dots + a_n b^n = 0    \]
Therefore $a^n \in (b)$, which, unless $b$ is a unit, contradicts the fact that $a, b$ are relatively prime. Therefore the only elements of the field of fractions that are integral over $A$ are those of $A$ itself.
\end{proof}

\begin{lem}
\label{L60}
Let $R$ be a domain with field of fractions $K$, and let $a_0, \dots, a_{n-1}$, $b_0, \dots, b_{m-1} \in R$. If $x^n + a_{n-1} x^{n-1} + \dots + a_0$ divides $x^m + b_{m-1} x^{m-1} + \dots + b_0$, then $a_i$ are integral over the $\Z$ subalgebra of $R$ generated by $\{b_j\}$.
\end{lem}
\begin{proof}
Choose some field extension $L$ of $K$ with $\beta_1, \beta_m \in L$ such that:
\[   x^m + b_{m-1}x^{m-1} + \dots + b_0 = \prod_{i=1}^m (x - \beta_i)  \]
Then by unique factorization in $L[x]$ we get:
\[     x^n + a_{n-1}x^{n-1} + \dots + a_0 = \prod_{j} (x - \beta_j)       \]
Where $j$ runs over a subset of $\{1, \dots, m\}$. But this means that:
\[     a_i \in \Z[b_0, \dots, b_{m-1} , \beta_1, \dots, \beta_m] \supset \Z[b_0, \dots, b_{m-1}]     \]
By Lemmas \ref{L1} and \ref{L55}, the inclusion is integral.
\end{proof}

\begin{lem}
\label{L61}
Let $R\subset A$ be a finite extension of domains, $R$ normal. For $a\in A$ we have:
\begin{enumerate} [(1)]
\item the coefficients of the minimal polynomial of $a$ over the field of fractions of $R$ are in $R$;
\item $\Nm (a) \in R$, where $\Nm$ denotes the norm.
\end{enumerate}
\end{lem}
\begin{proof}
Apply Lemma \ref{L60}. For example, $a$ must satisfy a monic polynomial with coefficients in $R$, and the minimal polynomial must divide that.
\end{proof}

\begin{lem}
\label{L62}
Suppose $R\subset A$ is a finite extension of domains, and $R$ is normal. Suppose also that $f\in A$, $\fr p \subset A$ prime with $V(f) \subset V(\fr p)$. Then setting $f_0 = \Nm_{\text{ff}(A)/\text{ff}R} (f)$ we have:
\begin{enumerate}
\item $f_0 \in R$;
\item $R \cap \fr p = \sqrt{(f_0)}$.
\end{enumerate} [(1)]
\end{lem}
\begin{proof}
$(1)$ follows by Lemma \ref{L61}. (See also the argument by Tate in Mumford's red book.) For part $(2)$, let
\[    x^d + r_1 x^{d-1} + \dots r_d    \]
be the minimal polynomial of $f$ over f.f.$(R)$, with $r_i \in R$. (This is possible by Lemma \ref{L61}.) Then $f_0 = r_d^e$ for some $e\geq 1$. So:
\[     f^d + r_1 f^{d-1} + \dots + r_d = 0   \Rightarrow r_d \in (f) \Rightarrow f_0 \in (f)   \]
We already know that $f \in \fr p$ by assumption $V(f) = V(\fr p)$, so we get $\sqrt{f_0} \subset R \cap \fr p$. Conversely, if $r \in R \cap \fr p$ we have $r^n\in (f)$ for some $n$, because $V(f) = V(\fr p)$. Say $r^n = af$, then:
\[         (r^n)^{[\text{f.f.}(A) : \text{f.f.}(R)]} = \Nm (r^n) = \Nm(a) \Nm(f) = \{\text{sth in }R\} f_0       \]
Then $r \in \sqrt{f_0}$.
\end{proof}
\begin{rem}
This lemma says that the image (under a finite map) of an irreducible hypersurface is an irreducible hypersurface.
\end{rem}
Now we use this lemma to prove the missing link from dimension theory.
\begin{thm}
\label{L63}
Given a finite type $k$-algebra $A$ which is a domain and a height $1$ prime $\fr p$, then:
\[       \dim A = \dim(A/\fr p) + 1   \]
\end{thm}
\begin{proof}
By Lemma \ref{L45}, $\fr p$ minimal over $(f)$ for some $f\in A$. Say $\fr p, \fr p_1, \dots, \fr p_n$ are all the distinct minimal primes over $(f)$. (Lemma \ref{L40} says they are finitely many.) Then $\fr p_1 \dots \fr p_n \not \subset \fr p$. (Prime avoidance.) Then we can pick $g\in p_1 \dots p_n$, $g\not \in \fr p$. After replacing $A$ by $A_g$ and $\fr p$ by $\fr p A_g$ and $f$ by $f/1$ we may assume $\fr p$ is the only prime minimal over $(f)$, i.e. $V(f) = V(\fr p)$. By an earlier remark, $\dim A = \dim A_g$, so the statement of the theorem doesn't change if we do the replacement. 
\\
\\
Now by Noether Normalization we choose a finite injective map $k[t_1, \dots, t_d] \hookrightarrow A$. Set $f_0 = \Nm(f) \in k[t_1, \dots, t_d]$, by Lemma \ref{L62} and $\fr p \subset k[t_1, \dots, t_d] = \sqrt{(f_0)}$, again by Lemma \ref{L62}. Since $k[t_1, \dots, t_d]$ is a UFD, we can write $f_0 = cf^e$, for some $c\in k^*, e\geq 1$. Then $\sqrt{(f_0)} = (f_0)$ and we see that $k[t_1, \dots, t_d]/(f_0) \hookrightarrow A/\fr p$ is a finite injective map. Thus:
\[       \text{trdeg}_k (\text{f.f.} (A/\fr p)) \subset \text{trdeg}_k (\text{f.f.} (k[t_1, \dots, t_d]/(f_0))) = d-1    \]
\end{proof}

\begin{exmp}
We compute the integral closure of $k[x,y]/(y^2 - x^3)$ in its field of fractions. (This is called ``normalization''.)
\[       k[x,y]/(y^2 - x^3) \subset \text{f.f.} (k[x,y]/(y^2 - x^3))    \]
To get some element in the integral closure which is not in the ring, we look at the equation:
\[       y^2 - x^3 = 0 \Rightarrow \frac{y}{x} = x^{1/2}    \]
We see that $t = y/x$ is both in the integral closure and in the field of fractions. We therefore add it to the ring and see what happens. We construct the map:
\begin{align*}      k[x,y]/(y^2 - x^3) &\to k[t] \\
x &\to t^2 \\
y &\to t^3
\end{align*}
We need to check that this induces an isomorphism of fraction fields, namely maps $y/x$ to $t$. We also need to check that the map is integral (it is, because $t$ is integral). We are now done, because $k[t]$ is a UFD and therefore a normal domain.
\end{exmp}

\section*{Class 10}

\begin{cor}
\label{L64}
Let $A$ be a finite type $k$-algebra. Then:
\begin{enumerate} [(1)]
\item $A$ domain $\Rightarrow$ any maximal chain of primes in $A$ has length $\dim A$;
\item $A$ domain and $\fr p \subset A$ prime $\Rightarrow \dim A = \dim(A/\fr p) + \text{ht}(\fr p)$;
\item $A$ domain and $\fr m \subset A$ maximal $\Rightarrow \dim(A_{\fr m}) = \dim A$ and all maximal chains of primes in $A_{\fr m}$ have length $\dim A$;
\item if $\fr p \subset \fr q \subset A$ then any maximal chain of primes between $\fr p$ and $\fr q$ has length $\trdeg_k (\text{f.f.}(\fr p)) - \trdeg_k (\text{f.f.}(\fr q))$;
\item if $\fr m \subset A$ maximal then $\dim(A_m) = \max_{\fr p \text{ minimal}} \dim(A/\fr p) = \dim_{\fr m} \Spec(A)$.
\end{enumerate}
\end{cor}

\begin{proof}
\begin{enumerate} [(1)]
\item Choose a chain of primes:
\[    (0) \subsetneq \fr p_1 \subsetneq \dots \subsetneq \fr p_l \subsetneq A    \]
We have ht$(\fr p_1) = 1$. Hence Proposition \ref{L63} shows that $\dim(A/\fr p_1) = \dim(A) - 1$ and thus we get a chain:
\[    (0) = \frac{\fr p_1}{\fr p_1} \subsetneq \frac{\fr p_2}{\fr p_1} \subsetneq \dots \subsetneq \frac{\fr p_l}{\fr p_1}    \]
By induction $l = \dim A$.
\item Let $h =$ ht$(\fr p)$ and $d = \dim(A/\fr p)$. Then we have chains:
\[           (0) = \frac{\fr p}{\fr p} \subsetneq \frac{\fr p_1}{\fr p} \subsetneq \dots \subsetneq \frac{\fr p_d}{\fr p} \subsetneq  \frac{A}{\fr p}      \]
\[     (0) \subsetneq \fr q_1 A_{\fr p}  \subsetneq \dots \subsetneq \fr q_h A_{\fr p}   \]
Then $(0) \subsetneq \dots \subsetneq \fr q_{h-1} \subsetneq \fr q_h = \fr p \subset \dots \fr p_d$ is a maximal chain of primes in $A$ of length $h+d$. By $(1)$ $\dim A = h+d$.
\item follows from (1).
\item follows from (3) by looking at $A/\fr p$ and $\fr q/\fr p$ and using Lemma \ref{L54}.
\item follows from (3) and Lemma \ref{L51}.
\end{enumerate}
\end{proof}

\begin{defn}
A \textbf{graded ring} is a ring $A$ together with a given direct sum decomposition $A=\oplus_{d\geq 0}A_d$ such that $A_dA_e\subset A_{d+e}$. A \textbf{graded module} $M$
over $A$ is an $A$-module $M$ equipped with a direct sum decomposition $M=\oplus_{d\in\Z}Md$ such that $A_dM_e\subset M_{d+e}$. We say that $B$ is a \textbf{graded $A$-algebra} if there is a direct sum decomposition as $R$-modules.
\end{defn}

\begin{exmp}
If we take $A=k[x_1,\ldots,x_d]$ and $A_d$ to be the homogeneous polynomials of degree $d$, we see that $A$ is a graded ring.
\end{exmp}

\begin{thm}
\label{L65}
Let $M$ be a finitely-generated, graded, $A$-module, where $A$ is a graded $k$-algebra which is generated (as a $k$-algebra) by a finite number of elements of degree 1. Then the function $d\mapsto \dim_k(M_d)$ is a numerical polynomial. This function is known as the \textbf{Hilbert polynomial}.
\end{thm}

\begin{defn}
A function $f:\Z\to\Z$ is a \textbf{numerical polynomial} iff there exists an $r\geq 0,a_i\in \Z$ such that
\[f(d)=\sum_{i=0}^r a_i\binom{d}{i}\]
for all $d\gg 0$.
\end{defn}

\begin{lem}
\label{L66}
If $f:\Z\to\Z$ is a function and $d\mapsto f(d)-f(d-1)$ is a numerical polynomial, then so is $f$.
\end{lem}

Let us now prove the theorem.

\begin{proof}
We may assume that $A=k[x_1,\ldots, x_d]$, graded as in the above example. The proof proceeds by induction on $n$. Let us consider three distinct cases. In the first, we suppose that $x_n$ is a nonzerodivisor on $M$. Then we have a short exact sequence
\[0\to M\overset{x_n}{\to}M\to M/x_nM\to 0.\]
Note that the multiplication by $x_n$ shifts the grading by 1 and that
\[-\dim M_{d-1}+\dim M_d-\dim \left( M/x_nM\right)_d=0.\]
Now $M/x_nM$ is a finitely-generated graded module as $k[x_1,\ldots,x_{d-1}]$ so we are done by induction and Lemma \ref{L66}.

Next consider the case where $x_n^eM=0$ for some $e\geq 0$. In this case we get a short exact sequence
\[0\to x_nM\to M\to M/x_nM\to 0.\]
Note that $x_n^{e-1}x_nM=0$. Hence we are done by induction on $e$ and $n$.

Finally, consider the general case. Let $N=\{m\in M | x_n^e m=0\text{ for some }e\}$. Then
we get an exact sequence
\[0\to N\to M\to M/N\to 0.\]
At $N$, this follows from the nilpotent cases, as $A$ is Noetherian and $M$ being finitely-generated implies that $N$ is. At $M/N$ this follows from the nonzerodivisor case.
\end{proof}

\begin{defn}
Let $(A,\fr m, k)$ be a Noetherian local ring. Set
\[\text{gr}_\fr m A=\bigoplus_{n\geq0}\fr m^n/\fr m^{n+1}.\]
This is a graded $k$-algebra generated by $\fr m/\fr m^2$ over $k$. Hence
$n\mapsto\dim_k(\fr m^n/\fr m^{n+1})$ is a numerical polynomial by Theorem \ref{L65}.
We denote by $d(A)$ the degree of this polynomial; if $\fr m^n=0$ for some $n$, we take -1.
\end{defn}

\begin{thm*}
For $(A,\fr m,k)$ a Noetherian local ring, $d(A)=\dim A-1$.
\end{thm*}
\begin{proof}
We will only sketch the proof of this theorem.
The result is clear for $\dim A=0$. Suppose $\dim A>0$. Suppose you can find an $x\in\fr m$ such that $x$ is a nonzerodivisor in $A$ and $\bar x\in\fr m/\fr m^2$ is a nonzerodivisor in $\text{gr}_\fr m A$. Then $\dim A/xA=\dim A-1$. The sequences
\[0\to\fr m^{n-1}/\fr m^n\overset{x}{\to}\fr m^n/\fr m^{n+1}\to \bar{\fr m}^n/\bar{\fr m}^{n+1}\to 0 \]
are exact, where $\bar{\fr m}\subset A/xA$ is the maximal ideal. Then $d(A/xA)=d(A)-1$, and we are done by induction on $\dim A$.

This argument will not work if such an $x$ does not exist. Roughly speaking, one finds an $x$ such that $\bar x$ is in the none of the minimal primes of $\text{gr}_\fr mA$ and one shows that $d(A/xA)$ does actually drop by 1.
\end{proof}

\begin{thm*}
If $B$ is a graded $k$-algebra generated by finitely many elements of degree 1, then $\dim B-1$ is the degree of the numerical polynomial $n\mapsto\dim_k B_n$.
\end{thm*}

\begin{cor*}
$\dim A=\dim\text{gr}_\fr m A.$
\end{cor*}

\section*{Class 11}

\begin{defn}
Let $A$ be a ring, $I\subset A$ an ideal, $M$ an $A$-module. Then we define the \textbf{completion of $M$} to be
\[\hat M=\varprojlim M/I^nM=\{(x_1,x_2,\ldots)\in \prod M/I^nM | x_{n+1}=x_n\mod I^nM\}.\]
There is a canonical map $M\to\hat M$ that takes $x\mapsto (x,x,x,\ldots)$.
\end{defn}

\begin{defn}
We say that $M$ is \textbf{I-adically complete} if $M\to \hat M$ is an isomorphism. Note that included is the condition $\cap_n I^nM=(0)$.
\end{defn}

\begin{lem*}
If $I$ is finitely-generated then $\hat M$ is I-adically complete and moreover $I^n\hat M=\ker(\hat M\to M/I^nM)$.
\end{lem*}

\begin{lem*}
If $A$ is Noetherian then $\hat A$ is Noetherian.
\end{lem*}
\begin{proof}
Say $I=(f_1,\ldots, f_r)$ then the map
$A[[x_1,\ldots,x_{r-1}]]\to \hat A$ that takes $\sum a_Ix^I\mapsto \sum A_If^I$ is surjective. So this lemma follows from the fact that $A[x_1,\ldots,x_r]]$ is Noetherian.
\end{proof}

\begin{lem*}
If $(A,\fr m)$ is local and Noetherian then the completion $\hat A$ of $A$ with respect to $\fr m$ is a Noetherian local ring with maximal ideal $\hat m=m\hat A$ such that $\text{gr}_{\hat{\fr m}}\cong\text{gr}_{\fr m}A$ and $\dim A=\dim\hat A$. This can be used to reduce problems to the complete local case.
\end{lem*}

\begin{thm*}[Cohen Structure Theorem for characteristic 0]
A complete Noetherian local ring $A$ containing a field of characteristic zero is isomorphic to $k[[x_1,\ldots,x_n]]/I$ for some field $k=A/\fr m$.
\end{thm*}

\begin{lem}[Artin-Rees Lemma]
Let $A$ be a Noetherian ring. Let $I\subset A$ be an ideal. Let $N\subset M$ be $A$-modules with $M$ finite. Then there exists $c\geq 0$ such that
\[I^nM\cap N=I^{n-c}(I^cM\cap N)\]
for all $n\geq c$.
\end{lem}

\begin{proof}
Consider $B=A\oplus I\oplus I^2\oplus\ldots$ (the Rees algebra). This is a finitely-generated $A$-algebra and hence is Noetherian. Consider $P=M\oplus IM\oplus I^2M\oplus\ldots$ By similar considerations, $P$ is a finite $B$-module, hence Noetherian (a.c.c. for submodules). Hence the $B$-submodule $N\oplus IM\cap N\oplus I^2M\cap N\oplus\ldots$ is a finitely-generated submodule. Thus there exist $r\geq 0$, $c_1,\ldots,c_r\geq 0$ and $x_i\in I^{c_i}M\cap N$ such that $N\oplus IM\cap N\oplus I^2M\cap N\oplus\ldots=\sum Bx_i$. Take $n\geq c=\max(c_i)$. Then
\[I^nM\cap N=\left(\sum Bx_i\right)_n=\sum I^{n-c_i}x_i\subseteq\sum I^{n-c_i}(I^{c_i}M\cap N)\subseteq I^{n-c}(I^cM\cap N).\]
Hence we've proved the $\subseteq$ inclusion. The reverse inclusion is trivial.
\end{proof}

Let us investigate some of the consequences of this lemma. First note that in a Noetherian local ring $(A,\fr m)$ we have $\cap\fr m^n=(0)$. We can prove this by setting $M=A$ and $N=\cap\fr m^n$ and hence $N=\fr m^n\cap N=\fr m^{n-c}(\fr m^c\cap N)=\fr m^{n-c}N$, which by Nakayama's lemma implies that $N=0$.

Moreover, if $A$ is Noetherian, $I$ an ideal, then the functor $M\to\hat M$ is exact on the category of finite modules.

Finally, note that $(A,\fr m)$ Noetherian local implies that $A\to\hat A$ is flat.

\begin{defn}
Let $k\subset K$ be fields. A \textbf{discrete valuation} on $K/k$ is a surjective map $v:K^\times\to\Z$ such that:
\begin{enumerate}[(i)]
\item $v(c)=0$ for all $c\in k^\times$;
\item $v(xy)=v(x)+v(y)$ for all $x,y\in K^\times$;
\item $v(x+y)\geq \min(v(x),v(y))$.
\end{enumerate}
\end{defn}

\begin{lem*}
If $v(x)\neq v(y)$ then $v(x+y)=\min(v(x),v(y))$.
\end{lem*}
\begin{proof}
Assume $v(y)>v(x)$. Then $v(y)=v(-y)$ because $-1\in k$. Then $v(x)=v(x+y-y)\geq\min(v(x+y),v(y))$.
\end{proof}

\begin{exmp}
Let $K=k(t)$. Then let $v(f)$ be the order of vanishing of $f$ at $t=0$. Then we see that
$v(t/(1+t))=1$ and $v((1+t)/(t^2+t^3))=-2$.
\end{exmp}

Now assume that $k$ is algebraically closed, i.e. $\bar k=k$. What are all the discrete valuations on $k(t)/k$? Let us proceed by cases.

In the first case, suppose that $v(f)<0$ for some $f\in k[t]$. Then it's clear that $f$ is non-constant and that we can add a constant to $f$ without changing $v(f)$ (by the lemma). Hence we may assume that $f=tg$ for some $g\in h[t]$. Then either $v(t)<0$ or $v(g)<0$. By induction we find that $v$ of some linear polynomial must be less than 0. Suppose $v(at+b)=-m$. Then $v(t+b/a)=-m$ and $v(t+\lambda)=-m$ for all $\lambda\in k$. This shows that $v(a_nt^n+\ldots+a_0)=-nm$, because $k=\bar k$. We conclude that $m=1$ because $v$ is surjective. So
\[v\left(\frac{at^n+\ldots}{bt^l+\ldots}\right)=l-n,\]
which gives us the order of vanishing at infinity.

For the other case, where $v(f)\geq 0$ for all $f\in k[t]$. In this case
$\fr m=\{f\in k[t] | v(f)>0\}$, is an ideal, prime, not the zero ideal, and hence maximal via $k[t]$ a PID. Hence $\fr m$ is a maximal ideal, i.e. $(t-\alpha)$ for some $\alpha$. $v(t-a)=m>0$. Then $v(f)$ is $m$ times the order of vanishing of $f$ at $t-\alpha$. Because $v$ is surjective, we get $m=1$ and $v(f)$ is the order of vanishing at $t=\alpha$.

\begin{defn}
The \textbf{projective line} $\mathbb{P}_k^1$ over $k=\bar k$ is the set of valuations on $k(t)/k$. We define its topology by closed sets being finite subsets of $\mathbf{P}_k^1$ and $\varnothing$ and $\mathbb{P}_k^1$. The \textbf{regular functions} given $U\subset\mathbb{P}_k^1$ open are $\mathcal{O}(U)=\{f\in k(t) | v(f)\geq 0 \forall v\in U\}$.
\end{defn}

For example, if $U=\mathbb{P}_k^1\setminus\{\infty\}$ then $\mathcal{O}(U)=k[t]$.

\section*{Class 12 - Algebraic Curves}

Throughout, let $k=\bar k$ be our algebraically closed ground field. Let $K$ be a finitely-generated field extension of $k$ with $\trdeg_k K=1$. Somehow we want to think of $K$ as a function field of an algebraic curve. We will follow van der Waerden's Algebra Vol. II, chapter 19.

We will denote by $C$ the set of discrete valuations of $K/k$ (and think about it as the set of points of $K/k$. For $v\in C$ we set
\begin{align*}
\mathcal{O}_v&=\{f\in K \mid v(f)\geq 0\}\\
\fr m_v&= \{ f\in K \mid v(f)>0 \}.
\end{align*}

\begin{lem}
\label{L68}
For a discrete valuation $v$ on a field $K$, $\mathcal{O}_v$ is a local domain with a maximal ideal $\fr m_v$. If $z\in\fr m_v$ has $v(z)=1$ (such $z$ exist by surjectivity of valuations) then $\fr m_v=(z)$ and every ideal in $\mathcal{O}_v$ is of the form $(z^n)=\fr m_v^n$ for some $n\geq 0$. Such a ring is called a \textbf{discrete valuation ring}.
\end{lem}
\begin{proof}
If $f\in\fr m_v$ then $v(f)\geq 1$ so $v(f/z)\geq 0$ so $f=(f/z)z$ as $f/z\in\mathcal{O}_v$, which proves the claim.
\end{proof}

\begin{lem}
\label{L69}
If $K/k$ is finitely-generated of transcendence degree 1 and $v\in C$ then $\kappa_v=\mathcal{O}_v/\fr m_v$ is equal to $k$.
\end{lem}
\begin{proof}
Pick $z\in K$ with $v(z)=1$. Then $z\notin k$ so $z$ is transcendental over $k$. Hence $n=[K:k(z)]<\infty$. Say $[\kappa_v:k]>1\implies [\kappa_v:k]=\infty$. This implies that there exist $u_1,\ldots,u_n\in\mathcal{O}_v$ such that $\bar u_1,\ldots,\bar u_{n+1}\in\kappa_v$ are $k$-linearly independent:
\[\sum_i a_i u_i=0\]
for some $a_i\in k(z)$ not all zero. Clearing denominators, we may assume that $a_i\in k[t]$ and not all in $(z)$. But then we obtain
\[\sum_i \bar a_i \bar u_i=0,\]
which is a non-trivial relation in $\kappa_v$, which is a contradiction.
\end{proof}

\begin{rem}
Now we know that $\mathcal{O}_v$ contains $k$, has residue field $k$, and has a \textbf{uniformizer} $z$ i.e. $\fr m_v=(z)$. Then it follows that $\hat{\mathcal{O}}_v=\varprojlim \mathcal{O}_v/\fr m_v^n\cong k[[z]].$ A special case is where $K=k(z)$, $v=\text{ord}_{z=0}$. Then $\mathcal{O}_v=k[z]_{(z)}$ and $\hat{\mathcal{O}}_v=k[[z]]$ in which we see $1/(1+z)\mapsto 1-z+z^2-z^3+\ldots$.
\end{rem}


\begin{lem}
\label{L70}
Let $v_1,\ldots,v_n$ be pairwise distinct discrete valuations on a field $K$. Then there exists an $f\in K$ such that $v_1(f)>0$ and $v_i(f)<0$ for $i=2,\ldots n$.
\end{lem}
\begin{proof}
If $n=1$, this is trivial. Suppose $n=2$ and the statement does not hold, i.e.
$v_1(f)>0$ and $v_2(f)\geq 0$. Pick $z\in K$ with $v_1(z)>0$. Then for any $f\in K$ we have $v_1(f^az^b)=av_1(f)+bv_1(z)$ and $v_2(f^az^b)=av_2(f)+bv_2(z)$. This implies that $v_1(f)>-bv_1(z)/a$ and $v_2(f)\geq -bv_2(f)/a$. Thinking a little bit, we can conclude that $v_2(f)\geq v_1(f)\frac{v_2(z)}{v_1(z)}$. Similarly for $f^{-1}$,
\[-v_2(f)=v_2(f^{-1})\geq v_1(f^{-1})\frac{v_2(z)}{v_1(z)}=-v_1(f)\frac{v_2(z)}{v_1(z)}.\]
By surjectivity of $v_1,v_2$ we find that the fraction above must be 1, and hence $v_1=v_2$.

Now consider $n>2$. Pick $f\in K$ such that $v_1(f)>0$ and $v_i(f)<0$ for $i=2,\ldots,n-1$ (by induction). If $v_n(f)<0$ then we are done. If not, pick $g\in K$ such that $f_1(g)>0$ and $v_r(g)<0$. Then $h_r=g(1+f^r)$, and for $r\gg 0$ we get $v_1(h_r)=v_1(g)$ because $f^r\in\fr m_v$ and $1+f^r$ is thus a unit. Now, for $2\geq i\leq n-1$, note that $v_i(h_r)=v_i(g)+v_i(1+f^r)=v_i(g)+rv_i(f)$ which goes to $-\infty$ as $r\to\infty$. Finally, we see that $v_n(h_r)=v)n(g)+v_n(1+f^r)<0$. We know that $v_n(f)\geq 0$. If $v_n(f)\geq 0$ then $1+f^r$ os a unit and $v(1+f^r)=0$. If $v_n(f)=0$ then $v_n(1+f^r)>0$ only if $f^r$ maps to $-1$ in $\kappa_{v_n}$. We are done as there is an infinite sequence of $r$ such that $f^r=-1$ (except possibly if the $\text{char }k=2$, which we leave as an exercise).
\end{proof}

\begin{defn}
Let $v_1,\ldots, v_n$ be pairwise distinct discrete valuations on a field $K$. Let $f_1,\ldots, f_n\in K$. An \textbf{approximation to order $N$} of $f_1,\ldots, f_n$
 at $v_1,\ldots, v_n$ is an $f\in K$ such that $v_i(f-f_i)\geq N$ for $i=1,\ldots,n$.
 \end{defn}

\begin{thm}
\label{L71}
Approximations exist.
\end{thm}
\begin{proof}
Pick $f\in K$ as in Lemma \ref{L70}. Then
\[h_M=\frac{1}{1+f^M}\]
for $M\gg 0$. This satisfies $h_M\in\mathcal{O}_v$, $h_M\equiv 1\mod \fr m_{v_1}^M$ and that $h_M\in \fr m_{v_i}^M$ for $i=2,\ldots, n$ (compute $v_i(h_M)$). So we see that $h_M\cdot f_1$ approximates $f_1,0,\ldots,0$ up to order $N$ if $M\gg N$. But since we can approximate the sum of vectors as well, we are done.
\end{proof}

\begin{lem}
\label{L72}
Let $K\subset L$ be a finite extension of fields. Let $w:L^\times\to\Z$ be a discrete valuation. Then $w|_{K^\times}=ev$ for some discrete valuation $v$ on $K$ and integer $e\geq 1$.
\end{lem}

\begin{proof}
It is clear that $w|_{K^\times}$ satisfies the properties of a discrete valuation except for possibly being surjective. So $e=[\Z:w(K^\times)]$ works provided that $w(K^\times)\neq 0$. But if $w(K^\times)=0$ then $K\subset\mathcal{O}_w\subset L$, which by the previous lemma $\mathcal{O}_w$ is a field, which is a contradiction.
\end{proof}

\begin{defn}
In the situation of Lemma \ref{L72} we say that $w$ \textbf{extends} $v$.
\end{defn}

\begin{lem}
\label{L73}
If $K\subset L$ is finite and $v:K^\times\to\Z$ given then there are at most a finite number of $w:L^\times\to\Z$ such that $w|_{K^\times}=e_N\cdot v$.
\end{lem}
\begin{proof}
Say we have pairwise distinct $w_1,\ldots, w_r$ on $L$ extending $v$. Pick (by a previous lemma) $f_i\in L$ such that $w_i(f_i)>0$ and $w_j(f_i)<0$ for $j\neq i$. We'll show that $f_1^{-1}\cdots f_n^{-1}$ are linearly independent, which suffices. Suppose $\sum_i a_if_i^{-1}=0$ for some $a_1,\ldots, a_r\in K$. Clearing denominators, we may assume that $a_i\in\mathcal{O}_v$ and that $a_{i_0}\notin\fr m_r$ for some $i_0$. Then
\[w_{i_0}\left(\sum_i a_if^{-1}_i\right)\geq\min w_{i_0}(a_if^{-1}_i)\geq\min e_{i_0}v(a_i)+w_{i_0}(f_i^{-1}).\]
But the first term is greater than or equal to zero (equal if $i=j$) and the second is always greater than equal to zero (equal if $i=i_0$). Hence this could not have been zero. So $\sum_i a_if_i^{-1}\neq 0$.
\end{proof}

\section*{Class 13}


\begin{defn}
$v\in C$ is a \textbf{zero} (respectively \textbf{pole}) of $f \in K^{\times}$ if $v(f) >0$, respectively $v(f)<0$. (The integers are called the order of the zero, and $(-1)\cdot$the order of the pole.)
\end{defn}
\begin{lem}
\label{L74}
Let $K$ be a finitely generated extension of $k$ with transcendence degree 1. If $f\in K^{\times}$, then $f$ has a finite number of zeros and poles.
\end{lem}
\begin{proof}
We look at the map $k(t) \to K$ that takes $t\to f$. Then $n = [K : k(t)] < \infty$. If $v$ is a zero of $f$ then $v$ is an extension to $K$ of the valuation $\ord_{t=0}$ on $k(t)$. By Lemma \ref{L73}, there exist finitely many of these.
\end{proof}

\begin{lem}
\label{L75}
Let $k\subset L$ be a finite extension of (arbitrary) fields. Let $v:K^{\times} \to \Z$ be a discrete valuation. Then there exists $w:L^{\times} \to \Z$ extending $v$.
\end{lem}
Last lecture we proved the finiteness of such extensions and this is the existence statement. We will not prove this fact yet; we will do so later using the technology of divisors. There are two approaches we could use to prove it even now, but it would be quite hard.
\begin{enumerate}
\item Look at the completion $\hat K$ of $K$. Then look at $\hat L = L \otimes_K \hat K$. This is not necessarily a field. Then choose a quotient $\hat L \twoheadrightarrow L'$ which is a field. Then we can take $v\circ \Nm_{L'/\hat K}$ to be the extension.
\item $\mathcal{O}_v \subset K$ can be extended to a valuation ring $\mathcal{O}_v \subset \mathcal{O}_w \subset L$ with f.f.$(\mathcal{O}_w) = L$. Then we show that $\mathcal{O}_w$ is a discrete valuation ring.
\end{enumerate}

\begin{lem}
\label{L76}
Let $K/k$ be finitely generated with transcendence degree 1, then any $f\in K, f\not \in k$ (i.e. nonconstant) has a pole.
\end{lem}
\begin{proof}
Look at $k(t) \to K$ that takes $t\to f$, then $v = \ord_{\infty} \in P^1_k$ has an extension $w$ to $K$. $w(f)<0$ because $\ord_{\infty}(t)<0$.
\end{proof}
\begin{exmp}
The poles of $\frac{t^2}{t+1}$ are $-1, \infty$. Its zeros are $0$, counted twice.
\end{exmp}

Let us now fix $k\subset K$ a function field over a curve $C=\{\text{discrete valuations on }K/k\}$.

\begin{defn}
A \textbf{divisor} is a formal sum $D=\sum_{v\in C}n_v v$ with $n_v\in\Z$ almost all zero. $D$ is said to be \textbf{effective} iff $n_v\geq0$ for all $v\in C$. Moreover, we say that $D_1\geq D_2$ iff $n_{1v}\geq n_{2v}$ for all $v\in C$, i.e. $D_1-D_2$ is effective. For $f\in K^\times$, the \textbf{principal divisor} associated to $f$ is $(f)=\sum_{v\in C}v(f)v$. This is well-defined by Lemma \ref{L74}. Observe that $(fg)=(f)+(g)$. Next we define the \textbf{zero divisor} of $f$ to be $(f)_0=\sum_{v(f)>0} v(f)v$ and the \textbf{pole divisor} of $f$ to be $(f)_\infty=\sum_{v(f)<0}v(f)v$. Observe that $(f)=(f)_0+(f)_\infty$. If $D$ is a divisor $D=\sum n_vv$ then its \textbf{degree} is $\deg(D)=\sum n_v\in\Z$.

Given a divisor $D=\sum n_vv$ we set
\[L(D)=\{f\in K^\times \mid v(f)\geq -n_v \forall n_v\in C\}\cup\{0\}.\]
This is a $k$-vector space. Let $\ell(D)=\dim_k L(D)$.
\end{defn}
  
\begin{lem}
\label{L77}
If $D_1\geq D_2$ then $L(D_2)\subset L(D_1)$ has codimension at most $\deg(D_1)-\deg(D_2)$.
\end{lem}
\begin{proof}
It suffices to prove this for $D_2 = D_1 - v$ for some $v\in C$, and then the result follows by induction. We have the exact sequence of vector spaces:
\[     0 \to L(D_2) \to L(D_1) \to \pi^{-n}\mathcal{O}_v/\pi^{-(n-1)}\mathcal{O}_v        \]
where $\pi \in \mathcal{O}_v$ is the uniformizer and $n$ is the coefficient of $v$ in $D_1$. We are done, because $\pi^{-n}\mathcal{O}_v/\pi^{-(n-1)}\mathcal{O}_v$ is a 1-dimensional $k$-vector space.
\end{proof}

\begin{lem}
\label{L78}
Say $D = D_1 - D_2$ with $D_1, D_2 \geq 0$. Then $l(D) \leq \deg(D_1) + 1$.
\end{lem}
\begin{proof}
Observe that $L(D) \subset L(D_1)$. Observe also that $L(0) = k$, because by Lemma \ref{L76} any nonconstant function has a pole. So the statement holds for the trivial divisor. Now we can induct: add points and use Lemma \ref{L77}. For example $L(v) \supset L(0)$ of codimension $\leq 1$. So either $l(v) = 0$ or $l(v) = 1$. Then $L(v_1 + v_2) \supset L(v_1)$ has codimension $\leq 1$, etc.
\end{proof}
\begin{cor*}
$l(D) \leq \infty$.
\end{cor*}
\begin{defn}
$D_1, D_2$ are called \textbf{rationally equivalent} (notation $D_1 \sim_{\text{rat}} D_2$) if there exists $f \in K^{\times}$ such that $D_1 = D_2 + (f)$. This is obviously an equivalence relation.
\end{defn}

\begin{lem}
\label{L79}
If $D_1 \sim_{\text{rat}} D_2$, then $l(D_1) = l(D_2)$.
\end{lem}
\begin{proof}
We can pass from $L(D_1)$ to $L(D_2)$ and viceversa by multiplying functions by $f^{-1}$ or $f$ respectively.
\end{proof}
\begin{exmp}
If $K = k(t)$, then the degree of a principal divisor is 0. Namely every nonzero rational function looks like:
\[       f = c \prod_{\lambda \in k} (t - \lambda)^{n_{\lambda}}       \]
Such that almost all $n_{\lambda} = 0$, and $c\in k^{\times}$. Then:
\[    (f) = \sum_{\lambda} n_{\lambda} \ord_{t=\lambda} - \left( \sum_{\lambda} n_{\lambda}\right) \cdot \infty   \]
Then $\deg(f) = \sum_{\lambda} n_{\lambda} - \sum_{\lambda} n_{\lambda} = 0$. Caution: in the equation above, the symbol $\infty$ represents the valuation which is the point at infinity on $P^1_K$. Clearly, any divisor $D$ is rationally equivalent to $n\cdot \infty$ for some $n$. (Substract some appropriately chosen $(f)$ to cancel out the other terms.) Then this $n$ is the degree of $D$. Then we can compute $l(D)$, since:
\begin{align*}   L(n\cdot \infty) &= \{ f\in k(t) : \ord_{t=\lambda}(f) \geq 0 , \ord_{t=\infty}(f) \geq -n  \}  \\
 &= \{ f \in k[t] : \ord_{t=\infty}(f) \geq -n \} \\
 &= \{ f \in k[t] : \deg(f) \leq n  \}
\end{align*}
This has dimension $n+1$. Hence the bound in Lemma \ref{L78} is optimal.


\end{exmp}




\section*{Lecture 14}
\begin{prop}
\label{L80}
Let $K/k$ be a function field and $f\in K$ be nonconstant. Set $n = [K:k(t)]$. Then $n = \deg(f)_0 = \deg(f)_{\infty}$.
\end{prop}
\begin{proof}
\textbf{STEP 1}. $n\geq \deg(f)_0$. To show this, say:
\[       (f)_0 = n_1v_1 + \dots n_r v_r \;\;\;\;\;\;\; n_i>0      \]
For $1\leq i \leq r$ and $1\leq j \leq n_r$ pick $z_{ij} \in K$ such that $z_{ij}$ has order of vanishing $j$ at $v_i$ and it has order of vanishing $>n_i$ at all other points. This is possible by Thm \ref{L71}. We claim that $z_{ij}$ are linearly independent over $k(f)$. If $\sum a_{ij} z_{ij} = 0$ we can clear denominators and get $a_{ij} \in k[f]$, with not all $a_{ij}$ zero at $f=0$. Say $a_{i_0j_0} \neq 0$, with $j_0$ minimal. Then we see that $v_{i_0}(\sum a_{ij}z_{ij}) = j_0$. This is because, if $a\in k[f]$ is zero at $f=0$ then $v_{i_0}(a)\geq n_{i_0}$ because $v_{i_0}(f) = n_{i_0}$. Therefore $\sum a_{ij} z_{ij} \neq 0$.
\\
\\
\textbf{STEP 2}. $n\geq \deg(f)_{\infty}$. To see this, apply step 1 to $f^{-1}$.
\\
\\
\textbf{STEP 3}. $n\leq \deg(f)_{\infty}$. To prove this, write $K = \oplus_{i=1}^n k(t) \cdot u_i$ for some $u_1, \dots, u_n \in K$. Suppose $v$ is a valuation of $K$ lying over $\ord_{f=\lambda} , \lambda \in k$ such that $v(u_i)<0$. Then after multiplying $u_i$ by $(f-\lambda)^{n_{u_i}}$ the valuation $v$ is no longer a pole of $u_i$, and we haven't introduced any new poles for $u_i$, except perhaps poles of $f$. After doing this finitely many times, we may assume that poles of $u_i \subset$ poles of $f$. This implies that $\exists m_i\geq 0$ such that $(u_i)_\infty \leq (m_i + 1)(f)_{\infty}$. Choose $m\geq \max \{m_i\}$. Then $u_i, fu_i, f^2u_i, \dots f^{m-m_i -1}u_i \in L\big(m(f)_{\infty}\big)$. Hence $l\big(  m(f)_{\infty} \big) \geq \sum_{i=1}^n (m-m_i) = nm - \sum_{m_i}$. We are done, because by Lemma \ref{L78} $l\big(  m(f)_{\infty} \big) \leq m \deg\big((f)_{\infty}) +1$.
\end{proof}

\begin{cor}
\hspace{1mm}
\label{L81} 
\begin{enumerate}
\item The degree of a principal divisor is 0.
\item If $D\sim_{\text{rat}} D'$ then $\deg(D) = \deg(D')$.
\end{enumerate}
\end{cor}

\begin{lem}
\label{L82}
Let $D$ be a divisor, TFAE:
\begin{enumerate}
\item $l(D)>0$;
\item $D$ is rationally equivalent to an effective divisor.
\end{enumerate}
\end{lem}
\begin{proof}
For $f\in K^{\times}$ we have $D+(f) \geq 0 \Leftrightarrow f \in L(D)$.
\end{proof}

\begin{prop}
\label{L83}
There exists a constant $c$ such that $\deg(D) - l(D) \leq c$ for all $D$.
\end{prop}
\begin{proof}
Let $f\in K$ be noncosntant, write $K = \oplus k(f) u_i$ with $(u_i)_{\infty} \leq (m_i + 1) (f)_{\infty}$ as in the proof of proposition \ref{L80}. We saw that:
\[     \deg(m(f)_{\infty}) - l(m (f)_{\infty}) \leq (\sum m_i) + 100   \]
Hence the proposition holds for $D = m(f)_{\infty}$ with $c = (\sum m_i) + 100$. Now let $D$ be arbitrary. By lemma \ref{L77} it suffices to find $D' \geq D$ such that $c\geq \deg(D') - l(D') \geq \deg(D) - l(D)$. We may also replace $D$ by $D + (f)$, as this doesn't change $\deg(D)$ or $l(D)$. By the same argument as in the proof of prop \ref{L80}, $D$ is rationally equivalent to a divisor $D'$ such that the support of the positive part of $D'$ is included in the poles of $f$. Then $D' \leq m(f)_{\infty}$ for some $m>>0$.
\end{proof}

\begin{defn}
The \textbf{genus} of $K/k$ is the smallest integer $g$ such that $\deg(D) - l(D) \leq g-1$ for all divisors $D$.
\end{defn}
\begin{rem}
$g\geq 0$ because for $D=0$ we have $l(D) = 1$.
\end{rem}
If we define the \textbf{speciality index} $h^1(D)$ by the formula:
\[      l(D) - h^1(D) = \deg(D) + 1 -g     \]
Then the \textbf{Riemann-Roch theorem} gives a meaning to $h^1(D)$. We have seen in the proof of proposition \ref{L83} that given any nonconstant $f\in K$ then the sequence of integers $\deg(m(f)_{\infty}) - l(m(f)_{\infty})$ is nondecreasing and has an upper bound. This upper bound is $g-1$.
\begin{rem}
In proposition \ref{L83} we have proved asymptotic Riemann-Roch.
\end{rem}

\begin{exmp}
The genus of $P_k^1$. We know that any divisor $D$ is rationally equivalent to $d\infty$ for some integer $d = \deg(D)$. Also:
\[       L(\infty) = \{ f\in k[t] : \deg (f) \leq d  \}  \]
So $l(d\infty) = d+1$ if $d>0$, and 0 otherwise. Then:
\[      \deg(d\infty) - l(d\infty) = \left\{  \begin{array} {c} -1 \text{ if } d\geq 0 \\ d \text{ if } d<0  \end{array}  \right.    \]
\end{exmp}
\begin{exmp}
$K = $ f.f.$(\C[x,y]/(y^2 - x^3 -x))$. What Riemann would say is that this is the surface corresponding to $\sqrt{x^3+x}$. We could try to compute $L(m\infty ')$, where we have that $2\cdot \infty' =(x)_{\infty}$, where $x$ is regarded as an element of $K = \C[x,y]$. Note also that elements of $K$ are of the form $a + by$, with $a,b\in \C(x)$.
\end{exmp}

\section*{Class 15}

Let $K/k$ be finitely generated of transcendence degree 1. For $v\in C$ an algebraic curve we define $K_v$ to be the v-adic topology $f.f.(\hat{O}_v)$, i.e. the set of all Cauchy sequences $(x_n)$ where $v(x_n-x_m)\to\infty$ as $n,m\to\infty$ (modulo null sequences).

\begin{lem}
\label{L84}
$K_v$ is a field. In fact, if $z\in K$ is a uniformizer at $v$ then $\hat{O}_v=k[[z]]$ and $K_v=k((z))$, the field of Laurent series in $z$ over $k$.
\end{lem}
\begin{proof}
Omitted. The idea is that there is a map $k[z]\to\mathcal{O}_v$ which induces an isomorphism on completions $k[[z]]\to\hat{\mathcal{O}}_v$.
\end{proof}


\begin{rem}
In the literature, one often finds written $\mathcal{O}_v$ for $\hat{\mathcal{O}}_v$.
\end{rem}

\begin{defn}
The \textbf{ring of adeles} is
\[\mathbb{A}_k=\{(x_v)\in\prod_{v\in C} K_v\mid \text{for almost all }v, x_v\in\hat{\mathcal{O}}_v\} \]
\end{defn}

\begin{lem}
\label{L85}
Let $k$ be a field. Let $L=k((t))$. Then $\Hom_{cts}(L,k)$, the set of $\lambda:L\to k$ $k$-linear such that $\lambda|_{t^nk[[t]]}\equiv 0$ for some $n$, is a one-dimensional vector space over $L$.
\end{lem}
\begin{proof}
Consider the map $\lambda_0$ given by $f\to\res_{t=0}\frac{fdt}{t}$ given by $\sum_i a_it^i\mapsto a_0$. We claim that any $\lambda$ is a multiple of $\lambda_0$. Namely, say $\lambda(t^m)= b_m\in k$ for $m\in\Z$. Then $b_m=0$ for $m\gg0$. So we can introduce $g=\sum b_mt^{-m}\in L$. We then claim that $\lambda(f)=\lambda_0(gf).$ It suffices to check this for $t^i$ (due to the basis of $L/t^nk[[t]]$). Then $\lambda(t^i)=b_i$ and $\lambda_0(t^i g)=\res_{t=0}\left(\sum t^{i-m}b_m\frac{dt}{t}\right)=b_i$.
\end{proof}

\begin{defn}
The module of \textbf{covectors} is the module
\[\mathbb{A}_k^*=\{\lambda=(\lambda_v)\in\prod_v\Hom_{cts}(K_v,k)\mid \text{for almost all }v: \lambda_v|_{\hat{\mathcal{O}}_v}=0\}.\]
\end{defn}

\noindent Recall now that we have defined a genus $g$ such that
\[\ell(D)-h^1(D)=\deg(D)+1-g\]
for some $h^1(D)\geq 0$.

\begin{lem}
\label{L86}
If $D'\geq D$ then $h^1(D')\leq h^1(D)$.
\end{lem}
\begin{proof}
This is a reformulation of Lemma \ref{L77}.
\end{proof}

\begin{lem}
\label{L87}
For every $D$ there exists a $D'$ with $D'\geq D$ and $h^1(D')=0$.
\end{lem}
\begin{proof}
By the construction of $g$, there exists a $D_0$ with $h^1(D_0)=0.$ Choose $D'\geq D_0$ and $D'\geq D$. Apply Lemma \ref{L86}.
\end{proof}

\noindent Now note that there exists a canonical map $K\to\mathbb{A}_k$ given by $f\mapsto(f,f,f,\ldots)$.

\begin{defn}
A \textbf{Weil differential} is a covector $\lambda\in\mathbb{A}_k^*$ such that $\lambda(f)=0$ for all $f\in K$.
\end{defn}

\begin{defn}
Given $\lambda=(\lambda_n)\in\mathbb{A}_k^*$ and $D$ a divisor, we say $\lambda\geq D$ if and only if $\lambda_v|_{\pi_v^{-n_v}\mathcal{O}_v}=0$ for all $v$.
\end{defn}

\begin{thm}
\label{L88}
$h^1(D)$ is precisely the dimension of the $k$-vector space of Weil differentials $\lambda$ such that $\lambda\geq D$.
\end{thm}
\begin{proof}
Pick any $D'=\sum_vn_v'v\geq D=\sum_vn_vv$ with $h^1(D')=0$ (by Lemma \ref{L86}). Then we have the short exact sequence
\[0\longrightarrow \frac{L(D')}{L(D)}\longrightarrow \prod_v\frac{\pi_v^{-n_v'}\mathcal{O}_v}{\pi_v^{-n_v}\mathcal{O}_v}\longrightarrow \text{v.s. of dim } h^1(D)\longrightarrow0.\]
Note that the dimensions work out by our formula for $g$ and that in fact
\[\frac{\pi_v^{-n_v'}\mathcal{O}_v}{\pi_v^{-n_v}\mathcal{O}_v}=\frac{\pi_v^{-n_v'}\hat{\mathcal{O}}_v}{\pi_v^{-n_v}\hat{\mathcal{O}}_v}.\]
We can put all these sequences together for varying $D'\geq D$ with $h^1(D')=0$:
\[0\longrightarrow\frac{K}{L(D)}\longrightarrow\prod_v\frac{K_v}{\pi_v^{-n_v}\mathcal{O}_v}\longrightarrow\text{fixed v.s. of dim }h^1(D)\longrightarrow 0.\]
\end{proof}

\begin{cor}
The $k$-vector space of regular Weil differentials, i.e. $\lambda\geq 0$, has dimension $g$.
\end{cor}

\begin{thm}
\label{L90}
The collection of all Weil differentials is a one-dimensional $K$-vector space.
\end{thm}
\begin{proof}
It is already clear that the dimension is nonzero. Suppose now that $\lambda_1,\lambda_2$ are $K$-linearly independent Weil differentials. Choose $D$ such that $\lambda_i\geq D$. Pick a divisor $E$ with huge degree. For $f_1,f_2\in L(E)$ we get that $\lambda=f_1\lambda_1+f_2\lambda_2\geq D-E$ (this is an easy local computation). Thus, $h^1(D-E)\geq 2\ell(E)$ by Theorem \ref{L88}, which by the formula for $g$ is greater than or equal to $2(\deg E+1-g)$. On the other hand, $\ell(D-E)-h^1(D-E)=\deg D-\deg E+1-g$. Putting these together we find that $\ell(D-E)\geq \deg(E)+\text{const}\cdot(\text{something depending on } D, g)$. This is a contradiction.
\end{proof}

Now pick a (fixed) non-zero differential $\lambda_0$. Set $K_c$ to be the largest divisor such that $\lambda_0\geq K_c$ (we leave it as an exercise to show that such a thing exists). In other words, if $K_c=\sum m_vv$ then $\lambda_{0,v}|_{\pi_v^{-m_v}\hat{\mathcal{O}}_v}=0$ but $\lambda_{0,v}|_{\pi_v^{-m_v-1}\hat{\mathcal{O}}_v}\neq0$.

\begin{lem}
\label{L91}
The rational equivalence class of $K_c$ is well-defined.
\end{lem}
\begin{proof}
Let $\lambda$ be another non-zero differential. Then by Theorem \ref{L90} we see that $\lambda=f\cdot \lambda_0$ and $K_c$ is changed by adding $(f)$.
\end{proof}

\begin{defn}
The divisor of $K_c$ is called the \textbf{canonical divisor of $C$}.
\end{defn}

\begin{thm}[Riemann-Roch]
For a divisor $D$, $\ell(D)-\ell(K_c-D)=\deg D+1-g$.
\end{thm}
\begin{proof}
The result follows easily by Theorems \ref{L88} and \ref{L90}:
\begin{align*}
\ell(K_c-D)&=\dim\left\{f\mid (f)+K_c-D\geq 0\right\}\\
&=\dim\left\{f\mid (f)+K_c\geq D\right\}\\
&=\dim\left\{f\mid f\lambda_0\geq D\right\}\\
&=\dim\left\{\text{Weil differentials }\lambda \mid \lambda\geq D\right\}\\
&=h^1(D)
\end{align*}
\end{proof}

\section*{Class 16}

\subsection*{Modules of differentials}

Let $A\to B$ be a ring map. Let $M$ be a $B$-module.

\begin{defn}
An \textbf{$A$-derivation} $D:B\to M$ is an $A$-linear map satisfying the Leibniz rule:
\[D(b_1b_2)=b_1D(b_2)+b_2D(b_1).\]
It is clear that $D(1)=0$.
\end{defn}

\begin{exmp}
Consider $A\to B=A[x_1,\ldots,x_n]$ and $D=(\partial_{x_1},\ldots,\partial_{x_n}):B\to\oplus_{i=1}^n B$.
\end{exmp}

\begin{lem}
\label{L93}
There exists a universal $A$-derivation $d:B\to\Omega_{B/A}$, i.e. for any $A$-derivation $D:B\to M$ there exists a unique $B$-linear map $\theta:\Omega_{B/A}\to M$ such that $D=\theta\circ d$.
\end{lem}
\begin{proof}
Let $\Omega_{B/A}$ be the free $B$-module on the symbols $d(b)$ for $b\in B$ and then quotient out by the $B$-submodule generated by $d(a)$ for $a\in A$, $d(b_1+b_2)-d(b_1)-d(b_2)$ for $b_1,b_2\in B$, and by $d(b_1b_2)-b_1d(b_2)-b_2d(b_1)$ for $b_1,b_2\in B$. Now it's clear that our map $\theta$ from $\Omega_{B/A}$ to $M$ should take $db$ to $D(b)$. 
\end{proof}

\begin{exmp}
If $B=A[x_1,\ldots,x_n]$ then $\Omega_{B/A}=\oplus_{i=1}^n A[x_1,\ldots,x_n]dx_i$. Then $d(f)=\sum_{i=1}^n\left(\frac{\partial f}{\partial x_i}\right)dx_i$.
\end{exmp}

\begin{exmp}
Suppose $P\in A[x,y]$ and $B=A[x,y]/(P)$. Then $\Omega_{B/A}=\text{coker}\left(B\longrightarrow Bdx\oplus Bdy \right)$, where the map takes $1\mapsto dP=\frac{\partial P}{\partial x}dx+\frac{\partial P}{\partial y}dy$.
\end{exmp}

\begin{lem}
\label{L94}
If $S\subset B$ is a multiplicative subset, then $\Omega_{S^{-1B/A}}=S^{-1}\Omega_{B/A}$ (as $S^{-1}B$ modules).
\end{lem}
\begin{proof}
See, in the Stacks Project Tag 00RT.
\end{proof}

\begin{exmp}
$\Omega_{k(t)/k}=k(t)dt$ because $\Omega_{k[t]/k}=k[t]dt$.
\end{exmp}

\begin{lem}
\label{L95}
Let $K/k$ be a finitely-generated field extension of transcendence degree 1, with $k$ algebraically closed. Then $\Omega_{K/k}$ is a one-dimensional vector space over $K$.
\end{lem}
\begin{proof}
By field theory there exists
\begin{enumerate}[(a)]
\item an element $x\in K$ such that $K/k(x)$ is finite and separable;
\item An element $y\in K$ such that $K$ is generated by $x,y$ over $k$ (theorem of the primitive element).
\end{enumerate}
Denote by $B$ the domain $k[x,y]/(P)$.
Let $P(x,y)\in k[x,y]$ be an irreducible polynomial such that $P(x,T)$ is $k(x)^\times$ multiple of the minimial polynomial for $y$ over $k(x)$. Then $K=S^{-1}(k[x,y]/(P))$ whence $\Omega_{K/k}=S^{-1}\Omega_{(k[x,y]/(P))/k}=S^{-1}\left(\text{coker}\left(B\overset{(\frac{\partial P}{\partial x},\frac{\partial P}{\partial y})}{\to}Bdx\oplus Bdy\right)\right)=\text{coker}\left(K\overset{(\frac{\partial P}{\partial x},\frac{\partial P}{\partial y})}{\to}Kdx\oplus Kdy\right)$. By $(a)$ above we see that $\frac{\partial P}{\partial y}\neq 0$ in $K$. Hence the map is not the zero map and the cokernel has dimension one.
\end{proof}

\begin{lem}[Scholium]
\label{L96}
If $x\in K$ and $K/k(x)$ is finite and separable then $dx$ is a basis for $\Omega_{K/k}$.
\end{lem}

Now let us return to Weil differentials and the setup from previous classes. Let $\omega\in\Omega_{K/k}$ be nonzero. Then we will construct a covector $\lambda_\omega\in\mathbb{A}_K^*$ by setting $\lambda_\omega((f_v))=\sum\res\left(f_v\omega\right)$. Let us investigate what this means. ``As my five-year old likes to say: what the heck?''

Fix $v\in C$ and pick a uniformizer $z_v$. Then $K_v=k[[z_v]]$, canonically (by Lemma \ref{L84}). There are maps
\begin{equation}
\begin{tikzcd}
\Omega_{K/k}\arrow{r}{\text{functoriality}}&\Omega_{k_v/k}\arrow{r}{\text{Taylor expansion}}&k((z_v))dz_v
\end{tikzcd}
\end{equation}
where the second map sense $\sum f_i dg_i\mapsto \sum\left(\frac{dg_i}{dz_v}\right)dz_v$.
Note carefully that the Taylor expansion is not an isomorphism. This can be fixed by replacing $\Omega_{K_v/k}$ by $\Omega_{K_v/k,\text{cont}}$, i.e. continuous differentials.
Then we set $\res_v(f_v\omega)=\res_{z_v=0}\left(\text{Taylor}_{z_v}(f_v\omega)\right)$.

\begin{lem}
\label{L97}
The RHS is independent of choice of local uniformizer.
\end{lem}
\begin{proof}
Suppose $\eta\in\Omega_{K_v/k}$ and $h(z_v)dz_v=\text{Taylor}_{z_v}(\eta)$ and $\tilde h(\tilde z_v)d\tilde z_v=\text{Taylor}_{\tilde z_v}(\eta)$.
Then $\tilde z_v=a_1z_v+a_2z_v^2+\cdots$ with $a_1\neq 0$. We are given that $h(z_v)dz_v=\tilde h(a_1z_v+a_2z_v^2+\cdots)d(a_1z_v+a_2z_b^2+\cdots)$. Then we see that $\tilde h=\sum b_j(\tilde z_v)^j$ is a Laurent series as it has finitely many negative terms and that $\res_{\tilde z_v=0}\left(\tilde h(\tilde z_v)d\tilde z_v\right)=b_{-1}$. It remains to show that the coefficient of $z_v^{-1}$ in
\[\sum b_j(a_1z_v+a_2z_v^2+\cdots)^j(a_1+2a_2z_v+\cdots)\]
is equal to $b_{-1}$. Factoring, we find
\[\sum a_1^{j+1}b_jz_v^j\left(1+\frac{a_2}{a_1}z_v+\cdots\right)^j\left(1+\frac{2a_2}{a_1}z_v+\cdots\right).\]
We claim that the coefficient of $z_v^{-1}$ is now $b_{-1}$.
\end{proof}

\begin{lem}
\label{L98}
For $\omega\in\Omega_{K/k}$ and $(f_v)\in\mathbb{A}_k$ we have $\res(f_v\omega)=0$ for almost all $v$.
\end{lem}
\begin{proof}
Say $K=\text{Frac}k[x,y]/(P)\left(\right)$ with $\frac{\partial P}{\partial y}\neq 0$ in $K$ as before. Then $\omega=fdx$ for some $f\in K$. Let $v$ be a discrete valuation of $K/k$ centered on $k[x,y]/(P)$, i.e. $v(h)\geq 0$ for all $h\in k[x,y]/(P)$ and there are only a finite number of $v$ which are not centered on $k[x,y]/(P)$. If $v$ is not a pole or zero of $f$ and $v$ is also not a zero of $\partial P/\partial y$ then 
Let $(\alpha,\beta)\in k^2$ be coordinates of the point in $\{P=0\}\subset k^2$ corresponding to $v$. Then
\begin{enumerate}[(a)]
\item $x-a$ is a uniformizer at $v$;
\item $\omega=cd(x-\alpha)$ with $c$ a unit in $\mathcal{O}_v^*$.
\end{enumerate}
Hence $\res(f_v\omega)=0$ if $f_v\in\hat{\mathcal{O}}_v$.
\end{proof}

How do we prove (a) and (b) above? Since $(\alpha,\beta)$ is a nonsingular point of $\{P=0\}$ (as defined in the exercises) we have that
\[R=\left(k[x,y]/(P)\right)_{(x-\alpha,y-\beta)}\]
is a regular local ring of dimension one and hence a discrete valuation ring. Since $R\subset\mathcal{O}_v$ is an inclusion of discrete valuation rings with same fraction field, we see that $R=\mathcal{O}_v$. So you can read off whether $x-\alpha$ is a uniformizer from the structure of $R$, i.e. show $x-\alpha\notin \fr m_{R}^2$.

Consequently, we see that $\lambda_\omega:\mathbb{A}_k\to k$ given by $(f_v)\mapsto \sum_v\res_v(f_v\omega)$ is a covector. If $\omega\neq 0$ then $\lambda_\omega\neq 0$ (easy).

\begin{thm}
\label{L99}
The covector $\lambda_\omega$ is a Weil differential and the assignment $\omega\mapsto\lambda_\omega$ is an isomorphism from $\Omega_{K/k}$ to the space of Weil differentials.
\end{thm}
\begin{proof}
It now suffices to show that if $f\in K$ then $\sum_{v\in C}\res_v(f_\omega)=0$. See below.
\end{proof}

\begin{thm}
\label{L100}
The sum of the residues of an element in $\Omega_{K/k}$ is zero. 
\end{thm}

\begin{exmp}
Take $K=k(t)$ and $\omega=dt$. There is a pole of order 2 at infinity: $d(1/s)=-1/s^2 ds$. It's clear that $\res_\infty$ is zero in this case.
Let's instead try $\omega=dt/t$. We see that $d(1/s)/(1/s)=-ds/s$ and hence the residue at infinity is -1 which cancels out the residue of 1 at zero.
\end{exmp}

Let us now prove this theorem (for characteristic zero).
\begin{proof}
Let us split this into two cases. For this first, assume $K=k(t)$. In this case
\[\omega=f(t)dt=\sum\]
Missing notes here
\end{proof}


\section*{Class 17}
We resume the proof of Thm \ref{L100}. We first recall the statement. Let $K/k$ be a function field, and let $\omega \in \Omega_{K/k}$. Then $\sum_{v\in C} \Res_v(\omega) = 0$.
\begin{proof}
\textbf{Case I.} $K = k(t)$. In this case, $\omega = f(t) dt = \sum \frac{c_i dt}{(t-\alpha_i)^{e_i}}$. We actually only need to check when $\omega = \frac{dt}{(t-\alpha_i)^{e_i}}$, and we did this last time. 
\\
\\
\textbf{Case II.} $K \supset k(t)$ separable and finite. Let $\omega = dt$. Then we claim that:
\[       \sum_{w \text{ valuation of } K/k} \Res_w(f\omega) = \sum_{v \text{ valuation of } k(t)/k} \Res_v (\Tr_{K/k(t)} (f) \cdot \omega)     \]
This claim follows from Lemma \ref{L101}. Then we are done, since the RHS is 0 by Case I.
\end{proof}

\begin{lem}
\label{L101}
Let $k((x)) \subset k((y))$ be a finite separable extension. For $f\in k((y))$ we have:
\[  \Res_{y=0}(\text{Taylor}_y(fdx)) = \Res_{x=0}(\text{Taylor}_x(\Tr_{k((y))/k((x))} (f) dx)) \]
\end{lem}
\begin{proof}
We prove this lemma for character 0. In this case, there exists a uniformizer $y' \in k((y))$ such that $x = (y')^e$ for some $e$. (Using the fact that $\overline{k((x))} = \cup_e k((x^{y_e}))$ if $k = \bar k$ has characteristic 0.) By Lemma \ref{L97}, the LHS does not depend on the choice of uniformizer. We may replace $y$ by $y'$ and assume $y^e = x$. Then we get:
\[      LHS = \Res_{y=0} (ef(y) y^{e-1} dy) = e( \text{coeff. of } y^{-e} \text{ in }f)       \]
\[      RHS = \Res_{x=0}(\sum_{i=0}^{e-1} f(\xi^i y) dx) =  e( \text{coeff. of } y^{-e} \text{ in }f)       \]
In the above equation $\xi$ is a primitive $e^{th}$ root of unity of $k$. The last equality follows since:
\[        y^n \to \sum_{i=0}^{e-1} \xi^{in} y^n = \left\{ \begin{array} {c} 0 \text{ if } \;\; e \not | n \\ ey^n \text{ if } e | n \end{array} \right.         \]
\end{proof}

Let us discuss a few applications. First note that the canonical divisor class equals the divisor class of a nonzero meromorphic differential form, i.e. if $\omega\in\Omega_{K/k}$ nonzero then
\[K_c\sim_{\text{rat}}\sum_{v\in C}(\text{ord}_v\omega)v.\]

Note also that we can use this to show that $\deg K_c=2g-2$. This is because $\ell(0)-\ell(K_c)=1-g$ and
$\ell(K_c)-\ell(0)=\deg K_c + 1-g$. Hence $0=\deg K_c+2-2g$.

Furthermore $\ell(K_c)=g$ (number of regular differential forms on $C$). Finally, note that if $\deg K_c\geq 2g-1$ then $\ell(D)=\deg D+1-g$ simply because $\deg(K_c-D)<0$ so $\ell(K_c-D)=0$.


\begin{lem}
\label{L102}
Let $S\subset C$ be a finite nonempty subset. Then,
\begin{enumerate}[(i)]
\item $A=\cap_{v\in C,v\notin S}\mathcal{O}_v$ is a normal domain with fraction field $K$ of finite type over $k$
\item $\dim A=1$
\item Closed points of $\Spec A$ are in one-to-one correspondence with points $C\setminus S$.
\end{enumerate}
\end{lem}
\begin{proof}
$A$ is a normal domain because it is the intersection of the normal domains $\mathcal{O}_v$. Once we show finite-type, $A$ will be Noetherian of dimension one, so all of its local rings will be DVRs, i.e. points of $A$ correspond to $v$. (Then $\varepsilon$ more to finish the proof). Let us prove finite-type. Say $S=\{v_1,\ldots,v_n\}$. For each $i$ we have
\[\ell(ev_i)=e+1-g, e\geq 2g-1.\]
Hence we can pick $f_{i,j}\in L(jv_i)\setminus L((j-1)v_i)$ for $j=2g,\ldots 4g-1$. Note that $f_{i,j}\in A$. We claim that $A$ is generated as a $k$-algebra by $f_{i,j}$ and $L((2g-1)(v_1+\ldots+v_n))$. Namely, if $f\in A$, then
\[(f)_\infty =\sum_{i=1}^n e_iv_i, e_i\geq 0.\]
If $e_i\geq 2g$ for some $i$. Then expanding in a uniformizer at $v_i$, we see
that $f'=f-c$ (some monomial $f_{i,j}$). Some $c\in k^*$ has a lower value of $e_i$. So $f'$ has a lower degree of pde divisor. Continue until we reach $f\in L((2g-1)(\sum v_i)).$
\end{proof}

\begin{rem}
Take $C$, $K/k$ as above. Put cofinite topology on $C$, and for $U\subset C$ open, set $\mathcal{O}_v(U)=\cap_{v\in U}\mathcal{O}_v$ (finite-type normal $k$-algebra with MaxSpec $U$). Then $(C,\mathcal{O}_v)$ is a ``variety over $k$.''
\end{rem}

Let us now discuss maps to projective spaces $\Proj^r=(k^{\oplus r+1}\setminus \{0\})/k^*$.
\begin{defn}
Let $C/k$ be a curve. A \textbf{linear system} on $C$ is a finite-dimensional $k$-subvector space $V\subset K$.
\end{defn}

Let $f_0,\ldots f_r\in V$ be a basis with $r\geq 1$ (i.e. $\dim V\geq 2$). Then we define $(f_0,\ldots, f_r):C\to \Proj^r$ given by $v\mapsto [f_0(v):\cdots:f_r(v)]$. For almost all $v$, $v$ is not a pole of $f_0,\ldots,f_r$, and some $f_i(v)\neq 0$. In general, let $i_0\in\{0,\ldots, r\}$ be the index with $-v(f_i)$ maximal, then $f_0/f_i,\ldots,f_r/f_i\in\mathcal{O}_v$ and $f_i/f_i=1$. So $((f_0/f_i)(v),\ldots,1,\ldots,(f_r/f_i)(v))\in k^{r+1}\setminus \{0\}$ gives a point  of $\Proj^r$.

For example, take $K=\C(t)/\C$ with $f_0=1/t(t+1), f_1=1/t^2(t+1)^2,f_2=t^2/(t+1)^{10}.$ $\phi=(f_0:f_1:f_2):\Proj^1\to\Proj^2$. If $t\neq 0,-1,\infty$ then just take the value. If $t=0$, $(f_0/f_1,1,f_2/f_1)=(t(t+1),1,t^4/(t+1)^8)|_{t=0}=(0,1,0)$. For $t=-1$ we get $(0,0,1)$ and for $t=\infty$
 we get $(1,0,0)$. We see that $\phi^{-1}(\text{first }\mathbb{A}^1)=\Proj^1\setminus\{0,-1\}=\text{MSpec } k[1/t,1/(t+1)]$.
 
 \begin{rem}
 The map to projective space defined by the linear system is the same up to choice of coordinates as the map defined by $f.V$ for $f\in K^*$.
 \end{rem}


\section*{Class 18}

Consider the homogeneous coordinates $x_0,\ldots, x_r$ on $\Proj^r$. These do not give functions on $\Proj^r$ but they do give functions up to $k^*$. More precisely, if $F\in k[x_0,\ldots,x_r]$ is \textbf{homogeneous} then
\[V(F)=\{[x_0:\ldots:x_n]\in\Proj^r \mid F(x_0,\ldots,x_n)=0\}.\]
is well defined. Moreover,
\[\Proj^r=(\Proj^r\setminus V(x_0))\cup \cdots\cup (\Proj^r\setminus V(x_r))\]
and $\Proj^r\setminus V(x_i)$ is an affine space with coordinates $x_j/x_i$ for $j=0,\ldots, i-1,i+1,r$. 

Take $\Proj^1$, for example. We get $\Proj^1\setminus V(X_0)$ with coordinate $x_1/x_0$ and $\Proj^1\setminus V(X_1)$ with coordinates $x_0/x_1$.

\begin{prop}
\label{L103}
A subset $Z\subset\Proj^r$ is called \textbf{Zariski closed} if the following equivalent conditions are satisfied.
\begin{enumerate}
\item For each $i=0,\ldots,r$ the intersection $Z\cap(\Proj^r\setminus V(x_i))$ is the zero set of a collection of polynomials $f_\alpha\in k[x_j/x_i]$;
\item There exists a collection of homogeneous polynomials $F_\beta\in k[x_0,\ldots,x_r]$ such that $Z=\cap_\beta V(F_\beta)$.
\end{enumerate}
\end{prop}
\begin{proof}
De(homogenize).
\end{proof}

\begin{exc}
The Zariski topology on $\Proj^r$ is Noetherian (d.c.c. for closed subsets) of dimension $r$.
\end{exc}

\begin{lem}
\label{L104}
Let $C$ be a curve with function field $K/k$ and let $V=kf_0+\ldots+kf_r$ be a linear system of dimension $r+1$. Then
\[\text{Im}\left([f_0:f_1:\cdots :f_r]\right):C\to\Proj^r\]
is Zariski closed.
\end{lem}
\begin{proof}
Pick $a\in\{0,\ldots, r\}$. Then consider
$k[x_0/x_i,\ldots, x_r/x_i]\to K$ given by $x_j/x_i\mapsto f_j/f_i$. The image is a finite-type $k$-algebra $B\subset K$. Let $B\subset A\subset K$ be the integral closure. Then $A$ is of finite-type over $k$ [Ha, 3.9A]. Then $A$ is a normal Noetherian domain of dimension one with fraction field $K$. Then $\fr m\in\text{MaxSpec } A$ yields a DVR $A_\fr m$ in $C$ with a valuatoin on $K/k$. By Lemma \ref{L15} we get a surjective map from $\text{MaxSpec } A\twoheadrightarrow \text{MaxSpec } B\subset\text{MaxSpec }k[x_0/x_i,\ldots,x_r/x_i]=\Proj^r-V(x_i)\subset\Proj^r$ and the diagram commutes (proof omitted).
\end{proof}

\begin{rem}
Of course, in the situation above, the dimension of the image is 1.
\end{rem}
\begin{rem}
If $f_0,\ldots, f_s\in V$ span but are linearly dependent, then we still get a map $C\to\Proj^s$ with closed image, but now $\text{Im }\Phi\subset V(\sum a_ix_i)$ for some $a_0,\ldots,a_r\in k$ not all zero. We say that the image is \textbf{linearly degenerate} in this case. 
\end{rem}

\begin{lem}
\label{L105}
Let $D\subset\Proj^r$ be an irreducible Zariski closed subset of dimension one. There $D$ is the image of a curve $C$ by a map as above.
\end{lem}
\begin{proof}
Pick $i\in\{0,\ldots,r\}$ such that $D(\Proj^r\setminus V(x_i))\neq\varnothing$. Then $D_i=D\cap(\Proj^r\setminus V(x_i))=\text{MaxSpec } k[x_j/x_i]/J$ where $J$ is the set of all polynomials in $x_j/x_i$ vanishing on $D_i$. By topology, $D$ irreducible implies that $D_i$ is irreducible, and hence $J$ is prime. This implies that $A=k[x_j/x_i]/J$ is a domain. Then, $\dim D=1$ implies that $\dim D_i=1$ and thus $\dim A=1$. By Lemma \ref{L54} we see that $K=\text{Frac } A$ is a finitely-generated field extension of $k$ with $\trdeg_k K=1$.

Let $V$ be $k$ times the class of $x_0/x_1$ plus ... plus $k$ times the class of $x_r/x_i$. Denote $x_j/x_i$ by $f_j$. Then consider $\Phi=[f_0:\ldots:f_r]:C\to\Proj^r$. If $h\in J\subset k[x_0/x_1,\ldots,x_r/x_i]$ then $h(f_0,\ldots,\hat{f_i},f_r)=0$ in $K$. Then, thinking a little bit, we see that $\Phi(C)\cap(\Proj^r\setminus V(x_i)$, which is contained in $D_i$. Some topology shows that $\Phi(C)=D$ (both closed irreducible and agree on an open).
\end{proof}

Let $V\subset K$ be a linear system. Let $f_0,\ldots, f_r$ be a basis. Let $D=\max((f_i)_\infty)-\min((f_i)_0)$. Then $V\subset L(D)$ and $D$ is minimal with this property.

\begin{defn}\hspace{1mm}
\begin{enumerate}[(a)]
\item We say that $V$ is a \textbf{complete linear system} if $V=L(D)$ with $D$ as above;
\item The \textbf{degree} of a linear system $V$ is $\deg D$;
\item The $r$ of the linear system $V$ is $\dim V-1$.
\end{enumerate}
In this situation, we say that $V$ is a $g_d^r$.
\end{defn}

\begin{lem}
\label{L106}
If $V$ is a $g^r_d$ then $\Phi^{-1}_V(\text{hyperplane }H\subset \Proj^r)$ is a \textbf{divisor} consisting of exactly $d$ points.
\end{lem}
\begin{proof}
Pick a minimal divisor $D$ such that $V\subset L(D)$ Say
\[H=V(a_0x_0+\ldots+a_rx_r).\]
Then $\Phi^{-1}(H)$ is the set of $v\in C$ such that $v(\sum a_i f_i)$ is greater than minus the coefficient of $v$ in $D$ as a divisor $(\sum a_i f_i)+D$. This is an effective divisor, as $\sum a_if_i\in V\subset L(D)$ and the degree is equal to $\deg D=d$ because the degree of a principal divisor is zero. Pick $i_0$ such that $-v(f_{i_0})$ is maximal. Then $\Phi(v)$ has coordinates of $f_i/f_{i_0}$ evaluated at $v$. So $v$ maps into $H$ if and only if $\sum a_if_i/f_{i_0}=0$ if and only if $\sum a_i f_i$ has lesser pole order than $f_{i_0}$ at $v$.
\end{proof}

\section*{Class 19}

Here's another way to think about $\Phi_V$. First, consider $\Proj^r$ as being the space of $r$-dimensional linear subspaces of $(k^{r+1})^*$. Given a vector space $V$, we will set in Grothendieck notation, $\Proj(V)$ to be the projective space of codimension one linear subspaces of $V$. Given a linear system $V\subset K$ choose $D$ a divisor minimal with $V\subset L(D)$. Then
\begin{align*}
\Phi_V:&C\longrightarrow \Proj(V)\\
&p \mapsto V(-P)=L(D-P)\cap V
\end{align*}
\begin{cor}
\label{L107}
The map associated to $V$ is \textbf{injective} if and only if for all $P\neq Q$ in $C$ we have $\dim V(-P-Q)=\dim V-2$. We say ``$V$ separates points.''
\end{cor}
\begin{proof}
$\Phi_V(P)\neq\Phi_v(Q)\leftrightarrow V(-P)\neq V(-Q)$ and we have
\[V(-P-Q)=V(-P)\cap V(-Q).\]
\end{proof}

There's a problem here: $\Phi_V$ can be injective without being an ``isomorphism'' onto its image. An example of this is the cuspidal curve. Take $K=k(t)$ and $V=k+kt^2+kt^3$. Then $\Phi:C=\Proj^1\to\Proj^2$ maps $\Proj^1$ to something with a singularity. More precisely, $\mathbb{A}^1=\Proj^1-\infty=\Phi^{-1}\left(\Proj^2-V(X_0)\right)\mapsto \Proj^2-V(X_0)=\mathbb{A}^2 $.
The image is the curve in $\mathbb{A}^2$ cut out by $y^2-x^3=0$. In characteristic  zero, getting a singularity in the image is the only problem. In characteristic $p>0$, on the other hand, let $V=k+kt^p$. Then we get a map $\Phi:\Proj^1\to\Proj^1$ given by $t\mapsto t^p$, which is injective and surjective. Namely, if $a,b\in k$ and $a^p=b^p$ then $a=b$. But this map is not an isomorphism, as it has degree $p>1$. We have $k(t)\supset k(t^p)$.


\begin{prop}
\label{L108}
Let $V$ be a linear system. If $V$ separates points and tangent vectors, i.e. for all $P,Q$ in $C$ (with $P=Q$ allowed) $\dim V(-P-Q)=\dim V=-2$. Then $\Phi_V:C\to\Proj^r$ is an isomorphism onto its image. In fact, $\Phi_V$ is a closed immersion, i.e. for every $i=0,\ldots, r$, the ring map
\[\mathcal{O}_C(\Phi_V^{-1}(\Proj^r-V(X_i))\longleftarrow k[x_0/x_i,\ldots x_r/x_i].\]
\end{prop}
\begin{proof}
Omitted. Roughly speaking, if we let $A$ be the left-hand side and $B$ be a quotient of the right-hand side, we see that $A$ is the integral closure of $B$ in $K$, $A$ is finite over $B$, and finally we use Lemma II 7.4 in Hartshorne (see also 7.3).
\end{proof}


\begin{defn}
Let $D$ be a divisor. We say that $D$ is \textbf{basepoint free} if $L(D-P)\neq L(D)$ for all $P\in C$. This really only makes sense as long as $L(D)\geq 1$.
\end{defn}
\begin{rem}
This exactly means that $L(D)$ is a $g^r_d$ with $d=\deg D$. Furthermore, this exactly means that $L(D)\subset L(D')\leftrightarrow D'\geq D$.
\end{rem}

\begin{lem}
\label{L109}
Let $C$ be a curve of genus $g$ and $D$ a divisor of degree $d$. Then
\begin{enumerate}
\item if $d>2g-1$ then $D$ is bpf;
\item if $d>2g$ then $L(D)$ separates points and tangent vectors.
\end{enumerate}
\end{lem}
\begin{proof}
If $E$ is a divisor with $\deg E>2g-2$ then $\ell(E)=\deg E+1-g$. For (2) we simply see that $L(D-P-Q)=d-2+1-g=(d+1-g)-2=\ell(D)-2$.
\end{proof}

\begin{exmp}
Let $g=0$. If $d=1$, we get an isomorphism from $\Proj^1\to\Proj^1$. If $d=2$ we get an isomorphism from $\Proj^1$ to some curve in $\Proj^2$ which meets everywhere line in $\Proj^2$ at two points (counted with multiplicity). Next, take $d=3$. We get a map $\Proj^1\to\Proj^3$ given by $[1:t]\mapsto [1:t:t^2:t^3]$ whose image is a curve that meets every plane at 3 points. This is called a rational normal curve of degree 3.

Now let $g=1$ and $d=3$. Our image is isomorphic to a curve $D\subset \Proj^2$ which meets every line in 3 points. We claim that $D=V(F)$ where $F$ is a cubic (homogeneous). To see this note that we know $D$ is irreducible of dimension one. In particular, $D\cap (\Proj^2-V(X_0))$ corresponds to a prime ideal in $k[x_1/x_0,x_2/x_0]$ which is not maximal and not $(0)$. But since we have a UFD, $\fr p=(f)$, $f$ an irreducible polynomial. Then $D=V(F)$ where $F$  is $X_0$ to the total degree of $(f)$ times $f(x_1/x_0.x_2/x_0)$. Now why is the degree of $F$ 3? Well given a squarefree homogeneous $F\in k[x_0,x_1,x_2]$ we can find a line $L\subset \Proj^2$ such that the number of points of $L\cap F$ is the degree of $F$. The idea for proving this is to dehomogenize. Then let $\Delta(x)$ be the discriminant of the dehomogenized polynomial. Clearly $\Delta(x)\neq 0$ everywhere because $F$ is squarefree. Take $\lambda\in k$ such that $\Delta(\lambda)\neq 0$ and now the line $x=\lambda$ i.e. $x_1/x_0=\lambda$ i.e. $X_1-\lambda X_0=0$ meets $V(F)$ in exactly $\deg F$ points.
\end{exmp}



\section*{Lecture 20}
In the previous lectures we proved the statements:
\begin{enumerate} [(1)]
\item If $C$ has $g^2_d$ then we get $\Phi : C \to \Proj^2$ such that every line meets $\Phi(C)$ in $d$ points (counted with multiplicity).
\item If $F$ homogenous of degree $d$ and squarefree then a general line meets $D = V(F) \subset \Proj^2$ in $d$ distinct points.
\end{enumerate}

\begin{lem}
\label{L110}
In situation (1) above $\Phi(C) = V(F)$, where $F$ is an irreducible homogenous polynomial of degree $e$, and $e|d$. In fact, $d/e$ is the degree of the function field of $C$ over the function field of $D$.
\end{lem}
\begin{proof}
Pick any $F\neq 0$ homogenous such that $\Phi(C) \subset V(F)$. This is possible because the image is closed. If we can factor $F = F_1 F_2$, then $\Phi(C) \subset V(F_1) \cup V(F_2)$, but $\Phi(C)$ irreducible so we have either $\Phi(C) \subset V(F_1)$ or $\Phi(C) \subset V(F_2)$. Therefore we may assume that $F$ is irreducible. But this implies $\Phi(C) = V(F)$. [There's also an alternative way to think about this. $\Phi : C \to \Proj^2$ is given by $V = k f_0 + kf_1 + kf_2$. Then $\Phi(C) \subset V(F) \Leftrightarrow F(f_0, f_1, f_2) = 0$ in $k$.]
\\
\\
By (2) we see that $e = \deg F \leq d$. But, to show that $e|d$, we need to use Lemma \ref{L111} below. This shows us that, for every line $H$, $\deg \Phi^{-1}(H) = [k(C) : k(D)] \cdot \deg(H\subset V(F))$. In conclusion, if $D$ is a degree 3 divisor on a genus 1 curve $C$, then its complete linear system $L(D)$ defines a closed embedding (injective immersion such that the domain is birational to the image):
\[      C \hookrightarrow \Proj^2      \]
whose image is a degree 3 curve.
\end{proof}

\begin{lem}
\label{L111}
Suppose $K_1 \supset K_2$ is a finite extension of an extension of $k$ which is f.g. and has transcendence degree 1. The corresponding map $C_1 \to C_2$ has degree $[K_1 : K_2]$ in the sense that its fibers have exactly that many points if you count with muptiplicity.
\end{lem}
\begin{proof}
If $v$ valuation on $K_2$, then $w_1, \dots, w_i$ extensions of $v$ to $K_1$, set $e_i = $ the integer such that $w_i|_{K_2} = e_i v$, then $\sum e_i = [K_1 : K_2]$.
\end{proof}
We also prove here the converse of Lemma \ref{L110}, namely that any nonsingular degree 3 curve has genus 1.
\begin{proof}
Let $F \in k[x_0, x_1, x_2]$ homogenous of degree 3 such that $C = V(F)$ is nonsingular, i.e. on the affine pieces you get a nonsingular curve. This means that $\frac{\p F}{\p x_1}, \frac{\p F}{\p x_2}, \frac{\p F}{\p x_3}, F$ have no common 0 in $\Proj^2$.
\\
\\
Now we work on the affine piece $C \cap (\Proj^2 \ V(x_0))$. Let $x = \frac{x_1}{x_0}, y= \frac{x_2}{x_0}$ such that we can write $f(x,y) = F(1,x,y)$. We know that $f_x, f_y$ are not simultaneously 0 on $f=0$. So we can look at:
\[       \omega = \frac{dx}{f_y} = - \frac{dy}{f_x} \in \Omega_{k(C)/k}        \]
The equality holds because $df = f_x dx + f_y dy = 0$. Thus, for every $P$ on $f=0$, the module of differentials is generated by $dx$ if $f_y(P) \neq 0$ or by $dy$ if $f_x(P) \neq 0$. We conclude that $\omega \in \Omega_{k(C)/k}$ vanishes nowhere and doesn't have a pole at any of the points of this open.
\\
\\
On the overlap with another affine open, say $k[u,v] = k[x_0/x_1, x_2/x_1]$, we have $u = x^{-1}$, $v = yx^{-1}$. Then write $g(u,v) = F(u,1,v)$. We have:
\[       F(1,x,y) = x^3 F(x^{-1}, 1, yx^{-1}) = x^3 g(x^{-1}, yx^{-1})        \]
\[      f_y = x^3 x^{-1} g_v = u^{-2} g_v      \]
For the purpose of generalizing to arbitrary degree $d$, which we will do shortly, note that the 2 which appears here is $d-1$.
\[     \omega = \frac{dx}{f_y} = \frac{u^{-2} du }{u^{-2} g_v} = \frac{du}{g_v}  \]
By similar reasoning, this has no zeros or poles. In conclusion, our $C$ has a global differential form $\omega \in \Omega_{k(C)/k}$ without poles or zeros. Therefore $\deg(K_C) = 0 = 2g_C - 2$, so $g_C = 1$.
\end{proof}

\begin{rem}
If $\deg F = d>3$, then we get:
\[      \omega = \frac{dx}{f_y} = - u^{\deg(F) - 3} \frac{du}{g_v}    \]
Assume we chose coordinates such that $C \cap V(x_0) \cap V(x_1) = \emptyset$. Then $(u=0) \cap (C\cap (\Proj^2 - V(x_1))) = V(x_0) \cap C$. This is a divisor of degree $d$. In conclusion, if $d>3$ then there exists a regular differential form whose divisor of zeros is $(d-3) \cdot$ divisor at $\infty$. This has degree $d(d-3)$. Then:
\[      2g_C - 2 = d(d-3) \Rightarrow g_C = \frac{(d-1)(d-2)}{2}    \]
\end{rem}

\subsection*{About other embeddings of genus 1 curves}
If $D$ is a divisor of degree $\geq 3$ on a genus 1 curve $C$ then the full linear system of $D$ is BPF and embeds $C$ as a degree $d$ curve in $\Proj^{d-1}$. If $d = \deg D = 4 = 2\cdot 2$, then we guess that we should get an intersection of two quadratics. To prove this, we set $V = L(D) \subset K$. We look at:
\[        \Sym^2(V) \to L(2D)     \]
By Riemann-Roch, $L(2D)$ has dimension $8+1 - g = 8$. The dimension of $\Sym^2(V)$ is $\binom{5}{2}$. Therefore the kernel has dimension $2$, so $C$ lies on two quadratics.


\section*{Lecture 21}
\begin{lem*}
Let $U.V,W$ be finite dimensional vector spaces. Let $\phi: V\otimes W \to U$ be a linear map such that $\phi(v\otimes w) \neq 0$ if $v,w\neq 0$. Then $\dim U \geq \dim V + \dim W - 1$.
\end{lem*}
\begin{proof}
We look at $X = \{ v \otimes w \in V\otimes W \} \hookrightarrow V \otimes W$, which is Zariski closed. We have that $\dim V + \dim W - 1$, where the $-1$ comes from the invariance to rescaling $v$ and $w$ simultaneously. If $U = k^{\oplus n}$, then $\phi = (\phi_1, \dots, \phi_n)$. Then:
\begin{align*}        
0 = \dim(X \cap \Ker(\phi)) &= \dim( X\cap \Ker \phi_1 \cap \dots \cap \Ker \phi_n)  \\
&= \dim(X) - 1 - 1 \dots -1 \\
&= \dim(X) - n
\end{align*}
\end{proof}

\begin{thm*}  [Clifford's theorem]
Let $D$ be a divisor with $l(D), l(K_C - D) \geq 1$. Then $l(D) \leq \frac{1}{2} \deg(D) + 1$.
\end{thm*}
\begin{proof}
Apply the lemma to the multiplication map:
\[     L(D) \otimes_k L(K_C - D) \to L(K_C)    \]
And get $l(D) + l(K_C - D) \leq g+1$. On the other hand, by RR, we have $l(D) - l(K_C - D) = \deg D + 1 - g$. Adding these two equations gives the desired result.
\end{proof}

\begin{thm*}
If $g(C) \geq 2$, thn $C$ has at most one $g_2^1$. (Up to multiplication by an element of $K^{\times}$.)
\end{thm*}
\begin{proof}
If $V$ is a $g_1^2$, then it's a complete linear system (by Clifford's theorem). Say we have two divisors $D,D'$ of degree 2, $l(D) = l(D') = 2$. We may assume $D,D'$ are effective. Then $l(D+D') \leq \frac{1}{2} \cdot 4 + 1 = 3$. (Note: to use Clifford's theorem here, we need to check that $l(K_C - D - D') >0$, but this is an easy consequence of Riemann-Roch.)
\\
\\
If $L(D)$ has basis $f_1, f_2$ and $L(D')$ has basis $g_1, g_2$, then:
\[         g_2^1 : C \overset{f_1/f_2}{\to} \Proj^1        \]
\[         g_2^1 : C \overset{g_1/g_2}{\to} \Proj^1        \]
We want to show that these maps are the same. There exist $a_{ij} \in k$ not all zero such that:
\[       \sum a_{ij} f_i g_j = 0  \text{     in } K     \]
The rank of $A = (a_{ij})$ cannot be 1, because then $L(D) \otimes_{k} L(D')$ would send a pure tensor to 0. Then we can always change basis $f_i, g_i$ such that:
\[       f_1 g_2 - f_2 g_1 = 0 \Leftrightarrow \frac{f_1}{f_2} = \frac{g_1}{g_2}   \]
\end{proof}

\begin{thm*} [Clifford, part 2]
If we have equality in Clifford's theorem,and $D \not \sim_{\text{rat}} 0, D' \not \sim_{\text{rat}} 0$, then $C$ is hyperelliptic and $D = mD'$, with $D'$ the unique $g_2^1$ on $C$.
\end{thm*}
\begin{proof}
See Harthshorne.
\end{proof}

\begin{defn}
Assume (unnecessarily, but in order to make Prof. de Jong less stressed out) that $k = \C$. The \textbf{gonality} of the curve $C$ is any of the following equivalent notions:
\begin{enumerate} [(1)]
\item The minimal degree of a morphism $C \to \Proj^1$.
\item The minimal degree of $[K:\C(f)]$ for any $f\in K - \C$.
\item The minimal integer $k$ such that $C$ has a $g^1_k$.
\end{enumerate}
\end{defn}
\begin{lem*}
The gonality of $C$ is at least the integer $k$ such that there exist $P_1, \dots, P_k$ in $C$ which do not impose independent conditions on $L(K_C)$,i.e.:
\[      l (K_c - P_1 - \dots - P_k) > g-k         \]
\end{lem*}
\begin{proof}
Suppose $f: C \to \Proj^1$ has degree $k$, let:
\[     f^{-1} (\{t\}) = \{ P_1, \dots, P_k \}    \]
for $t\in \Proj^1(\C)$. Then a few moments of thought give:
\[       l(P_1 + \dots + P_k) \geq 2     \]
\[       \Rightarrow l(K_C - P_1 - \dots - P_k) \geq 2 - k - 1 + g = 1 + g -k       \]
Conversely, if $l(K_C - P_1 - \dots - P_k) >  g -k$, then $l(P_1 + \dots + P_k) \geq 2$ and we get a $g_k^1$.
\end{proof}

\begin{thm*} [1979]
A smooth plane curve has gonality $d-1$, where $d\geq 4$ is the degree.
\end{thm*}
\begin{proof}
By projection from a point of $C$ we see that the gonality is $\leq d-1$ (because a line intersects $C$ in $d$ points, and one of them is the point we project from). For the converse we use the fact from last time:
\[       K_C \sim (d-3) H \cap C     \]
$H \cap \Proj^2$ is the hyperplane at $(x)$, so $H = V(x_0)$ and $g_c = \frac{(d-1)(d-2)}{2}$. Hence for every $h \in \C[x,y]$ of total degree $\leq d-3$ we obtain an element $L(K_C)$:
\[     h \to h\omega = \frac{h dx}{f_y} - \frac{h dy}{f_x}   \]
Where $f(x,y) = 0$ defines $C \cap (\Proj^2 - V(x_0))$. This map is injective because $h \not \in (f)$ by degree reasons.
\[      \dim \{h\} = 1 + 2 + \dots + d-2 = \frac{(d-1)(d-2)}{2}     \]
Where, for example, the $d-2$ is the dimension of $\C x^{d-3} + \dots + \C y^{d-3}$. Hence $\{h\} \cong L(K_C)$. Now if $C \overset{\phi}{\Proj^1}$ is a $g_k^1$ then picking $t \in \Proj^1 (\C)$ general, we see that $\phi^{-1} (\{t\}) = \{ P_1, \dots, P_k \} $ are all going to be in $C - C\cap H \subset \C^2$. But $k\leq d-2$ points in $\C^2$ pose independent conditions on polynomials of degree $d-3$. So the gonality of $C$ is $\geq d-1$.
\end{proof}

\begin{cor*}
A smooth plane curve of degree $d\geq 4$ is not hyperelliptic.
\end{cor*}

\newpage
\section*{Class 22}

Let's approach schemes backwards - let's define schemes and then work backwards to see what it means. Hence, there will be undefined terms at any given point.

\begin{defn}\hspace{1mm}
\begin{enumerate}[(i)]
\item A \textbf{scheme} is a \textbf{locally ringed space} such that every point has an open neighborhood isomorphic to an \textbf{affine scheme}.
\item A \textbf{morphism of schemes} is a \textbf{morphism of locally ringed spaces}.
\item A locally ringed space is a \textbf{ringed space} $(X,O_X)$ such that the \textbf{stalks} of the \textbf{structure sheaf} $O_X$ are local rings for $x\in X$
\item A morphism of locally ringed spaces is a \textbf{morphism of ringed spaces} $f:(X,O_X)\to(Y,O_Y)$ such that for all $x\in X$ the induced map $f_x^\#:O_{Y,f(x)}\to O_{X,x}$ is a local homorphism of local rings: $f_x^\#(\fr m_{f(x)})\subset\fr m_x$.
\item A \textbf{ringed space} $(X,O_X)$ is a pair consisting of a topological space $X$ and a \textbf{sheaf of rings} $O_X$ on $X$ called the \textbf{structure sheaf} of $X$
\item A morphism of ringed spaces $f:(X,O_X)\to(Y,O_Y)$ is given by a continuous map $f:X\to Y$ and for all $V\subset Y$ open, a map $O_Y(V)\overset{f^\#}{\to}O_x(f^{-1}V)$ of rings compatible with the \textbf{restriction mapping} of $O_X$ and $O_Y$.
\item Let $X$ be a topological space. A \textbf{presheaf} $\mathcal{F}$ of some target category $\mathcal{C}$ on $X$ is a rule which assigns to every open $U\subset X$ an object of the target category $\mathcal{F}(U)$ and for every $V\subset U$ a restriction morphism $\rho^U_V:\mathcal{F}(U)\to\mathcal{F}(V)$ of the target cateory such that $\rho^U_U=\text{id}$ and $\rho^U_V\circ\rho^W_U=\rho^W_V$ if $V\subset U\subset W$ open in $X$.  Often used is the notation that $s|_V=\rho^U_V(s)$.
\item A presheaf $\mathcal{F}$ is called a sheaf if given an open covering $U_i$ of any open $U\subset X$, we have that $\mathcal{F}(U)$ maps bijectively to \[\{(s_i)\in\prod_{i\in J}\mathcal{F}(U_i)\mid \rho^{U_i}_{U_i\cap U_j}(s_i)=\rho^{U_j}_{U_i\cap U_j}(s_j)\text{ for all }i,j\in I\]
with $s\mapsto \rho_{U_i}^U(s)$.
This is called the \textbf{sheaf condition}. This is essentially saying that local data plus gluing implies global data.
\item Given a presheaf $\mathcal{F}$ of $\mathcal{C}$ and a point $x\in X$, \textbf{the stalk of $\mathcal{F}$ at $x$} is
\[\mathcal{F}_x=\{(U,s)\}/\sim\]
where $(U,s)$ are pairs consisting of an open neighborhood $U$ of $x$ and $s\in\mathcal{F}(U)$ with the equivalence
\[(U,s)\sim(U',s')\Leftrightarrow \exists x\in U''\subset U\cap U' \text{ open s.t. } \rho_{U''}^U(s)=\rho_{U''}^{U'}(s'). \]
Note that $\mathcal{F}_x$ is an object in $\mathcal{C}$.
\item A \textbf{morphism of presheaves} $\mathcal{F}\overset{\phi}{\to}\mathcal{G}$ is given by a collection of maps $\phi(U):\mathcal{F}(U)\to\mathcal{G}(U)$ for $U\subset X$ open compatible with restriction mappings: whenever $V\subset U$ open, the diagram
\begin{equation*}
\begin{tikzcd}
\mathcal{F}(U)\arrow{r}{\phi(U)}\arrow[swap]{d}{\rho_V^U} &\mathcal{G}(U)\arrow{d}{\rho_V^U}\\
\mathcal{F}(V)\arrow{r}{\phi(V)} & \mathcal{G}(V)
\end{tikzcd}
\end{equation*}
commutes. A \textbf{morphism of sheaves} is simple a morphism of presheaves.
\end{enumerate}
\end{defn}

Our next goal is to construct affine schemes from rings. The idea is that $\Spec A$ has as a basis of opens the standard opens
\[D(f)=\{\fr p\subset A\mid f\notin\fr p\}\]
for $f\in A$. We will want (because it works)
\[O_{\Spec A}(D(f))=A_f.\]

Let $\mathcal{B}$ be a basis for the topology of a topological space $X$ and assume (for simplicity) that if $U,V\in\mathcal{B}$ then $U\cap V\in\mathcal{B}$. Define \textbf{presheaves on $\mathcal{B}$} exactly as before. Furthermore, define \textbf{sheaves on $\mathcal{B}$} exactly as before, i.e. the sheaf condition for coverings $U=\cup_{i\in I}U_i$ when $U_i\in\mathcal{B}$.

\begin{prop}
Let $X$ and $\mathcal{B}$ as above. Then there is an equivalence of categories $\fr{Sh}(X)\Leftrightarrow \fr{Sh}(b)$ with $\mathcal{F}\mapsto\mathcal{F}|_\mathcal{B}$. This is in general not correct for presheaves.
\end{prop}
Read this in the Stacks project or in FOAG (Ravi's notes). ``Ravi claims that he covers ALL OF THAT in one year. And that his students do all the exercises. It's not possible. I don't believe it... well maybe it's possible.'' - de Jong.

\begin{proof}
If $\mathcal{G}$ is a sheaf on $\mathcal{B}$ and $U\subset X$ is an arbitrary open then we can choose a covering $U=\cup U_i$ with $U_i\in\mathcal{B}$. Then the sheaf $\mathcal{F}$ on $X$ corresponding to $\mathcal{G}$ should satisfy
\[\mathcal{F}(U)=\{(s_i)\in\prod \mathcal{G}(U_i)\mid \rho^{U_i}_{U_i\cap U_j}(s_i)=\rho^{U_j}_{U_i\cap U_j}(s_j).\]
Then we just define $\mathcal{F}$ by this formula! ``...and now you have to write up sort of infinitely long checks.''
\end{proof}

\begin{exmp}
$X=\Spec A$ with the Zariski topology.
\[\mathcal{B}=\{D(f) \mid f\in A\}.\]
This satisfies our assumption $D(f)\cap D(g)=D(fg)$. We want to consider $\mathcal{B}\ni D(f)\mapsto A_f$. But this is nonsense! It might happen that $D(f)=D(g)$ without $f=g$.
\end{exmp}

\begin{lem}
\label{SL1}
If $f,g\in A$ and $D(f)\supset D(G)$ then there is a unique $A$-algebra map $A_f\to A_g$. If $D(f)=D(g)$ then this map is an isomorphism.
\end{lem}
\begin{proof}
Recall that $\Spec A_f\to \Spec A$ induces a homeomorphism onto $D(g)$. Thus the assumption $D(f)\supset D(g)$ implies that $f$ maps to an element of $A_g$ not in any prime ideal. So $f$ is invertible in $A_g$. Hence we get our unique canonical map by the universal property of localization.
\end{proof}

\section*{Class 23}

\begin{lem}
\label{SL2}
Let $\mathcal{B}$ be the standard basis of opens of $\Spec A$. Given an $A$-module $M$ the rule $\mathcal{B}\to $A$-\text{modules}$ given by $U=D(f)\mapsto M_f=M\otimes_A A_f$ is a sheaf of $A$-modules on $\mathcal{B}$.
\end{lem}
\begin{proof}
Lemma \ref{SL1} shows that this is well-defined and gives the restriction mappings. It is left as homework to prove the sheaf condition.
\end{proof}

\begin{defn}
The \textbf{structure sheaf} of $\Spec A$ is the sheaf of rings $O_{\Spec A}$ which corresponds via the proposition to the rule $D(f)\mapsto A_f$ on the basis $\mathcal{B}$ of standard opens.
\end{defn}

\begin{rem}
Similarly, we have a sheaf $\tilde M$ corresponding to $D(f)\mapsto M_f$. Observe that $\tilde M$ is a sheaf of $O_{\Spec A}$-modules.
\end{rem}

Let us look at the stalk of the structure sheaf at $\fr p$. Note that since $\mathcal{B}$ is a basis for the topology, to compute the stalk we need only consider pairs $(D(f),s)$ where $\fr p\in D(f)$ and $s\in A_f$, i.e. $f\in A\setminus\fr p$ and $s=a/f^n$. Then two such pairs $(D(f),a/f^n)$ and $(D(g),b/g^m)$ give the same element of the stalk if and only if there exists $h\in A\setminus \fr p$ such that $D(h)\subset D(f), D(h)\subset D(g)$ and $a/f^n$ and $b/g^m$ map to the same element of $A_h$. Contemplate the diagram:
\begin{equation*}
\begin{tikzcd}
    {}&A\arrow{ld}\arrow{rd}&\\
A_f\arrow{rd}\arrow{rdd} && A_g\arrow{ld}\arrow{ldd}\\
&A_h\arrow{d}&\\
&A_{\fr p}&
\end{tikzcd}
\end{equation*}
and conclude that we get a well-defined map $O_{\Spec A,\fr p}\to A_\fr p$ that is both injective and surjective. In fact, it is an algebraic fact that the colimit
\[\varinjlim_{f\in A\setminus\fr p}=A_\fr p\]
and that
\[O_{\Spec A,\fr p}=\varinjlim_{U\ni\fr p}O(U)=\varinjlim_{U\in\mathcal{B},\fr p\in U}\mathcal{F}(U).\]
In particular $O_{\Spec A,\fr p}=A_\fr p$ is a local ring

\begin{lem}
\label{SL3}
The stalk of $O_{\Spec A}$ at $\fr p$ is $A_\fr p$. The stalk of $\tilde M$ at $\fr p$ is $M_\fr p$.
\end{lem}

\begin{defn}
An \textbf{affine scheme} is a locally ringed space isomorphic\footnote{As a (locally) ringed space. It doesn't matter which, because if $\phi:A\to B$ is an isomorphism of rings, and $A$ and $B$ are local, then $\phi$ is local.} to $(\Spec A,O_{\Spec A})$ for some ring $A$.

\end{defn}

Now recall that a scheme $X$ is a locally ringed space that at every point $x\in X$ has an open neighborhood isomorphic to an affine scheme.

\begin{rem}
If $(X,O_X)$ is a (locally) ringed space and $U\subset X$ is open, then $(U,O_X|_U)$ is a (locally) ringed space, let us call it an \textbf{open subspace}. Moreover, there is an inclusion morphism
$j:(U,O_X|_U)\to (X,O_X)$ of (locally) ringed spaces.
\end{rem}

\begin{rem}
Open subspaces of schemes are again schemes. To see this, it is enough to show that
\[(D(f),\OO_{\Spec A}|_{D(f)})\cong (\Spec A_f,\OO_{\Spec A_f}),\]
which we shall do below.
\end{rem}

\subsection*{Ring maps and morphisms}

Let $\phi: A\to B$ be a ring map and $\Spec \phi:\Spec B\to \Spec A$ be the associated continuous map of topological spaces. Moreover, if $f\in A$, then
\[\Spec\phi^{-1}(D(f))=D(\phi(f)).\]

\begin{prop*}
Let $f:X\to Y$ be a continuous map of topological spaces. Let $\cal B$, resp. $\cal C$, be a basis for the topology on $X$, resp. $Y$, bot closed under intersections. Assume $f^{-1}V\in\cal B$ for all $V\in\cal C$. Then, given sheaves $\cal F$, resp. $\cal G$, on $X$, resp. $Y$, to give a collection of maps $\phi(V):\mathcal{G}(V)\to\mathcal{F}(f^{-1}V)$ for all $V$ open in $Y$ compatible with restriction mappings is the same thing as giving a collection of maps $\phi(V):\mathcal{G}(V)\to\mathcal(F)(f^{-1}V)$ for all $V\in \cal C$ compatible with restriction maps.
\end{prop*}
\begin{rem}
Such a collection of maps $\phi=\{\phi(V)\}$ is called an $f$-map from $\cal G$ to $\cal F$.
\end{rem}
\begin{proof}
Given $\phi(V)$ defined for $V\subset \cal C$ and $W\subset Y$ open, we choose an open covering $W=\cup V_i$ for $V_i\in\cal C$ and then define $\phi(W)$ by taking
\[\mathcal{G}(W)=\{(s_i)\in\prod \mathcal{G}(V_i)\mid \rho^{V_i}_{V_i\cap V_j}(s_i)=\rho^{V_j}_{V_i\cap V_j}(s_j)\}\]
to
\[\mathcal{F}(f^{-1}(W))=\{(t_i)\in\prod \mathcal{F}(f^{-1}V_i)\mid \cdots\}\]
One now has to do an infinite number of things to show that this works, and this concludes the proof.
\end{proof}

Going back to our ring map $A\to B$ we let $\Spec\phi:(\Spec B,\OO_{\Spec B})\to(\Spec A,\OO_{\Spec A})$ defined by the rules:
\begin{enumerate}
\item $(\Spec \phi)(\fr q)=\phi^{-1}(\fr q)$ for $\fr q\in\Spec B$.
\item $\OO_{\Spec A}(D(f))=A_f\to\OO_{\Spec B}(D(\phi(f)))=B_{\phi(f)}$ defined by $a/f^n\mapsto \phi(a)/\phi(f)^n$.
\end{enumerate}
This gives us a morphism of ringed spaces. To check that it is indeed a morphism of schemes, we must check that the induced maps
\[\OO_{\Spec A,\phi^{-1}(\fr q)}A_{\phi^{-1}(\fr q)}\to B_\fr q=\OO_{\Spec B,\fr q}\]
is a local homorphism of local rings. This does hold, as it is the map induced by $\phi$.

\begin{rem}
Suppose we have $f:X\to Y$ and $\phi:\mathcal{G}\to\mathcal{F}$ and $f$-map. We need to get an induced map on the stalks $\phi_x:\mathcal{G}_{f(x)}\to \mathcal{F}_x$ for some $x\in X$. We can do this by sending $(V,t)\mapsto (f^{-1}V,\phi(V)(t))$.
\end{rem}

\begin{lem}
\label{SL4}
Let $A$ be a ring and $f\in A$. The ring map $A\to A_f$ induces an isomorphism 
\begin{equation*}
\begin{tikzcd}
(\Spec A_f,\OO_{\Spec A_f})\arrow{rr}\arrow{rd}{A\to A_f}&& (D(f),\OO_{\Spec A_f}|_{D(f)})\arrow{ld}{j}\\
&(\Spec A,\OO_{\Spec A})&
\end{tikzcd}
\end{equation*}
\end{lem}
\begin{proof}
Omitted.
\end{proof}

\begin{lem}
\label{SL5}
Let $f:(X,O_X)\to (Y,O_Y)$ be a morphism of (locally) ringed spaces. Then $f$ is an isomorphism iff $f$ is a homeomorphism and $f$ induces isomorphisms on stalks: for $x\in X$, $\OO_{Y,f(x)}\to\OO_{X,x}$.
\end{lem}
\begin{proof}
Obvious by Lemma \ref{SL6}.
\end{proof}

\begin{lem}
\label{SL6}
Let $\mathcal{F}\overset{\alpha}{\to}\mathcal{G}$ be a map of sheaves on a topological space $X$. Then $\alpha$ is an isomorphism if and only if $\alpha_x:\mathcal{F}_x\to\mathcal{G}_x$ is an isomorphism for all $x\in X$.
\end{lem}
\begin{proof}
We have to construct a $\beta:\mathcal{G}\to\mathcal{F}$ inverse to $\alpha$. To do this, it is enough if $\alpha(U):\mathcal{F}(U)\to\mathcal{G}(U)$ is bijective for all $U\subset X$ open. Let us first show injectivity. Suppose $\alpha(s)=\alpha(s')$ for some $s,s'\in\mathcal{F}(U)$. Then $(U,\alpha(s))$ and $(U,\alpha(s'))$ define the same element of the stalk $\mathcal{G}_x$ for all $x\in U$. By assumption, this shows that $(U,s)$ and $(U,s')$ define the same element of $\mathcal{F}_x$ for all $x\in U$. By by definition, for all $x\in U$, there exists $x\in U_X\subset U$ such that $s|_{U_x}=s'|_{U_x}$. But then $U=\cup_{x\in U}U_x$ is an open covering and the sheaf condition for $\mathcal{F}$ shows that $s=s'$. Surjectivity is similarly shown.
\end{proof}

\section*{Class 24}

\subsection*{Examples of Schemes}

\begin{exmp}[Scheme associated to an abstract curve]
Let $k=\bar k$ be an algebraically closed field, $K/k$ a finitely generated extension of transcendence degree one. As a set,
\[X=\{\text{discrete valuations on }K/k\}\cup\{\eta\},\]
where $\eta$ is a generic point. Note that the closed subsets are $\varnothing, X$, and finite collections of discrete valuations of $K/k$. Additionally, if $U=X\setminus\{v_1,\ldots,v_n\}$ is open then we take
\[O_X(U)=\{f\in K\mid v(f)\geq 0 \forall v\in U\}.\]
This is indeed a scheme, with every open $U\subset X, U\neq X$ affine, the rings $O_X(U)$ always of finite type over $k$, the local ring of $O_X$ at $v$ is $O_{X,v}=O_v=\{f\in K\mid v(f)\geq 0\}$, and the local ring of $O_X$ at $\eta$ is $K$.
\end{exmp}

\begin{exmp}[Affine scheme over a ring $R$]
\[\mathbb{A}^n_R=\Spec R[x_1,\ldots, x_n]\longrightarrow\Spec R\]
\end{exmp}

\begin{exmp}[Spectrum of the integers]
Consider $\Spec \Z=\{(0),(2),(3),\ldots\}$. The closed subsets are $\varnothing,\Spec\Z$, and finite subsets of $D(0)$. Any open is of the form $D(f)$ where $f\in\Z$ and $O_{\Spec \Z}(D(f))=\Z[1/f]$. Note that $\Spec\Z$ is a final object in the category of schemes.
\end{exmp}

\begin{exmp}[Proj of a graded ring]
Let $A=\oplus_{d\geq 0}A_d$ a graded ring and $A_+=\oplus_{d>0}A_d\subset A$ be the irrelevant ideal. We define $\Proj A$ to be the graded prime ideals $\fr p\subset A$ such that $\fr p$ does not contain $A_+$. As a topology, for $f\in A_+$, we take a homogeneous set
\[D_+(f)=\{\fr p\in\Pro A\mid f\notin\fr p\}.\]
These subsets are a basis for a topology on $\Pro A$. We take
\[O_{\Pro A}(D_+(f))=A_{(f)},\]
which is simply the degree 0 part of $A_f$ (which is indeed a ring). Observe that $A_f$ is a $\Z$-graded ring. This yields a sheaf of rings on this basis and hence a sheaf of rings on $\Pro A$. It turns out that $D_+(f)$ is affine, hence $D_+(f)\cong \Spec A_{(f)}$. Finally, if $\fr p\in \Pro A$, then
\[O_{\Pro A,\fr p}=A_{(\fr p)}.\]
\end{exmp}

\begin{rem}
Warning: the $\Pro$ construction is not functorial (in the variable $A$). E.g.
\[A=R[x_0,x_1]\to R[y_0,y_1,y_2]=B\]
given by $x_0\mapsto y_0,x_1\mapsto y_1$. Then $\fr q=(y_0,y_1)\in\Pro B$ but its image under the map $\Spec B\to\Spec A$ is not in $\Pro A$. Namely, it is $(x_0,x_1)=A_+$, which is disallowed. Geometrically, one can think of trying to map $\Proj^2\to\Proj^1$ but the point $(0:0:1)$ gives us trouble as it is sent to $(0:0)$.
\end{rem}

\begin{exmp}[Projective space over a ring $R$]
We define
\[\Proj^n_R=\Pro R[x_1,\ldots, x_n]\longrightarrow\Spec R\]
where the grading is the usual one. Since $R[x_0,\ldots, x_n]_+=(x_0,\ldots, x_n)$, by the definition of $\Pro$, we see that
\[\Proj^n_R=D_+(x_0)\cup\cdots\cup D_+(x_n).\]
and
\[D_+(X_i)=\Spec R[\frac{x_0}{x_1},\cdots,\frac{x_n}{x_i}].\]
\end{exmp}

\subsection*{Key facts on affine schemes}

Let $(X,O_X)=(\Spec A,O_{\Spec A})$ be an affine scheme.

\begin{enumerate}
\item $\Gamma(X,O_X)=O_X(X)=A$;
\item $O_X(D(f))=A_f$;
\item For any scheme $(S,O_S)$, we have
\[\text{Mor}_{\text{Schemes}}((S,O_S),(X,O_X))\Leftrightarrow \Hom_{\text{Rings}}(A,O_S(S))\]
given by
$f\mapsto f^\#:A=O_X(X)\to O_S(S)$
\item In particular, the full subcategory (of schemes) of affine schemes is anti-equivalent to the category of rings. Note that this and the above are incorrect if in the definition of morphisms of schemes one does not require ``locally.''
\end{enumerate}

From point 3, let us construct the map in the other direction. Suppose we are given a ring map $\phi: A\to O_S(S)$. We want to construct $f:(S,O_S)\to(X,O_X)$. As a map of sets, let $s\in S$ and consider
\[A\overset{\phi}{\to}O_S(S)\to O_{S,s}\]
where we have $\fr m_s\subset O_{S,s}$ the unique maximal and a prime $\fr p\subset A$. Hence we set $f(s)=\fr p$ to be the inverse image of $\fr m_s$ under this composition. Now why is $f$ continuous? Say $g\in A$. It suffices to show that the inverse image of a basis element is open, i.e. $f^{-1}(D(g))$ is open in $S$. But $f^{-1}D(g)$ is the set of all $s\in S$ such that $g$ is not contained in the inverse image of $m_s$, or equivalently, the global section $\phi(g)$ in $O_S(S)$ does not map into $\fr m_s$.

\begin{lem*}
Let $(T,O_T)$ be a locally ringed space. Let $g\in O_T(T)$. Then
\[T_g=U=\{t\in T\mid g\notin \fr m_t\}\]
is open in $T$ and moreover, there exists a unique $h\in O_T(U)$ such that $g|_U\cdot h=1$.
\end{lem*}
\begin{proof}
Idea: if $t\in U$ then $g$ is invertible in $O_{T,t}$. There exists an $h_t\in O_{T,t}$ such that $gh_t=1$. Then, by definition of stalks, there exists an open neighborhood $U_t$ of $t$ and $h_t\in O_T(U_t)$ such that $g|_{U_t}h_t=1$. Then $U=\cup U_t$ is open, where the union is over $t$ such that $g\notin\fr m_t$. Moreover, by uniqueness of inverses in Rings, and the sheaf condition, the local $h_t$'s glue to an $h$.
\end{proof}

Now we define $f^\#$. For $D(g)\subset \Spec A$ open we let \[f^\#:O_{\Spec A}(D(g))=A_g\to O_S(f^{-1}D(S))\] be the unique ring map extending \[A\overset{\phi}{\to} O_S(S)\to O_S(f^{-1}D(g)).\]
This works because the above lemma tells us that $g$ is mapped to an invertible element.

We have to now check that the above constructions do really give a morphism of locally ringed spaces, and we have to check that we indeed do have a inverse.

\begin{exmp}[Surjective morphism $\mathbb{A}^1_\C\to\Proj_\C^1$]
Take $t\mapsto [t:t^2+1]$ at the level of points. At the level of primes, we take $(0)\mapsto (0)\subset\C[x_0,x_1]$ and take $(x-t)\mapsto ((t^2+1)x_0-tx_1)\subset\C[x_0,x_1]$. At the level of rings, not that $\Proj_\C^1=D_+(x_0)\cup D_+(x_1)$.
\[f^{-1}(D_+(x_0))=D(x)=\Spec\C[x,x^{-1}]\to\Spec \C[\frac{x_1}{x_0}]\]
with $x_1/x_0$ on the right mapped to $(x^2+1)/x$ on the left. Similarly,
\[f^{-1}(D_+(x_1))=D(x^2+1)=\Spec\C[x,1/(x^2+1)]\to D_+(x_1)=\Spec\C[x_0/x_1]\]
where $x_0/x_1$ on the right is mapped to $x/(x^2+1)$ on the left. We claim that this is surjective, as $x/(x^2+1)$ takes on all values while $(x^2+1)/x$ takes on every value but zero.
\end{exmp}

\section*{Class 25}

\begin{exmp}
Consider $A^1_\Z\to\Spec\Z$ where $A^1_\Z=\Spec\Z[x]$. Note that
\[\dim A^1_\Z=2.\]
The points of $A^1_\Z$ are as follows. There is a generic point corresponding to $(0)\subset \Spec \Z[x]$. There are codimension one points corresponding to irreducibles $(f)$ (i.e. primes or irreducible nonconstant polynomials). The closed points correspond to $(p, f)$ where $p$ is a prime number and $f$ is a polynomial which is irreducible modulo $p$.

One has a very similar picture for
\begin{equation*}
\begin{tikzcd}
A^1_{\C[t]}\arrow{r}\arrow{d}{\cong}&\Spec\C[t]\arrow{d}{\cong}\\
A^2_\C\arrow{r}&A_\C^1
\end{tikzcd}
\end{equation*}

Observe that if $Z\subset A^1_\Z$ is closed, the its image in $\Spec\Z$ is either a finite set of closed points or open in $\Spec \Z$. Similarly for the complex case.
\end{exmp}

\begin{exmp}
Consider $A^2_\C\to A^2_\C$ with the map $\C[x,y]\leftarrow\C[x,y]$ given by $xy\leftarrow x,y\leftarrow y$.
``On points,'' $(x,y)\mapsto (xy,y)$.
In the arithmetic case we take $A^1_\Z\overset{g}{\to} A^1_\Z$ and $\Z[x]\leftarrow\Z[x]$ given by $13x\leftarrow x$.
In the geometric case we have $\text{Im}(g)=D(y)\cup\text{closed pt }(0,0)$.
In the arithmetic case, $\text{Im}(g)=D(13)\cup\{(13,x)\}$.
\end{exmp}

\begin{defn}
Let $X$ be a Noetherian topological space. We say $E\subset X$ is \textbf{constructible} if $E$ is a finite union of locally closed subsets (closed of an open).
\end{defn}

\begin{thm}[Chevalley]
Let $f:X\to Y$ be a finite-type morphism of Noetherian schemes. Then the image of a constructible set under $f$ is constructible.
\end{thm}
We will not define these terms here, but one may look at Hartshorne or the Stacks project. The translation into algebra is the following: given $R\to S$ a finite-type ring map of Noetherian rings, then $\Spec S\to\Spec R$ sends constructibles to constructibles.

\begin{lem}
Let $R$ be a Noetherian ring. $E\subset\Spec R$ is constructible if and only if $E$ is a finite union of subsets of the form $D(f)\cap V(g_1,\ldots, g_m)$ with $f,g_1,\ldots,g_m\in R$.
\end{lem}

\begin{proof}
The right-to-left statement is clear. Hence we assume that $E$ is locally closed, i.e. $E=U\cap V^c$ where $U,V\subset\Spec R$ are open. Then $U=D(f_1)\cup\ldots\cup D(f_n)$ and $V=D(g_1)\cup\ldots\cup D(g_m)$ as $\Spec R$ is Noetherian so every subset is quasicompact. Then
\[U\cap V^c=(D(f_1)\cap V(g_1,\ldots,g_m))\cup\cdots\cup (D(f_n)\cap V(g_1,\ldots,g_m))\]
(look this up).
\end{proof}

\begin{lem*}
Let $R$ be a Noetherian ring, $f\in R, S=R_f$. The the result holds.
\end{lem*}
\begin{proof}
In this case $\Spec S\to\Spec R$ is a homeomorphism onto an open, hence locally closed subsets map to locally closed subsets.
\end{proof}

\begin{lem*}
Let $R$ be a Noetherian ring, $I\subset R$ and ideal with $S=R/I$. Then the result holds.
\end{lem*}
\begin{proof}
In this case $\Spec S\to\Spec R$ is a homeomorphism onto an closed subset, hence locally closed subsets map to locally closed subsets.
\end{proof}

\begin{lem*}
Let $R$ be a Noetherian ring. The map $\Spec R[x]\to\Spec R$ is open.
\end{lem*}
\begin{proof}
It is enough to show that the image of $D(f)$ is open in $\Spec R$ for $f\in R[x]$. Suppose $f=a_dx^d+\cdots+a_0,$ for $a_i\in R$. Then we claim that the image of $D(f)$ is $D(a_0)\cup\cdots\cup D(a_d)$ (which is open). Let $\fr p\subset R$ be a prime ideal. Let $\bar f\in\kappa(\fr p)[x]$ be the image of $f$. We claim that
\[(R[x]_f)_{\fr p}/\fr p(R[x]_f)_{\fr p}\cong \kappa(\fr p)[x]_{\bar f}.\]
The claim follows by some work based on the stuff we did with fibers early in the semester. Hence we see that $\fr p$ is in the image if and only if the ring is not zero. In other words $\fr p$ is in the image if and only if $\bar f\neq 0$.
\end{proof}

\begin{lem*}
Let $R$ be a Noetherian ring and $f,g\in R[x]$. Assume the leading coefficient of $g$ is a unit in $R$. Then the image of $D(f)\cap V(g)$ in $\Spec R$ is open.
\end{lem*}
\begin{proof}
Write $g=ux^d+a_{d-1}x^{d-1}+\ldots+a_0$ with $u\in R^\times$. Set $S=R[s]/(g)$. As an $R$-module $S$ is free with basis $1,x,\ldots, x^{d-1}$. Consider multiplication by $f$ on $S$ - this is an $R$-linear map so we get a matrix as well as a characteristic polynomial $P(T)\in R[R]$. We write
\[P(T)=T^d+r_{d-1}T^{d-1}+\cdots+r_0.\]
We claim that the image of $D(f)\cap V(g)$ is $D(r_0)\cup\cdots\cup D(r_{d-1}).$
Suppose $\fr q\in D(f)\cap V(g)$ and $\fr p=R\cap\fr q$. Then there is a map $S_\fr p/\fr pS_\fr p\to\kappa(\fr q)$ compatible with multiplication by $f$. Since $f$ acts as a unit on $\kappa(\fr q)$, we see that $f$ is not nilpotent on $S_\fr p/\fr pS_\fr p$ so $\fr p\in D(r_0)\cup\cdots\cup D(r_{d-1})$ (by the lemma below).

Let us now prove the converse. Suppose $r_i\notin \fr p$ for some $i$. Then multiplication by $f$ is not nilpotent on $S_\fr p/\fr pS_\fr p$. This implies that there exists a maximal ideal $\bar{\fr q}\subset S_\fr p/\fr pS_\fr p$ not containing the image of $f$, which in turn implies that the inverse image $\fr q\subset R[x]$ of $\bar{\fr q}$ is a point of $D(f)\cap V(g)$ mapping to $\fr p$.
\end{proof}

\begin{lem*}
We have that $\fr p\in V(r_0,\ldots, r_{d-1})$ if and only if multiplication by $f$ on $S_\fr p/\fr pS_\fr p$ is nilpotent.
\end{lem*}
\begin{proof}
Omitted. Hint: $S_\fr p/\fr pS_\fr p$ is free with basis $1,x,\ldots,x^{d-1}$ over $\kappa(\fr p)$.
\end{proof}

\begin{thm}[Chevalley's theorem]
Let $R\to S$ be a finite-type ring map. Let $R$ be Noetherian. Then if $E\subset\Spec S$ is constructible then the image of $E$ under the map is constructible.
\end{thm}

\begin{proof}
Take $S=R[x_1,\ldots,x_n]/(f_1,\ldots, f_m)$. Can factor
\[R\to R[x_1]\to R[x_1,x_2]\to\ldots\to R[x_1,\ldots,x_n]\twoheadrightarrow S\]
and so may assume $S=R[x]$ (as the statement has been shown for surjective maps in the lemmas above). We know that $E$ is a finite union of sets of the form $D(f)\cap V(g_1,\ldots,g_m)$ and hence reduce to $E=D(f)\cap V(g_1,\ldots,g_m)$. Note that if $c\in R$ then $\Spec R=D(c)\sqcup V(c)$ and it suffices to show our set intersected with either piece is construcible (by previous lemmas). The claim is that the shape of $E$ does not change under this procedure, i.e. that the image of $(D(f)\cap V(g))\cap D(c)$ is the image of $D(f_c)\cap V(g_c)$ in $\Spec R_c$ where $f_c,g_c\in R_c[x]$ are the images of $f$ and $g$. Similarly for $V(c)$.

We use induction on $m$ and on the degrees of the $g_i$. Let $d$ be the degree of $g_1$ where $g_1$ has leading coefficient $c$. Cut up $R$ as above. On the $R_c$ part we get that $g_1$ has invertible leading coefficient, while on the $R/c$ part, the degree of $g_1$ drops so we are done by induction. For the localization case, we can lower the degrees of $g_2,\ldots, g_m$ if they are bigger or equal to $d$ ($V(g_1,\ldots, g_m)=V(g_1,g_2-hg_1,\ldots)$) again we are done if some $g_i$ has degree $\geq d$. If not I swap $g_1,g_2$ and go back to start. (The two base cases were $D(f)\cap V(g)$ and $D(f)$ and were done in the lemmas)
\end{proof}

\section*{Class 26}

The homework over winter break is to read Hartshorne Chapter II, Sections 1,2,3,4, and to do some of the contained exercises.

\begin{defn}
A morphism of schemes $f:X\to Y$ is \textbf{proper} if it is of finite-type, separated, and universally closed (i.e. closed after any base change).
\end{defn}

Today we will discuss this definition for $\Proj_S^n\to S$.

\begin{prop*}
For any scheme $S$ the projection morphism $\Proj_S^n\to S$ is closed.
\end{prop*}
\begin{proof}
We have immediate reduction to the case $S=\Spec R$.
Recall that $\Proj_R^n=D_+(X_0)\cup\cdots\cup D_+(X_n)$ where $D_+(X_i)=\Spec R_i$ where $R_i=R[x_0/x_i,\ldots, x_n/x_i]$.
Set $R_{ij}=(R_i)_{x_j/x_i}=(R_j)_{x_i/x_j}=R[x_0/x_i\ldots, x_n/x_i,x_i/x_j]=R[x_0/x_j,\ldots, x_n/x_j,x_j/x_i]$ where we take the degree zero part.
Let $Z\subset \Proj_R^n$ be a closed subset. Then $Z\cap D_+(X_i)=V(I_i)$ for a unique radical ideal $I_i\subset R_i$.
Then $(Z\cap D_+(X_i))\cap D_+(X_j)=Z\cap D_+(X_iX_j)=(Z\cap D_+(X_j))\cap D_+(X_i)$.
Thus we see by the correspondence between the radicals ideals of $R_{ij}$ and the closed subsets of $\Spec R_{ij}$ that $I_iR_{ij}=I_jR_{ij}$ for all $i\neq j$.
Note that closed subschemes of $\Proj_R^n$ correspond exactly to ideals $I_i\subset R_i$ (not necessarily radical) such that the just-stated correspondence holds.

Now set $I\subset R[x_0,\ldots, x_n]$ to be the graded ideal generated by homogeneous elements $F$ of positive degree such that $F/X_i^{\deg F}\in I_i$ for $i=0,\ldots, n$. If we denote  $V_+(F)=\Proj_R^n\setminus D_+(F)$, this condition is equivalent to saying that $F\in I\iff Z\subset V_+(F)$.

Let $\fr p\in\Spec R$ not in the image of $Z$ under $\Proj_R^n\to\Spec R$. This implies that for all $i=0,\ldots,n$, $(R_i/I_i)_{\fr p}/\fr p(R_i/I_i)_{\fr p}=0=(R_i)_{\fr p}/((I_i)_\fr p+\fr p(R_i)_{\fr p})$, which in turn implies (by clearing denominators) that for all $i=0,\ldots, n$, there exist $g_i\in R\setminus\fr p$ such that $f_i\in I_i$,
\[g_i=f_i+\sum_i a_{i,t}+f_{i,t}\]
where $a_{i,t}\in\fr p,f_{i,t}\in R_i$. By the second lemma below (and homogenizing) we find that
\[g_ix_i^N=F_i+\sum_i a_{i,t}F_{i,t}\]
with $F_i\in I_N$ and $F_{i,t}\in (R[X_0,\ldots, X_n])_N$ where this notation denotes the $N$th graded part of the objects. Then $X_i^N\in (I_N)_{\fr p}+\fr p(R[x_0,\ldots,x_n]_N)_{\fr p}$. Then $(R[x_0,\ldots,x_n]_{(n+1)N})_{\fr p}=(I_{(n+1)N})_{\fr p}+\fr p(R[X_0,\ldots, X_n)]_{(n+1)N})_{\fr p}$. The left hand side becomes $(R_{\fr p}[X_0,\ldots X_n])_{(n+1)N}$, which is finite and free as $R_{\fr p}$-modules on the basis of the monomials. By Nakayama's lemma, we see that $(R[X_0,\ldots,X_n]_{(n+1)N})_{\fr p}=(I_{(n+1)N})_{\fr p}$, which implies that $h_iX_i^{(n+1)N}\in I_{(n+1)N}$ for some $h_i\in R\setminus\fr p$.
Aside: this means that $Z\subset V_+(h_iX_i^{(n+1)N})=V(h_iX_i)$ (uniquely).
This in turn implies that the image of $Z$ to $\Spec R$ is disjoint from $D(h_0)\cap\cdots\cap D(h_n)=D(h_0\cdots h_n)$ which is an open neighborhood of $\fr p$. Hence $V_+(X_0)\cap\cdots\cap V_+(X_n)=\varnothing$, and we are done.
\end{proof}

\begin{lem*}
If $f\in R_i$ and the image of $f$ is in $I_iR_{ij}$, then $X_j/X_i\cdot f\in I_i$.
\end{lem*}
\begin{proof}
$f\in I_iR_{ij}$ means that $f$ vanishes on $Z\cap D_+(X_iX_j)$. Then $X_i/X_j\cdot f$ vanishes on $Z\cap D_+(X_iX_j)$ and $V_+(X_j)\cap D_+(X_i)=V(X_j/X_i)\subset D_+(X_i)=\Spec R_i$. So $X_j/X_i$ vanishes on $Z\cap D_+(X_i)$, so $X_j/X_i\cdot f\in I_i$.
\end{proof}

\begin{lem*}
If $F$ is homogeneous of positive degree, and $F/X_i^{\deg F}\in I_i$ for some $i$, then $X_iF\in I$.
\end{lem*}
\begin{proof}
Immediate consequence of the lemma above.
\end{proof}

\end{document}

