\documentclass{../../mathnotes}

\usepackage{tikz-cd}
\usepackage{amsmath}
\usepackage{todonotes}
\usetikzlibrary{decorations.pathmorphing}


\title{Lecture 1: Motivating abstract nonsense}
\author{Nilay Kumar}
\date{May 28, 2014}


\begin{document}

\maketitle

This summer, the UMS lectures will focus on the basics of category theory. Rather dry and substanceless
on its own (at least at first), category theory is difficult to grok without proper motivation. With the proper
motivation, however, category theory becomes a powerful language that can concisely express and abstract away
the relationships between various mathematical ideas and idioms. This is not to say that it is purely a tool
-- there are many who study category theory as a field of mathematics with intrinsic interest, but this is
beyond the scope of our lectures.

\subsection*{Functoriality}

Let us start with some well-known mathematical examples that are unrelated in content but similar in ``form''.
We will then make this rather vague similarity more precise using the language of categories.

\begin{exmp}[Galois theory]\hfill\\
    Recall from algebra the notion of a finite, Galois field extension $K/F$ and its associated Galois group $G=\Gal(K/F)$.
    The fundamental theorem of Galois theory tells us that there is a bijection between subfields $E\subset K$
    containing $F$ and the subgroups $H$ of $G$. The correspondence sends $E$ to all the elements of
    $G$ fixing $E$, and conversely sends $H$ to the fixed field of $H$. Hence we draw the following diagram:
    \begin{equation*}
        \begin{tikzcd}
            K\ar[leftrightsquigarrow]{r} & 0\ar[hook]{d}\\
            E\ar[hook]{u}\ar[leftrightsquigarrow]{r} & H\ar[hook]{d}\\
            F\ar[hook]{u}\ar[leftrightsquigarrow]{r} & G
        \end{tikzcd}
    \end{equation*}
    I denote the Galois correspondence by squiggly arrows instead of the usual arrows. Unlike most bijections
    we usually see between say elements of two sets (or groups, vector spaces, etc.), the fundamental theorem gives
    us a one-to-one correspondence between two very different types of objects: fields and groups. This is
    precisely why Galois theory is so useful; we can reduce difficult questions about, say, solvability by
    radicals of polynomials over a field, to possibly simpler questions about the solvability of a group, \emph{without
    losing any structure}.
\end{exmp}

\begin{exmp}[The fundamental group]\hfill\\
    In topology it is often useful to be able to distinguish between various topological spaces. This drives
    the study of topological invariants: data associated to homeomorphism (or homotopy) classes of spaces.
    
    One of the simplest examples is the fundamental group $\pi_1(X,x_0)$, the group of all ``homotopy classes of loops''
    in $X$ based at $x_0$. A loop is a continuous map $\gamma:S^1\to X$, and two loops $\gamma,\delta$ are said to be homotopic if there
    is a continuous map $F:S^1\times [0,1]\to X$ such that $F(x,0)=\gamma$ and $F(x,1)=\delta$. Hence, as a set,
    $\pi_1(X,x_0)$ is the set of all loops in $X$ upto continuous deformation. Indeed, $\pi_1(X,x_0)$ can be given a 
    group structure. The product of two loops is the loops obtained by traversing each in turn, and so the identity
    is the loop given by $S^1\mapsto \{pt\}$. Let's look at a few examples.

    Let $X=\R^n$. It's pretty clear that for any choice of basepoint $x_0\in X$ any two loops can
    be continuously deformed to each other, i.e. there is only one homotopy class of loops. Thus,
    $\pi_1(X,x_0)=0$, the trivial group. The first non-trivial example is the circle, $S^1$. Intuitively, one sees
    that a loop wrapping around the circle once cannot be continuously deformed to a loop wrapping around the circle
    twice. And indeed, it can be shown (although it takes some work) that $\pi_1(S^1,x_0)\cong\Z$.
    With this result in hand, it is not hard to compute fundamental groups of other run-of-the-mill spaces.

    This formalism is quite useful; it is easy to see that if $X\cong Y$ are homeomorphic spaces then
    $\pi_1(X,x_0)\cong\pi_1(Y,y_0)$ (where $x_0$ is identified with $y_0$).\footnote{In fact, more is true.
    If $X$ and $Y$ are only homotopy equivalent, i.e. there exist $f:X\to Y$ and $g:Y\to X$ such that
    $g\circ f$ is homotopic to $\id_X$ and $f\circ g$ is homotopic to $\id_Y$, then they have the same fundamental
    group.} 
    Thus any two spaces with different fundamental groups cannot be homeomorphic!
    I want to pause here for a moment and note that we have assigned to each topological space a group. This assignment
    is in fact well-defined on homeomorphism classes of spaces, which is why the fundamental group is useful as
    an invariant. Hence we reduce a topological problem of distinguishing two spaces to a problem of computing (and
    comparing) their fundamental groups, a task that can often be reduced to algebra.
    Compare this to the case of Galois theory above: the fundamental group, like the fundamental theorem
    of Galois theory, associates objects from two very different worlds (``categories'') in some structure-preserving
    (``functorial'') way that turns out to be quite useful.
\end{exmp}

Let us now make the similarity between the two examples above more precise. We define a \textbf{category} to consist
of ``objects'' and ``morphisms,'' where every object has a distinguished identity morphism, and morphisms can be composed 
in the usual way. We've already seen a few examples: there is the category $\catname{Top}$ with topological spaces as
objects and continuous maps as morphisms, the category $\catname{Grp}$ with groups as objects and group homomorphisms
as morphisms, and the category $\catname{Field}$ with fields as objects and ring homomorphisms (i.e. field extensions)
as morphisms. Even simpler is the category $\catname{Set}$ with (you guessed it) sets as objects and set maps as morphisms.

But the idea of a category is old hat to the point where it seems almost trivial -- we're used to restricting our domain
of objects of interest to the objects of a single category. What's actually interesting about the above examples can be neatly
encapsulated in the following definition. A \textbf{functor} $F:\mathcal{C}\to\mathcal{D}$ from a category $\mathcal{C}$ to a category
$\mathcal{D}$ is a map that takes every object $c\in\mathcal{C}$ to an object $F(c)$ in $\mathcal{D}$ \emph{and} every morphism
$\alpha:c_1\to c_2$ of objects $c_1,c_2\in\mathcal{C}$ to a morphism $F\alpha:F(c_1)\to F(c_2)$ in $\mathcal{D}$.

In this sense, the fundamental theorem of Galois theory provides us with two functors: let $K/F$ be Galois.
Let $\mathcal{L}$ be the category whose objects are intermediate fields between $F$ and $K$ and whose morphisms
are field extensions. Let $\mathcal{G}$ be the category whose objects are subgroups of $G=\Gal(K/F)$ and morphisms
are inclusions. Then there the fundamental theorem provides us with a functor $S:\mathcal{L}\to \mathcal{G}$ taking
a field extension $E$ of $F$ to the Galois group $\Gal(K/E)$ and taking the inclusion $F\hookrightarrow E$ to the
inclusion $\Gal(K/E)\hookrightarrow \Gal(K/F)$. Moreover, we have a functor $S':\mathcal{G}\to \mathcal{L}$
taking a subgroup $H\subset G$ to the fixed field $K^H$ and the inclusion $H\hookrightarrow G$ to the inclusion
$F\hookrightarrow K^H$. Note that these functors are inclusion-reversing -- we call such functors \textbf{contravariant}.
The two functors $S$ and $S'$ in this case are ``inverses'' in a sense (that we will not elaborate on today), which
gives Galois theory its power.

Now consider the fundamental group. It takes a topological space $X$ together with a fixed point $x_0\in X$ to
a group. Consider the pair $(X,x_0)$, a so-called pointed topological space. We write the category of pointed topological
spaces (together with continuous maps $f:(X,x_0)\to(Y,y_0)$ with $f(x_0)=y_0$) cleverly as $\catname{Top.}$. We thus
think of the fundamental group as a functor $\pi_1:\catname{Top.}\to \catname{Grp}$.\footnote{As mentioned in the previous
    footnote, $\pi$ cannot distinguish between homotopy equivalent spaces, and hence
    one can think of it as a functor $\pi_1:\catname{Toph.}\to\catname{Grp}$ from homotopy types of pointed topological spaces
    to groups.}
It is not hard to check that
a continuous map $f:(X,x_0)\to (Y,y_0)$ induces a homomorphism $f_*:\pi_1(X,x_0)\to \pi_1(Y,y_0)$. Note that the order
of domain/codomain is preserved, unlike in the case of Galois theory; we say that $\pi_1$ is a \textbf{covariant} functor.
Note that unlike in the case of Galois theory, there is no functor going the other way, which is why the fundamental group
is not a complete topological invariant: there exist non-homeomorphic spaces with isomorphic fundamental group.

\subsection*{Mathematical reuse}

Before I wrap up this lecture, let me briefly discuss another nice feature of categorical language. Often when one
is learning about or working with new, unfamiliar objects, it is not clear why certain constructions deserve the name
that they do. Ravi Vakil offers the following example:
\begin{quote}
    For example, we will define the notion of \emph{product} of schemes. We could just give a definition of product,
    but then you should want to know why this precise definition deserves the name of ``product''\dots We will be creating
    some new mathematical objects (such as schemes, and certain kinds of sheaves), and we expect them to act like objects
    we have seen before.
\end{quote}
The point is that category theory allows us to say precisely what it means to ``act like'' a product object, for example.
Indeed, one can define a product of two objects in some category $\mathcal{C}$ via the following \textbf{universal property of the product}:
an object $X$ is the product of $X_1$ and $X_2$, denoted $X_1\times X_2$, if and only if there exist morphisms $\pi_1:X\to X_1$,
$\pi_2:X\to X_2$ such that for every object $Y$ and pair of morphisms $f_1:Y\to X_1,f_2:Y\to X_2$, there exists a unique morphism
such that the following diagram commutes:
\begin{equation*}
    \begin{tikzcd}
        \;& Y\ar[swap]{dl}{f_1}\ar{dr}{f_2}\ar[dashed]{d}{f} &\\
        X_1 & X_1\times X_2\ar{l}{\pi_1}\ar[swap]{r}{\pi_2} & X_2
    \end{tikzcd}
\end{equation*}
In line with the usual intuition of product objects, the morphism $f$ is called the product of $f_1$ and $f_2$ and
$\pi_1,\pi_2$ are called projection morphisms. You can check, in your favorite category, that this does indeed agree with the
definition of product that you may be used to. The beautiful thing about such an abstract definition is that it captures
the ``idea'' of what a product is. Because it's a definition via the ``behavior'' of an object, the universal property
can be applied in any category.\footnote{It turns out that not all categories have products (consider, say, $\catname{Field}$),
but in those that do, the object defined by the universal property is the unique such object up to unique isomorphism.
In other words, the universal property guarantees uniqueness but not existence. This is a general fact about universal properties
that follows from some rather formal manipulation.}

This example of using a universal property as a definition exemplifies the type of ``reuse'' that category theory affords.
There are several, more drastic examples of reuse. Take, for example, \textbf{abelian categories}. Such categories generalize
the notion of the category $\catname{Ab}$ of abelian groups. Any abelian category has, among other things, a zero object, 
products and coproducts, and kernels and cokernels. This generalization is useful because many interesting categories
(such as the category of modules over a ring or the category of sheaves of abelian groups on a topological space) are abelian:
one can prove general results for an arbitrary abelian category, instead of rederiving such results in isolation when working
in any given category. This turns out to be very convenient and quite powerful in algebraic topology and geometry;
this is the subject of homological algebra.

\subsection*{Categories and beyond}

There's a whole lot to category theory, as we will see in this seminar, but I hope that these examples give you
an introductory overview of how category theory captures patterns and constructions that arise frequently in mathematics.

\begin{thebibliography}{9}

    \bibitem{lamport94}
        Ravi Vakil,
        \emph{Foundations of Algebraic Geometry}.
        June 11, 2013 version.

\end{thebibliography}


\end{document}
