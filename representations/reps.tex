\documentclass{../mathnotes}

\title{Representations of Finite Groups}
\author{Nilay Kumar}
\date{Last updated: \today}


\begin{document}

\maketitle

\section{Actions of groups on sets}

\begin{defn}
    Let $G$ be a group, and let $X$ be a set. An \textbf{action} of $G$ on $X$ is a map from the Cartesian product $G\times X$ to $X$ sending
    $(g,x)$ to $gx$ and satisfying the following two conditions:
    \begin{align}
        gh\cdot x=g\cdot hx \nm{and} 1\cdot x = x 
        \label{eq:actioncond}
    \end{align}
    for all $g,h\in G$ and $x\in X$. Here 1 is the identity element of $G$.
\end{defn}

\begin{thm}
    Given an action of $G$ on $X$, each of the maps $\pi_g: x\to gx$ from $X$ to $X$ is a permutation of $X$ (i.e. a bijection from $X$ to $X$),
    and the map $g\to\pi_g$ is a homomorphism from $G$ to $S_X$, the symmetric group of all permutations of $X$.

    Conversely, given any homomorphism $\pi:g\to\pi_g$ from $G$ to $S_x$ we get an action of $G$ on $X$ by putting $gx=\pi_g(x)$ for $g\in G,x\in X$.
\end{thm}
\begin{proof}
    The conditions on group actions in Eq.~(\ref{eq:actioncond}) yield
    \begin{align}
        \label{eq:actionhomo1}
        \pi_{gh}(x)&=\pi_g(\pi_h(x)) \nm{and} \pi_1(x)=x\\
        \pi_{gh}&=\pi_g\circ\pi_h \nm{and} \pi_1=\id_X
        \label{eq:actionhomo2}
    \end{align}
    for all $g,h\in G$ and $x\in X$. It follows, then, that
    \begin{align*}
        \pi_g\circ\pi_g^{-1}=\id_X
    \end{align*}
    for all $g\in G$, which implies that each $\pi_g$ is surjective and each $\pi_g^{-1}$ is injective. Replacing $g$ by $g^{-1}$, it follows
    that each $\pi_g$ is bijective, i.e. each $\pi_g\in S_x$. Eq.~(\ref{eq:actionhomo1}) shows that $\pi:g\to\pi_g$ is a homomorphism from $G$ to $S_X$.
    
    For the converse, note that each homomorphism $\pi:g\to \pi_g$ from $G$ to $S_X$ satisfies Eq.~(\ref{eq:actionhomo2}), i.e. Eq.~(\ref{eq:actionhomo1}).
    Putting $gx=\pi_g(x)$, we clearly get an action of $G$ on $X$.
\end{proof}

\begin{rem}
    In short, an action of $G$ on $X$ ``is'' a homomorphism $\pi: G\to S_X$.
\end{rem}

\begin{defn}
    Given an action of a group $G$ on a set $X$, for each $x\in X$, $Gx=\left\{ gx:g\in G \right\}$ is the \textbf{orbit} of $x$ and
    $G_x=\left\{ g\in G:gx=x \right\}$ is the \textbf{stabilizer} of $x$.
\end{defn}

\begin{exc}
    Prove that for each action of $G$ on $X$ and $x, x_1, x_2\in X$,
    \begin{enumerate}
        \item $x\in Gx$
        \item either $Gx_1\cap Gx_2=\varnothing$ or $Gx_1=Gx_2$
    \end{enumerate}
\end{exc}
\begin{proof}
    To prove the first, simply note that the identity in $G$ takes $x$ to itself, so $1\cdot x=x$ is in $Gx$.

    For the second, we have two cases - $Gx_1$ and $Gx_2$ have either some or no points in common. If they have no points in common, they are disjoint.
    Otherwise, if there are some points in common, we have
    \begin{align}
        gx_1=hx_2
        \label{eq:ex1}
    \end{align}
    for at least one pair $g,h\in G$. To show that $Gx_1=Gx_2$, let us show that $Gx_1\subset Gx_2$ and $Gx_2\subset Gx_1$. Multiplying both sides by $g^{-1}$, we have
    \begin{align*}
        x_1=g^{-1}hx_2.
    \end{align*}
    If we now act on both sides with any $k\in G$, we get
    \begin{align*}
        kx_1=kg^{-1}hx_2.
    \end{align*}
    Since the right side of this equation is some element of $Gx_2$, we have shown that any element of the orbit of $x_1$ is an element in $Gx_2$; in other words,
    $Gx_1\subset Gx_2$. To show that $Gx_2\subset Gx_1$, we simply multiply both sides of Eq.~(\ref{eq:ex1}) by $h^{-1}$ and proceed in a similar fashion as above.
    Consequently, either the two orbits share points, in which case they are identically equal, or they do not, i.e. they are disjoint, and we are done.
\end{proof}

\begin{thm}
    For each action of a group $G$ on a set $X$, the disjoint orbits \textbf{partition} $X$, i.e. the orbits are nonempty, pairwise disjoint, and they cover $X$.
\end{thm}
\begin{proof}
    This follows from both parts of the previous exercise.
\end{proof}

\begin{thm}
    For each action of a group $G$ on a set $X$, each stabilizer is a subgroup of a $G$, and for each $x$ there is a bijection $Gx\to G/G_x$ from the orbit
    of $x$ to the set of left cosets of $G_x$ in $G$.
\end{thm}
\begin{cor}
    For each action of a finite group $G$ on a set $X$, and each $x\in X$, we have $|Gx|=(G:G_x)$, i.e. the size of the orbit of $x$ is the index of the stabilizer
    of $x$.
\end{cor}
\begin{proof}
    First, let us show that $G_x$ is a subgroup of $G$. Then, for $g,h\in G_x$, we know that $gx=x$ and $hx=x$
    \begin{align*}
        gh\cdot x=g\cdot hx=gx=x,
    \end{align*}
    which means $gh\in G_x$. Note also that the inverse of $g\in G_x$ is also in $G_x$:
    \begin{align*}
        g^{-1}x=g^{-1}\cdot gx=g^{-1}g\cdot x=1\cdot x=x.
    \end{align*}
    Finally, the identity element is obviously in $G_x$, since $1x=x$.

    To prove the existence of the bijection, we take $x\in X$ and $g,g'\in G$,
\end{proof}

\end{document}
