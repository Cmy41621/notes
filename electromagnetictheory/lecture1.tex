\section{Lecture 1}

\subsection{Radiation from a moving charge}

Given a particle trajectory $\vec{S}(t)$, we wish to examine the (fully relativistic case) of radiation from a moving charge in the radiation zone. Let us define:
\begin{align*}
    \vec{\beta(t)}&=\frac{\vec{v}(t)}{c}\\
    \kappa(t)&=1-\hat{r}\cdot \vec{\beta}(t)\\
    \nm{Detection time}&=t\\
    \nm{Retarded time} &= t'=t-\frac{r}{c}+\frac{\hat{r}\cdot\vec{S}(t')}{c}\\
\end{align*}
Note that this yields $t'$ as a function of $t, r, \theta, \phi$. Although this can be hard to solve,there is a theorem that for $v/c\leq1$ there is only one solution
for $t'$. There is then the \textbf{Lienard-Wiechert radiation field}, along with the power emitted/detected:
\begin{align*}
    \vec{E}(\vec{x}, t)&=\frac{q}{r c^2}\frac{\hat{r}\times\left[ \left( \hat{r}-\vec{\beta}(t') \right)\times \vec{a}(t') \right]}{\left( 1-\hat{r}\cdot\vec{\beta}(t') \right)^3}\\
    \frac{dP(t)}{d\Omega}=\frac{d^2\mathcal{E}}{dt d\Omega}&=\frac{\nm{energy detected}}{\nm{(detection time)(solid angle)}}\\
    \frac{dP'(t')}{d\Omega}=\frac{d^2\mathcal{E}}{dt' d\Omega}&=\frac{\nm{energy emitted}}{\nm{(emission time)(solid angle)}}\\
    \frac{d^2\mathcal{E}}{dt d\Omega}&=\frac{d^2\mathcal{E}}{dt'd\Omega}\left( \frac{dt'}{dt} \right)
\end{align*}
Since t is defined relative to $t'$ as above, we can find that
\begin{align*}
    \frac{dt}{dt'}=1-\hat{r}\cdot\frac{\dot{\vec{S}}(t')}{c}=\kappa(t')
\end{align*}
Thus, we arrive at the result
\begin{equation*}
    \frac{dP'(t')}{d\Omega}=\frac{dP(t)}{d\Omega}\kappa(t')
\end{equation*}
Furthermore, we have that
\begin{align*}
    \frac{dP(t)}{d\Omega}=r^2(\vec{S}\cdot\hat{r})=\frac{cr^2}{4\pi}\vec{E}\cdot\vec{E}=\frac{q^2}{4\pi c^3}\frac{\big|\hat{r}\times\left[ (\hat{r}-\vec{\beta}(t'))\times\vec{a}(t') \right]\big|^2}{\kappa^6(t')}
\end{align*}
Thus we reach our final result that
\begin{align*}
    \frac{dP'(t')}{d\Omega}=\frac{q^2}{4\pi c^3}\frac{\big|\hat{r}\times\left[ (\hat{r}-\vec{\beta}(t'))\times\vec{a}(t') \right]\big|^2}{\kappa^5(t')}
\end{align*}

In the non-relativistic limit, we have
\begin{align*}
    \frac{v}{c}&\ll1\\
    \kappa&\approx 1\\
    (\hat{r}-\vec{\beta})&=\hat{r}\\
    \vec{E}(\vec{x}, t)=\frac{q}{rc^2}&\left[ \hat{r}\times\left( \hat{r}\times \vec{a}(t') \right) \right]
\end{align*}
where the last equation is evaluated at $t=t'$. We now have:
\begin{align*}
    \frac{dP'(t')}{d\Omega}&=\frac{dP(t)}{d\Omega}\\
    &=\frac{q^2}{4\pi c^3}\big|\hat{r}\times\left( \hat{r} \times \vec{a} \right)\big|\\
    &=\frac{q^2}{4\pi c^3}\big|\vec{a}\big|^2\sin^2\Theta
\end{align*}
i.e. the power emitted is precisely the power detected. We then have that
\begin{equation*}
    P(t)=\int \frac{dP(t)}{d\Omega} d\Omega=\frac{2q^2}{3c^2}\big|\vec{a}\big|^2
\end{equation*}
which is the \textbf{Larmor formula}.

Let us now do a relativistically correct example. Let the velocity be parallel or antiparallel to the acceleration. Let us choose the velocity to be along the $\hat{z}$ direction and
thus the acceleration to go as either $\hat{z}$ or $-\hat{z}$. Note that $\vec{\beta}\times\vec{a}=0$ and $\kappa=1-\hat{r}\cdot\vec{\beta}=1-\frac{v}{c}\cos\theta$. With these, we can now write the
power emitted per solid angle.
\begin{align*}
    \frac{dP'(t')}{d\Omega}&=\frac{q^2}{4\pi c^3}\frac{\big|\hat{r}\times(\hat{r}\times \vec{a})\big|^2}{\left( 1-\frac{v}{c}\cos\theta \right)^5}\\
    %=\frac{q^2}{4\pi c^3}\frac{|\vec{a}|^2\sin^2\theta}{\left( 1-\frac{v}{c}\cos\theta \right)^5}\\
    &=\frac{q^2\big|\vec{a}(t')\big|^2\sin^2\theta}{4\pi c^3\left( 1-\frac{v(t')}{c}\cos\theta \right)^5}
\end{align*}
Of course, this formula is independent of $\phi$, as there is azimuthal symmetry in the way we set up the velocity and acceleration.
For the exact result, $\frac{dP'(t')}{d\Omega}$ is a maximum when $\theta=\theta_{\rm max}$, with
\begin{equation*}
    \cos\theta_{\rm max}=\frac{1}{3\beta}\left[ \sqrt{1+15\beta^2}-1 \right]
\end{equation*}
For the non-relativistic case, with $\beta\ll1$, it turns out that the cosine goes to zero and $\theta_{\rm max}=\pi/2$. For the highly relativistic case, on the other hand, (after expanding in powers of
$\gamma^{-1}$) we can find that $\theta_{\rm max}\approx 1/2\gamma$.
If we integrate over the $4\pi$ solid angle to compute the total power emitted (left as an exercise), we find that
\begin{align*}
    P'(t')=\frac{2q^2}{3c^3}\big|\vec{a}\big|^2\gamma^6,
\end{align*}
which is the relativistic generalization of the Larmor formula for this example.

Let us now move on to the case where $\vec{a}\perp\vec{v}$. Let the velocity be along $\hat{z}$ and the acceleration be along $\hat{x}$. It is clear that this set up breaks azimuthal symmetry,
and thus we will obtain an answer dependent on both $\theta$ and $\phi$. We still have that $\kappa=1-v/c \cos\theta$, with
\begin{align*}
    \frac{dP'(t')}{d\Omega}&=\frac{q^2}{4\pi c^3}\frac{\big|\hat{r}\times\left( (\hat{r}-\vec{\beta})\times \vec{a}\right)\big|^2}{\left( 1-\frac{v}{c}\cos\theta \right)^5}\\
    &=\frac{q^2}{4\pi c^3}\frac{\big|(\hat{r}-\vec{\beta})(\vec{a}\cdot \hat{r})-\vec{a}(\hat{r}\cdot(\hat{r}-\vec{\beta}))\big|^2}{\left( 1-\frac{v}{c}\cos\theta \right)^5}\\
    &=\frac{q^2}{4\pi c^3}\frac{\big|(r-\beta)|\vec{a}|\sin\theta\cos\phi-\vec{a}\kappa\big|^2}{\left( 1-\frac{v}{c}\cos\theta \right)^5}\\
    &=\frac{q^2\big|\vec{a}(t')\big|^2}{4\pi c^3}\frac{\left[ 1-\frac{\sin^2\theta\cos^2\theta}{\gamma^2(1-\beta\cos\theta)^2} \right]}{\left(1-\beta\cos\theta\right)^3}
\end{align*}
where we have used the triple vector product identity. In the highly relativistic case, this formula shows that the radiation peaks in the forward direction about $\theta=\pi/2$,
unlike for the previous example, where the radiation directly forward was zero (due to the $\sin^2\theta$ in the numerator). Note that this case is what we call \textbf{synchrotron radiation},
as the particle is undergoing circular motion in
the $xz$ plane. In this sense, for the highly relativistic particle in a synchrotron, there is radiation observed roughly tangentially to the circle (roughly, due to the non-trivial
$\phi$ dependence -- there is a slight tilt inwards, towards $\vec{a}$). The angular spread in this radiation is of order $1/\gamma$. This is the so-called \textbf{flashlight effect}.
This is the cause of the ``blue ominous light'' that one hopes not to be detecting by eye; one would see quick flashes of light in short succession, all of which are identical. Of course, one has
to distinguish between the emitted and the detected radiation, but this is the rough idea.

\subsection{Frequency distribution of detected radiation}

Suppose we have a particle moving in some fashion, localized in a finite region of space. We arbitrarily choose an origin in this region, and enclose the region in a detector. Furthermore,
let us place a frequency filter in front of our detector that allows detection of one frequency in range $[\omega,\omega+d\omega]$. What we wish to do is to Fourier analyze
$\vec{E}(\vec{x}, t)$ in the time variable.
\begin{align*}
    \vec{E}(\vec{x}, t)=\int_{-\infty}^{\infty} \frac{d\omega}{\sqrt{2\pi}} e^{-i\omega t}\vec{E}_\omega(\vec{x})
\end{align*}
Unfortunately, we are out of time, so we shall continue this topic next lecture.
